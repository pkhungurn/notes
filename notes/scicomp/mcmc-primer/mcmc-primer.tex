\documentclass[10pt]{article}
\usepackage{fullpage}
\usepackage{amsmath}
\usepackage[amsthm, thmmarks]{ntheorem}
\usepackage{amssymb}
\usepackage{graphicx}
\usepackage{enumerate}
\usepackage{verse}
\usepackage{tikz}
\usepackage{verbatim}
\usepackage{hyperref}

\newtheorem{lemma}{Lemma}
\newtheorem{theorem}[lemma]{Theorem}
\newtheorem{definition}[lemma]{Definition}
\newtheorem{proposition}[lemma]{Proposition}
\newtheorem{corollary}[lemma]{Corollary}
\newtheorem{claim}[lemma]{Claim}
\newtheorem{example}[lemma]{Example}

\def\sc#1{\dosc#1\csod}
\def\dosc#1#2\csod{{\rm #1{\small #2}}}

\newcommand{\dee}{\mathrm{d}}
\newcommand{\Dee}{\mathrm{D}}
\newcommand{\In}{\mathrm{in}}
\newcommand{\Out}{\mathrm{out}}
\newcommand{\pdf}{\mathrm{pdf}}
\newcommand{\Cov}{\mathrm{Cov}}
\newcommand{\Var}{\mathrm{Var}}

\newcommand{\ve}[1]{\mathbf{#1}}
\newcommand{\mrm}[1]{\mathrm{#1}}
\newcommand{\ves}[1]{\boldsymbol{#1}}
\newcommand{\etal}{{et~al.}}
\newcommand{\sphere}{\mathbb{S}^2}
\newcommand{\modeint}{\mathcal{M}}
\newcommand{\azimint}{\mathcal{N}}
\newcommand{\ra}{\rightarrow}
\newcommand{\mcal}[1]{\mathcal{#1}}
\newcommand{\X}{\mathcal{X}}
\newcommand{\Y}{\mathcal{Y}}
\newcommand{\Z}{\mathcal{Z}}
\newcommand{\x}{\mathbf{x}}
\newcommand{\y}{\mathbf{y}}
\newcommand{\z}{\mathbf{z}}
\newcommand{\tr}{\mathrm{tr}}
\newcommand{\sgn}{\mathrm{sgn}}
\newcommand{\diag}{\mathrm{diag}}
\newcommand{\Real}{\mathbb{R}}
\newcommand{\sseq}{\subseteq}
\newcommand{\ov}[1]{\overline{#1}}
\DeclareMathOperator*{\argmax}{arg\,max}
\DeclareMathOperator*{\argmin}{arg\,min}

\title{A Primer on Markov Chain Monte Carlo Algorithms}
\author{Pramook Khungurn}

\begin{document}
\maketitle

This note is a primer on Markov chain Monte Carlo (MCMC) algorithms. I take materials from \cite{Andrieu:2003}, \cite{Gareth:2004}, and \cite{Neal:2011}.

\section{Introduction}

\begin{itemize}
  \item We are given an unnormalized density function $\pi_u$ defined on a measurable space $(\Omega,\Sigma)$ such that $0 < \int_\Omega \pi_u(x)\, \dee x < \infty$.
  
  \item The above density gives rise to a probability measure $P$, which is given by
  \begin{align*}
    P(A) = \frac{\int_A \pi_u( x)\, \dee x}{\int_\Omega \pi_u( x)\, \dee x}
  \end{align*}
  for any $A \in \Sigma$. The corresponding probability density function is
  \begin{align*}
    \pi( x) = \frac{\pi_u( x)}{\int_\Omega \pi_u( x)\, \dee x}.
  \end{align*}
  We collectively call $P$ and $\pi$ the {\bf target distribution}.

  \item MCMC algorithms allow us to perform the following two following tasks:
  \begin{itemize}
    \item Sample elements from $\Omega$ according to $P$.
    \item Compute an estimate of 
    \begin{align*}
    \pi(f) 
    = E_{ x \sim \pi}[f( x)]
    = \int_\Omega f( x) \pi( x)\, \dee  x
    = \frac{\int_\Omega f( x) \pi_u( x)\, \dee  x}{\int_\Omega \pi_u( x)\, \dee x}.
    \end{align*}
  \end{itemize}

  \item There are a number of motivations for doing this.
  \begin{itemize}
    \item In statistictical mechanics, the probability density of state $ x$ is given by
    \begin{align*}
      \pi( x) = \frac{1}{Z} \exp\bigg(-\frac{E( x)}{kT}\bigg)
    \end{align*}
    where $E( x)$ is the Hamiltonian of $s$, $k$ is the Boltzmann's constant, and $T$ is the temperature of the system. The constant $Z$ is called the {\bf partition function}, and it is given by
    \begin{align*}
      Z = \int_{\Omega} \exp\bigg( -\frac{E( x)}{kT} \bigg)\, \dee  x.
    \end{align*}    
    We see that this setting is exactly the same as the one we just discuss after taking $\pi_u( x) = e^{-E( x)/(kT)}$. So, MCMC is useful for computing expectations arising from these probabilities.

    \item In Bayesian parameter estimation, we are given data $x$, and we want to estimate parameter $\theta$ of our model, which includes the how to compute the prior $p(\theta)$, and the likelihood $p( x|x)$. The posterior is given by
    \begin{align*}
      p(\theta | x) = \frac{p(x|\theta)p(\theta)}{\int p(x|\theta)p(\theta)\, \dee \theta}
    \end{align*}
    In many cases, we also want to compute expectations such as
    \begin{align*}
      E_{\theta \sim p(\theta|x)}[f(\theta)] 
      = \int f(\theta)p(\theta|x)\, \dee \theta
      = \frac{\int f(\theta)p(x|\theta)p(\theta)\, \dee\theta}{\int p(x|\theta)p(\theta)\, \dee \theta}.
    \end{align*}
    So, if we let $\Theta$ be the set of the $\theta$'s, then we may set $\Omega = \{ (\theta,x) : \theta \in \Theta \}$, and $\pi_u(\theta,x) = p(x|\theta)p(\theta)$ be our unnormalized density.
  \end{itemize}

  \item In the above situations, what prevents us from computing probabilities and integrals is the difficulty of computing the integral $\int_\Omega \pi_u( x)\, \dee x$.
  
  \item MCMC gives a way to sample $ x$ according to the target distribution without explicity normalizing $\pi_u( x)$. The idea is to constructs a Markov chain $\{ X_i \}_{i=0}^\infty$ whose stationary distribution is the target distribution $\pi(x)$. To do so, we need to find a transition kernel $K$, whose density is $k$, such that
  \begin{align*}
    \pi(y) = \int_\Omega \pi(x) k(x, y)\, \dee x.
  \end{align*}
  
  \item Then, we can run the Markov chain for a long time (starting from somewhere). When $n$ is large, the distribution of $X_n$ will be approximately $\pi$. We can now restart the Markov chain from $X_n$, run the chain for $M$ iterations to obtain $M$ samples $x_1$, $ x_2$, $\dotsc$, $ x_M$, and then compute the estimate
  \begin{align*}
    \int_\Omega f( x) \pi( x)\, \dee x \approx \widehat{\pi}(f) = \frac{1}{M} \sum_{i=1}^M f( x_i).
  \end{align*}
  This is an unbiased estimate, and the error $(\pi(f) - \widehat{\pi}(f))/\sqrt{M}$ has Gaussian distribution. We have this property despite the fact that the samples are correlated.\footnote{\url{https://en.wikipedia.org/wiki/Markov_chain_central_limit_theorem}}

  \item We say that a Markov chain $\{ X_i \}_{i=0}^\infty$ with kernel density function $k$ is {\bf reversible} with respect to probability density function $\pi$ if
  \begin{align} \label{eqn:detailed-balance}
    \pi(x) k(x,y) &= \pi(y) k(y,x) 
  \end{align}
  for all $x,y \in \Omega$. The condition \eqref{eqn:detailed-balance} is called {\bf detailed balance}.  

  \item Note that when $\pi$ satisfies detailed balance, we have that $\pi$ is a stationary distribution. This is because
  \begin{align*}
    \int_\Omega \pi(x) k(x, y)\, \dee x = \int_\Omega \pi(y)k(y, x)\, \dee x = \pi(y) \int_\Omega k(y, x)\, \dee x = \pi(y).
  \end{align*}

  \item Hence, the goal would be to find a kernel that satisfies detailed balance with respect to the target distribution. The most popular way to do so is the Metropolis--Hastings algorithm.
\end{itemize}

\section{The Metropolis--Hastings Algorithm} \label{sec:metropolis-hastings}

\begin{itemize}
  \item The algorithm presupposes a Markov chain over $\Omega$. Let $Q$ denote its transition kernel, and let $q$ denote the transition kernel's density. This density can be unnormalized, just like $\pi_u$, but we require a way to sample $y$ according to $q(x,y)$ for any $x$. We often call $q$ the {\bf proposal distribution}.
  
  \item The algorithm proceeds as follows.
  \begin{enumerate}
    \item Choose some $X_0$ as the starting point.
    \item Given $X_n$, generates a proposal $Y_{n+1}$ according to $q(X_n, \cdot)$.
    \item Compute $\alpha(X_n, Y_{n+1})$ where
    \begin{align*}
      \alpha(x,y) = \min\bigg( 1, \frac{\pi_u(y)q(y,x)}{\pi_u(x)q(x,y)} \bigg)
    \end{align*}
    is the {\bf acceptance probability}.
    Here, if $\pi(x)q(x,y) = 0$, we set $\alpha(x,y) = 1$.
    \item Sample $\xi$ uniformly from $[0,1)$.
    \item If $\xi < \alpha(x,y)$, then set $X_{n+1} = Y_{n+1}$. Otherwise, set $X_{n+1} = X_n$.
    \item Repeat Step 2 to Step 5 until you are satisfied.
  \end{enumerate}

  \item The algorithm creates a Markov chain whose transition density is given by $k(x,y) = q(x,y)\alpha(x,y)$. 
  
  \item To see that $\pi$ is the stationary distribution of this above Markov chain, we need to show that $k$ satisfies detailed balance with respect to $\pi$. In other words,
  \begin{align*}
    \pi(x)k(x,y) = \pi(y)k(y,x).
  \end{align*}
  To see this, let $\pi(x) = C^{-1} \pi_u(x,y)$ where $C = \int_\Omega p_u(x)\, \dee x$ is the normalizing constant. We have that
  \begin{align*}
    \pi(x) k(x,y)
    &= \pi(x) q(x,y) \alpha(x,y) \\
    &= C^{-1}\pi_u(x) q(x,y) \min\bigg( 1, \frac{\pi_u(y)q(y,x)}{\pi_u(x)q(x,y)} \bigg)\\
    &= C^{-1} \min\bigg( \pi_u(x) q(x,y) , \pi_u(y)q(y,x) \bigg).    
  \end{align*}
  Notice that the expression on the RHS is symmetric in $x$ and $y$. Hence, we have that $\pi(x)k(x,y) = \pi(y)k(y,x)$ as required.

  \item The main question for employing the Metropolis--Hastings algorithm is how to choose the proposal distribution $q(\cdot, \cdot)$.
  
  \item Of course, $q$ needs to be such that the Markov chain is irreducible and aperiodic so that it would converge to the stationary distribution regardless of where we start.
    
  \item Note, though, that transition kernel $k$ allows for rejection, so the chain is aperiodic by default. As a result, we only need to make sure that it covers the whole support of $\pi$. In other words, if $\pi_u(x) > 0$ and $\pi_u(y) > 0$, then $q(x,y)$ should be greater than $0$ too.  

  \item There are many approaches to construct the proposal distribution.
  \begin{itemize}
    \item {\bf Symmetric Metropolis algorithm.} Here, $q$ is chosen such that $q(x,y) = q(y,x)$. The acceptance probability simplifies to
    \begin{align*}
      \alpha(x,y) = \min\bigg( 1, \frac{\pi_u(y)}{\pi_u(x)} \bigg).
    \end{align*}
    One way to arrange for this is to use $y \sim \mcal{N}(x, \sigma^2)$.

    \item {\bf Random walk Metropolis algorithm.} Here, $q(x,y) = q(y-x)$. For example, we may have $y \sim \mrm{Uniform}(x-1, x+1)$.
    
    \item {\bf Independence sampler.} Here, $q(x,y) = q(y)$. The acceptance probability thus reduces to
    \begin{align*}
      \alpha(x,y) 
      = \min\bigg( 1, \frac{\pi_u(y)q(x)}{\pi_u(x)q(y)} \bigg)
      = \min\bigg( 1, \frac{w(y)}{w(x)} \bigg)
    \end{align*}
    where $w(x) = \pi_u(x)/q(x)$ for all $x$.

    \item {\bf Langevin dynamics.} The proposal distribution is given by
    \begin{align*}
      \ve{y} \sim \mcal{N}\Big(\ve{x} + \frac{\sigma}{2} \nabla \log \pi_u(\ve{x}), \sigma^2 I \Big) 
    \end{align*}
    for some small $\sigma > 0$.    
  \end{itemize}

  \item Heuristics on the standard deviation of the proposal distribution.
  \begin{itemize}
    \item If it is too low (i.e., the proposal distribution is too narrow), then only a few modes of the target distribution $\pi(\cdot)$ would be visited.
    \item If it is too wide, then there's a high chance of new samples being rejected, resulting in high correlation between samples.
  \end{itemize}

  \item If all modes of the target distribution is visited while the acceptance probability is high, we say that the chain {\bf mixes well}.
\end{itemize}

\section{Combining Markov Chains}

\subsection{Mixture MCMC}

\begin{itemize}
  \item If transition kernel density $k_1$ and $k_2$ both result in the same stationary distribution $\pi$, then so does the kernel $$\nu k_1 + (1 - \nu)k_2$$ for any $0 \leq \nu \leq 1$.
  
  \item The update step of the MCMC algorithm that uses the mixed kernel would be as follows:
  \begin{itemize}
    \item[] Sample $\xi \sim \mrm{Uniform}(0,1)$.
    \item[] {\bf if} $\xi < \nu$ {\bf then}
    \item[] $\qquad$ Generate the next state $X_{n+1}$ using kernel $k_1$.
    \item[] {\bf else}
    \item[] $\qquad$ Generate the next state $X_{n+1}$ using kernel $k_2$.
    \item[] {\bf end if}
  \end{itemize}

  \item Mixing more than two kernels is, of course, possible.

  \item Mixture MCMC is useful when the target distribution has many narrow peaks. In this case, we can have a ``global'' kernel that has large variance and a ``local'' kernel that has small variance. The global proposal would jump between peaks, and the local proposals would allow one to explore the space around each peak.
\end{itemize}

\subsection{Cycle MCMC}

\begin{itemize}
  \item If we have multiple transition kernel densities $k_1$, $k_2$, $\dotsc$, $k_m$ that have $\pi$ as the stationary distribution of their Markov chains, then the Markov chain obtained by applying the kernels in the round robin fashion, would also have $\pi$ as its stationary distribution.
  
  \item Another use of this type of MCMC algorithm is when a sample $\ve{x}$ can be divided into $m$ blocks $\ve{x} = (\ve{x}^{(1)} | \ve{x}^{(2)} | \dotsb | \ve{x}^{(m)})$. If the kernel $k_i$ only change the $\ve{x}^{(i)}$ block, then applying all the kernels in succession would give us a kernel that change all components of the sample, and this can be engineered to have $\pi$ as the statinoary distribution.
  
  \item The update step of the cycle MCMC algorithm be as follows:
  \begin{itemize}
    \item[] Use $k_1$ to get a state $Y^{(1)}_{n+1}$ from $X_n$.
    \item[] Use $k_2$ to get a state $Y^{(2)}_{n+1}$ from $Y^{(1)}_{n+1}$.
    \item[] $\qquad \vdots$
    \item[] Use $k_m$ to get a state $Y^{(m)}_{n+1}$ from $Y^{(1)}_{n+1}$. 
    \item[] Set $X_{n+1}$ to $Y^{(m)}_{n+1}$.
  \end{itemize}

  \item The expression for the overall kernel $k$ would be
  \begin{align*}
    k(x,y) = \int \int \dotsm \int k_1(x, x^{(1)})k_2(x^{(1)}, x^{(2)}) \dotsb k_m(x^{(m-1)}, y)\, \dee x^{(1)}\dee x^{(2)} \dotsm \dee x^{(m)}.
  \end{align*}
  However, if $k_i$ only changes the $i$th block independent of the other blocks, then we can write
  \begin{align*}
    k(\ve{x},\ve{y}) = \prod_{i=1}^m k_i(\ve{x}^{(i)},\ve{y}^{(i)}).
  \end{align*}
  Weirdly, all the papers that I read write the overall kernel as
  \begin{align*}
    k = k_1 k_2 \dotsb k_m.
  \end{align*}
\end{itemize}

\section{The Gibbs Sampler}

\begin{itemize}
  \item Suppose that $\Omega = \Real^d$. So, we denote an elements of $\Real^d$ by $\ve{x} = (x_1, x_2, \dotsc, x_d)$.
  
  \item Let $\ve{x}_{-i}$ denote $\ve{x}$ with the component $x_i$ removed. In other words,
  \begin{align*}
    \ve{x}_{-i} = (x_1, \dotsc, x_{i-1}, x_{i+1}, \dotsc, x_d) = (\ve{x}[1:i-1], \ve{x}[i+1:d]).
  \end{align*}

  \item Suppose that, in addition to an expression for $\pi_u(\ve{x})$, we are also given, for each $i$, an algorithm for sampling $x_i$ given all the other components (i.e., given $\ve{x}_{-i}$) according to the conditional probability distribution
  \begin{align*}
    \pi(x_i | \ve{x}_{-i}) = \frac{\pi_u(\ve{x})}{\int \pi_u(\ve{x}[1:i-1], x, \ve{x}[i+1,d])\, \dee x}.
  \end{align*}
  
  \item For further derivativation, let us denote the marginal distribution of $\ve{x}_{-i}$ by
  \begin{align*}
    \pi(\ve{x}_{-i}) = \frac{1}{C} \pi_u(\ve{x}_{-i}) = \frac{1}{C} \int \pi_u(\ve{x}[1:i-1], x, \ve{x}[i+1,d])\, \dee x.
  \end{align*}
  where $C = \int \pi_u(\ve{x}')\, \dee \ve{x}'$ is the normalizing constant. It follows that $$\pi(\ve{x}) = \pi(x_i|\ve{x}_{-i}) \pi(\ve{x}_{-i}),$$ which is just the standard Bayes' rule. Moreover, $$\pi_u(\ve{x}) = C\pi(\ve{x}) = C \pi(x_i|\ve{x}_{-i}) \pi(\ve{x}_{-i}) = \pi(x_i|\ve{x}_{-i}) \pi_u(\ve{x}_{-i}).$$
  
  \item The $i$th component Gibbs sampler is a kernel $k_i$ that leaves all components besides $x_i$ unchanged and replaces $x_i$ by a sample drawn from the above expression. In other words, the proposal distribution $q_i(\ve{x},\ve{y})$ is defined only when $\ve{x}_{-i} = \ve{y}_{-i}$, and it is given by
  \begin{align*}
    q(\ve{x},\ve{y}) = \pi(y_i | \ve{x}_{-i}).
  \end{align*}

  \item When used with the Metropolis--Hastings algorithm, the acceptance probability is
  \begin{align*}
    \alpha_i(\ve{x}, \ve{y}) 
    &= \min\bigg( 1, \frac{\pi_u(\ve{y}) q_i(\ve{y},\ve{x}) }{\pi_u(\ve{x})q_i(\ve{x},\ve{y}) } \bigg)
    = \min\bigg( 1, \frac{\pi_u(\ve{y}) \pi(x_i|\ve{y}_{-i}) }{\pi_u(\ve{x}) \pi(y_i|\ve{x}_{-i}) } \bigg) \\
    &= \min\bigg( 1, \frac{\pi_u(\ve{y}_{-i}) \pi(y_i|\ve{y}_{-i}) \pi(x_i|\ve{y}_{-i}) }{ \pi_u(\ve{x}_{-i}) \pi(x_i|\ve{x}_{-i}) \pi(y_i|\ve{x}_{-i}) } \bigg)
  \end{align*}
  Again, this is defined only when $\ve{x}_{-i} = \ve{y}_{-i}$. So,
  \begin{align*}
    \alpha_i(\ve{x}, \ve{y}) 
    = \min\bigg( 1, \frac{\pi_u(\ve{x}_{-i}) \pi(y_i|\ve{x}_{-i}) \pi(x_i|\ve{x}_{-i}) }{ \pi_u(\ve{x}_{-i}) \pi(x_i|\ve{x}_{-i}) \pi(y_i|\ve{x}_{-i}) } \bigg)
    = 1.
  \end{align*}
  This means that $k_i(\ve{x},\ve{y}) = q_i(\ve{x},\ve{y}) = \pi(y_i|\ve{x}_{-i})$ given that $\ve{x}_{-i} = \ve{y}_{-i}$.

  \item With the above property, we have that, if $\ve{x}_{-i} = \ve{y}_{-i}$, 
  \begin{align*}
    \pi(\ve{x}) k_i(\ve{x},\ve{y}) 
    &= \pi(x_i|\ve{x}_{-i}) \pi(\ve{x}_{-i}) \pi(y_i|\ve{x}_{-i}) 
    = \pi(x_i|\ve{y}_{-i}) \pi(\ve{y}_{-i}) \pi(y_i|\ve{y}_{-i})
    = \pi(x_i|\ve{y}_{-i}) \pi(\ve{y}) \\
    &= \pi(\ve{y}) k_i(\ve{y},\ve{x}).
  \end{align*}
  So, the kernel $k_i$ is reversiable with respect to the conditional probability distribution $\pi(\ve{x}|\ve{x}_{-i})$.
  
  \item The above derivation implies that $k_1$, $k_2$, $\dotsc$, $k_d$ can be combined to obtain a kernel $k$ that has $\pi$ as the stationary distribution. There are at least two ways to combine them.
  \begin{itemize}
    \item The {\bf deterministic-scan Gibbs sampler} is
    \begin{align*}
      k = k_1 k_2 \dotsm k_d.
    \end{align*} 

    \item The {\bf random-scan Gibbs sampler} is
    \begin{align*}
      k = \frac{k_1 + k_2 + \dotsb + k_d}{d}.
    \end{align*}
  \end{itemize}
  We note that the random-scan Gibbs sampler is reversible, but the deterministic-scan one is not. Moreover, for the deterministic scan one, $X_n$ is usable only when $d$ divides $n$.
\end{itemize}


\section{Hamiltonian Monte Carlo}

\begin{itemize}
  \item Hamiltonian Monte Carlo (HMC) is an algorithm that uses the gradient of the target distribution to create proposals, which allows a new proposal $Y_{n+1}$ to be far away from $X_n$ while still attaining high acceptance probability.
  
  \item The algorithm uses ideas from analytical mechanics and statistical physics.
\end{itemize}

\subsection{Hamiltonian Dynamics}

\begin{itemize}  
  \item In analytical mechanics, the state of a system can be described by its {\bf position vector} $\ve{x} \in \Real^d$ and its {\bf momentum vector} $\ve{p} \in \Real^d$. 
  
  \item The set of all $(\ve{x},\ve{p})$ pairs are called the {\bf phase space}. 
  
  \item The {\bf Hamiltonian}, denoted by $H(\ve{x},\ve{p})$, is a scalar function of $\ve{x}$ and $\ve{p}$. It is normally defined to be the total energy of the system: the sum of {\bf potential energy} $U(\ve{x},\ve{p})$ and {\bf kinetic energy} $K(\ve{x},\ve{p})$.
  \begin{align*}
    H(\ve{x},\ve{p}) = U(\ve{x},\ve{p}) + K(\ve{x},\ve{p}).
  \end{align*}

  \item However, for our use case, we shall require that the potential energy depends only on $\ve{x}$, so we can write $U(\ve{x})$ instead of $U(\ve{x},\ve{p})$. 
  
  \item We will also use a specific form of kinetic energy
  \begin{align*}
    K(\ve{x},\ve{p}) = K(\ve{p}) = \frac{1}{2} \ve{p}^T M^{-1} \ve{p}
  \end{align*}
  where $M$ is a symmetric positive-definite matrix called the {\bf mass matrix}. The mass matrix is typically diagonal and a scalar multiple of the identity matrix: $M = mI$. In such a case, we have that
  \begin{align*}
    K(\ve{p}) = \frac{\| \ve{p}\|^2}{2m},
  \end{align*}
  which agrees with the standard definition of kinetic energy in physics (i.e., $K = m\|\ve{v}\|^2/2$).
  
  \item Given the Hamiltonian, the dynamics of the system is determined by {\bf Hamilton's equations}:
  \begin{align*}
    \frac{\dee x_i}{\dee t} &= \frac{\partial H}{\partial p_i}, \\
    \frac{\dee p_i}{\dee t} &= -\frac{\partial H}{\partial x_i}.
  \end{align*}
  Hamilton's equations are equivalent to Newton's laws, given that the energies included in the Hamiltonian account for all the forces in system.

  \item Plugging in $H(\ve{x},\ve{p}) = U(\ve{x}) + \| \ve{p}^2 \|/(2m)$, we have
  \begin{align*}
    \frac{\dee x_i}{\dee t} &= \frac{p_i}{m}, \\
    \frac{\dee p_i}{\dee t} &= -\frac{\partial U}{\partial x_i}.
  \end{align*}

  \item For any $t \geq 0$, Hamilton's equations give rise to the ``time evolution'' function $\mcal{E}_t: \Real^{2d} \rightarrow \Real^{2d}$ where $\mcal{E}_t((\ve{x},\ve{p}))$ gives the state of the system after letting it evolve according to Hamilton's equations for duration $t$, starting at state $(\ve{x},\ve{p})$. 

  \item Hamiltonian dynamics has several nice properties.
  \begin{itemize}
    \item It is reversible in the sense that $$\mcal{E}_t((\ve{x},\ve{p})) = (\ve{x}',\ve{p}') \iff \mcal{E}_t((\ve{x}',-\ve{p}')) = (\ve{x},-\ve{p}).$$
    In other words, we can run the dynamics backward by starting at the final position and negating the momentum.
    \begin{itemize}
      \item This property implies that the time evolution function $\mcal{E}_t$ for any fixed $t$ is a bijection. The inverse function is given by
      \begin{align*}
        \mcal{E}^{-1}_t((\ve{x},\ve{p})) = \mrm{flip}(\mcal{E}_t(\mrm{flip}((\ve{x},\ve{p}))))
      \end{align*}
      where $\mrm{flip}$ is a function that negates the momentum of the input: $\mrm{flip} (\ve{x},\ve{p}) = (\ve{x},-\ve{p}).$
    \end{itemize}
    
    \item The Hamiltonian does not change in time as the result of the dynamics. In other words,
    \begin{align*}
      \mcal{E}_t((\ve{x},\ve{p})) = (\ve{x}',\ve{p}') \implies H(\ve{x},\ve{p}) = H(\ve{x}',\ve{p}').
    \end{align*}
    If we imagine the phase space as being partition into level sets of $H$, we have that Hamiltonian dynamics preserves them: a point is always mapped to another point in the same level set.
    
    \item It perserves volume in phase space. In other words, $A$ and $B$ be Lebesgue-measurable set in the phase space, and let $v$ denote the Lebesgue (i.e., volume) measure. Then, by abusing the notation a little, we have that
    \begin{align*}
      \mcal{E}_t(A) = B \implies v(A) = v(B).
    \end{align*}
    \begin{itemize}
      \item This property can be restated as
      \begin{align*}
        \det(D\mcal{E}_t) = 1
      \end{align*}
      where $D$ is the derivative opeartor, and so $D\mcal{E}_t$ is the Jacobian of $\mcal{E}_t$.
    \end{itemize}
  \end{itemize}

  \item The last property has an important implication when we use Hamiltonian dynamics as a way to create proposals in the Metropolis--Hastings algorithm.

  \item To discuss the above implication, however, let us recall a fact regarding transformations and probability densities. Let $\pi$ be a probability density defined on the phase space, and let $f: \Real^{2d} \times \Real^{2d}$ be a bijection on the phase space. Consider the following random process:
  \begin{itemize}
    \item Sample $(\ve{x},\ve{p})$ according to $\pi$.
    \item Compute $(\ve{x}',\ve{p}') \gets f((\ve{x},\ve{p})).$
  \end{itemize}
  If we let $\pi'$ denote the probability density of the resulting $(\ve{x}',\ve{p}')$, it follows that
  \begin{align*}
    \pi'((\ve{x}', \ve{p}')) = \frac{\pi((\ve{x},\ve{p}))}{|\det(Df((\ve{x},\ve{p})))|}.
  \end{align*}
  You can find more details about this at \cite{Khungurn:2022}.

  \item Now, if we replace the bijection $f$ with $\mcal{E}_t$, we have that $\pi'((\ve{x}',\ve{p}')) = \pi(\ve{x},\ve{p})$ because $\det(D\mcal{E}_t) = 1$. In other words, when we transform a point in phase space using time evolution according to Hamiltonian dynamics, we do not need to include any correction factors in our calculation.
\end{itemize}

\subsection{The HMC Algorithm}

\begin{itemize}
  \item The HMC algorithm uses time evolution according to Hamiltonian dynamics to create proposals in the Metropolis--Hastings algorithm. In other words, we want to use $\mcal{E}_t$ to construct the proposal distribution $q$.
  
  \item However, the construction is not that straightforward. To do so, we must solve the following problems.
  \begin{enumerate}
    \item[(a)] Recall that we are given only $\pi_u$, which is a function of $\ve{x}$. How should we define $\ve{p}$? Also, what would be the probability distribution of $(\ve{x},\ve{p})$?
    
    \item[(b)] How exactly should we define the proposal distribution $q(\cdot, \cdot)$?
    
    \item[(c)] How do we compute $\mcal{E}_t$? In other words, how do we simulate Hamiltonian dynamics?
  \end{enumerate}
\end{itemize}

\subsubsection{Where Does the Momentum Come From?}

\begin{itemize}
  \item To design an MCMC algorithm, we first imagine the stationary distribution that we want the Markov chain to be converge to.

  \item Here, we borrow an idea from statistical mechanics. When a system with state $\ve{x}$ of energy $E(\ve{x})$ is connected to a heat bath of temperature $T$, the probability density of observing a state $\ve{x}$ is given by the {\bf canonical distribution}
  \begin{align*}
    \pi(\ve{x}) = \frac{1}{Z}\exp\bigg( \frac{-E(\ve{x})}{k_B T} \bigg)
  \end{align*}
  where $k_B$ is the Boltzmann's constant, and $\ve{Z}$ is the partition function
  \begin{align*}
    Z = \int \exp\bigg( \frac{-E(\ve{x})}{k_B T} \bigg)\, \dee\ve{x},
  \end{align*}
  which is a constant needed to normalize the probability density so that it integrates to $1$ over the domain.

  \item For simplicty, let us assume that $k_B T = 1$ from now on.
  
  \item The Hamiltonian $H(\ve{x},\ve{p})$ is the total energy of the system. So, we can just plug it into the canonical distribution to get a joint probability distributon of $\ve{x}$ and $\ve{p}$.
  \begin{align*}
    \pi(\ve{x},\ve{p}) = \frac{1}{Z} \exp( -H(\ve{x},\ve{p}) )
  \end{align*}

  \item Taking $H(\ve{x},\ve{p}) = U(\ve{x}) + K(\ve{p})$, we have that
  \begin{align} \label{eqn:phase-space-prob}
    \pi(\ve{x},\ve{p}) = \frac{1}{Z} \exp(-U(\ve{x})) \exp(-K(\ve{p})). 
  \end{align}
  We can see from this that the marginal probability distributions for $\ve{x}$ and $\ve{p}$ are independent.

  \item What should $U(\ve{x})$ be then? Recall that we are interested in sampling from a probability distribution $\pi(\ve{x})$ that is proportional to $\pi_u(\ve{x})$. We can set the potential energy $U(\ve{x})$ to be $-\log \pi_u(\ve{x})$, which gives
  \begin{align*}
    \exp(-U(\ve{x})) = \exp(-(-\log \pi(\ve{x}))) = \pi_u(\ve{x}).
  \end{align*}
  Now, we can see that the marginal distribution of $\ve{x}$ would be proportional to $\pi_u(\ve{x})$, which means that it would be exactly $\pi(\ve{x})$, our target distribution. 

  \item Equation~\eqref{eqn:phase-space-prob} also gives us a hint where $\ve{p}$ should come from: we can just sample $\ve{p}$ out of nowhere because $\ve{x}$ and $\ve{p}$ are independent of each other!
  
  \begin{itemize}
    \item This is possible because there's nothing in our problem statement that tells us what the distribution of $\ve{p}$ should be, so we can choose a distribution that is easy to sample from.

    \item So, we can pick $K(\ve{p}) = \| \ve{p} \|^2/2m$, which makes that the marginal distribution of $\ve{p}$ the Gaussian distribution $\mcal{N}(\ve{0}, mI)$.
  \end{itemize}

  \item Moreover, because $\ve{x}$ and $\ve{p}$ are independent, we can drop $\ve{p}$ at any time, and we would not lose any information on the distribution of $\ve{x}$.
  
  \item The above reasoning suggests an MCMC algorithm of the following form.
  \begin{itemize}
    \item Start with an arbitrary position $\ve{x}_0$.
    \item {\bf for} $n = 0, 1, 2, \dotsc$ {\bf do}
    \begin{itemize}
      \item Sample $\ve{p}_n$ according to $\mcal{N}(0,mI)$.
      \item Transition $(\ve{x}_n,\ve{p}_n)$ to $(\ve{x}_n',\ve{p}'_n)$ in such a way that satisfies detailed balance with the canonical distribution \eqref{eqn:phase-space-prob}.
      \item Set $\ve{x}_{n+1} \gets \ve{x}'_n$.
    \end{itemize}
    \item[] {\bf end for}
  \end{itemize}

  \item We can see that the Markov chain above would have the target distribution $\pi(\ve{x})$ as the stationary distribution because the transition $(\ve{x}_n,\ve{p}_n) \rightarrow (\ve{x}_n',\ve{p}'_n)$ would preserve $\pi(\ve{x},\ve{p})$.
  
  \item Moreover, because we sample a new momentum in every step and because $\ve{N}(0,mI)$ has infinite support, we can say that the Markov chain is irreducible and aperiodic with respect to $\ve{p}$, which is a good thing.
  \begin{itemize}
    \item On the other hand, if we decide to keep maintain $\ve{p}_n$ along with $\ve{x}_n$, we would have to create some rules to let $\ve{p}_n$ vary through all of its domain, which would make the algorithm somewhat more complicated.
  \end{itemize}

  \item Note that our MCMC algorithm has an interesting behavior. Our Markov chain generate points $\ve{x}_0, \ve{x}_1, \dotsc$ in position space. However, in order to generate the next position, we ``lift'' the position to a random point in phase space. We then transition to another point in the phase space and then ``drop'' the point down to position space again.
\end{itemize}

\subsubsection{How to Define the Proposal Distribution}

\begin{itemize}
  \item Given a position $(\ve{x},\ve{p})$ in phase space, we need to figure out how to transition it to another point $(\ve{x}',\ve{p}')$ in such a way that satisfies detailed balance with the canonical distribution
  \begin{align*}
    H(\ve{x},\ve{p}) = \frac{1}{Z} \exp(-U(\ve{x}))\exp(-V(\ve{p})) = \frac{1}{Z} \pi_u(\ve{x}) \exp\bigg( \frac{-\| \ve{p} \|^2}{2m} \bigg).
  \end{align*}
  
  \item To do so, we employ the Metropolis--Hastings proposal+acceptance step.

  \item The idea of HMC is to fix a duration $t$ and propose $(\ve{x}',\ve{p}') = \mcal{E}_t(((\ve{x},\ve{p}))$.
  
  \item Because $\mcal{E}_t$ is deterministic, we have that the transition kernel would be given by
  \begin{align*}
    q((\ve{x},\ve{p}), (\ve{x}',\ve{p}')) = \delta((\ve{x}',\ve{p}') - \mcal{E}_t((\ve{x},\ve{p})))
  \end{align*}
  where $\delta(\cdot)$ is Dirac's delta function. Note that there's no correction term on the RHS. This is because $\mcal{E}_t$ preserves volume in phase space.

  \item However, this turns out to not be a good idea. Let's look at the acceptance probability
  \begin{align*}
    \alpha((\ve{x},\ve{p}), (\ve{x}',\ve{p}')) = \frac{H(\ve{x}',\ve{p}') q((\ve{x'},\ve{p}'),(\ve{x},\ve{p}))}{H(\ve{x},\ve{p})q((\ve{x},\ve{p}),(\ve{x}',\ve{p}'))}
  \end{align*}
  Taking $(\ve{x}',\ve{p}') = \mcal{E}_t((\ve{x},\ve{p}))$, we would have that the nominator would almost always be $0$ because we cannot expect that we would arrive back at $(\ve{x},\ve{p})$ if we start simulating the dynamics from $(\ve{x}',\ve{p}')$. So, the Metropolis--Hastings algorithm would not accept any proposal!
  
  \item To solve the above problem, we propose $$(\ve{x}',\ve{p}') = \mrm{flip}(\mcal{E}_t((\ve{x},\ve{p}))).$$ In other words, we run the Hamiltonian dynamics, and we simply negate the momentum after we finish.
  
  \item The great thing about the above proposal is that, if we take the end result of the above process $(\ve{x}',\ve{p}')$ and try to get a proposal out of it, we come back to the starting point $(\ve{x},\ve{p})$! Symbolically,
  \begin{align} \label{eqn:flip-evolution-reversibility}
    (\ve{x}',\ve{p}') = \mrm{flip}(\mcal{E}_t((\ve{x},\ve{p}))) \implies (\ve{x},\ve{p}) = \mrm{flip}(\mcal{E}_t((\ve{x}',\ve{p}'))).
  \end{align}
  This is the case because
  \begin{align*}
    \mcal{E}_t^{-1}(\cdot) = \mrm{flip}(\mcal{E}_t(\mrm{flip}(\cdot))).
  \end{align*}
  So,
  \begin{align*}
    \mrm{flip}(\mcal{E}_t((\ve{x}',\ve{p}')))
    = \mrm{flip}(\mcal{E}_t(\mrm{flip}(\mcal{E}_t((\ve{x},\ve{p})))))
    = \mcal{E}_t^{-1}(\mcal{E}_t((\ve{x},\ve{p}))) 
    = (\ve{x},\ve{p}).
  \end{align*}

  \item The transition kernel for $\mrm{flip}(\mcal{E}_t((\ve{x},\ve{p})))$ is
  \begin{align*}
    q((\ve{x},\ve{p}),(\ve{x}',\ve{p}')) = \delta((\ve{x}',\ve{p}') - \mrm{flip}(\mcal{E}_t((\ve{x},\ve{p})))).
  \end{align*}
  Note that there's no scalar correction factor here either. This is because the Jacobian of $\mrm{flip}(\cdot)$ has determinant $(-1)^d$, so the absolute value is $1$.

  \item Now, the acceptance probability becomes
  \begin{align*}
    \alpha((\ve{x},\ve{p}), (\ve{x}',\ve{p}')) 
    &= \min\bigg(1, \frac{H(\ve{x}',\ve{p}') q((\ve{x'},\ve{p}'),(\ve{x},\ve{p}))}{H(\ve{x},\ve{p})q((\ve{x},\ve{p}),(\ve{x}',\ve{p}'))} \bigg) \\
    &= \min\bigg( 1, \frac{H(\ve{x}',\ve{p}') \delta((\ve{x}',\ve{p}') - \mrm{flip}(\mcal{E}_t((\ve{x},\ve{p})))) }{H(\ve{x},\ve{p}) \delta((\ve{x},\ve{p}) - \mrm{flip}(\mcal{E}_t((\ve{x}',\ve{p}')))) } \bigg)\\
    &= \min\bigg( 1, \frac{H(\ve{x}',\ve{p}')}{H(\ve{x},\ve{p})} \bigg).
  \end{align*}
  Here, the last line follows from \eqref{eqn:flip-evolution-reversibility}.

  \item If we can compute $\mcal{E}_t$ perfectly, the acceptance probability would be $1$ because the Hamiltonian would be perfectly preserved. However, this does not happen in real computation, so we keep the $H$ terms there.

  \item So, now, the algorithm is as follows.
  \begin{itemize}
    \item Start the simulation from some $\ve{x}_0$.
    
    \item {\bf for} $n = 0, 1, 2, \dotsc$ {\bf do}
    
    \begin{itemize}
      \item Sample $\ve{p}_n$ from $\mcal{N}(\ve{0}, mI)$.
      
      \item Starting from the phase space point $(\ve{x}_n, \ve{p}_n)$, simulate the for a fixed time $t$ to obtain $(\ve{x}_n^*, \ve{p}^*_n) = \mcal{E}_t((\ve{x}_n, \ve{p}_n))$.
      
      \item Flip the momentum by setting $$(\ve{x}_n', \ve{p}_n') \gets (\ve{x}_n^*, -\ve{p}_n^*).$$      
      
      \item Compute the acceptance propability 
      \begin{align*} 
        \alpha(\ve{x}_n, \ve{x}'_n) = \min\bigg(1 , \frac{\exp(-H(\ve{x}_n', \ve{p}'_n))}{\exp(-H(\ve{x}_n, \ve{p}_n))} \bigg)
      \end{align*}      
      
      \item With probability $\alpha(\ve{x}_n, \ve{x}_n')$, set $\ve{x}_{n+1} \gets \ve{x}_n'$. Otherwise, set $\ve{x}_{n+1} \gets \ve{x}_n$.
    \end{itemize}

    \item[] {\bf end for}
  \end{itemize}
\end{itemize}

\subsubsection{How to Simulate Hamiltonian Dynamics}

\begin{itemize}
  \item We now have to figure out how to simulate Hamiltonian dynamics. There are many ways to do so, including Euler's method, midpoint method, and Runge--Kutta method.
  
  \item The HMC algorithm as discussed in \cite{Neal:2011} uses an integration scheme called {\bf leapfrog} to simulate the Hamiltonian dynamics.
  
  \item The leapfrog method has several nice properties that make it very suitable as a component of an MCMC algorithm. More specifically, let $\mcal{L}_{\Delta t}((\ve{x},\ve{p}))$ denote the result of running the leapfrog method starting from state space point $(\ve{x},\ve{p})$ for a single time step $\Delta t$. We have that:
  \begin{itemize}
    \item The leapfrog method is a second-order method, meaning that, if we simulate the dynamics with step size $\Delta t$, the error would be of order $(\Delta t)^2$. In other words,
    \begin{align*}
      \mcal{L}_{\Delta t}((\ve{x},\ve{p})) = \mcal{E}_{\Delta t}((\ve{x},\ve{p})) + O((\Delta t)^2).
    \end{align*}
    
    \item It is reversible in time in the same exact sense that Hamiltonian dynamics is reversible.
    \begin{align*}
      \mcal{L}_{\Delta t}((\ve{x},\ve{p})) = (\ve{x}',\ve{p}') 
      \implies \mcal{L}_{\Delta t}((\ve{x}',-\ve{p}')) = (\ve{x},-\ve{p}).
    \end{align*}
    
    \item It also preserves volume in the phase space.
    \begin{align*}
      \mcal{L}_{\Delta t}(A) = B \implies v(A) = v(B).
    \end{align*}
  \end{itemize}
  See \cite{Young:2014} for justification of these properties.
  
  \item The last two properties allow us to show that the transition kernel $q$ defined by $\mcal{L}_{\Delta t}$ instead of $\mcal{E}_{\Delta t}$ satisfies the property
  \begin{align*}
    q((\ve{x},\ve{p}),(\ve{x}',\ve{p}')) = \delta((\ve{x}',\ve{p}') - \mrm{flip}(\mcal{L}_{\Delta t}((\ve{x},\ve{p})))).
  \end{align*}
  In other words, the RHS is just the Dirac's delta function without any scalar correction factor. Hence, the acceptance probability is simply
  \begin{align*}
    \alpha((\ve{x},\ve{p}),(\ve{x}',\ve{p}')) = \min\bigg(1, \frac{H(\ve{x}',\ve{p}')}{H(\ve{x},\ve{p})}\bigg)
  \end{align*}

  \item Here, we describe the leapfrog algorithm, running for one timestep of duration $\Delta t$. The input is $(\ve{x}_0,\ve{p}_0)$, and the output will be $(\ve{x}_1,\ve{p})$. The algorithm would first calculate the half way momentum, use it to compute $\ve{x}_1$, and then compute the momentum $\ve{p}_1$. Assuming that $H(\ve{x},\ve{p}) = U(\ve{x}) + K(\ve{p})$, the algorithm goes as follow.
  \begin{align*}
    \ve{p}_{1/2} & \gets \ve{p}_0 - \frac{\Delta t}{2} \frac{\partial U}{\partial \ve{x}}(\ve{x}_0) \\
    \ve{x}_1 & \gets \ve{x}_0 + \Delta t \frac{\partial K}{\partial \ve{p}}(\ve{p}_{1/2}) \\
    \ve{p}_1 & \gets \ve{p}_{1/2} - \frac{\Delta t}{2} \frac{\partial U}{\partial \ve{x}}(\ve{x}_1)
  \end{align*}

  \item However, when running the leapfrog algorithm for multiple time steps, it is not necessary to calculate the momentum at interger indices except for the first and the last one.
  \begin{itemize}
    \item[] Start with input phase space point $(\ve{x}_0, \ve{p}_0)$.
    \item[] Compute $\ve{p}_{1/2} \gets \ve{p}_0 - \frac{\Delta t}{2} \frac{\partial U}{\partial \ve{x}}(\ve{x}_0)$.
    \item[] {\bf for} $n = 0, 1, 2, \dotsc, N-1$ {\bf do}
    \begin{itemize}
      \item[] Compute $\ve{x}_{n+1} \gets \ve{x}_n + \Delta t \frac{\partial K}{\partial \ve{p}}(\ve{p}_{n+1/2}).$
      \item[] Compute $\ve{p}_{n+3/2} \gets \ve{p}_{n+1/2} - \Delta t \frac{\partial U}{\partial \ve{x}}(\ve{x}_{n+1}).$
    \end{itemize}
    \item[] {\bf end for}
    \item[] Compute $\ve{p}_N \gets \ve{p}_{N+1/2} + \frac{\Delta t}{2} \frac{\partial U}{\partial \ve{x}}(\ve{x}_{N})$
    \item[] {\bf return} $(\ve{x}_N, \ve{p}_N)$.
  \end{itemize}
\end{itemize}

\subsubsection{Putting It All Together}

\begin{itemize}
  \item The HMC algorithm is as follows.
  \begin{itemize}
    \item Start the simulation from some $\ve{x}_0$.
    
    \item {\bf for} $n = 0, 1, 2, \dotsc$ {\bf do}
    
    \begin{itemize}
      \item Sample $\ve{p}_n$ from $\mcal{N}(\ve{0}, mI)$.
      
      \item Starting from point $(\ve{x}_n, \ve{p}_n)$, simulate Hamiltonian dynamics with the leapfrog algorithm for $N$ step of size $\Delta t$ to obtain $(\ve{x}_n^*, \ve{p}^*_n)$.
      
      \item Flip the momentum by setting $$(\ve{x}_n', \ve{p}_n') \gets (\ve{x}_n^*, -\ve{p}_n^*).$$      
      
      \item Compute the acceptance propability 
      \begin{align*} 
        \alpha(\ve{x}_n, \ve{x}'_n) = \min\bigg(1 , \frac{\exp(-H(\ve{x}_n', \ve{p}'_n))}{\exp(-H(\ve{x}_n, \ve{p}_n))} \bigg)
      \end{align*}      
      
      \item With probability $\alpha(\ve{x}_n, \ve{x}_n')$, set $\ve{x}_{n+1} \gets \ve{x}_n'$. Otherwise, set $\ve{x}_{n+1} \gets \ve{x}_n$.
    \end{itemize}

    \item[] {\bf end for}
  \end{itemize}

  \item Conceptually, the Markov chain simulates movement of a particle. It above keeps the current position $\ve{x}_n$ of the particle in memory. To get to the text positive, it gives the partical a random kick $\ve{p}_n$ and simulate Hamiltonian dynamics to see where the particle lands after time $N\Delta t$.
\end{itemize}

\section{Langevin Dynamics}

\begin{itemize}
  \item Previously in Section~\ref{sec:metropolis-hastings}, we said that Langevin dynamics is a special case of the Metropolis--Hastings algorithm where the proposal $\ve{y}$ is given by
  \begin{align*}
    \ve{y} \sim \mcal{N}\bigg( \ve{x} + \frac{\sigma}{2} \nabla \log \pi_u(\ve{x}), \sigma^2 I \bigg),
  \end{align*}
  which gives the proposal distribution
  \begin{align*}
    q(\ve{x},\ve{y}) = \frac{1}{\sqrt{2\pi} \sigma} \exp\bigg( - \frac{\ve{y} - \sigma \nabla \log \pi_u(\ve{x})/2}{2\sigma^2} \bigg).
  \end{align*}
  We will look further into this algorithm in this section.
\end{itemize}

\bibliographystyle{apalike}
\bibliography{mcmc-primer}  
\end{document}