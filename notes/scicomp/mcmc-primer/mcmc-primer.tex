\documentclass[10pt]{article}
\usepackage{fullpage}
\usepackage{amsmath}
\usepackage[amsthm, thmmarks]{ntheorem}
\usepackage{amssymb}
\usepackage{graphicx}
\usepackage{enumerate}
\usepackage{verse}
\usepackage{tikz}
\usepackage{verbatim}
\usepackage{hyperref}

\newtheorem{lemma}{Lemma}
\newtheorem{theorem}[lemma]{Theorem}
\newtheorem{definition}[lemma]{Definition}
\newtheorem{proposition}[lemma]{Proposition}
\newtheorem{corollary}[lemma]{Corollary}
\newtheorem{claim}[lemma]{Claim}
\newtheorem{example}[lemma]{Example}

\def\sc#1{\dosc#1\csod}
\def\dosc#1#2\csod{{\rm #1{\small #2}}}

\newcommand{\dee}{\mathrm{d}}
\newcommand{\Dee}{\mathrm{D}}
\newcommand{\In}{\mathrm{in}}
\newcommand{\Out}{\mathrm{out}}
\newcommand{\pdf}{\mathrm{pdf}}
\newcommand{\Cov}{\mathrm{Cov}}
\newcommand{\Var}{\mathrm{Var}}

\newcommand{\ve}[1]{\mathbf{#1}}
\newcommand{\mrm}[1]{\mathrm{#1}}
\newcommand{\ves}[1]{\boldsymbol{#1}}
\newcommand{\etal}{{et~al.}}
\newcommand{\sphere}{\mathbb{S}^2}
\newcommand{\modeint}{\mathcal{M}}
\newcommand{\azimint}{\mathcal{N}}
\newcommand{\ra}{\rightarrow}
\newcommand{\mcal}[1]{\mathcal{#1}}
\newcommand{\X}{\mathcal{X}}
\newcommand{\Y}{\mathcal{Y}}
\newcommand{\Z}{\mathcal{Z}}
\newcommand{\x}{\mathbf{x}}
\newcommand{\y}{\mathbf{y}}
\newcommand{\z}{\mathbf{z}}
\newcommand{\tr}{\mathrm{tr}}
\newcommand{\sgn}{\mathrm{sgn}}
\newcommand{\diag}{\mathrm{diag}}
\newcommand{\Real}{\mathbb{R}}
\newcommand{\sseq}{\subseteq}
\newcommand{\ov}[1]{\overline{#1}}
\DeclareMathOperator*{\argmax}{arg\,max}
\DeclareMathOperator*{\argmin}{arg\,min}

\title{A Primer of Markov Chain Monte Carlo Algorithms}
\author{Pramook Khungurn}

\begin{document}
\maketitle

This note is a primer on Markov chain Monte Carlo (MCMC) algorithms. I take materials from \cite{Gareth:2004} and \cite{Andrieu:2003}.

\section{Introduction}

\begin{itemize}
  \item We are given an unnormalized distributon function $\pi_u$ defined on a measurable space $(\Omega,\Sigma)$ such that $0 < \int_\Omega p_u\, \dee x < \infty$.
  
  \item The above density gives rise to a probability measure $P$, which is given by
  \begin{align*}
    P(A) = \frac{\int_A \pi_u( x)\, \dee x}{\int_\Omega \pi_u( x)\, \dee x}
  \end{align*}
  for any $A \in \Sigma$. The corresponding probability distribution function is
  \begin{align*}
    \pi( x) = \frac{\pi_u( x)}{\int_\Omega \pi_u( x)\, \dee x}.
  \end{align*}

  \item MCMC algorithms allow us to perform the following two following tasks:
  \begin{itemize}
    \item Sample elements from $\Omega$ according to $P$.
    \item Compute an estimate of 
    \begin{align*}
    \pi(f) 
    = E_{ x \sim \pi}[f( x)]
    = \int_\Omega f( x) \pi( x)\, \dee  x
    = \frac{\int_\Omega f( x) \pi_u( x)\, \dee  x}{\int_\Omega \pi_u( x)\, \dee x}.
    \end{align*}
  \end{itemize}

  \item There are a number of motivations for doing this.
  \begin{itemize}
    \item In statistictical mechanics, the probability density of state $ x$ is given by
    \begin{align*}
      \pi( x) = \frac{1}{Z} \exp\bigg(-\frac{E( x)}{kT}\bigg)
    \end{align*}
    where $E( x)$ is the Hamiltonian of $s$, $k$ is the Boltzmann's constant, and $T$ is the temperature of the system. The constant $Z$ is called the {\bf partition function}, and it is given by
    \begin{align*}
      Z = \int_{\Omega} \exp\bigg( -\frac{E( x)}{kT} \bigg)\, \dee  x.
    \end{align*}    
    We see that this setting is exactly the same as the one we just discuss after taking $\pi_u( x) = e^{-E( x)/(kT)}$. So, MCMC is useful for computing expectations arising from these probabilities.

    \item In Bayesian parameter estimation, we are given data $x$, and we want to estimate parameter $\theta$ of our model, which includes the how to compute the prior $p(\theta)$, and the likelihood $p( x|x)$. The posterior is given by
    \begin{align*}
      p(\theta | x) = \frac{p(x|\theta)p(\theta)}{\int p(x|\theta)p(\theta)\, \dee \theta}
    \end{align*}
    In many cases, we also want to compute expectations such as
    \begin{align*}
      E_{\theta \sim p(\theta|x)}[f(\theta)] 
      = \int f(\theta)p(\theta|x)\, \dee \theta
      = \frac{\int f(\theta)p(x|\theta)p(\theta)\, \dee\theta}{\int p(x|\theta)p(\theta)\, \dee \theta}.
    \end{align*}
    So, if we let $\Theta$ be the set of the $\theta$'s, then we may set $\Omega = \{ (\theta,x) : \theta \in \Theta \}$, and $\pi_u(\theta,x) = p(x|\theta)p(\theta)$ be our unnormalized density.
  \end{itemize}

  \item In the above situations, what prevents us from computing probabilities and integrals is the difficulty of computing the integral $\int_\Omega \pi_u( x)\, \dee x$.
  
  \item MCMC gives a way to sample $ x$ according to the normalized version of $\pi_u( x)$ without explicity normalizing $\pi_u( x)$ itself. Given this ability, we may use MCMC to sample $ x_1$, $ x_2$, $\dotsc$, $ x_K$ according to $\pi( x)$ and then compute the estimate
  \begin{align*}
    \int_\Omega f( x) \pi( x)\, \dee x \approx \widehat{\pi}(f) = \frac{1}{K} \sum_{i=1}^K f( x_i).
  \end{align*}
  This is an unbiased estimate, having standard deviation of order $O(1/\sqrt{K})$.

  \item The idea of MCMC is to constructs a Markov chain $\{ X_i \}_{i=0}^\infty$ whose stationary distribution is $\pi(x) = \pi_u(x) / (\int_\Omega \pi_u(x)\, \dee x)$. To do so, we need to find a transition kernel $K$, whose density is $k$, such that
  \begin{align*}
    \pi(y) = \int_\Omega \pi(x) k(x, y)\, \dee x.
  \end{align*}
  Then, we can run the Markov chain for a long time (starting from somewhere). When $n$ is large, the distribution of $X_n$ will be approximately $\pi$. We can then restart the Markov chain from $X_n$ and obtain $K$ values to compute the estimate $\widehat{\pi}(f)$.

  \item We say that a Markov chain $\{ X_i \}_{i=0}^\infty$ with kernel density function $k$ is {\bf reversible} with probability density function $\pi$ if
  \begin{align} \label{eqn:detailed-balance}
    \pi(x) k(x,y) &= \pi(y) k(y,x) 
  \end{align}
  for all $x,y \in \Omega$. The condition \eqref{eqn:detailed-balance} is called {\bf detailed balance}.  

  \item Note that when $\pi$ satisfies detailed balance, we have that $\pi$ is a stationary distribution. This is because
  \begin{align*}
    \int_\Omega \pi(x) k(x, y)\, \dee x = \int_\Omega \pi(y)k(y, x)\, \dee x = \pi(y) \int_\Omega k(y, x)\, \dee x = \pi(y).
  \end{align*}
\end{itemize}

\section{The Metropolis--Hastings Algorithm}

\begin{itemize}
  \item The Metropolis--Hastings algorithm is the most popular MCMC algorithm.

  \item Back to the problem setting, we are given $\pi_u$, which is the unnormalized version of a probability density $\pi$.
  
  \item The Metropolis--Hasting algorithm presupposes a Markov chain over $\Omega$. Let $Q$ denote its transition kernel, and let $q$ denote the transition kernel's density. This density can be unnormalized, just like $\pi_u$, but we require a way to sample $y$ according to $q(x,y)$ for any $x$. We often call $q$ the {\bf proposal distribution}.
  
  \item The algorithm proceeds as follows.
  \begin{enumerate}
    \item Choose some $X_0$ as the starting point.
    \item Given $X_n$, generates a proposal $Y_{n+1}$ according to $q(X_n, \cdot)$.
    \item Compute $\alpha(X_n, Y_{n+1})$ where
    \begin{align*}
      \alpha(x,y) = \min\bigg( 1, \frac{\pi_u(y)q(y,x)}{\pi_u(x)q(x,y)} \bigg)
    \end{align*}
    is the {\bf acceptance probability}.
    Here, if $\pi(x)q(x,y) = 0$, we set $\alpha(x,y) = 1$.
    \item Sample $\xi$ uniformly from $[0,1)$.
    \item If $\xi < \alpha(x,y)$, then set $X_{n+1} = Y_{n+1}$. Otherwise, set $X_{n+1} = X_n$.
    \item Repeat Step 2 to Step 5 until you are satisfied.
  \end{enumerate}

  \item The algorithm creates a Markov chain whose transition density is given by $q^*(x,y) = q(x,y)\alpha(x,y)$. 
  
  \item To see that $\pi$ is the stationary distribution of this above Markov chain, we need to show that
  \begin{align*}
    \pi(x)q^*(x,y) = \pi(y)q^*(y,x).
  \end{align*}
  To see this, let $\pi(x) = C^{-1} \pi_u(x,y)$ where $C = \int_\Omega p_u(x)\, \dee x$ is the normalizing constant. We have that
  \begin{align*}
    \pi(x) q^*(x,y)
    &= \pi(x) q(x,y) \alpha(x,y) \\
    &= C^{-1}\pi_u(x) q(x,y) \min\bigg( 1, \frac{\pi_u(y)q(y,z)}{\pi_u(x)q(x,y)} \bigg)\\
    &= C^{-1} \min\bigg( \pi_u(x) q(x,y) , \pi_u(y)q(y,z) \bigg).    
  \end{align*}
  Notice that the expression on the RHS is symmetric in $x$ and $y$. Hence, we have that $\pi(x)q^*(x,y) = \pi(y)q^*(y,x)$ as required.

  \item The main question for employing the Metroplis--Hastings algorithm is how to choose the propose distribution $q(\cdot, \cdot)$.
  
  \item Of course, $q$ needs to be such that the Markov chain is irreducible and aderiodic so that the Markov chain would converge to the stationary distribution regardless of where we start the chain.
  \begin{itemize}
    \item Note, though, that $q^*$ allows for rejection, so the Markov chain is aperiodic by default. As a result, we only need to make sure that the Markov chain covers the whole set where $\pi(\cdot)$ is defined. In other words, $\pi_u(y)q(y,x) > 0$ for all $x$ and $y$ such that $\pi_u(x) > 0$ and $\pi_u(y) > 0$.
  \end{itemize}

  \item There are many approaches to construct the proposal distribution.
  \begin{itemize}
    \item {\bf Symmetric Metropolis algorithm.} Here, $q$ is chosen such that $q(x,y) = q(y,x)$. The acceptance probability simplies to
    \begin{align*}
      \alpha(x,y) = \min\bigg( 1, \frac{\pi_u(y)}{\pi_u(x)} \bigg).
    \end{align*}
    One way to arrange for this is to use $y \sim \mcal{N}(x, \sigma^2)$.

    \item {\bf Random walk Metropolis algorithm.} Here, $q(x,y) = q(y-x)$. For example, we may have $y \sim \mrm{Uniform}(x-1, x+1)$.
    
    \item {\bf Independence sampler.} Here, $q(x,y) = q(y)$. The acceptance probability thus reduces to
    \begin{align*}
      \alpha(x,y) 
      = \min\bigg( 1, \frac{\pi_u(y)q(x)}{\pi_u(x)q(y)} \bigg)
      = \min\bigg( 1, \frac{w(y)}{w(x)} \bigg)
    \end{align*}
    where $w(x) = \pi_u(x)/q(x)$ for all $x$.

    \item {\bf Langevin algorithm.} The proposal distribution is given by
    \begin{align*}
      \ve{y} \sim \mcal{N}\Big(\ve{x} + \frac{\sigma}{2} \nabla \log \pi_u(\ve{x}), \sigma^2 I \Big) 
    \end{align*}
    for some small $\sigma > 0$.    
  \end{itemize}

  \item Heuristics on the standard deviation of the proposal distribution.
  \begin{itemize}
    \item If it is too low (i.e., the proposal distribution is too narrow), then only a few modes of the target distribution $\pi(\cdot)$ would be visited.
    \item If it is too wide, then there's a high chance of new samples being rejected, resulting in high correlation between samples.
  \end{itemize}

  \item If all modes of the target distribution is visited while the acceptance probability is high, we say that the chain {\bf mixes well}.
\end{itemize}

\section{Combining Markov Chains}

\subsection{Mixture MCMC}

\begin{itemize}
  \item If transition kernel density $k_1$ and $k_2$ both result in the same stationary distribution $\pi$, then so does the kernel $$\nu k_1 + (1 - \nu)k_2$$ for any $0 \leq \nu \leq 1$.
  
  \item The update step of the MCMC algorithm that uses the mixed kernel would be as follows:
  \begin{itemize}
    \item[] Sample $\xi \sim \mrm{Uniform}(0,1)$.
    \item[] {\bf if} $\xi < \nu$ {\bf then}
    \item[] $\qquad$ Generate the next state $X_{n+1}$ using kernel $k_1$.
    \item[] {\bf else}
    \item[] $\qquad$ Generate the next state $X_{n+1}$ using kernel $k_2$.
    \item[] {\bf end if}
  \end{itemize}

  \item Mixing more than two kernels is, of course, possible.

  \item Mixture MCMC is useful when the target distribution has many narrow peaks. In this case, we can have a ``global'' kernel that has large variance and a ``local'' kernel that has small variance. The global proposal would jump between peaks, and the local proposals would allow one to explore the space around each peak.
\end{itemize}

\subsection{Cycle MCMC}

\begin{itemize}
  \item If we have multiple transition kernel densities $k_1$, $k_2$, $\dotsc$, $k_m$ that have $\pi$ as the stationary distribution of their Markov chains, then the Markov chain obtained by applying the kernels in the round robin fashion, would also have $\pi$ as its stationary distribution.
  
  \item Another use of this type of MCMC algorithm is when a sample $\ve{x}$ can be divided into $m$ blocks $\ve{x} = (\ve{x}^{(1)} | \ve{x}^{(2)} | \dotsb | \ve{x}^{(m)})$. If the kernel $k_i$ only change the $\ve{x}^{(i)}$ block, then applying all the kernels in succession would give us a kernel that change all components of the sample, and this can be engineered to have $\pi$ as the statinoary distribution.
  
  \item The update step of the cycle MCMC algorithm be as follows:
  \begin{itemize}
    \item[] Use $k_1$ to get a state $Y^{(1)}_{n+1}$ from $X_n$.
    \item[] Use $k_2$ to get a state $Y^{(2)}_{n+1}$ from $Y^{(1)}_{n+1}$.
    \item[] $\qquad \vdots$
    \item[] Use $k_m$ to get a state $Y^{(m)}_{n+1}$ from $Y^{(1)}_{n+1}$. 
    \item[] Set $X_{n+1}$ to $Y^{(m)}_{n+1}$.
  \end{itemize}

  \item The expression for the overall kernel $k$ would be
  \begin{align*}
    k(x,y) = \int \int \dotsb \int k_1(x, x^{(1)})k_2(x^{(1)}, x^{(2)}) \dotsb k_m(x^{(m-1)}, y)\, \dee x^{(1)}\dee x^{(2)} \dotsb \dee x^{(m)}.
  \end{align*}
  However, if $k_i$ only changes the $i$th block independent of the other blocks, then we can write
  \begin{align*}
    k(\ve{x},\ve{y}) = \prod_{i=1}^m k_i(\ve{x}^{(i)},\ve{y}^{(i)}).
  \end{align*}
  Weirdly, all the papers that I read write the overall kernel as
  \begin{align*}
    k = k_1 k_2 \dotsb k_m.
  \end{align*}
\end{itemize}

\subsection{The Gibbs Sampler}

\begin{itemize}
  \item Suppose that $\Omega = \Real^d$. So, we denote an elements of $\Real^d$ by $\ve{x} = (x_1, x_2, \dotsc, x_d)$.
  
  \item The $i$th component Gibbs sampler is a kernel $k_i$ that leaves all components besides $x_i$ unchanged and replaces $x_i$ by a draw from a distribution conditional on all the other components.
\end{itemize}


\section{Hamiltonian Monte Carlo}

\begin{itemize}
  \item 
\end{itemize}


\bibliographystyle{apalike}
\bibliography{mcmc-primer}  
\end{document}