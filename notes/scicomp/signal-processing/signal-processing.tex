\documentclass[10pt]{article}
\usepackage{fullpage}
\usepackage{amsmath}
\usepackage[amsthm, thmmarks]{ntheorem}
\usepackage{amssymb}
\usepackage{graphicx}
\usepackage{enumerate}
\usepackage{verse}
\usepackage{tikz}
\usepackage{verbatim}
\usepackage{hyperref}

\newtheorem{lemma}{Lemma}
\newtheorem{theorem}[lemma]{Theorem}
\newtheorem{definition}[lemma]{Definition}
\newtheorem{proposition}[lemma]{Proposition}
\newtheorem{corollary}[lemma]{Corollary}
\newtheorem{claim}[lemma]{Claim}
\newtheorem{example}[lemma]{Example}

\newcommand{\dee}{\mathrm{d}}
\newcommand{\Dee}{\mathrm{D}}
\newcommand{\In}{\mathrm{in}}
\newcommand{\Out}{\mathrm{out}}
\newcommand{\pdf}{\mathrm{pdf}}
\newcommand{\Cov}{\mathrm{Cov}}
\newcommand{\Var}{\mathrm{Var}}

\newcommand{\ve}[1]{\mathbf{#1}}
\newcommand{\ves}[1]{\boldsymbol{#1}}
\newcommand{\mrm}[1]{\mathrm{#1}}
\newcommand{\etal}{{et~al.}}
\newcommand{\sphere}{\mathbb{S}^2}
\newcommand{\modeint}{\mathcal{M}}
\newcommand{\azimint}{\mathcal{N}}
\newcommand{\ra}{\rightarrow}
\newcommand{\mcal}[1]{\mathcal{#1}}
\newcommand{\X}{\mathcal{X}}
\newcommand{\Y}{\mathcal{Y}}
\newcommand{\Z}{\mathcal{Z}}
\newcommand{\x}{\mathbf{x}}
\newcommand{\y}{\mathbf{y}}
\newcommand{\z}{\mathbf{z}}
\newcommand{\tr}{\mathrm{tr}}
\newcommand{\sgn}{\mathrm{sgn}}
\newcommand{\diag}{\mathrm{diag}}
\newcommand{\Real}{\mathbb{R}}
\newcommand{\sseq}{\subseteq}
\newcommand{\ov}[1]{\overline{#1}}
\newcommand{\N}{\mathcal{N}}
\newcommand{\data}{\mathrm{data}}
\newcommand{\SNR}{\mathrm{SNR}}

\DeclareMathOperator*{\argmin}{arg\,min}

\title{A Pocket Reference to Signal Processing}
\author{Pramook Khungurn}

\begin{document}
\maketitle

\section{Signals}

\begin{itemize}
  \item In this note, a {\bf signal} is a 1D real or complex function.

  \item The independent variable of a signal often denotes time, so it is commonly refered to as ``time''and denoted by $t$.
\end{itemize}
  
\subsection{Continous-Time and Discrete-Time Signals}

\begin{itemize}  
  \item A {\bf continuous-time signal} is a function $x(t)$ whose independent variable $t$ is a continous real variable.
  
  \item On the other hand, a {\bf discrete-time signal} is a function where time is  a discrete variable.
  \begin{itemize}
    \item We denote a discrete-time signal by $\{ x_n \}$ where $n \in \mathbb{Z}$ when we want to treat it as a sequence of numbers.
    
    \item Each individual term of the sequence is denoted by $x[n]$.
  \end{itemize}

  \item A discrete-time signal $x[n]$ may be obtained by {\bf sampling} a continous-time signal $x(t)$ at specific times:
  \begin{align*}
    x_n = x[n] = x(t_n).
  \end{align*}
  Each value above is called a {\bf sample}. The time interval between two consective samples is called the {\bf sampling interval}.

  \item When all time internvals are equal, we have the important special case of {\bf uniform sampling}:
  \begin{align*}
    x_n = x[n] = x(nT_s)
  \end{align*}
  where the constant $T_s$ is the sampling interval.
\end{itemize}

\subsection{Analog and Digital Signals}

\begin{itemize}
  \item An {\bf analog signal} is a continous-time signal whose range is continous.
  
  \item A {\bf digitial signal} is a discrete-time signal whose range is discrete and finite.
  
  \item Obviously, digital computers can only process digital signals.
\end{itemize}

\subsection{Odd and Even Signals}

\begin{itemize}
  \item For brevity, we will discuss only continous-time signals in subsection and the next. Definintions for discrete-time signals can be written in a similar fashion.

  \item A signal $x(t)$ {\bf even} if
  \begin{align*}
    x(-t) &= x(t) 
  \end{align*}
  for all $t$.

  \item A signel $x(t)$ {\bf odd} if 
  \begin{align*}
    x(-t) &= -x(t)
  \end{align*}
  for all $t$.

  \item Any signal can be decomposed as a sum of an even and an odd signal:
  \begin{align*}
    x(t) = x_{\mrm{even}}(t) + x_{\mrm{odd}}(t)
  \end{align*}
  where
  \begin{align*}
    x_{\mrm{even}} &= \frac{x(t) + x(-t)}{2},\\
    x_{\mrm{odd}} &= \frac{x(t) - x(-t)}{2}.
  \end{align*}
\end{itemize}

\subsection{Periodic Signals}

\begin{itemize}
  \item A signal $x(t)$ is said to be {\bf periodic} if there is a positive real number $T$ such that $$x(t + T) = x(t)$$ for all $t$. The number $T$ is called a {\bf period} of $x(t)$.
  
  \item The smallest period of a signal is called the {\bf fundamental period}. It is denoted by $T_0$.  
  \begin{itemize}
    \item A constant continuous-time signal $x(t) = c$ does not have a fundamental period because we cannot pinpoint the smallest positive real number.
    
    \item On the other hand, for a constant discrete-time signal $x[n] = c$, its fundamental period is $1$.
  \end{itemize}

  \item A signal that is not periodic is called {\bf aperiodic}.
\end{itemize}

\bibliographystyle{alpha}
\bibliography{signal-processing}  
\end{document}