\documentclass[10pt]{article}
\usepackage{fullpage}
\usepackage{amsmath}
\usepackage[amsthm, thmmarks]{ntheorem}
\usepackage{amssymb}
\usepackage{graphicx}
\usepackage{epstopdf}
\usepackage{enumerate}
\usepackage{verse}
\usepackage{tikz}
\usepackage{upgreek}


\newtheorem{lemma}{Lemma}[section]
\newtheorem{theorem}[lemma]{Theorem}
\newtheorem{definition}[lemma]{Definition}
\newtheorem{proposition}[lemma]{Proposition}
\newtheorem{corollary}[lemma]{Corollary}
\newtheorem{claim}[lemma]{Claim}
\newtheorem{example}[lemma]{Example}

\newcommand{\dee}{\mathrm{d}}
\newcommand{\Dee}{\mathrm{D}}
\newcommand{\In}{\mathrm{in}}
\newcommand{\Out}{\mathrm{out}}
\newcommand{\pdf}{\mathrm{pdf}}

\newcommand{\ve}[1]{\mathbf{#1}}
\newcommand{\mrm}[1]{\mathrm{#1}}
\newcommand{\etal}{{et~al.}}
\newcommand{\sphere}{\mathbb{S}^2}
\newcommand{\modeint}{\mathcal{M}}
\newcommand{\azimint}{\mathcal{N}}
\newcommand{\ra}{\rightarrow}
\newcommand{\mcal}[1]{\mathcal{#1}}
\newcommand{\likelihood}{\mathcal{L}}
\newcommand{\X}{\mathcal{X}}
\newcommand{\Y}{\mathcal{Y}}
\newcommand{\Z}{\mathcal{Z}}
\newcommand{\x}{\mathbf{x}}
\newcommand{\y}{\mathbf{y}}
\newcommand{\z}{\mathbf{z}}
\newcommand{\tr}{\mathrm{tr}}
\newcommand{\sgn}{\mathrm{sgn}}
\newcommand{\diag}{\mathrm{diag}}
\newcommand{\new}{\mathrm{new}}
\newcommand{\Arg}{\mathrm{Arg\,}}
\newcommand{\Log}{\mathrm{Log\,}}
\newcommand{\RE}{\mathrm{Re\,}}
\newcommand{\IM}{\mathrm{Im\,}}
\newcommand{\Res}{\mathrm{Res}}
\newcommand{\pv}{\mathrm{p.v.}}
\newcommand{\Real}{\mathbb{R}}
\newcommand{\sseq}{\subseteq}
\newcommand{\II}{\mathrm{II}}
\DeclareMathOperator{\Bd}{Bd}
\newcommand{\ov}[1]{\overline{#1}}
\newcommand{\metre}{\mathrm{m}}
\newcommand{\second}{\mathrm{s}}
\newcommand{\sterad}{\mathrm{sr}}
\newcommand{\kg}{\mathrm{kg}}
\newcommand{\Watt}{\mathrm{W}}
\newcommand{\group}{\mathrm{gr}}

\title{Vector Fields in Cylindrical and Spherical Coordinates}
\author{Pramook Khungurn}

\begin{document}
  \maketitle  

  \section{Vector Fields in Cartesian Coordinate}  
  \begin{itemize}
    \item A vector field $\ve{f}$ in 3 dimentions is a function from $\mathbb{R}^3$ to $\mathbb{R}^3$.

    \item To specify a vector field in Cartesian coordinate, you have to specify three scalar functions $f_x$, $f_y$, and $f_z$ so that
    \begin{align*}
      \ve{f}(x,y,z) = f_x(x,y,z) \hat{\ve{i}} + f_y(x,y,z) \hat{\ve{j}} + f_z(x,y,z) \hat{\ve{k}}
    \end{align*}    
  \end{itemize}

  \section{Vector Fields in Cylindrical Coordinate}
  \begin{itemize}
    \item The cylindrical coordinate system specifies a point in $\mathbb{R}^3$ by three numbers: $r$, $\theta$, and $z$. 

    \item The point $(r, \theta, z)$ in cylindrical coordinate is equivalent to the point $(x,y,z)$ in Cartesian coordinate where
    \begin{align*}
      x &= r \cos \theta \\
      y &= r \sin \theta \\
      z &= z.
    \end{align*}

    \item A vector field can also be specified in cylindrical coordinate system. What we mean by this is that, for any given point $(r, \theta, z)$ in $\ve{R}^3$, we construct a coordinate system specific to that point such that:
    \begin{itemize}
      \item the point $(r,\theta,z)$ being the origina,
      \item there are three orthonomal basis vectors, corresponding to the $r$, $\theta$, and $z$ corodinate, and
      \item each basis vector points in the direction where the corresponding coordinate is increasing.
    \end{itemize}

    \item Let
    \begin{align*}
      \ve{r}(r, \theta, z) = \begin{bmatrix}
        r \cos \theta \\
        r \sin \theta \\
        z
      \end{bmatrix}.
    \end{align*}
    
    \item If $u$ is one of $r$, $\theta$, and $z$, then the vector $\partial \ve{r} / \partial u$ points in the direction where $u$ is increasing. We define the basis vector as:    
    \begin{align*}
      \hat{u} = \frac{\partial \ve{r} / \partial u}{\| \partial \ve{r} / \partial u \|}.
    \end{align*}

    \item So, in case of spherical coordinate, we have
    \begin{align*}
      \frac{\partial \ve{r}}{\partial r} &= \begin{bmatrix}
        \cos \theta \\ \sin\theta \\ 0
      \end{bmatrix},
      &\frac{\partial \ve{r}}{\partial \theta} &= \begin{bmatrix}
        -r \sin \theta \\ r\cos\theta \\ 0
      \end{bmatrix},
      &\frac{\partial \ve{r}}{\partial z} &= \begin{bmatrix}
        0 \\ 0 \\ 1
      \end{bmatrix}\\
      \hat{r} &= \begin{bmatrix}
        \cos \theta \\ \sin\theta \\ 0
      \end{bmatrix}
      = \frac{\partial \ve{r}}{\partial r},
      &\hat{\theta} &= \begin{bmatrix}
        - \sin \theta \\ \cos\theta \\ 0
      \end{bmatrix}
      = \frac{1}{r} \frac{\partial \ve{r}}{\partial \theta},
      &\hat{z} &= \begin{bmatrix}
        0 \\ 0 \\ 1
      \end{bmatrix}
      = \frac{\partial \ve{r}}{\partial z}
    \end{align*}

    \item A vector field $\ve{f}$, then can be specified by specifying three scalar functoins $f_r$, $f_\theta$, and $f_z$ so that
    \begin{align*}
      \ve{f}(r, \theta, z) = f_r(r, \theta, z) \hat{\ve{r}}(r, \theta, z) +
        f_\theta(r, \theta, z) \hat{\ve{\theta}}(r, \theta, z) +
        f_z(r, \theta, z) \hat{\ve{z}}.
    \end{align*}

    \item When calculating the div, the grad, or the curl, it helps to have the derivative of the basis vectors:
    \begin{align*}
      \frac{\dee \hat{r}}{\dee t}
      &= \frac{\dee}{\dee t} \begin{bmatrix}
        \cos \theta \\ \sin\theta \\ 0
      \end{bmatrix}
      = \begin{bmatrix}
        -\sin \theta \\ \cos\theta \\ 0
      \end{bmatrix}\frac{\dee \theta}{\dee t}
      = \hat{\ve{\theta}} \dot \theta \\
      \frac{\dee \hat{\ve{\theta}}}{\dee t}
      &= \frac{\dee}{\dee t} \begin{bmatrix}
        -\sin \theta \\ \cos\theta \\ 0
      \end{bmatrix}
      = \begin{bmatrix}
        -\cos \theta \\ -\sin\theta \\ 0
      \end{bmatrix}\frac{\dee \theta}{\dee t}
      = -\hat{r} \dot\theta \\
      \frac{\dee \hat{z}}{\dee t} &= \ve{0}.
    \end{align*}    
  \end{itemize}

  \section{Vector Field in Spherical Coordinate}
  \begin{itemize}
    \item With spherical coordinate, we specify a position by three parameters---$r$, $\theta$, $\varphi$---with the function:
    \begin{align*}
      \ve{r}(r, \theta, \varphi) = \begin{bmatrix}
        r \sin\theta \cos\varphi \\
        r \sin\theta \sin\varphi \\
        r \cos\theta
      \end{bmatrix}.
    \end{align*}

    \item As a result,
    \begin{align*}
      \frac{\partial \ve{r}}{\partial r} &= \begin{bmatrix}
        \sin \theta \cos \varphi \\
        \sin \theta \sin \varphi \\
        \cos \theta
      \end{bmatrix}, &
      \frac{\partial \ve{r}}{\partial \theta} &= \begin{bmatrix}
        r \cos \theta \cos \varphi \\
        r \cos \theta \sin \varphi \\
        -r \sin \theta
      \end{bmatrix}, &
      \frac{\partial \ve{r}}{\partial \varphi} &= \begin{bmatrix}
        -r \sin \theta \sin \varphi \\
        r \sin \theta \cos \varphi \\
        0
      \end{bmatrix}.
    \end{align*}
    So,  
    \begin{align*}
      \hat{r} &= \begin{bmatrix}
        \sin \theta \cos \varphi \\
        \sin \theta \sin \varphi \\
        \cos \theta
      \end{bmatrix}
      = \frac{\partial \ve{r}}{\partial r}, \\
      \hat\theta 
      &= \frac{1}{r}\frac{\partial \ve{r}}{\partial \theta}
      = \frac{1}{r} \begin{bmatrix}
        r \cos \theta \cos \varphi \\
        r \cos \theta \sin \varphi \\
        -r \sin \theta        
      \end{bmatrix}
      = \begin{bmatrix}
        \cos\theta\cos\varphi \\
        \cos\theta\sin\varphi \\
        -\sin\theta
      \end{bmatrix}, \\
      \hat\varphi 
      &= \frac{1}{r\sin\theta} \frac{\partial \ve{r}}{\partial \varphi}
      = \frac{1}{r \sin\theta}\begin{bmatrix}
        -r \sin \theta \sin \varphi \\
        r \sin \theta \cos \varphi \\
        0
      \end{bmatrix}
      = \begin{bmatrix}
        -\sin\varphi \\
        \cos\varphi \\
        0
      \end{bmatrix}.
    \end{align*}

    \item For derivatives, we have
    \begin{align*}
      \frac{\dee \hat r}{\dee t} 
      &= \frac{\partial \hat r}{\partial \theta} \frac{\dee \theta}{\dee t} + \frac{\partial \hat r}{\partial \varphi} \frac{\dee \varphi}{\dee t}
      = \begin{bmatrix}
        \cos\theta \cos\varphi \\
        \cos\theta \sin\varphi \\
        -\sin\theta
      \end{bmatrix}
      \dot\theta
      + \begin{bmatrix}
        -\sin\theta \sin\varphi \\
        \sin\theta \cos\varphi\\
        0
      \end{bmatrix}
      \dot \varphi
      = \hat \theta \dot \theta + \hat \varphi \dot \varphi \sin\theta\\
      \frac{\dee \hat \theta}{\dee t} 
      &= \frac{\partial \hat \theta}{\partial \theta} \frac{\dee \theta}{\dee t} + \frac{\partial \hat \theta}{\partial \varphi} \frac{\dee \varphi}{\dee t}
      = \begin{bmatrix}
        -\sin\theta \cos\phi \\
        -\sin\theta \sin\phi \\
        -\cos\theta
      \end{bmatrix} \dot\theta + 
      \begin{bmatrix}
         -\cos\theta \sin\varphi\\
         \cos\theta \cos\varphi\\
         0
       \end{bmatrix} \dot\phi
      = -\hat r \dot\theta + \hat\varphi \dot\varphi \cos\theta\\
      \frac{\dee \hat\varphi}{\dee t} 
      &= \begin{bmatrix}
        -\cos\varphi \\
        -\sin\varphi \\
        0
      \end{bmatrix} =
      \begin{bmatrix}
        -(\sin^2\theta + \cos^2\theta)\cos\varphi \\
        -(\sin^2\theta + \cos^2\theta)\sin\varphi \\
        -\sin\theta\cos\theta + \cos\theta\sin\theta
      \end{bmatrix}
      \dot\varphi
      = \Bigg( \begin{bmatrix}      
        -\sin^2\theta \cos\phi \\
        -\sin^2\theta \sin\phi \\
        -\sin\theta \cos\phi
      \end{bmatrix}
      + \begin{bmatrix}      
        -\cos^2\theta \cos\phi \\
        -\cos^2\theta \sin\phi \\
        +\sin\theta \cos\theta
      \end{bmatrix}
      \Bigg) \dot\varphi\\
      &= \Bigg( -\sin\theta \begin{bmatrix}      
        \sin \theta \cos\phi \\
        \sin\theta \sin\phi \\
        \cos\phi
      \end{bmatrix}
      -\cos\theta \begin{bmatrix}      
        \cos \theta \cos\phi \\
        \cos \theta \sin\phi \\
        -\sin\theta
      \end{bmatrix}
      \Bigg) \dot\varphi
      = -(\hat r \sin\theta + \hat \theta \cos\theta) \dot\varphi.
    \end{align*}
  \end{itemize}  
  
\end{document}

 