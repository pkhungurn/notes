\documentclass[10pt]{article}
\usepackage{fullpage}
\usepackage{amsmath}
\usepackage[amsthm, thmmarks]{ntheorem}
\usepackage{amssymb}
\usepackage{graphicx}
\usepackage{epstopdf}
\usepackage{enumerate}
\usepackage{verse}
\usepackage{tikz}
\usepackage{upgreek}
\usepackage{hyperref}

\newtheorem{lemma}{Lemma}[section]
\newtheorem{theorem}[lemma]{Theorem}
\newtheorem{definition}[lemma]{Definition}
\newtheorem{proposition}[lemma]{Proposition}
\newtheorem{corollary}[lemma]{Corollary}
\newtheorem{claim}[lemma]{Claim}
\newtheorem{example}[lemma]{Example}

\newcommand{\dee}{\mathrm{d}}
\newcommand{\Dee}{\mathrm{D}}
\newcommand{\In}{\mathrm{in}}
\newcommand{\Out}{\mathrm{out}}
\newcommand{\pdf}{\mathrm{pdf}}

\newcommand{\ve}[1]{\mathbf{#1}}
\newcommand{\mrm}[1]{\mathrm{#1}}
\newcommand{\etal}{{et~al.}}
\newcommand{\sphere}{\mathbb{S}^2}
\newcommand{\modeint}{\mathcal{M}}
\newcommand{\azimint}{\mathcal{N}}
\newcommand{\ra}{\rightarrow}
\newcommand{\mcal}[1]{\mathcal{#1}}
\newcommand{\likelihood}{\mathcal{L}}
\newcommand{\X}{\mathcal{X}}
\newcommand{\Y}{\mathcal{Y}}
\newcommand{\Z}{\mathcal{Z}}
\newcommand{\x}{\mathbf{x}}
\newcommand{\y}{\mathbf{y}}
\newcommand{\z}{\mathbf{z}}
\newcommand{\tr}{\mathrm{tr}}
\newcommand{\sgn}{\mathrm{sgn}}
\newcommand{\diag}{\mathrm{diag}}
\newcommand{\new}{\mathrm{new}}
\newcommand{\Arg}{\mathrm{Arg\,}}
\newcommand{\Log}{\mathrm{Log\,}}
\newcommand{\RE}{\mathrm{Re\,}}
\newcommand{\IM}{\mathrm{Im\,}}
\newcommand{\Res}{\mathrm{Res}}
\newcommand{\pv}{\mathrm{p.v.}}
\newcommand{\Real}{\mathbb{R}}
\newcommand{\sseq}{\subseteq}
\newcommand{\II}{\mathrm{II}}
\DeclareMathOperator{\Bd}{Bd}
\newcommand{\ov}[1]{\overline{#1}}
\newcommand{\metre}{\mathrm{m}}
\newcommand{\second}{\mathrm{s}}
\newcommand{\sterad}{\mathrm{sr}}
\newcommand{\kg}{\mathrm{kg}}
\newcommand{\Watt}{\mathrm{W}}
\newcommand{\group}{\mathrm{gr}}

\title{Sample Variance: Why divide by $n-1$?}
\author{Pramook Khungurn}

\begin{document}
  \maketitle
  
  As someone with little training in statistics, it often puzzles me why, when estimatiing the sample/empirical variance of a random varianble, we divide the sum by $n-1$ instead of $n$.

  Let $X$ be a random variable. Suppose we have $n$ i.i.d.~samples of $X$, which we denote by $X_1$, $X_2$, $\dotsc$, $X_n$. Let $\bar{X}_n$ denote the empirical mean:
  \begin{align*}
    \bar{X}_n = \frac{1}{n} \sum_{i=1}^n X_i.
  \end{align*}
  Of course, we know that $\bar{X}_n$ is an unbiased estimator of $E[X]$.

  We want to construct an unbiased estimate of $Var(X)$. Now, we might be tempted to use the estimator:
  \begin{align*}
    S_n^2 = \frac{1}{n} \sum_{i=1}^n (X_i - \bar{X}_n)^2.
  \end{align*}
  However, this estimator is biased. One thing to notice is that $\bar{X}_n$ is not exactly $E[X]$. So, what is the expectation of this estimator? We have
  \begin{align*}
    E[S_n^2] 
    &= E\bigg[ \frac{1}{n} \sum_{i=1}^n ( X_i - \bar{X}_n)^2 \bigg]
    = \frac{1}{n} E\bigg[ \sum_{i=1}^n ( X_i^2 - 2X_i \bar{X}_n + \bar{X}_n^2 ) \bigg]
    = \frac{1}{n} \bigg( \sum_{i=1}^n E[X_i^2] - 2 \sum_{i=1}^n E[X_i \bar{X}_n] + \sum_{i=1}^n E[\bar{X}_n^2] \bigg)\\
    &= \frac{1}{n} \bigg( n E[X^2] - 2 \sum_{i=1}^n E\bigg[ X_i \bigg( \frac{1}{n} \sum_{j=1}^n X_j \bigg) \bigg] + n E\bigg[ \bigg( \frac{1}{n} \sum_{j=1}^n X_j \bigg)^2 \bigg] \bigg)\\
    &= \frac{1}{n} \bigg( n E[X^2] - \frac{2}{n} \sum_{i=1}^n \sum_{j=1}^n E[X_i X_j] + n E\bigg[ \frac{1}{n^2} \bigg( \sum_{i=1}^n \sum_{j=1}^n X_iX_j \bigg) \bigg] \bigg)\\
    &= \frac{1}{n} \bigg( n E[X^2] - \frac{2}{n} \sum_{i=1}^n \sum_{j=1}^n E[X_i X_j] + \frac{1}{n} \sum_{i=1}^n \sum_{j=1}^n E[  X_iX_j] \bigg)\\
    &= \frac{1}{n} \bigg( n E[X^2] - \frac{1}{n} \bigg( \sum_{i=1}^n E[X_i^2] + \sum_{i \neq j} E[X_iX_j] \bigg) \bigg)\\
    &= \frac{1}{n} \bigg( n E[X^2] - E[X^2] - (n-1) (E[X])^2 \bigg)\\
    &= \frac{n-1}{n} \bigg( E[X^2] - (E[X])^2 \bigg) = \frac{n-1}{n} Var(X).
  \end{align*}
  Therefore, the unbiased estimator is given by:
  \begin{align*}
    S_{n-1}^2 := \frac{n}{n-1} S_n^2 = \frac{1}{n-1} \sum_{i=1}^n (X_i - \bar{X}_n)^2.
  \end{align*}
\end{document}