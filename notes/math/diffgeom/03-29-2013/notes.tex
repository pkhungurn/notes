\documentclass[10pt]{article}
\usepackage{fullpage}
\usepackage{amsmath}
\usepackage[amsthm, thmmarks]{ntheorem}
\usepackage{amssymb}
\usepackage{graphicx}
\usepackage{epstopdf}
\usepackage{enumerate}
\usepackage{verse}
\usepackage{tikz}

\newtheorem{lemma}{Lemma}[section]
\newtheorem{theorem}[lemma]{Theorem}
\newtheorem{definition}[lemma]{Definition}
\newtheorem{proposition}[lemma]{Proposition}
\newtheorem{corollary}[lemma]{Corollary}
\newtheorem{claim}[lemma]{Claim}
\newtheorem{example}[lemma]{Example}

\newcommand{\dee}{\mathrm{d}}
\newcommand{\In}{\mathrm{in}}
\newcommand{\Out}{\mathrm{out}}
\newcommand{\pdf}{\mathrm{pdf}}

\newcommand{\ve}[1]{\mathbf{#1}}
\newcommand{\mrm}[1]{\mathrm{#1}}
\newcommand{\etal}{{et~al.}}
\newcommand{\sphere}{\mathbb{S}^2}
\newcommand{\modeint}{\mathcal{M}}
\newcommand{\azimint}{\mathcal{N}}
\newcommand{\ra}{\rightarrow}
\newcommand{\mcal}[1]{\mathcal{#1}}
\newcommand{\likelihood}{\mathcal{L}}
\newcommand{\X}{\mathcal{X}}
\newcommand{\Y}{\mathcal{Y}}
\newcommand{\Z}{\mathcal{Z}}
\newcommand{\x}{\mathbf{x}}
\newcommand{\y}{\mathbf{y}}
\newcommand{\z}{\mathbf{z}}
\newcommand{\tr}{\mathrm{tr}}
\newcommand{\sgn}{\mathrm{sgn}}
\newcommand{\diag}{\mathrm{diag}}
\newcommand{\new}{\mathrm{new}}
\newcommand{\Arg}{\mathrm{Arg\,}}
\newcommand{\Log}{\mathrm{Log\,}}
\newcommand{\RE}{\mathrm{Re\,}}
\newcommand{\IM}{\mathrm{Im\,}}
\newcommand{\Res}{\mathrm{Res}}
\newcommand{\pv}{\mathrm{p.v.}}
\newcommand{\Real}{\mathbb{R}}
\newcommand{\sseq}{\subseteq}

\title{Differential Geometry Notes of 03/29/2013}
\author{Pramook Khungurn}

\begin{document}
  \maketitle

  \section{Christoffel Symbols}

  \begin{itemize}
    \item $S$ will denote a regular, orientable, and oriented surface. Let $\ve{x} : U \sseq \Real^2 \ra S$ be a parameterization in the orientation of $S$.

    \item It is possible to assign each point of $\ve{x}(U)$ a basis given by the vectors $\ve{x}_u$, $\ve{x}_v$ and $N$.

    \item We can now express the derivatives of the vectors $\ve{x}$, $\ve{y}$, and $N$ in this basis.
    \begin{align*}
      \ve{x}_{uu} &= \Gamma_{11}^1 \ve{x}_u + \Gamma_{11}^2\ve{x}_v + L_1 N\\
      \ve{x}_{uv} &= \Gamma_{12}^1 \ve{x}_u + \Gamma_{12}^2\ve{x}_v + L_2 N\\
      \ve{x}_{vv} &= \Gamma_{22}^1 \ve{x}_u + \Gamma_{22}^2\ve{x}_v + L_3 N\\
      N_u &= a_{11} \ve{x}_u + a_{21} \ve{x}_v\\
      N_v &= a_{12} \ve{x}_u + a_{22} \ve{x}_v
    \end{align*}
    The coefficients $\Gamma_{ij}^k$ are  called the {\bf Christoffel symbols} of $S$ in the parameterization $\ve{x}$.

    \item Now, we have that
    \begin{align*}
      \langle \ve{x}_{uu}, N \rangle &= \Gamma_{11}^1 \langle \ve{x}_u, N \rangle + \Gamma_{11}^2 \langle \ve{x}_v, N \rangle + L_1 \langle N, N \rangle\\
      e &= L_1.
    \end{align*}
    Similar, by computing $\langle \ve{x}_{uv}, N \rangle$ and $\langle \ve{x}_{vv}, N \rangle$, we have that $L_2 = f$ and $L_3 = g$. Hence, we can rewrite the equations as:
    \begin{align*}
      \ve{x}_{uu} &= \Gamma_{11}^1 \ve{x}_u + \Gamma_{11}^2\ve{x}_v + e N\\
      \ve{x}_{uv} &= \Gamma_{12}^1 \ve{x}_u + \Gamma_{12}^2\ve{x}_v + f N\\
      \ve{x}_{vv} &= \Gamma_{22}^1 \ve{x}_u + \Gamma_{22}^2\ve{x}_v + g N\\
      N_u &= a_{11} \ve{x}_u + a_{21} \ve{x}_v\\
      N_v &= a_{12} \ve{x}_u + a_{22} \ve{x}_v
    \end{align*}

    \item Now,
    \begin{align*}
      E_u = \frac{\partial}{\partial u} \langle \ve{x}_u, \ve{x}_u \rangle = 2\langle \ve{x}_{uu}, \ve{x}_u \rangle.
    \end{align*}  
    So,
    \begin{align*}
      \frac{1}{2}E_u = \langle \ve{x}_{uu}, \ve{x}_u \rangle
    \end{align*}
    Now,
    \begin{align*}
      \frac{1}{2} E_u = \langle \ve{x}_{uu}, \ve{x}_u \rangle
      = \Gamma_{11}^1 \langle \ve{x}_u, \ve{x}_u \rangle + \Gamma_{11}^2 \langle \ve{x}_v, \ve{x}_u \rangle + e \langle N, \ve{x}_u \rangle
      = \Gamma_{11}^1 E + \Gamma_{11}^2 F
    \end{align*}
    Moreover,
    \begin{align*}
      F_u &= \frac{\partial}{\partial u} \langle \ve{x}_u, \ve{x}_v \rangle = \langle \ve{x}_{uu}, \ve{x}_v \rangle + \langle \ve{x}_{uv}, \ve{x}_{u} \rangle\\
      E_v &= \frac{\partial}{\partial v} \langle \ve{x}_u, \ve{x}_u \rangle = 2 \langle \ve{x}_{uv}, \ve{x}_u \rangle
    \end{align*}
    So,
    \begin{align*}
      \langle \ve{x}_{uu}, \ve{x}_v \rangle = F_u - \frac{1}{2} E_v.
    \end{align*}
    Now,
    \begin{align*}
      F_u - \frac{1}{2} E_v 
      = \langle \ve{x}_{uu}, \ve{x}_v \rangle 
      = \Gamma_{11}^1 \langle \ve{x}_u, \ve{x}_v \rangle + \Gamma_{11}^2 \langle \ve{x}_v, \ve{x}_v \rangle + e \langle N, \ve{x}_v \rangle
      = \Gamma_{11}^1 F + \Gamma_{11}^2 G 
    \end{align*}
    As a result, we have the following equations:
    \begin{align*}
      \frac{1}{2} E_u &= \Gamma_{11}^1 E + \Gamma_{11}^2 F\\
      F_u - \frac{1}{2} E_v &= \Gamma_{11}^1 F + \Gamma_{11}^2 G.
    \end{align*}
    In other words,
    \begin{align*}
      \begin{bmatrix}
        \frac{1}{2}E_u \\
        F_u - \frac{1}{2} E_v        
      \end{bmatrix}
      &=
      \begin{bmatrix}
        E & F\\
        F & G
      \end{bmatrix}
      \begin{bmatrix}
        \Gamma_{11}^1\\
        \Gamma_{11}^2\\
      \end{bmatrix}\\
      \begin{bmatrix}
        \Gamma_{11}^1\\
        \Gamma_{11}^2\\
      \end{bmatrix}
      &=
      \begin{bmatrix}
        E & F\\
        F & G
      \end{bmatrix}^{-1}
      \begin{bmatrix}
        \frac{1}{2}E_u \\
        F_u - \frac{1}{2} E_v        
      \end{bmatrix}
    \end{align*}
    Now that the determinat of the matrix being inverted is $EG-F^2$, which is always non-zero because this is the differential of the area of the regular surface. 

    Hence, {\bf we can write $\Gamma_{11}^1$ and $\Gamma_{11}^2$ in terms of $E$, $F$, $G$, and their derivatives.}

    \item With similar derivation as in the last item, we have that 
    \begin{align*}
      \Gamma_{12}^1 E + \Gamma_{12}^2 F &= \frac{1}{2} E_v\\
      \Gamma_{12}^1 F + \Gamma_{12}^2 G &= \frac{1}{2} G_u
    \end{align*}
    and
    \begin{align*}
      \Gamma_{22}^1 E + \Gamma_{22}^2 F &= F_v - \frac{1}{2} G_u\\
      \Gamma_{22}^1 F + \Gamma_{22}^2 G &= \frac{1}{2} G_v
    \end{align*}
    These equations tell us that {\bf we can write all the Christoffel symbols in terms of $E$, $F$, $G$ and their derivatives.}

    \item The consequence is that {\bf all geometric concepts and properties expressed in terms of the Christoffel symbols are invariant under isometries.}
  \end{itemize}

  \section{Surface of Revolution}

  \begin{itemize}
    \item In this section, we shall compute the Christoffel symbols for the surface of revolution given by the parameterization:
    \begin{align*}
      \ve{x}(u,v) = (f(v) \cos u, f(v) \sin u, g(v))
    \end{align*}
    where $f(v) \neq 0$.

    \item It can be easily shown that
    \begin{align*}
      E &= (f(v))^2\\
      F &= 0\\
      G &= (f'(v))^2 + (g'(v))^2
    \end{align*}
    So,
    \begin{align*}
      E_u &= 0\\
      E_v &= 2ff'\\
      F_u &= F_v = 0\\
      G_u &= 0\\
      G_v &= 2(f'f'' + g'g'')
    \end{align*}
    Now, using the approach discussed in the last section, we have that
    \begin{align*}
      \Gamma_{11}^1 &= 0 & \Gamma_{11}^2 &= - \frac{ff'}{(f')^2 + (g')^2}\\
      \Gamma_{12}^1 &= \frac{ff'}{f^2} & \Gamma_{12}^2 &= 0\\
      \Gamma_{22}^2 &= 0 & \Gamma_{22}^2 &= \frac{f'f''+g'g''}{(f')^2 + (g')^2}
    \end{align*}
  \end{itemize}

  \section{Gauss Formula}
  \begin{itemize}
    \item We shall rewrite the following identity using the expansion of $\ve{x}_{uu}$, $\ve{x}_{uv}$, and $\ve{x}_{vv}$ in the basis $\{ \ve{x}_u, \ve{x}_v, N \}$:
    \begin{align*}
      (\ve{x}_{uu})_v - (\ve{x}_{uv})_u &= 0.
    \end{align*}

    \item Starting with $(\ve{x}_{uu})_v$, we have
    \begin{align*}
      (\ve{x}_{uu})_v 
      &= (\Gamma_{11}^1 \ve{x}_u + \Gamma_{11}^2 \ve{x}_v + e N)_v\\
      &= (\Gamma_{11}^1)_v \ve{x}_u + \Gamma_{11}^1 \ve{x}_{uv} + (\Gamma_{11}^2)_v \ve{x}_v + \Gamma_{11}^2 \ve{x}_{vv} + e_v N + e N_v\\
      &= (\Gamma_{11}^1)_v \ve{x}_u + \Gamma_{11}^1 (\Gamma_{12}^1 \ve{x}_u + \Gamma_{12}^2 \ve{x}_v + f N)\\
      &\phantom{\ =\ } + (\Gamma_{11}^2)_v \ve{x}_v + \Gamma_{11}^2 (\Gamma_{22}^1 \ve{x}_u + \Gamma_{22}^2 \ve{x}_v + g N)\\
      &\phantom{\ =\ }+ e_v N + e (a_{12} \ve{x}_u + a_{22} \ve{x}_v)\\
      &= [(\Gamma_{11}^1)_v + \Gamma_{11}^1 \Gamma_{12}^1 + \Gamma_{11}^2 \Gamma_{22}^1 + a_{12} e] \ve{x}_u\\
      &\phantom{\ =\ } + [\Gamma_{11}^1 \Gamma_{12}^2 + (\Gamma_{11}^2)_v + \Gamma_{11}^2 \Gamma_{22}^2 + a_{22} e] \ve{x}_v\\
      &\phantom{\ =\ } + [f \Gamma_{11}^1 + g\Gamma_{11}^2 + e_v]N 
    \end{align*}
    Now, for $(\ve{x}_{uv})_u$, we have
    \begin{align*}
      (\ve{x}_{uv})_u
      &= (\Gamma_{12}^1 \ve{x}_u + \Gamma_{12}^2 \ve{x}_v + f N)_u\\
      &= (\Gamma_{12}^1)_u \ve{x}_u + \Gamma_{12}^1 \ve{x}_{uu} + (\Gamma_{12}^2)_u \ve{x}_v + \Gamma_{12}^2 \ve{x}_{uv} + f_u N + f N_u\\      
      &= (\Gamma_{12}^1)_u \ve{x}_u + \Gamma_{12}^1 (\Gamma_{11}^1 \ve{x}_u + \Gamma_{11}^2 \ve{x}_v + e N)\\
      &\phantom{\ =\ } + (\Gamma_{12}^2)_u \ve{x}_v + \Gamma_{12}^2 (\Gamma_{12}^1 \ve{x}_u + \Gamma_{12}^2 \ve{x}_v + f N)\\
      &\phantom{\ =\ }+ f_u N + f (a_{11} \ve{x}_u + a_{21} \ve{x}_v)\\      
      &= [(\Gamma_{12}^1)_u + \Gamma_{12}^1 \Gamma_{11}^1 + \Gamma_{12}^1 \Gamma_{12}^2 + a_{11} f] \ve{x}_u\\
      &\phantom{\ =\ } + [\Gamma_{12}^1 \Gamma_{11}^2 + (\Gamma_{12}^2)_u + (\Gamma_{12}^2)^2 + a_{21} f] \ve{x}_v\\
      &\phantom{\ =\ } + [e \Gamma_{12}^1 + f\Gamma_{12}^2 + f_u]N.
    \end{align*}
    Because $(\ve{x}_{uu})_v - (\ve{x}_{uv})_u = 0$ and because $\ve{x}_u$, $\ve{x}_v$, and $N$ are linearly independent, it must be the case that the coefficient of $\ve{x}_v$ in the expression of $(\ve{x}_{uu})_v - (\ve{x}_{uv})_u$ must be 0. In other words,
    \begin{align*}
      \Gamma_{11}^1 \Gamma_{12}^2 + (\Gamma_{11}^2)_v + \Gamma_{11}^2 \Gamma_{22}^2 + e a_{22} - \Gamma_{12}^1 \Gamma_{11}^2 - (\Gamma_{12}^2)_u - (\Gamma_{12}^2)^2 - f a_{21}
      &= 0
    \end{align*}
    Taking,
    \begin{align*}
      a_{21} &= \frac{eF - fE}{EG-F^2}\\
      a_{22} &= \frac{fF - gE}{EG-F^2},
    \end{align*}
    we have
    \begin{align*}
      \Gamma_{11}^1 \Gamma_{12}^2 + (\Gamma_{11}^2)_v + \Gamma_{11}^2 \Gamma_{22}^2 + e \frac{fF - gE}{EG-F^2} - \Gamma_{12}^1 \Gamma_{11}^2 - (\Gamma_{12}^2)_u - (\Gamma_{12}^2)^2 - f\frac{eF - fE}{EG-F^2}
      &= 0.
    \end{align*}
    In other words,
    \begin{align*}
      \Gamma_{11}^1 \Gamma_{12}^2 + (\Gamma_{11}^2)_v + \Gamma_{11}^2 \Gamma_{22}^2 - \Gamma_{12}^1 \Gamma_{11}^2 - (\Gamma_{12}^2)_u - (\Gamma_{12}^2)^2 
      &= f\frac{eF - fE}{EG-F^2} - e \frac{fF - gE}{EG-F^2}\\
      (\Gamma_{11}^2)_v -  (\Gamma_{12}^2)_u + \Gamma_{11}^1 \Gamma_{12}^2 + \Gamma_{11}^2 \Gamma_{22}^2 - \Gamma_{12}^1 \Gamma_{11}^2 - \Gamma_{12}^2 \Gamma_{12}^2
      &=\frac{efF - f^2E - ef F + eg E}{EG-F^2}\\
      (\Gamma_{11}^2)_v -  (\Gamma_{12}^2)_u + \Gamma_{11}^1 \Gamma_{12}^2 + \Gamma_{11}^2 \Gamma_{22}^2 - \Gamma_{12}^1 \Gamma_{11}^2 - \Gamma_{12}^2 \Gamma_{12}^2
      &= E \frac{eg -  f^2}{EG-F^2}\\
      (\Gamma_{11}^2)_v -  (\Gamma_{12}^2)_u + \Gamma_{11}^1 \Gamma_{12}^2 + \Gamma_{11}^2 \Gamma_{22}^2 - \Gamma_{12}^1 \Gamma_{11}^2 - \Gamma_{12}^2 \Gamma_{12}^2
      &= EK.
    \end{align*}
    This last equation is called the {\bf Gauss formula}.

    \item The last equation tells us that the Gaussian curvature $K$ can be written as an expression of $E$, $F$, $G$, and their derivatives.
    
    \item \begin{theorem}[Gauss]
      The Gaussian curvature $K$ of a surface is invariant by local isometries.
    \end{theorem}
    \begin{proof}
      Let $\ve{x} : U \sseq \Real^2 \ra S$ is a parameterization of $p \in S$. Let $\varphi: V \sseq S \ra S$, where $V \sseq \ve{x}(U)$ is a neighborhood of $p$, be a local isometry. Define $\ve{y} = \varphi \circ \ve{x}$. We have that $\ve{y}$ is a parameterization around $\varphi(p)$

      Since $\varphi$ is an isometry, the coefficients of the first fundamental forms in the parameterization $\ve{x}$ and $\ve{y}$ agree at corresponding points $q$ and $\varphi(q)$ for all point $q \in V$. It follows that $K(q) = K(\varphi(q))$ for all $q \in V$ because the Gaussian curvature can be written as an expression of the coefficients of the first fundamental forms and their derivatives.
    \end{proof}
  \end{itemize}

  \section{Mainardi--Cordazzi Equations}
  \begin{itemize}
    \item By setting the coefficient of $N$ in $(\ve{x}_{uu})_v - (\ve{x}_{uv})_u = 0$ to 0, we have that
    \begin{align} \label{mainardi-cordazzi-1}
      e_v - f_u = e\Gamma_{12}^1 + f(\Gamma_{12}^2 - \Gamma_{11}^1) - g \Gamma_{11}^2.
    \end{align}

    \item Also, by setting the coefficient of $N$ in $(\ve{x}_{vv})_u - (\ve{x}_{uv})_v = 0$ to 0, we have that
    \begin{align} \label{mainardi-cordazzi-2}
      f_v - g_u = e\Gamma_{22}^1 + f(\Gamma_{22}^2-\Gamma_{12}^1) - g\Gamma_{12}^2.
    \end{align}

    \item Equation \eqref{mainardi-cordazzi-1} and \eqref{mainardi-cordazzi-2} are known collectively as the {\bf Mainardi--Cordazzi} equations.

    \item The Gauss formula and the Mainardi--Cordazzi equations are known under the name of the {\bf compatibility equations of the theory of surfaces}.

    \item The compatibility equations assert relations among the coefficients of the first and second fundamental forms of a regular surfaces.  

    \item The converse is also true. If a collection of six functions satisfy the compatibility equations, then there exists a surface having them as the coefficients of the first and second fundamental forms.

    \begin{theorem}[Bonnet]
      Let $E, F, G, e, f, g$ be differentiable functions defined in an open set $V \sseq \Real^2$, with $E > 0$ and $G > 0$. Assume that the given functions satisfy the compatibility equations and that $EG - F^2 > 0$. Then, for every $q \in V$, there exists a neighborhood $U \sseq V$ of $q$ and a diffeomorphism $\ve{x} : U \ra \Real^3$ such that $\ve{x}(U)$ is a regular surface that has $E, F, G, e, f, g$ as coefficients of the first and second fundamental forms, respectively.

      Furthermore, if $U$ is connected and if $\bar{\ve{x}}: U \ra \Real^3$ is another diffeomorphism satisfying the same conditions, then there exists a translation $T$ and a proper linear orthogonal transformation $\rho$ in $\Real^3$ such that $\ve{x} = T \circ \rho \circ \ve{x}$. (In other words, the coefficients of the first and second fundamental forms determine the surface up to a rigid motion.)
    \end{theorem}

    \item The Mainardi--Cordazzi equations simplify when the coordinate neighborhoods contains no umbilical points and the coordinate curves are lines of curvature ($F = 0 = f$). The equations becomes:
    \begin{align*}
      e_v &= e \Gamma^1_{12} - g\Gamma^2{11}\\
      g_u &= g \Gamma^2_{12} - e\Gamma^1{22}.
    \end{align*}
    Also,
    \begin{align*}
      \Gamma_{11}^2 &= -\frac{1}{2}\frac{E_v}{G}\\
      \Gamma_{12}^1 &= \frac{1}{2}\frac{E_v}{E}\\
      \Gamma_{12}^2 &= \frac{1}{2}\frac{G_u}{G}\\
      \Gamma_{22}^1 &= -\frac{1}{2}\frac{G_u}{E}.
    \end{align*}
    Hence,
    \begin{align*}
      e_v &= \frac{E_v}{2}\bigg( \frac{e}{E} + \frac{g}{G} \bigg)\\
      g_u &= \frac{G_u}{2}\bigg( \frac{e}{E} + \frac{g}{G} \bigg)
    \end{align*}
  \end{itemize}
 \end{document}
