\documentclass[10pt]{article}
\usepackage{fullpage}
\usepackage{amsmath}
\usepackage[amsthm, thmmarks]{ntheorem}
\usepackage{amssymb}
\usepackage{graphicx}
\usepackage{epstopdf}
\usepackage{enumerate}
\usepackage{verse}
\usepackage{tikz}

\newtheorem{lemma}{Lemma}[section]
\newtheorem{theorem}[lemma]{Theorem}
\newtheorem{definition}[lemma]{Definition}
\newtheorem{proposition}[lemma]{Proposition}
\newtheorem{corollary}[lemma]{Corollary}
\newtheorem{claim}[lemma]{Claim}
\newtheorem{example}[lemma]{Example}

\newcommand{\dee}{\mathrm{d}}
\newcommand{\In}{\mathrm{in}}
\newcommand{\Out}{\mathrm{out}}
\newcommand{\pdf}{\mathrm{pdf}}

\newcommand{\ve}[1]{\mathbf{#1}}
\newcommand{\mrm}[1]{\mathrm{#1}}
\newcommand{\etal}{{et~al.}}
\newcommand{\sphere}{\mathbb{S}^2}
\newcommand{\modeint}{\mathcal{M}}
\newcommand{\azimint}{\mathcal{N}}
\newcommand{\ra}{\rightarrow}
\newcommand{\mcal}[1]{\mathcal{#1}}
\newcommand{\likelihood}{\mathcal{L}}
\newcommand{\X}{\mathcal{X}}
\newcommand{\Y}{\mathcal{Y}}
\newcommand{\Z}{\mathcal{Z}}
\newcommand{\x}{\mathbf{x}}
\newcommand{\y}{\mathbf{y}}
\newcommand{\z}{\mathbf{z}}
\newcommand{\tr}{\mathrm{tr}}
\newcommand{\sgn}{\mathrm{sgn}}
\newcommand{\diag}{\mathrm{diag}}
\newcommand{\new}{\mathrm{new}}
\newcommand{\Arg}{\mathrm{Arg\,}}
\newcommand{\Log}{\mathrm{Log\,}}
\newcommand{\RE}{\mathrm{Re\,}}
\newcommand{\IM}{\mathrm{Im\,}}
\newcommand{\Res}{\mathrm{Res}}
\newcommand{\pv}{\mathrm{p.v.}}
\newcommand{\Real}{\mathbb{R}}
\newcommand{\sseq}{\subseteq}

\title{Differential Geometry Notes of 02/17/2013}
\author{Pramook Khungurn}

\begin{document}
  \maketitle

  \section{The Tangent Plane}

  \begin{itemize}
    \item By a {\bf tangent vector} to a regular surface $S$, at a point $p \in S$, we mean the tangent vector $\alpha'(0)$ of a differentiable parameterized curve $\alpha: (-\epsilon, \epsilon) \ra S$ with $\alpha(0) = p$.

    \item \begin{proposition}
      Let $\ve{x}: U \sseq \Real^2 \ra S$ be a parameterization of regular surface $S$. Let $q \in U$. The vector subspace of dimension $2$, 
      \begin{align*}
        \dee \ve{x}_q(\Real^2) \sseq \Real^3
      \end{align*}
      coincides with the set of tangent vectors to $S$ at $\ve{x}(q)$.
    \end{proposition} 
    \begin{proof}
      Let $w$ be a tangent vector at $\ve{x}$. That is, let $w = \alpha'(0)$, where $\alpha:(\epsilon, -\epsilon) \ra \ve{x}(U) \sseq S$ is differentiable and $\alpha(0) = \ve{x}(q).$ Because $\ve{x}^{-1}$ is a differentiable function (See Example 2 of Section 2-3 of Do Carmo.), we have that $\beta = \ve{x}^{-1} \circ \alpha: (-\epsilon, \epsilon) \ra U$ is a differentiable function. By the definition of differentials, we have that $\dee \ve{x}_q(\beta'(0)) = w.$ Hence, $w \in \dee \ve{x}_q(\Real^2).$

      On the other hand, let $w = \dee \ve{x}_q(v)$, where $v \in \Real^2$. It is lcear that $v$ is the velocity vector of the curve $\gamma : (-\epsilon, \epsilon) \ra U$ given by:
      \begin{align*}
        \gamma(t) = tv + q.
      \end{align*}
      By the definition of the differential, $w = \alpha'(0)$ where $\alpha = \ve{x} \circ \gamma$.
    \end{proof}

    \item By the above proposition, the plane $\dee \ve{x}_q(\Real^2)$ does not depend on the parametermization $\ve{x}$.\\
    We call this plane the {\bf tangent plane} to $S$ at $p$.\\
    We denote the plane by the symbol $T_p(S)$.

    \item The choice of parametermization $\ve{x}$ around $p$ determine the basis vectors $(\partial\ve{x}/\partial u)(q)$ and $(\partial\ve{y}/\partial v)(q)$ of $T_p(S)$.\\
    We call them the {\bf basis associated to} $\ve{x}$\textbf{}.

    \item We sometimes write $\partial\ve{x}/\partial u$ as $\ve{x}_u$ and $\partial\ve{x}/\partial v$ as $\ve{x}_v$.

    \item If $\alpha = \ve{x} \circ \beta$ where $\beta: (-\epsilon, \epsilon) \ra U$ is given by $\beta(t) = (u(t),v(t))$, then the tangent vector $\alpha'(0)$ can be written in terms of the above basis vectors as follows:
    \begin{align*}
      \alpha'(0) = \frac{\dee (\ve{x} \circ \beta)}{\dee t} (0) = \frac{\dee \ve{x}(u(t), v(t))}{\dee t} (0) = \frac{\partial \ve{x}}{\partial u} \bigg|_q u'(0) + \frac{\partial \ve{x}}{\partial v} \bigg|_q v'(0) = \ve{x}_u(q) u'(0) + \ve{x}_v(q) v'(0).
    \end{align*}
    So, the vector $\alpha'(0)$ has coordinate $(u'(0), v'(0))$ in the basis $\{ \ve{x}_u(q), \ve{x}_v(q) \}$.    
  \end{itemize}
  
  \section{Differentials of Maps between Surfaces}
  \begin{itemize}
    \item Let $S_1$ and $S_2$ be two regular surfaces.\\
    Let $\varphi: V \sseq S_1 \ra S_2$ be a differentiable mapping of an open set $V$ of $S_1$ to $S_2$.

    \item If $p \in V$, we know that every tangent vector $w \in T_p(S)$ is the velocity vector $\alpha'(0)$ for a differentiable curve $\alpha: (-\epsilon, \epsilon) \ra V$ where $\alpha(0) = p$.

    \item Let $\beta = \varphi \circ \alpha$. We have that $\beta(0) = \varphi(p)$.\\
    As a result $\beta'(0)$ is a vector of $T_{\varphi(p)}(S_2)$

    \item We now define $\dee \varphi_{p}: T_p(S_1) \ra T_{\varphi(p)}(S_2)$ as follows:
    \begin{align*}
      \dee \varphi_p(w) = \beta'(0)
    \end{align*}
    for any differentiable curve $\alpha: (-\epsilon, \epsilon) \ra S_1$ such that $\alpha(0) = p$ and $\alpha'(0) = w$ and $\beta = \varphi \circ \alpha.$

    \item \begin{proposition}
      Given $w$, the definition above does not depend on the choice of $\alpha$. Moreover, the map $\dee \varphi_p$ is linear.
    \end{proposition}
    \begin{itemize}
      \item 
    \end{itemize}
  \end{itemize}
\end{document}