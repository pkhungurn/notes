\documentclass[10pt]{article}
\usepackage{fullpage}
\usepackage{amsmath}
\usepackage[amsthm, thmmarks]{ntheorem}
\usepackage{amssymb}
\usepackage{graphicx}
\usepackage{epstopdf}
\usepackage{enumerate}
\usepackage{verse}
\usepackage{tikz}

\newtheorem{lemma}{Lemma}[section]
\newtheorem{theorem}[lemma]{Theorem}
\newtheorem{definition}[lemma]{Definition}
\newtheorem{proposition}[lemma]{Proposition}
\newtheorem{corollary}[lemma]{Corollary}
\newtheorem{claim}[lemma]{Claim}
\newtheorem{example}[lemma]{Example}

\newcommand{\dee}{\mathrm{d}}
\newcommand{\In}{\mathrm{in}}
\newcommand{\Out}{\mathrm{out}}
\newcommand{\pdf}{\mathrm{pdf}}

\newcommand{\ve}[1]{\mathbf{#1}}
\newcommand{\mrm}[1]{\mathrm{#1}}
\newcommand{\etal}{{et~al.}}
\newcommand{\sphere}{\mathbb{S}^2}
\newcommand{\modeint}{\mathcal{M}}
\newcommand{\azimint}{\mathcal{N}}
\newcommand{\ra}{\rightarrow}
\newcommand{\mcal}[1]{\mathcal{#1}}
\newcommand{\likelihood}{\mathcal{L}}
\newcommand{\X}{\mathcal{X}}
\newcommand{\Y}{\mathcal{Y}}
\newcommand{\Z}{\mathcal{Z}}
\newcommand{\x}{\mathbf{x}}
\newcommand{\y}{\mathbf{y}}
\newcommand{\z}{\mathbf{z}}
\newcommand{\tr}{\mathrm{tr}}
\newcommand{\sgn}{\mathrm{sgn}}
\newcommand{\diag}{\mathrm{diag}}
\newcommand{\new}{\mathrm{new}}
\newcommand{\Arg}{\mathrm{Arg\,}}
\newcommand{\Log}{\mathrm{Log\,}}
\newcommand{\RE}{\mathrm{Re\,}}
\newcommand{\IM}{\mathrm{Im\,}}
\newcommand{\Res}{\mathrm{Res}}
\newcommand{\pv}{\mathrm{p.v.}}
\newcommand{\Real}{\mathbb{R}}
\newcommand{\sseq}{\subseteq}

\title{Differential Geometry Notes of 02/08/2013}
\author{Pramook Khungurn}

\begin{document}
  \maketitle

  \section{Differentiable Functions on Surfaces} % (fold)
  \label{sec:differentiable_functions_on_surfaces}

  \begin{itemize}
    \item We are interested in defining differential functions on a surface.

    \item A natural approach to defining differentiability on a surface is as follows. We say that a function $f: S \ra \Real$ is differentiable at point $p \in S$ if there is a coordinate neighborhood parameterized by $u$ and $v$ such that $f$'s expression in terms of $u$ and $v$ admits partial derivatives of all orders.

    \item However, the problem with this approach is that there can be many coordinate neighborhoods around $p$. Some may satisfy the conditions. Some may not.

    \item For the above definition to make sense, the differentiability should not depend on the chosen coordinate neighborhood. As such, we must show that if $p$ belongs to two coordinate neighborhoods---one with parameters $(u,v)$ and other with $(\xi, \eta)$---it is possible to pass from one to the other by means of a differentiable transformation.

    \item \begin{proposition}[Change of Parameters]
      Let $p$ be a point of a regular surface $S$.\\
      Let $\ve{x}:U \sseq \Real^2 \ra S$ and $\ve{y}:V \sseq \Real^2 \ra S$ be two parameterizations of $S$ such that $p \in \ve{x}(U) \cap \ve{x}(V) = W$.\\
      Then, the ``change of coordinates''
      \begin{align*}
        h = \ve{x}^{-1} \circ \ve{y},
      \end{align*}
      which maps $\ve{y}^{-1}(W)$ to $\ve{x}^{-1}(W)$, is a diffeomorphism. That is, it is differentiable and has a differentiable inverse $h^{-1}$.
    \end{proposition}

    \begin{proof}
      First, we note that $\ve{x}$ and $\ve{y}$ are homeomorphisms. They are both one-to-one and have continuous inverses. Thus, $h = \ve{x}^{-1} \circ \ve{y}$ is a homeomorphism.

      However, we cannot conclude that $h$ is differentiable yet. The problem is that, while $\ve{y}$ is differentiable by definition, we do not know how to differentiate $\ve{x}^{-1}: S \ra U$ because its domain $S$ is not an open set.

      The trick is to extend $\ve{x}$ so that we know that its inverse is differentiable. This is done by applying the inverse function theorem. So, let $\ve{x} = (x(u,v), y(u,v), z(u,v))$. We define $\bar{\ve{x}}: \Real^3 \ra \Real^3$ as follows:
      \begin{align*}
        \bar{\ve{x}}(u,v,t) = (x(u,v), y(u,v), z(u,v) + t).
      \end{align*}
      Let $r = \ve{y}^{-1}(p)$ and let $q = h(r)$ so that $\ve{x}(q) = p$.
      Because $\dee \ve{x}_q$ is injective, one of its Jacobian determinant is not zero. WLOG, let us assume that
      \begin{align*}
        \renewcommand{\arraystretch}{1.5}
        \bigg| \frac{\partial(x,v)}{\partial(u,v)} \bigg|
        = \begin{vmatrix}
          \frac{\partial x}{\partial u} & \frac{\partial x}{\partial v} \\
          \frac{\partial y}{\partial u} & \frac{\partial y}{\partial v}
        \end{vmatrix}
        \neq 0.
      \end{align*}
      Hence, we have that the Jacobian $\dee \bar{\ve{x}}_q$ is given by:
      \begin{align*}
        \renewcommand{\arraystretch}{1.5}
        \begin{vmatrix}
          \frac{\partial x}{\partial u} & \frac{\partial x}{\partial v} & \frac{\partial x}{\partial t} \\
          \frac{\partial y}{\partial u} & \frac{\partial y}{\partial v} & \frac{\partial v}{\partial t} \\
          0 & 0 & 1
        \end{vmatrix}
        = \bigg| \frac{\partial(x,v)}{\partial(u,v)} \bigg| \neq 0.
      \end{align*}
      As such, there's a neighborhood $M$ around $p$ such that $\bar{\ve{x}}^{-1}$ is defined and is differentiable. 

      Let $N = W \cap M$. Consider the function $F = \bar{\ve{x}}^{-1} \circ \ve{y}$ from $\ve{y}^{-1}(N) \ra \Real^3$. We have that $F$ is differentiable, and $\ve{y}^{-1}(N)$ contains $r$. Now, for all point $n \in N$, we have that $\bar{\ve{x}}^{-1}$ is of the form $(*,*,0)$ where the first two coordinates must agree with $\ve{x}^{-1}(n).$ Hence, in a neighborhood of $q$, we can say that $h = \pi \circ \bar{\ve{x}}^{-1} \circ \ve{y}$ where $\pi$ is the projection that drops the last component. Since $\pi$, $\bar{\ve{x}}$, and $\ve{y}$ are all differentiable, we have that $h$ is differentiable.
    \end{proof}

    \item \begin{definition}
      Let $f: V \sseq S \ra \Real$ be a function defined on an open subset $V$ of a regular surface $S$. Then $f$ is said to be differentiable at point $p \in V$ if, for some parameterization $\ve{x}: U \sseq \Real^2 \ra S$ with $p \in \ve{x}(U) \sseq V$, the composition $f \circ \ve{x}: U \sseq \Real^2 \ra \Real$ is differentiable at $\ve{x}^{-1}(p)$.

      We say that $f$ is differentiable in $V$ if it is differentiable at all points of $V$.
    \end{definition}

    \item From now on, when $f$ is a function from a surface to the reals, we will sometimes write $f(u,v)$ instead of $f(\ve{x}(u,v))$ for some coordinate function $\ve{x}$.  
  \end{itemize}
  
  % section differentiable_functions_on_surfaces (end)  
  
\end{document}