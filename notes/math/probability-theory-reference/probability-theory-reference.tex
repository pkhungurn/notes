\documentclass[10pt]{article}
\usepackage{fullpage}
\usepackage{amsmath}
\usepackage[amsthm, thmmarks]{ntheorem}
\usepackage{amssymb}
\usepackage{graphicx}
\usepackage{enumerate}
\usepackage{verse}
\usepackage{tikz}
\usepackage{verbatim}
\usepackage{hyperref}

\newtheorem{lemma}{Lemma}
\newtheorem{theorem}[lemma]{Theorem}
\newtheorem{definition}[lemma]{Definition}
\newtheorem{proposition}[lemma]{Proposition}
\newtheorem{corollary}[lemma]{Corollary}
\newtheorem{claim}[lemma]{Claim}
\newtheorem{example}[lemma]{Example}

\def\sc#1{\dosc#1\csod}
\def\dosc#1#2\csod{{\rm #1{\small #2}}}

\newcommand{\dee}{\mathrm{d}}
\newcommand{\Dee}{\mathrm{D}}
\newcommand{\In}{\mathrm{in}}
\newcommand{\Out}{\mathrm{out}}
\newcommand{\pdf}{\mathrm{pdf}}
\newcommand{\Cov}{\mathrm{Cov}}
\newcommand{\Var}{\mathrm{Var}}

\newcommand{\ve}[1]{\mathbf{#1}}
\newcommand{\mrm}[1]{\mathrm{#1}}
\newcommand{\ves}[1]{\boldsymbol{#1}}
\newcommand{\etal}{{et~al.}}
\newcommand{\sphere}{\mathbb{S}^2}
\newcommand{\modeint}{\mathcal{M}}
\newcommand{\azimint}{\mathcal{N}}
\newcommand{\ra}{\rightarrow}
\newcommand{\mcal}[1]{\mathcal{#1}}
\newcommand{\X}{\mathcal{X}}
\newcommand{\Y}{\mathcal{Y}}
\newcommand{\Z}{\mathcal{Z}}
\newcommand{\x}{\mathbf{x}}
\newcommand{\y}{\mathbf{y}}
\newcommand{\z}{\mathbf{z}}
\newcommand{\tr}{\mathrm{tr}}
\newcommand{\sgn}{\mathrm{sgn}}
\newcommand{\diag}{\mathrm{diag}}
\newcommand{\Real}{\mathbb{R}}
\newcommand{\sseq}{\subseteq}
\newcommand{\ov}[1]{\overline{#1}}
\DeclareMathOperator*{\argmax}{arg\,max}
\DeclareMathOperator*{\argmin}{arg\,min}

\title{A Pocket Reference to Probability and Measure Theory}
\author{Pramook Khungurn}

\begin{document}
\maketitle

Materials are from \cite{Bartle:1995}.

\section{Basic Definitions}

\begin{itemize}
  \item it is convenient to work with the {\bf extended real number system} $\overline{\Real} = \Real \cup \{-\infty, \infty\}$.
  \begin{itemize}
  \item For any $x \in \Real$, we have that $-\infty < x < \infty$.
  \item We say that the length of the real line is $\infty$.
  \item We define the supremum of non-empty set of real numbers which does not have an upper bound to be $\infty$, and the infemum of the a non-empty set of real numbers which does not have a lower bound to be $-\infty$.
  \begin{itemize}
    \item In this way, all non-empty sets of real numbers (or subsets of $\overline{\Real}$) have unique supremums and infemums.
  \end{itemize}
  \item The arithematic operations between the infiniites and real numbers are as follows:
  \begin{align*}
      (\pm \infty) + (\pm \infty) = x + (\pm \infty) = (\pm \infty) + x &= \pm \infty \\
      (\pm \infty) (\pm \infty) &= +\infty \\
      (\pm \infty) (\mp \infty) &= -\infty \\        
      (\pm \infty) x = x (\pm \infty) &= \begin{cases}
          \pm \infty, & \mbox{if } x > 0, \\
          0, &\mbox{if } x = 0, \\
          \mp \infty, & \mbox{if } x < 0
      \end{cases}        
  \end{align*}
  for any (finite) real number $x$.  
  \item Note that we do not define $(\pm \infty) - (\pm \infty)$. We also do not define quotients when the denominators are $\pm \infty$.  
  \end{itemize}

  \item Let $S$ be a set. The power set of $S$ is denoted by $2^S$.
  
  \item \begin{definition}
    Let $S$ be a set. A collection of sets $\mcal{S} \subseteq 2^S$ is called a {\bf $\sigma$-algebra on $S$} if it satisifies the following properties.
    \begin{enumerate}
      \item $\emptyset, S \in \mcal{S}$.
      \item It is closed under complementation: if $A \in \mcal{S}$, then $A^c = S-A \subseteq \mcal{S}.$
      \item It is closed under countable unions: if $\{A_n : n \in \mathbb{N} \}$ is a countable collection of sets in $\mcal{S}$, then $\bigcup_{n=1}^\infty A_n \in \mcal{S}$.
    \end{enumerate}    
  \end{definition}

  \item Let $\mcal{S}$ be a $\sigma$-algebra on $S$. One can easily show that, if $\{A_n : n \in \mathbb{N} \}$ be a countable collection of sets in $\mcal{S}$, then $\bigcap_{n=1}^\infty A_n \in \mcal{S}$ as well. So, a $\sigma$-algebra is also closed under finite intersection.

  \item A {\bf measurable space} is a tuple $(S,\mcal{S})$ where $S$ is a set, and $\mcal{S}$ is $\sigma$-algebra on $S$.  

  \item \begin{definition}
    Let $\mcal{A}$ be a non-empty collection of subsets of $S$. The {\bf $\sigma$-algebra generated by $\mcal{A}$}, denoted by $\sigma(\mcal{A})$ is the smallest $\sigma$-algebra that contains $\mcal{A}$. In other words,
    \begin{align*}
      \sigma(\mcal{A}) = \bigcap\Big\{\tilde{\mcal{A}} \subseteq 2^S : \mcal{A} \subseteq \tilde{\mcal{A}} \mbox{ and $\tilde{\mcal{A}}$ is a $\sigma$-algebra}\Big\}.
    \end{align*}
  \end{definition}
  
  \item \begin{definition} The {\bf Borel algebra} on $\Real^d$ is the $\sigma$-algebra $\mcal{B}(\Real^d)$ generated by the set of ``rectangles,'' which are sets of the form
  \begin{align*}
    [a_1, b_1] \times [a_2, b_2] \times \dotsb \times [a_d, b_d]
  \end{align*}
  where $-\infty < a_i < b_i < \infty$ for $i = 1, 2, \dotsc, n$.
  Any set in $\mcal{B}(\Real^d)$ is called a {\bf Borel set}.  
  \end{definition}

  \item Alternatively, one can define $\mcal{B}(\Real^d)$ to be the $\sigma$-algebra generated by open sets in $\Real^d$. This is equivalent to the above definition because every open set is a countable union of rectangles.

  \begin{definition}
    Let $(S, \mcal{S})$ be a measurable space. A {\bf measure} is a function $\mu: \mcal{S} \rightarrow [0, \infty]$ with the following properties.
    \begin{enumerate}
      \item $\mu(\emptyset) = 0$.
      \item $\mu$ is countably additive. That is, for a sequence $(E_n)$ of disjoint sets, it holds that
      \begin{align*}
        \mu\bigg( \bigcup_{n=1}^\infty E_n \bigg) = \sum_{n=1}^\infty \mu(E_n).
      \end{align*}
    \end{enumerate}
  \end{definition}

  \item If $\mu(E) < \infty$ for all $E \in \mcal{S}$, we say that $\mu$ is {\bf finite}.
  
  \item A {\bf probability measure} is a finite measure with $\mu(S) = 1$.
  
  \item If there exists a sequence $(E_n)$ of sets in $\mcal{S}$ with $\bigcup_{i=1}^\infty E_n = S$ and such that $\mu(E_n) < \infty$ for all $n$, then we say that $\mu$ is {\bf $\sigma$-finite}.
  
  \item \begin{proposition}
    Let $\mu$ be a measure defined on a $\sigma$-algebra $\mcal{S}$. 
    \begin{enumerate}
      \item If $E, F \in \mcal{S}$ and $E \subseteq F$, then $\mu(E) \leq \mu(F)$. If $\mu(E) < \infty$, then $\mu(F - E) = \mu(F) - \mu(E)$.
      
      \item Let ${E_n \in \mcal{S}: n \in \mcal{N}}$ be a collection of sets where $E_n \subset E_{n+1}$ for all $n$, then
      \begin{align*}
        \mu\bigg( \bigcup_{n=1}^\infty E_n \bigg) = \lim_{n \rightarrow \infty} \mu(E_n)
      \end{align*}

      \item Let ${E_n \in \mcal{S}: n \in \mcal{N}}$ be a collection of sets where $E_n \supset E_{n+1}$ for all $n$, and $\mu(E_1) < \infty$, then
      \begin{align*}
        \mu\bigg( \bigcap_{n=1}^\infty E_n \bigg) = \lim_{n \rightarrow \infty} \mu(E_n)
      \end{align*}
    \end{enumerate}
  \end{proposition}

  \item \begin{theorem}
    There is a unique measure $\lambda^*$ defined on the Borel algebra $\mcal{B}(\Real^n)$ such that
    \begin{align*}
      \lambda^*([a_1,b_1] \times [a_2,b_2] \times \dotsc [a_d,b_d]) = (b_1 - a_1)(b_2 - a_2) \dotsm (b_d - a_d)
    \end{align*}
    for all rectangles $[a_1,b_1] \times [a_2,b_2] \times \dotsc [a_d,b_d]$. This measure is called the {\bf Lebesgue measure}.
  \end{theorem}

  \item The Lebesgue measure is not finite ($\lambda^*(\Real^d) = \infty$), but it is $\sigma$-finite because $\Real^d$ can be covered by a countable collection of rectangles of volume 1.

  \item A {\bf measure space} is a triple $(S, \mcal{S}, \mu)$ where $S$ is a non-empty set, $\mcal{S}$ is a $\sigma$-algebra on $S$, and $\mu$ is a measure on $\mcal{S}$.
  
  \item \begin{definition}
    In a measure space $(S, \mcal{S}, \mu)$, a set $N \in \mcal{S}$ is set to be of {\bf measure zero} or a {\bf null set} if $\mu(N) = 0$. A property that holds on $N^c$ is said to hold {\bf $\mu$-almost everywhere}. In the context where $\mu$ is clear, we says that a property holds just {\bf almost everywhere}.
  \end{definition}
  
  \item \begin{definition}
    A {\bf probability space} is a triple $(\Omega, \Sigma, P)$ where $\Omega$ is a non-empty set, $\Sigma$ is a $\sigma$-algebra on $\Omega$, and $P$ is a probability measure on $\Sigma$. We often call $\Omega$ the {\bf sample space}. An element of $\Sigma$ is called an {\bf event}. If $E$ is an event, $P(E)$ is referred to as the {\bf probability} of $E$.
  \end{definition}
\end{itemize}

\section{Measurable Functions and Random Variables}

\begin{itemize}
  \item \begin{definition}
    Let $(S, \mcal{S})$ be a measurable space. A real-valued function $f:S \rightarrow \Real$ is {\bf $\mcal{S}$-measurable} if, for all $B \in \mcal{B}(\Real)$, we have that $f^{-1}(B) = \{ x \in X : f(x) \in B \} \in \mcal{S}$. When the $\sigma$-algebra in the context is clear, we will simply say that the function is measurable.
  \end{definition}

  \item Examples of easy measurable functions.
  \begin{itemize}
    \item The constant function $f(x) = c$ for some $c \in \Real$.
    \item The identity function $f: \Real \rightarrow \Real$ where $f(x) = x$. Here, the measurable space is $(\Real, \mcal{B}(\Real))$.
    \item Let $A \in \mcal{S}$. The {\bf indicator function} $\chi_A$ is given by
    \begin{align*}
      \chi_A(x) = \begin{cases}
        1, & x \in A, \\
        0, & x \not\in A.
      \end{cases}
    \end{align*}    
  \end{itemize}

  \item \begin{proposition}
    If $f$ and $g$ are measurable functions and $c \in \Real$, then 
    $$ cf, \qquad f^2, \qquad f+g, \qquad fg, \qquad |f|, \qquad 1/f, \qquad \min(f,g), \qquad \max(f,g) $$
    are also measurable. For the case of $1/f$, we assume that $f(x) \neq 0$ for all $x$.
  \end{proposition}

  \item We may define measurability for an extended real-valued function $f: S \rightarrow \overline{\Real}$ in a similar manner in exactly the same way as we define measurablity of a real-valued function because the $\mcal{B}(\Real)$ include intervals of the form $(-\infty, a]$, $(b,\infty)$ and $(-\infty,\infty)$. 
  
  \item The collection of all extended real-valued $\mcal{S}$-measurable function on $X$ is denoted by $M(S,\mcal{S})$. The collection of non-negative functions in $M(S, \mcal{S})$ is denoted by $M^+(S,\mcal{S})$.
  
  \item \begin{proposition}
    An extended real-valued function $f$ is measurable if and only if the set $\{ x \in X : f(x) = \infty\}$ and $\{ x \in X : f(x) = -\infty\}$ are measurable, and the real-valued function $f_0$ defined by
    \begin{align*}
      f_0(x) = \begin{cases}
        f(x), & f(x) \not\in \{-\infty,\infty\} \\
        0, & f(x) \in \{-\infty,\infty\}
      \end{cases}
    \end{align*}
    is measurable.
  \end{proposition}

  \item As a consequence of the above proposition, if $f$ and $g$ are measurable extended-real valued functions and $c \in Real$, then
  \begin{align*}
    cf, \qquad f^2, \qquad fg, \qquad |f|, \qquad 1/f, \qquad \min(f,g), \qquad \max(f,g)
  \end{align*}
  are also measurable with the usual caveat that $f$ should not be $0$ when assessing the measurability of $1/f$.

  \item The measurability of $f+g$ needs greater care because we cannot say anything about it if there is an $x$ where $f(x) = \pm\infty$ and $g(x) = \mp\infty$. Otherwise, $f+g$ is measurable given that $f$ and $g$ are measurable.
  
  \item \begin{proposition}
    Let $\{ f_n : n \in \mathbb{N} \}$ be a sequence of functions in $M(S,\mcal{S})$. Then, all of the functions
    \begin{align*}
      \underline{f}(x) &= \inf_{n \geq 1} f_n(x),\\
      \overline{f}(x) &= \sup_{n \geq 1} f_n(x), \\
      \underline{F}(x) &= \liminf_{n \in \mathbb{N}} f_n(x) = \sup_{n \geq 1} \left\{ \inf_{m \geq n} f_m(x) \right\}, \\
      \overline{F}(x) &= \limsup_{n \in \mathbb{N}} f_n(x) = \inf_{n \geq 1} \left\{ \sup_{m \geq n} f_m(x) \right\}.
    \end{align*}
    also belong to $M(S,\mcal{S})$. Moreover, if $\{ f_n : n \in \mathbb{N} \}$ converges to a function $f$, then $f \in M(S,\mcal{S})$.
  \end{proposition}

  \item In the context of a probability space $(\Omega, \Sigma, P)$, a $\Sigma$-measurable real-valued function is called a {\bf random variable}.
  
  \item Notational conventions.
  \begin{itemize}
    \item We denote a random variable by captical letters such as $X$, $Y$, $Z$. 
    \item We typically do not write it in functional forms such as $X(\omega)$, $Y(\omega)$, and so on. 
    \item We write $P(X^{-1}(B))$ as $P(X \in B)$.
  \end{itemize}
  
  \item \begin{proposition}
    Let $X: \Omega \rightarrow \Real$ be a random variable. Then, $\{ X^{-1}(B) : B \in \mcal{B}(\Real )\}$ is a $\sigma$-algebra called the {\bf $\sigma$-algebra generated by $X$}. It is denoted by $\sigma(X)$. It is a the smallest $\sigma$-algebra where $X$ is measurable.
  \end{proposition}

  \item \begin{proposition}
    Let $X$ be a random variable. The function $P_X: \Sigma \rightarrow [0,1]$ given by $P_X(A) = P(X \in A)$ is a probability measure on $\Sigma$ and $\sigma(X)$. It is called the {\bf probability distribution measure of $X$} .
  \end{proposition}

  \item \begin{definition}
    Let $X$ be a random variable the {\bf cumulative distribution function (CDF) of $X$} is given by $F_X(x) = P(X \leq x)$ for any $x \in \Real$.
  \end{definition}

  \item The CDF has the following properties.
  \begin{itemize}
    \item $F_X$ is monotone.
    \item $F_X$ is right continuous. In other words, $\lim_{x \rightarrow c^+} F_X(x) = F_X(c)$.
    \item $\lim_{x \rightarrow \infty} F_X(x) = 1$.
    \item $\lim_{x \rightarrow -\infty} F_X(x) = 0$.
  \end{itemize}  

  \item \begin{definition}
    Let $X$ be a probability distribution. If the CDF $F_X$ is differentiable, then its derivative is called the {\bf probability density function (PDF) of $X$} and is denoted by
    \begin{align*}
      f_X(x) = \frac{\dee F_X}{\dee x}.
    \end{align*}    
  \end{definition}  

  \item We have that
  \begin{align*}
    P(X \leq a) &= F_X(a) = \int_{-\infty}^a f_X(x)\, \dee x \\
    P(a \leq X \leq b) &= F_X(b) - F_X(a) = \int_a^b f_X(x)\, \dee x.
  \end{align*}
\end{itemize}

\section{Lebesgue Integration}

\begin{itemize}
  \item \begin{definition}
    A measurable real-valued function $\varphi$ is {\bf simple} if it attains a finite number of values.
  \end{definition}

  \item A simple function can be written as a linear combination of indicator functions of measurable sets.
  \begin{align*}
    \varphi(x) = \sum_{i=1}^n \alpha_i \chi_{A_i}
  \end{align*}
  where each $\alpha_i \in \Real$ and each $A_i$ is measurable. There is a unique ``standard representation'' where the $\alpha_i$'s are distinct, and the $A_i$'s are disjoint from 
  one another.

  \item \begin{definition} Let $(S,\mcal{S},\mu)$ be a measure space. We define the integral of functions $M(S,\mcal{S})$ in multiple steps.
  \begin{enumerate}
    \item Let $\varphi \in M^+(S,\mcal{S})$ be simple. The {\bf (Lebesgue) integral of $\varphi$ with respect to $\mu$} (or simply the integral) is the extended real number
    \begin{align*}
      \int \varphi\, \dee\mu = \sum_{i=1}^n \alpha_i \mu(A_i)
    \end{align*}
    where the $\alpha_i$'s and the $A_i$'s form the standard representation of $\varphi$.

    \item Let $f \in M^+(S,\mcal{S})$. Then,
    \begin{align*}
      \int f\, \dee\mu = \sup \bigg\{ \int \varphi\, \dee\mu\ \bigg| \  \varphi \in M(S,\mcal{S})\mbox{ and }0 \leq \varphi(x) \leq f(x)\mbox{ for all }x\in S \bigg\}. 
    \end{align*}

    \item Let $f \in M(S,\mcal{S})$. Then,
    \begin{align}
      \int f\, \dee\mu = \int f^+\, \dee\mu - \int f^-\, \dee\mu.
      \label{eqn:integral-definition}
    \end{align}
    where $f^+ = \max(f,0)$ and $f^- = -\min(f,0)$. The integral is defined only if the RHS does not turn into $\infty - \infty$.
  \end{enumerate}  
  \end{definition}

  \item \begin{definition}
    The collection $L = L(S,\mcal{S},\mu)$ of {\bf integrable functions} consists of all real-valued $\mcal{S}$-measurable functions $f$ defined on $X$, such that both the positive and negative parts ($f^+$ and $f^-$) of $f$ have finite integrals with respective to $\mu$. In other words, the RHS of \eqref{eqn:integral-definition} does not involve any infinities.
  \end{definition}

  \item \begin{definition}
    Let $(S,\mcal{S},\mu)$, $A \in \mcal{S}$, and $f \in M(S,\mcal{S})$. The {\bf integral of $f$ over $A$ with respect to $\mu$} is given by
    \begin{align*}
      \int_A f\, \dee\mu = \int f\,\chi_E\, \dee\mu.
    \end{align*}
  \end{definition}

  \item In literature, there are several different equivalent notations for the integral:
  \begin{align*}
    \int_A f\, \dee\mu = \int_A f(x)\, \dee\mu = \int_A f(x)\,\dee\mu(x) = \int_A f(x)\, \mu(\dee x).
  \end{align*}

  \item \begin{proposition}
    The integral satisfies the following properties.
    \begin{enumerate}
      \item If $f,g \in M^+(S, \mcal{S})$ and $f(x) \leq g(x)$ for all $x$, then $\int f\, \dee\mu \leq \int g\, \dee\mu$.
      
      \item If $f \in M^+(S,\mcal{S})$, $A,B \in \mcal{S}$, and $A \subseteq B$, then $\int_A f\,\dee\mu \leq \int_B f\,\dee\mu$.
      
      \item If $f \in L(S,\mcal{S},\mu)$ and $c \in \Real$, then $$\int cf\, \dee\mu = c \int f\, \dee\mu.$$
      
      \item If $f,g \in L(S,\mcal{S},\mu)$, then $$\int (f+g)\, \dee\mu = \int f\, \dee\mu + \int g\, \dee\mu.$$
      \item $f \in L(S,\mcal{S},\mu)$ if and only if $|f| \in L(S,\mcal{S},\mu)$. If $f \in L(S,\mcal{S},\mu)$, we have that
      \begin{align*}
        \bigg|\int f\, \dee\mu\bigg| \leq \int |f|\, \dee\mu.
      \end{align*}

      \item Let $A_1, A_2, \dotsc, A_n$ be disjoint sets such that $\bigcup_{i=1}^n A_i = S$. Then, for any $f \in L(S,\mcal{S},\mu)$, we have that
      \begin{align*}
        \int f\, \dee\mu = \sum_{i=1}^n \int_{A_i} f\, \dee\mu.
      \end{align*}

      \item If $f \in M^+(S,\mcal{S})$, then $f(x) = 0$ $\mu$-almost everywhere if and only if $\int f\, \dee\mu = 0$.      
    \end{enumerate}
  \end{proposition}

  \item \begin{definition}
    Let $\mcal{X}$ be a $\sigma$-algebra on a set $X$. A function $\mu: \mcal{X} \rightarrow 
    \Real$ is said to be a {\bf charge} on $\mcal{X}$ if the following properties are satisfied.
    \begin{enumerate}
      \item $\mu(\emptyset) = 0$.
      \item $\mu$ is countably additive. This is, for a sequence $(E_n)$ of disjoint sets, it holds that
      \begin{align*}
        \mu\bigg( \bigcup_{n=1}^\infty E_n \bigg) = \sum_{n=1}^\infty \mu(E_n).
      \end{align*}
    \end{enumerate}
  \end{definition}

  \item The difference between a charge and a measure is that a measure is always non-negative, but a charge can be negative. Moreover, a charge cannot take infinite values.

  \item \begin{proposition} \label{proposition:change-from-integral}
    If $f \in L(S,\mcal{S},\mu)$, then the function $$ \lambda(E) = \int_E f\, \dee\mu $$
    is a charge. If $f \in M^+(S,\mcal{S})$, then $\lambda$ is a measure.
  \end{proposition}

  \item \begin{definition}
    Consider two measures $\lambda$ and $\mu$ on $(S,\mcal{S})$. We say that $\lambda$ is {\bf absolutely continuous with respect to} $\mu$ if $\mu(E) = 0$ implies that $\lambda(E) = 0$. If this is the case, we write $\lambda \ll \mu$.
  \end{definition}

  \item The measure $\lambda(E) = \int_E f\,\dee\mu$ defined from $f \in M^+(S,\mcal{S})$ is absolutely continuous with respect to $\mu$.  
\end{itemize}

\section{Convergence Theorems}

\begin{itemize}
  \item  
\end{itemize}

\section{Random Variables}

\section{Conditional Probability}

\section{Conditional Expectation}

\bibliographystyle{apalike}
\bibliography{probability-theory-reference}  
\end{document}