\documentclass[10pt]{article}
\usepackage{fullpage}
\usepackage{amsmath}
\usepackage[amsthm, thmmarks]{ntheorem}
\usepackage{amssymb}
\usepackage{graphicx}
\usepackage{enumerate}
\usepackage{verse}
\usepackage{tikz}
\usepackage{verbatim}
\usepackage{hyperref}
\usepackage{bbm}

\newtheorem{lemma}{Lemma}
\newtheorem{theorem}[lemma]{Theorem}
\newtheorem{definition}[lemma]{Definition}
\newtheorem{proposition}[lemma]{Proposition}
\newtheorem{corollary}[lemma]{Corollary}
\newtheorem{claim}[lemma]{Claim}
\newtheorem{example}[lemma]{Example}

\newcommand{\dee}{\mathrm{d}}
\newcommand{\Dee}{\mathrm{D}}
\newcommand{\In}{\mathrm{in}}
\newcommand{\Out}{\mathrm{out}}
\newcommand{\pdf}{\mathrm{pdf}}
\newcommand{\Cov}{\mathrm{Cov}}
\newcommand{\Var}{\mathrm{Var}}

\newcommand{\ve}[1]{\mathbf{#1}}
\newcommand{\ves}[1]{\boldsymbol{#1}}
\newcommand{\mrm}[1]{\mathrm{#1}}
\newcommand{\etal}{{et~al.}}
\newcommand{\sphere}{\mathbb{S}^2}
\newcommand{\modeint}{\mathcal{M}}
\newcommand{\azimint}{\mathcal{N}}
\newcommand{\ra}{\rightarrow}
\newcommand{\mcal}[1]{\mathcal{#1}}
\newcommand{\X}{\mathcal{X}}
\newcommand{\Y}{\mathcal{Y}}
\newcommand{\Z}{\mathcal{Z}}
\newcommand{\x}{\mathbf{x}}
\newcommand{\y}{\mathbf{y}}
\newcommand{\z}{\mathbf{z}}
\newcommand{\tr}{\mathrm{tr}}
\newcommand{\sgn}{\mathrm{sgn}}
\newcommand{\diag}{\mathrm{diag}}
\newcommand{\Real}{\mathbb{R}}
\newcommand{\sseq}{\subseteq}
\newcommand{\ov}[1]{\overline{#1}}
\newcommand{\data}{\mathrm{data}}

\DeclareMathOperator*{\argmin}{arg\,min}


\title{Quaternion Exponentiation and Logarithm}
\author{Pramook Khungurn}

\begin{document}
\maketitle

This note is about quaternion exponentiation. I'm basing this note on the note by Glenn Rowe \cite{Rowe:2025}.

\section{Quaternions}

\begin{itemize}
    \item A quaternion is a mathematical object of the form
    \begin{align*}
        a + b\ve{i} + c\ve{j} + d\ve{k}
    \end{align*}
    where $a, b, c, d$ are real numbers, and $\ve{i}, \ve{j}, \ve{k}$ are basis vectors that conform to the multiplication rules below:
    \begin{align*}
        \ve{i}^2 &= -1, & \ve{i}\ve{j} &= \ve{k}, & \ve{j}\ve{k} &= -\ve{i}, \\ 
        \ve{j}\ve{i} &= -\ve{k}, & \ve{j}^2 &= -1, & \ve{j}\ve{k} &= \ve{i}, \\
        \ve{k}\ve{i} &= \ve{j}, & \ve{k}\ve{j} &= -\ve{i}, & \ve{k}^2 &= -1.
    \end{align*}

    \item Let us make note of an interesting property. Let $s = b\ve{i} + c\ve{j} + d\ve{k}$. That is, $s$ is a quaternion without a real part, which means that it is \emph{purely imaginary}. Then,
    \begin{align*}
        s^2 
        &= (b\ve{i} + c\ve{j} + d\ve{k})^2\\
        &= b^2\ve{i}^2 + c^2\ve{j}^2 + d^2\ve{k}^2 + bc\ve{i}\ve{j} + bc\ve{j}\ve{i} + cd\ve{j}\ve{k} + cd\ve{k}\ve{j} + bd\ve{k}\ve{i} + bd\ve{i}\ve{k} \\
        &= -b^2 -c^2 -d^2 + bc\ve{k} - bc\ve{k} + cd\ve{i} - cd\ve{i} + bd\ve{j} - bd\ve{j} \\
        &= -(b^2 + c^2 + d^2)
    \end{align*}

    \item The norm of the quaternion $q$ is defined as
    \begin{align*}
        \|q\| = \sqrt{a^2 + b^2 + c^2 + d^2}.
    \end{align*}

    \item If $s$ is a purely imaginary quaternion, then $$s^2 = -\|s\|^2.$$
    In particular, for $k \in \mathbb{N} \cup \{0\}$,
    \begin{align*}
        s^k = \begin{cases}
            (-1)^{k/2} \|s\|^{k}, & \mbox{$k$ is even} \\
            (-1)^{(k-1)/2} \|s\|^{k-1} s, & \mbox{$k$ is odd}
        \end{cases}.
    \end{align*}

    \item Another way to denote the above fact is to write $s = u\theta$ where $\theta = \| s \|$ and $u$ is a unit vector in $\Real^3$ that makes the equation true. (In other words, $u$ is uniquely determined if $\| s \| \neq 0$, but we can pick any unit vector if $\| s \| = 0$.) We have that
    \begin{align*}
        u^k = \begin{cases}
            (-1)^{k/2} , & \mbox{$k$ is even} \\
            (-1)^{(k-1)/2} u, & \mbox{$k$ is odd}
        \end{cases}.
    \end{align*}
    So,
    \begin{align*}
        s^k = (u \theta)^k = \begin{cases}
            (-1)^{k/2} \theta^k , & \mbox{$k$ is even} \\
            (-1)^{(k-1)/2} u \theta^k, & \mbox{$k$ is odd}
        \end{cases}.
    \end{align*}    
\end{itemize}

\section{Quaternion Exponentiation}

\begin{itemize}
    \item Let $s = u\theta$ be a purely imaginary quaternion. We have that
    \begin{align*}
        e^s = e^{u\theta}
        &= \sum_{k=0}^\infty \frac{(u\theta)^k}{k!} \\
        &= \sum_{k=0}^\infty \frac{(u\theta)^{2k}}{(2k)!} + \sum_{k=0}^\infty \frac{(u\theta)^{2k+1}}{(2k+1)!} \\
        &= \sum_{k=0}^\infty \frac{(-1)^k \theta^{2k}}{(2k)!} + \sum_{k=0}^\infty \frac{(-1)^k u \theta^{2k+1}}{(2k+1)!} \\        
        &= \sum_{k=0}^\infty \frac{(-1)^k \theta^{2k}}{(2k)!} + u \sum_{k=0}^\infty \frac{(-1)^k \theta^{2k+1}}{(2k+1)!} \\
        &= \cos \theta + u \sin \theta.
    \end{align*}

    \item As a result, for $q = a + b\ve{i} + c\ve{j} + d\ve{k} = a + s = a + u\theta$, we have that
    \begin{align*}
        e^q = e^{a+u\theta} = e^a e^{u\theta} = e^a \left( \cos \theta + u \sin \theta \right).
    \end{align*}
\end{itemize}

\section{Quaternion Logarithm}

\begin{itemize}
    \item Let $q$ be a unit quaternion. We can always find $\theta \in \Real$ and $u = x\ve{i} + y\ve{j} + z\ve{k}$ such that $\| u \| = 1$ such that
    \begin{align*}
        q = \cos\theta + u \sin\theta.
    \end{align*}

    \item From the last section, we know that $e^{u\theta} = \cos\theta + u \sin\theta$. As a result, we may say that
    \begin{align*}
        \log q = \log (\cos \theta + u \sin \theta) = u\theta.
    \end{align*}

    \item For a general quaternion $q$, we may write $q = \|q\| (\cos \theta + u \sin \theta)$. Hence,
    \begin{align*}
        \log q = \log \big( \|q\| (\cos \theta + u \sin \theta) \big) = \log \|q\| + \log (\cos \theta + u \sin \theta) = \log \|q\| + u\theta.
    \end{align*}
\end{itemize}

\section{Rotation and Logarithm}

\begin{itemize}
    \item The \emph{conjugate} of the quaternion $q = a + b\ve{i} + c\ve{j} + d\ve{k}$ is defined as
    \begin{align*}
        q^* = a - b\ve{i} - c\ve{j} - d\ve{k}.
    \end{align*}
    
    \item If we write $q = a + s$ where $s = b\ve{i} + c\ve{j} + d\ve{k}$. Then, $q^* = a - s$. Moreover,
    \begin{align*}
        qq^* = (a + s)(a - s) = a^2 - s^2 = a^2 + \| s \|^2 = a^2 + b^2 + c^2 + d^2 = \|q\|^2.
    \end{align*}
    We can show that $q^*q = \| q \|^2 = qq^*$ as well.

    \item Let $q = \cos (\theta/2) + u \sin (\theta/2)$ be a unit quaternion. For any purely imaginary quaternion $v = b \ve{i} + c \ve{j} + d \ve{k}$, it is well-known that $qvq^*$ is the rotation of $\ve{v}$ around the axis $u$ by an angle of $\theta$. As a result, a rotation in $\Real^3$ can be represented by a unit quaternion.
    
    \item We can go even further. When we represent a rotation by a unit quaternion $q$, we can take the logarithm of $q$ to get a vector $u \theta/2 \in \Real^3$. So, a rotation in $\Real^3$ can also be represented by a vector in $\Real^3$.
    
    \item Note, however, that the logarithm representation is not unique. This is because $e^{u\theta} = e^{u(\theta + 2\pi k)}$ for any integer $k$.
\end{itemize}


\bibliographystyle{arxivalpha}
\bibliography{quaternion-exp}  
\end{document}