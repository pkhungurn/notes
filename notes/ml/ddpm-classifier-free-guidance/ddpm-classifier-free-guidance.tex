\documentclass[10pt]{article}
\usepackage{fullpage}
\usepackage{amsmath}
\usepackage[amsthm, thmmarks]{ntheorem}
\usepackage{amssymb}
\usepackage{graphicx}
\usepackage{enumerate}
\usepackage{verse}
\usepackage{tikz}
\usepackage{verbatim}
\usepackage{hyperref}

\newtheorem{lemma}{Lemma}
\newtheorem{theorem}[lemma]{Theorem}
\newtheorem{definition}[lemma]{Definition}
\newtheorem{proposition}[lemma]{Proposition}
\newtheorem{corollary}[lemma]{Corollary}
\newtheorem{claim}[lemma]{Claim}
\newtheorem{example}[lemma]{Example}

\newcommand{\dee}{\mathrm{d}}
\newcommand{\Dee}{\mathrm{D}}
\newcommand{\In}{\mathrm{in}}
\newcommand{\Out}{\mathrm{out}}
\newcommand{\pdf}{\mathrm{pdf}}
\newcommand{\Cov}{\mathrm{Cov}}
\newcommand{\Var}{\mathrm{Var}}

\newcommand{\ve}[1]{\mathbf{#1}}
\newcommand{\mrm}[1]{\mathrm{#1}}
\newcommand{\ves}[1]{\boldsymbol{#1}}
\newcommand{\etal}{{et~al.}}
\newcommand{\sphere}{\mathbb{S}^2}
\newcommand{\modeint}{\mathcal{M}}
\newcommand{\azimint}{\mathcal{N}}
\newcommand{\ra}{\rightarrow}
\newcommand{\mcal}[1]{\mathcal{#1}}
\newcommand{\X}{\mathcal{X}}
\newcommand{\Y}{\mathcal{Y}}
\newcommand{\Z}{\mathcal{Z}}
\newcommand{\x}{\mathbf{x}}
\newcommand{\y}{\mathbf{y}}
\newcommand{\z}{\mathbf{z}}
\newcommand{\tr}{\mathrm{tr}}
\newcommand{\sgn}{\mathrm{sgn}}
\newcommand{\diag}{\mathrm{diag}}
\newcommand{\Real}{\mathbb{R}}
\newcommand{\sseq}{\subseteq}
\newcommand{\ov}[1]{\overline{#1}}
\DeclareMathOperator*{\argmax}{arg\,max}
\DeclareMathOperator*{\argmin}{arg\,min}
\newcommand{\N}{\mathcal{N}}
\newcommand{\data}{\mathrm{data}}
\newcommand{\SNR}{\mathrm{SNR}}
\newcommand{\sigmoid}{\mathrm{sigmoid}}

\title{Classifer-Free Diffusion Guidance}
\author{Pramook Khungurn}

\begin{document}
\maketitle

This note is written as I read the paper ``Classifer-Free Diffusion Guidance'' by Ho and Salimans \cite{Ho:2022}.

\section{Introduction}

\begin{itemize}
\item Classifier guidance.
\begin{itemize}
  \item First derived by Sohl-Dicksteing \etal\ in 2015 \cite{SohlDickstein:2015}.
  
  \item Popularized by Dhariwal and Nichol in 2021 \cite{Dhariwal:2021}.
  
  \item The technique allows us to force a DDPM to generate images of a certain class.
  
  \item It also allows a DDPM to trade diversity of generated data items for their fidelity.
  \begin{itemize}
    \item This helps improve metrics such as FID, sFID, and precision.
    \item It is similar to the way GANs can improve their metrics by utilizing the ``trucation trick.''
  \end{itemize}

  \item We train a DDPM as usual. However, we also need to train a classifier $p_{\ves{\phi}}(y|\ves{x}^{(t)})$ on degraded samples $\ve{x}^{(t)}$.
  
  \item In the backward process, when sampling $\ve{x}^{(t-1)}$ from $\ve{x}^{(t)}$, we use the distribution
  \begin{align*}
    p_{\ves{\theta}, \ves{\phi}}(\ve{x}^{(t-1)}|\ve{x}^{(t)}, y)
    &= \mcal{N}\big(\ve{x}^{(t-1)}; \ves{\mu} + s \Sigma \nabla \log p_{\ves{\theta}}(y | \ves{\mu})\big)
  \end{align*}
  where
  \begin{itemize}
    \item $\ves{\mu} = \ves{\mu}_{\ves{\theta}}(\ve{x}^{(t)},t)$ is the mean of the backward transition,
    \item $\Sigma = \Sigma_{\ves{\theta}}(\ve{x}^{(t)},t)$ is the covariance matrix of the backward transition, and
    \item $s$ is a scaling factor.
  \end{itemize}
  
  \item See more details in another note of mine \cite{Khungurn:2022a}.
\end{itemize}

\item Research question: can guidance be performed without a classifier?
\begin{itemize}
  \item To use a classifier guidance, we need to train a classifier from scratch.
  \begin{itemize}
    \item Existing ones cannot be used because we need to train them on degraded samples.
  \end{itemize}

  \item Their use in DDPM sampling is also cumbersome.
  \begin{itemize}
    \item Need to evaluate the gradient of the log of the probability.
    
    \item The most convenient way to do this is to use automatic differentiation in deep learning packages. Still, this is still a hassle.
  \end{itemize}

  \item Ho and Salimans also worry that classifier guidance makes DDPMs similar to GANs.
    \begin{itemize}
      \item Classifier guidance is an ``attack'' on a certain classifier because it tries to fool that classifier to think that the generated samples are from a certain class.
      
      \item So, in a sense, classifier guidance is similar to adversarial training.
      
      \item Do we have a non-GAN-like way to trade diversity for fidelity?
    \end{itemize}
\end{itemize}
\end{itemize}

\section{Background}

\begin{itemize}
  \item Ho and Salimans use the formalism developed in Kingma \etal's paper \cite{Kingma:2021} with some modifications.  

  \begin{itemize}
    \item $\ve{x}$ denoted a data item that is sampled from a data distribution $p(\ve{x})$.
    
    \item The time parameter $t$ is dropped in favor of $\lambda$. A degraded data item is denoted by $\ve{z}_\lambda$, which is indexed by $\lambda \in [\lambda_{\min}, \lambda_{\max}] \subseteq \Real$ like this $\ve{z}_\lambda$. Here, $\lambda$ can be negative.
    
    \item The forward process is denoted by $q$, so we have probabilities like $q(\ve{z}_\lambda | \ve{x})$ and $q(\ve{z}_{\lambda} | \ve{z}_{\lambda'})$.

    \item We define
    \begin{align*}
      \alpha_{\lambda}^2 
      &= \sigmoid(\lambda) = \frac{1}{1 + e^{-\lambda}}, 
      \\
      \sigma_\lambda^2       
      &= 1 - \alpha^2_\lambda = \frac{e^{-\lambda}}{1 + e^{-\lambda}} = \frac{1}{1 + e^{\lambda}} = \sigmoid(-\lambda), 
      \\
      \alpha^2_{\lambda_0|\lambda_1} 
      &= \frac{\alpha^2_{\lambda_0}}{\alpha^2_{\lambda_1}} = \frac{1 + e^{-\lambda_1}}{1 + e^{-\lambda_0}}, 
      \\
      \sigma^2_{\lambda_0|\lambda_1}
      &= \sigma^2_{\lambda_0} - \alpha^2_{\lambda_0|\lambda_1} \sigma^2_{\lambda_1} 
      = 1 - \alpha^2_{\lambda_0} - \frac{\alpha_{\lambda_0}^2}{\alpha_{\lambda_1}^2} (1 - \alpha^2_{\lambda_1}) 
      = 1 - \alpha_{\lambda_0}^2 - \frac{\alpha_{\lambda_0}^2}{\alpha_{\lambda_1}^2} + \alpha_{\lambda_0}^2
      = 1 - \frac{\alpha_{\lambda_0}^2}{\alpha_{\lambda_1}^2} 
      \\
      &= 1 - \frac{1 + e^{-\lambda_1}}{1 + e^{-\lambda_0}} 
      = \frac{e^{-\lambda_0} - e^{-\lambda_1}}{1 + e^{-\lambda_0}}
      = \frac{e^{\lambda_0}(e^{-\lambda_0} - e^{-\lambda_1})}{e^{\lambda_0}(1 + e^{-\lambda_0})}
      = \frac{1 - e^{\lambda_0-\lambda_1'}}{1 + e^{\lambda_0}} 
      \\
      &= (1 - e^{\lambda_0-\lambda_1'}) \sigma_{\lambda_0}^2.
    \end{align*}
    Note that the formula for $\sigma_{\lambda_0|\lambda_1}^2$ works only when $\lambda_0 \leq \lambda_1$. Otherwise, we would have $\sigma^2_{\lambda_0|\lambda_1} < 0$.

    \item The signal-to-noise ratio (SNR) is given by
    \begin{align*}
      \SNR(\lambda) = \frac{\alpha^2_\lambda}{\sigma^2_\lambda} = \frac{1 + e^{\lambda}}{1 + e^{-\lambda}} = \frac{e^{\lambda}(1 + e^{-\lambda})}{1 + e^{-\lambda}} = e^{\lambda}.
    \end{align*}
    So,
    \begin{align*}
      \lambda = \log \frac{\alpha_\lambda^2}{\sigma^2_\lambda} = \log \SNR(\lambda).
    \end{align*}

    \item As a result, the forward process starts from $\lambda_{\max}$ (highest SNR, least noise) and goes to $\lambda_{\min}$ (lower SNR, most noise). The backward process does the opposite.
    
    \item As usual, we set
    \begin{align*}
      q(\ve{z}_\lambda | \ve{x}) &= \N(\alpha_\lambda \ve{x}, \sigma_\lambda^2 I), \\
      q(\ve{z}_{\lambda_0}|\ve{z}_{\lambda_1}) &= \N(\alpha_{\lambda_0|\lambda_1} \ve{x}, \sigma^2_{\lambda_0|\lambda_1} I).
    \end{align*}
    for any $\lambda$, $\lambda_0$, and $\lambda_1$ with the condition that $\lambda_0 < \lambda_1$.
  
    \item Given $\lambda_0 < \lambda_1$, we can show that
    \begin{align*}
      q(\ve{z}_{\lambda_1} | \ve{z}_{\lambda_0}, \ve{x})
      &= \N(\ve{z}_{\lambda_1}; \ves{\mu}_{\lambda_1|\lambda_0}(\ve{z}_{\lambda_0}, \ve{x}), \tilde{\sigma}^2_{\lambda_1|\lambda_0}I)
    \end{align*}
  \end{itemize}

  
\end{itemize}

\appendix

\section{Gaussian Identities}

\begin{itemize}
  \item Many of these identities come from a lecture note by Marc Toussaint \cite{Toussaint:2011}.
  
  \item A multivariate Gaussian with mean $\ves{\mu}$ and covariance matrix $\Sigma$, denoted by $\N(\ves{\mu},\Sigma)$ is the distribution:
  \begin{align*}
  \N(\ve{x}; \ves{\mu}, \Sigma) = \frac{1}{(\det 2\pi\Sigma)^{1/2}} \exp\bigg(-\frac{1}{2} (\ve{x}-\ves{\mu})^T \Sigma^{-1} (\ve{x}-\ves{\mu}) \bigg).
  \end{align*}
  It is defined only if the covariance matrix is positive definite.  
  
  \item \begin{proposition}
    For any invertible matrix $A$ and any vector $\ve{b}$, we have that
    \begin{align*}
    \N(A\ve{x} + \ve{b}; \ves{\mu}, \Sigma) = \frac{1}{|\det A|} \N(\ve{x}, A^{-1}(\ves{\mu} - \ve{b}), A^{-1}\Sigma A^{-T}).
    \end{align*}
  \end{proposition}

  \begin{proof}
    \begin{align*}
    \N(\ve{x}; \ves{\mu}, \Sigma) 
    &= \frac{1}{(\det 2\pi\Sigma)^{1/2}} \exp\bigg(-\frac{1}{2} (A\ve{x} + \ve{b} -\ves{\mu})^T \Sigma^{-1} (A\ve{x} + \ve{b} -\ves{\mu}) \bigg) \\
    &= \frac{1}{(\det 2\pi\Sigma)^{1/2}} \exp\bigg(-\frac{1}{2} (A\ve{x} + \ve{b} -\ves{\mu})^T A^{-T} A^T \Sigma^{-1} A A^{-1} (A\ve{x} + \ve{b} -\ves{\mu}) \bigg) \\
    &= \frac{1}{(\det 2\pi\Sigma)^{1/2}} \exp\bigg(-\frac{1}{2} (\ve{x} - A^{-1}(\ves{\mu} - \ve{b}))^T (A^{-1} \Sigma A^{-T})^{-1} (\ve{x} + A^{-1}(\ves{\mu} - \ve{b})) \bigg) \\
    &= \frac{1}{(\det AA^T)^{1/2}} \frac{1}{ (\det A^{-1}A^{-T})^{1/2} (\det 2\pi\Sigma)^{1/2}} \\
    & \qquad \exp\bigg(-\frac{1}{2} (\ve{x} - A^{-1}(\ves{\mu} - \ve{b}))^T (A^{-1} \Sigma A^{-T})^{-1} (\ve{x} + A^{-1}(\ves{\mu} - \ve{b})) \bigg) \\
    &= \frac{1}{|\det A|} \frac{1}{ (\det 2\pi A^{-1}\Sigma A^{-T})^{1/2}} \\
    & \qquad \exp\bigg(-\frac{1}{2} (\ve{x} - A^{-1}(\ves{\mu} - \ve{b}))^T (A^{-1} \Sigma A^{-T})^{-1} (\ve{x} + A^{-1}(\ves{\mu} - \ve{b})) \bigg) \\
    &= \frac{1}{|\det A|} \N(\ve{x}, A^{-1}(\ves{\mu} - \ve{b}), A^{-1}\Sigma A^{-T})
    \end{align*}
    as required.
  \end{proof}  

  \item \begin{corollary}
    if $a \in \Real$ and $\ve{b} \in \Real^d$ is a vector, then
    \begin{align*}
      \N(a\ve{x} + \ve{b}; \ves{\mu}, \Sigma)
      &= \frac{1}{|a|^d} \N\bigg(\ve{x}; \frac{\ves{\mu} - \ve{b}}{a}, \frac{\Sigma}{a^2}\bigg).
    \end{align*}
  \end{corollary}

  \item \begin{proposition}
    \begin{align*}
      \N(\ve{x};\mu_1, \Sigma_1) \N(\ve{x};\mu_2, \Sigma_2) = \N(\mu_1; \mu_2, \Sigma_1 + \Sigma_2) \N(\ve{x}; \mu_3, \Sigma_3)
  \end{align*}
  where
  \begin{align*}
  \mu_3 &= \Sigma_2(\Sigma_1 + \Sigma_2)^{-1} \mu_1 + \Sigma_1(\Sigma_1 + \Sigma_2)^{-1} \mu_2, \\
  \Sigma_3 &= \Sigma_1 (\Sigma_1 + \Sigma_2)^{-1} \Sigma_2.
  \end{align*}
  \end{proposition}

  We will not prove this proposition. It looks painful.

  \item \begin{proposition} \label{gaussian-convolution}
    \begin{align*}
    \mcal{N}(x;\mu_1, \sigma_1^2) * \mcal{N}(x;\mu_2, \sigma_2^2)
    &= \int \mcal{N}(t;\mu_1, \sigma_1^2) \mcal{N}(x-t;\mu_1, \sigma_1^2)\, \dee t
    = \mcal{N}(x; \mu_1 + \mu_2, \sigma_1^2 + \sigma_2^2).
    \end{align*}
  \end{proposition}
  \begin{proof}
    Let $X_1$ be a random variable distributed according to $\mcal(\mu_1, \sigma_1^2)$, and let $X_2$ be a random variable distributed according to $\mcal(\mu_2, \sigma_2^2)$. Let $X_1$ and $X_2$ be indepdendnet of each other. Define $X = X_1 + X_2$. Then,
    we have that the probability distribution function of $X$ is given by
    \begin{align*}
      p_X(x) = \mcal{N}(x;\mu_1, \sigma_1^2) * \mcal{N}(x;\mu_2, \sigma_2^2).
    \end{align*}
    Observe that 
    \begin{align*}
      E[X] &= E[X_1] + E[X_2] = \mu_1 + \mu_2.
    \end{align*}
    Moreover,
    \begin{align*}
      \Var(X_1 + X_2) = \Var(X_1) + \Var(X_2) = \sigma_1^2 + \sigma_2^2.
    \end{align*}
    So, the proposition would be true if we can show that $p_X(x)$ is a Gaussian. In other words, if we can show that there are constants $A$, $B$, and $C$ such that
    \begin{align*}
      p_X(x) = A \exp\big(-B(x - C)^2\big)
    \end{align*}    
    with $A, B > 0$, then we would be done.

    So, let us that that
    \begin{align*}
      \mcal{N}(t;\mu_1, \sigma^2) &= A_1 \exp\big(-B_1(t - C_1)^2\big), \\
      \mcal{N}(x - t;\mu_2, \sigma^2) &= A_2 \exp\big(-B_2(x - t - C_2)^2\big)
    \end{align*}
    with $A_1, B_1, A_2, B_2 > 0$.
    It follows that
    \begin{align*}
      p_X(x) 
      &= \int A_1 \exp\big(-B_1(t - C_1)^2\big)A_2 \exp\big(-B_2(x - t - C_2)^2\big)\, \dee t \\
      &= A_1 A_2 \int \exp\big(-B_1(t - C_1)^2 - B_2(x - t - C_2)^2 \big)\, \dee t.
    \end{align*}
    Now, we have that
    \begin{align*}
      &-B_1(t-C_1)^2 - B_2(x - t - C_2)^2 \\
      &= -B_2x^2 + 2B_2 C_2 x - (B_1 + B_2)t^2 +2(B_1 C_1 + B_2 x - B_2C_2) t - B_1C_1^2 -B_2C_2^2.
    \end{align*}
    Now, let's simplify the expression by given the constants new names.
    \begin{align*}
      -B_1(t-C_1)^2 - B_2(x - t - C_2)^2
      &= -B_2 x^2 + 2D_2 x - (B_1 + B_2) t^2 + 2(B_2x + D_4)t + D_5.
    \end{align*}
    We know that $D_3 = B_1 + B_2 > 0$. Let's complete the square of the polynomial that involves $t$. We have that
    \begin{align*}
      &-(B_1 + B_2) t^2 + 2(B_2x + D_4)t \\
      &= -(B_1 + B_2)\bigg( t^2 - 2\frac{B_2 x + D_4}{B_1 + B_2} t \bigg) \\
      &= -(B_1 + B_2)\bigg( t^2 - 2\frac{B_2 x + D_4}{B_1 + B_2} t + \frac{(B_2 x + D_4)^2}{(B_1 + B_2)^2} - \frac{(B_2 x + D_4)^2}{(B_1 + B_2)^2} \bigg) \\
      &= -(B_1 + B_2)\bigg( t^2 - 2\frac{B_2 x + D_4}{B_1 + B_2} t + \frac{(B_2 x + D_4)^2}{(B_1 + B_2)^2}  \bigg) + \frac{(B_2 x + D_4)^2}{B_1 + B_2} \\
      &= -(B_1 + B_2)\bigg( t - \frac{B_2 x + D_4}{B_1 + B_2} \bigg)^2 + \frac{(B_2 x + D_4)^2}{B_1 + B_2} \\
      &= -(B_1 + B_2)( t - D_6 x - D_7)^2 + \frac{B_2^2}{B_1+B_2} x^2 + 2D_8x + D_9.
    \end{align*}
    So,
    \begin{align*}
      &-B_2 x^2 + 2D_2 x - (B_1 + B_2) t^2 + 2(B_2x + D_4)t + D_5 \\
      &= -B_2 x^2 + 2D_2 x -(B_1 + B_2)( t - D_6 x - D_7)^2 + \frac{B_2^2}{B_1+B_2} x^2 + 2D_8x + D_9 + D_5 \\
      &= -\bigg( B_2 - \frac{B_2^2}{B_1 + B_2} \bigg) x^2  + 2(D_2 + D_8) x -(B_1 + B_2)( t - D_6 x - D_7)^2 + D_{10} \\
      &= -\bigg( \frac{B_1 B_2}{B_1 + B_2} \bigg) x^2  + 2(D_2 + D_8) x -(B_1 + B_2)( t - D_6 x - D_7)^2 + D_{10}.
    \end{align*}
    Note that, because $B_1, B_2 > 0$, we have that $B_1B_2/(B_1 + B_2) > 0$. Now, we can complete the square of the terms that involve $x$:
    \begin{align*}
      -\bigg( \frac{B_1 B_2}{B_1 + B_2} \bigg) x^2  + 2(D_2 + D_8) x
      &= -\bigg( \frac{B_1 B_2}{B_1 + B_2} \bigg)(x - D_{11})^2 + D_{12}.
    \end{align*}
    This gives
    \begin{align*}
      &-\bigg( \frac{B_1 B_2}{B_1 + B_2} \bigg) x^2  + 2(D_2 + D_8) x -(B_1 + B_2)( t - D_6 x - D_7)^2 + D_{10} \\
      &= -\bigg( \frac{B_1 B_2}{B_1 + B_2} \bigg)(x - D_{11})^2 -(B_1 + B_2)( t - D_6 x - D_7)^2 + D_{10} + D_{12} \\
      &= -\bigg( \frac{B_1 B_2}{B_1 + B_2} \bigg)(x - D_{11})^2 -(B_1 + B_2)( t - D_6 x - D_7)^2 + D_{13}.
    \end{align*}
    In other words,
    \begin{align*}
      -B_1(t-C_1)^2 - B_2(x - t - C_2)^2 = -D_{14}(x - D_{11})^2 - D_{15}(t - D_6x - D_7)^2 + D_{13}
    \end{align*}
    for some $D_{14}, D_{15} > 0$.

    Hence,
    \begin{align*}
      p_X(x) 
      &= A_1 A_2 \int \exp\big( -D_{14}(x - D_{11})^2 - D_{15}(t - D_6x - D_7)^2 + D_{13}  \big)\, \dee t \\
      &= A_1 A_2 e^{D_{13}} \bigg( \int \exp( - D_{15}(t - D_6x - D_7)^2 )\, \dee t \bigg) \exp(-D_{14}(x - D_{11})^2) .
    \end{align*}
    The integral on the last line is a positive constant because it is an integral of a Gaussian. So, $p_X(x)$ is a Gaussian. We are done.
  \end{proof}
\end{itemize}

\bibliographystyle{alpha}
\bibliography{ddpm-classifier-free-guidance}  
\end{document}