\documentclass[10pt]{article}
\usepackage{fullpage}
\usepackage{amsmath}
\usepackage[amsthm, thmmarks]{ntheorem}
\usepackage{amssymb}
\usepackage{graphicx}
\usepackage{enumerate}
\usepackage{verse}
\usepackage{tikz}
\usepackage{verbatim}
\usepackage{hyperref}

\newtheorem{lemma}{Lemma}
\newtheorem{theorem}[lemma]{Theorem}
\newtheorem{definition}[lemma]{Definition}
\newtheorem{proposition}[lemma]{Proposition}
\newtheorem{corollary}[lemma]{Corollary}
\newtheorem{claim}[lemma]{Claim}
\newtheorem{example}[lemma]{Example}

\newcommand{\dee}{\mathrm{d}}
\newcommand{\Dee}{\mathrm{D}}
\newcommand{\In}{\mathrm{in}}
\newcommand{\Out}{\mathrm{out}}
\newcommand{\pdf}{\mathrm{pdf}}
\newcommand{\Cov}{\mathrm{Cov}}
\newcommand{\Var}{\mathrm{Var}}

\newcommand{\ve}[1]{\mathbf{#1}}
\newcommand{\ves}[1]{\boldsymbol{#1}}
\newcommand{\mrm}[1]{\mathrm{#1}}
\newcommand{\etal}{{et~al.}}
\newcommand{\sphere}{\mathbb{S}^2}
\newcommand{\modeint}{\mathcal{M}}
\newcommand{\azimint}{\mathcal{N}}
\newcommand{\ra}{\rightarrow}
\newcommand{\mcal}[1]{\mathcal{#1}}
\newcommand{\X}{\mathcal{X}}
\newcommand{\Y}{\mathcal{Y}}
\newcommand{\Z}{\mathcal{Z}}
\newcommand{\x}{\mathbf{x}}
\newcommand{\y}{\mathbf{y}}
\newcommand{\z}{\mathbf{z}}
\newcommand{\tr}{\mathrm{tr}}
\newcommand{\sgn}{\mathrm{sgn}}
\newcommand{\diag}{\mathrm{diag}}
\newcommand{\Real}{\mathbb{R}}
\newcommand{\sseq}{\subseteq}
\newcommand{\ov}[1]{\overline{#1}}
\newcommand{\N}{\mathcal{N}}
\newcommand{\data}{\mathrm{data}}
\newcommand{\SNR}{\mathrm{SNR}}

\title{Variational Diffusion Models}
\author{Pramook Khungurn}

\begin{document}
\maketitle

This note is written as I read the paper ``Variational Diffusion Models'' by Kingma \etal.~\cite{Kingma:2021}.

\section{Introduction}

\begin{itemize}
  \item The paper lists two contributions.
  
  \item First, it proposes a new family of diffusion-based generative models.
  \begin{itemize}
    \item It incorporate Fourier features.
    \item It can joinly optimize the noise schedule together with the rest of the model.
    \item It can be easily casted to continuous time settings.
  \end{itemize}

  \item Second, it contributes new theoretical understanding of diffusion-based generative models.
  \begin{itemize}
    \item Derive a simple expression of the variational lower bound (VLB) in terms of the signal-to-noise ratio.
    \item Prove a new invariance in the continuous time setting.
    \item Show that various diffusion models in literature are equivalent up to a trivial time-dependent rescaling of the data.
  \end{itemize}

  \item The end result is that the authors got a model that achieved SOTA log likelihood at the time.
  \begin{itemize}
    \item However, FID score was not the best when compared to other models, so the method might not lead to the best looking images.
    
    \item Their model is also large, deep, and kind of impossible to train if you don't have enough resource.
  \end{itemize}
\end{itemize}

\section{Model}

\begin{itemize}
  \item A data item is represented by $\ve{x} \in \Real^d$.
  
  \item The data distribution is denoted by $p(\ve{x})$, which we want to model.
\end{itemize}

\subsection{Forward Time Diffusion Process}

\begin{itemize}
  \item We start with a data item $\ve{x}$ sampled according to $p(\ve{x})$.
  
  \item We define a sequence increasingly noisy versions of $\ve{x}$, which we call the {\bf latent variables} $\ve{z}_t$.
  \begin{itemize}
    \item Here, $t$ runs from $t=0$ (least noisy) to $t=1$ (most noisy).
  \end{itemize}

  \item The distribution of the latent variable $\ve{z}_t$ conditioned on $\ve{x}$ is given by
  \begin{align}
    q(\ve{z}_t|\ve{x}) = \N(\ve{z}_t ; \alpha_t \ve{x}, \sigma_t^2 I) \label{eq:zt-given-x}
  \end{align}
  where $\alpha_t$ and $\sigma^2_t$ are strictly positive scalar-valued functions of $t$.

  \item We also assume that $\alpha_t$ and $\sigma_t$ are smooth.
  \begin{itemize}
    \item In other words, they have continuous first derivatives with respect to $t$, and the derivatives are finite.
  \end{itemize}  

  \item Define the {\bf signal-to-noise radio (SNR)} to be
  \begin{align*}
    \SNR(t) = \alpha_t^2 / \sigma_t^2.
  \end{align*}
  
  \item The SNR should be monotonically decreating in time.
  \begin{itemize}
    \item In other words, $t > s$ implies $\SNR(t) < \SNR(s)$.
    \item This formalizes the notion that, as $t$ increases, the latent variable should become noisier.
  \end{itemize}  

  \item In the original DDPM paper \cite{Ho:2020}, we have that $\alpha_t = \sqrt{1 - \sigma_t^2}$.
  \begin{itemize}
    \item So, $\alpha_t^2 + \sigma_t^2 = 1$ for all $t$.
    \item As a result, we call such a model {\bf variance preserving}.
  \end{itemize}

  \item In the paper by Song \etal\ on the SDE formulation of score-based models \cite{Song:2020:SDE}, we have a model where $\alpha_t = 1$ for all $t$.
  \begin{itemize}
    \item As $t \rightarrow 1$, $\sigma_t^2$ must increase in order for the SNR to decrease.
    \item This means that $\alpha_t^2 + \sigma_t^2 = 1 + \sigma_t^2$, which increase as $t$ increase.
    \item As a result, we call such a model {\bf variance exploding}.
    \item In fact, the SDE for such a model is calle the variance-exploding SDE (VE-SDE).
  \end{itemize}
  
  \item We also require that the forward time process also satisfies the following properties.
  \begin{enumerate}
    \item For any $0 \leq s < t \leq 1$, we have that
    \begin{align}
      q(\ve{z}_t | \ve{z}_s) = \N(\ve{z}_t ; \alpha_{t|s}\ve{z}_s; \sigma^2_{t|s} I) \label{eq:zt-given-zs}
    \end{align}
    where $\alpha_{t|s} = \alpha_t / \alpha_s$ and $\sigma^2_{t|s} = \sigma_t^2 - \alpha^2_{t|s} \sigma^2_{s}.$

    \item The joint distribution $(\ve{z}_s, \ve{z}_t, \ve{z}^{u})$ for any $0 \leq s < t < u \leq 1$ is Markov. In other words,
    \begin{align*}
      q(\ve{z}_u|\ve{z}_t, \ve{z}_s) = q(\ve{z}_u|\ve{z}_t).
    \end{align*}    
  \end{enumerate}

  \item We want the model to be consistent. In other words, it should be the case that
  \begin{itemize}
    \item \eqref{eq:zt-given-x} should be consistent with \eqref{eq:zt-given-zs}, and
    \item \eqref{eq:zt-given-zs} should be consistent with itself.
  \end{itemize}
  This is indeed the case, and the proofs can be found in Appendix~\ref{sec:model-properties}.

  \item It can be shown that, for any $0 \leq s < t \leq 1$, we have that
  \begin{align*}
    q(\ve{z}_s|\ve{z}_t, \ve{x}) = \N(\ve{z}_s; \ves{\mu}_Q(\ve{z}_t, \ve{x}; s, t), \sigma^2_Q(s,t) I )
  \end{align*}
  where
  \begin{align*}
    \sigma_Q^{2}(s,t) & = \sigma_{t|s}^2 \sigma_s^2 / \sigma_t^2, \\
    \ves{\mu}_Q(\ve{z}_t, \ve{x}; s, t) &= \frac{\alpha_{t|s}\sigma_s^2}{\sigma_t^2} \ve{z}_t + \frac{\alpha_s \sigma_{t|s}^2}{\sigma_t^2} \ve{x}.
  \end{align*}
  See a proof also in Appendix~\ref{sec:model-properties}.
\end{itemize}

\subsection{Noise Schedule}

\begin{itemize}
  \item In works such as \cite{Ho:2020}, the nosie schedule has a fixed form.
  
  \item The paper proposes learning the noise schedule through the parameterization
  \begin{align*}
    \sigma_t^2 = \mrm{sigmoid}(\gamma_{\ves{\eta}}(t)) = \frac{1}{1 + \exp(-\gamma_{\ves{\eta}}(t))}
  \end{align*}
  where $\gamma_{\ves{\eta}}(t)$ is a monotonic neural network with parameter $\ves{\eta}$.
  
  \item The specification of $\gamma_{\ves{\eta}}(t)$.
  \begin{itemize}
    \item It has 3 linear layers with weights that are restricted to be positive.
    
    \item Let us call the layers $l_1$, $l_2$, and $l_3$. Then,
    \begin{align*}
      \gamma_{\ves{\eta}}(t) := l_1(t) + l_3(\phi(l_2(l_1(t))))
    \end{align*}
    where $\phi$ is the sigmoid function.

    \item $l_2$ has 1024 outputs while other layers have only a single output.
  \end{itemize}

  \item The paper fixes $\alpha_t = \sqrt{1 - \sigma_t^2}$, subscribing to the variance-perserving camp.
  \begin{itemize}
    \item However, we will show later that variance-preserving models and variance-exploding models are equivalent.    
  \end{itemize}

  \item We now have that
  \begin{align*}
    \alpha_t^2 
    &= 1 - \sigma_t^2 
    = 1 - \frac{1}{1 + \exp(-\gamma_{\ves{\eta}}(t))}
    = \frac{\exp(-\gamma_{\ves{\eta}}(t))}{1 + \exp(-\gamma_{\ves{\eta}}(t))}
    = \frac{1}{1 + \exp(\gamma_{\ves{\eta}}(t))} \\
    &= \mrm{sigmoid}(-\gamma_{\ves{\eta}}(t)), \\
    \SNR(t)
    &= \frac{\alpha^2_t}{\sigma^2_t}
    = \frac{\exp(-\gamma_{\ves{\eta}}(t))}{1 + \exp(-\gamma_{\ves{\eta}}(t))} (1 + \exp(-\gamma_{\ves{\eta}}(t))) \\
    &= \exp(-\gamma_{\ves{\eta}}(t)).
  \end{align*}
\end{itemize}

\subsection{Reverse Time Generative Process}

\begin{itemize}
  \item The generative model is defined by inverting the forward time process.
  
  \item It samples a sequence of latent variables $\ve{z}_t$ with time running backward from $t = 1$ to $t = 0$.

  \item The model can be defined in the discrete time and continuous time setting. We will discuss the discrete time setting first.
  
  \item Definitions for the discrete time settings.
  \begin{itemize}
    \item Let $T$ be a positive integer. 
    \item We split the time interval $[0,1]$ into $T$ segments, each with width $\tau = 1/T$.
    \item Define $s(i) = (i-1)/T$ and $t(i) = i/T$.
    \item The generative model for data item $\ve{x}$ is given by:
    \begin{align*}
      p(\ve{x}) = \int_{z} p(\ve{z}_1) p(\ve{x}|\ve{z}_0) \prod_{i=1}^T p(\ve{z}_{s(i)}|\ve{z}_{t(i)})\, \dee\ve{z}.
    \end{align*}
    Here, $\ve{z}$ denotes $(\ve{z}_0, \ve{z}_{1/T}, \ve{z}_{2/T}, \dotsc, \ve{z}_1)$.
  \end{itemize}

  \item With the variance preserving setting and sufficiently small $\SNR(1)$, we have that $q(\ve{z}_1|\ve{x}) \approx \N(\ve{z}_1 ; \ve{0}, I)$. So, we can model the marginal distribution of $\ve{z}_1$ with $\N(\ve{0},I)$. In other words,
  \begin{align*}
    p({\ve{z}_1}) = \N(\ve{z}_1; \ve{0}, I).
  \end{align*}

  \item For $p(\ve{x}|\ve{z}_0)$, the paper factors the terms into independent components. Let the $i$th component of $\ve{x}$ and $\ve{z}_0$ be denoted by $x_i$ and $z_{0,i}$, respectively. We set
  \begin{align*}
    p(\ve{x}|\ve{z}_0) = \prod_{i=1}^d p(x_i|z_{0,i})
  \end{align*}
  and 
  \begin{align*}
    p(x_i | z_{0,i}) = \frac{q(z_{0,i}|x_i)}{\sum_{x=0}^{255} q(z_{0,i}| x)}
  \end{align*}
  taking into account that each $x_{i}$ is an 8-bit pixel value. The last equation is just applying Bayes' rule assuming that each pixel value is equally likely.

  \item Lastly, we choose
  \begin{align*}
    p(\ve{z}_s|\ve{z}_t) = q(\ve{z}_t |\ve{z}_t, \ve{x} = \hat{\ve{x}}_{\ves{\theta}}(\ve{z}_t; t)).
  \end{align*}
  This is the same as $q(\ve{z}_s;\ve{z}_t, \ve{x})$ we discussed in the last section but the sampled data $\ve{x}$ is replaced by a {\bf denoising model} $\hat{\ve{x}}_{\ves{\theta}}(\ve{z}_t, t)$ that predicts $\ve{x}$ from $\ve{z}_t$. 
  
  \item To be more concrete, we can also rewrite $p(\ve{z}_s | \ve{z}_t)$ as
  \begin{align*}
    p(\ve{z}_s|\ve{z}_t) 
    &= \N(\ve{z}_s; \ves{\mu}_{\ves{\theta}}(\ve{z}_t; s, t), \sigma_Q^2(s,t) I)
  \end{align*}
  where
  \begin{align*}
    \ves{\mu}_{\ves{\theta}}(\ve{z}_t; s, t)
    &= \ves{\mu}_Q(\ve{z}_t, \hat{\ve{x}}_{\ves{\theta}}(\ve{z}_t; t); s, t)
    = \frac{\alpha_{t|s} \sigma_s^2}{\sigma_t^2} \ve{z}_t + \frac{\alpha_s \sigma_{t|s}^2}{\sigma_t^2} \hat{\ve{x}}_{\ves{\theta}}(\ve{z}_t; t).
  \end{align*}

  \item The mean of the backward step $\ves{\mu}_{\ves{\theta}}(\ve{z}_t; s, t)$ can also be written as
  \begin{align*}
    \ves{\mu}_{\ves{\theta}}(\ve{z}_t; s, t)
    &= \frac{1}{\alpha_{t|s}} \ve{z}_t - \frac{\sigma^2_{t|s}}{\alpha_{t|s}\sigma_t} \hat{\ves{\xi}}_{\ves{\theta}}(\ve{z}_t;t) 
    = \frac{1}{\alpha_{t|s}} + \frac{\sigma_{t|s}^2}{\alpha_{t|s}} \ve{s}_{\ves{\theta}}(\ve{z}_t; t)
  \end{align*}
  where
  \begin{align*}
    \hat{\ves{\xi}}_{\ves{\theta}}(\ve{z}_t;t) = \frac{\ve{z}_t - \alpha_t \hat{\ve{x}}_{\ves{\theta}}(\ve{z}_t;t)}{\sigma_t}
  \end{align*}
  is the {\bf noise prediction model} that predicts that Gaussian noise $\ves{\xi} \sim \N(0,I)$ that is used to make $\ve{z}_t = \alpha_t \ve{x} + \sigma_t \ves{\xi}$, and
  \begin{align*}
    \ve{s}_{\ves{\theta}}(\ve{z}_t;t) = -\frac{\hat{\ves{\xi}}_{\ves{\theta}}(\ve{z}_t; t)}{\sigma_t}
  \end{align*}
  is the {\bf score model} that predicts the score $\nabla \log q(\ve{z}_t)$ from $\ve{z}_t$.

  \item Moreover, we can simplify $\ves{\mu}_{\ves{\theta}}(\ve{z}_t; s, t)$ and $\sigma^2_Q(s,t)$ further:
  \begin{align*}
    \ves{\mu}_{\ves{\theta}}(\ve{z}_t; s, t) 
    &= \frac{\ve{z}_t + \sigma_t \mrm{expm1}( \gamma_{\ves{\eta}}(s) - \gamma_{\ves{\eta}}(t)) \hat{\ves{\xi}}_{\ves{\theta}}(\ve{z}_t; t)}{\alpha_{t|s}} \\
    \sigma_Q^2(s,t) &= -\sigma_s^2 \mrm{expm1}(\gamma_{\ves{\eta}}(s) - \gamma_{\ves{\eta}}(t))
  \end{align*}
  where $\mrm{expm1}(u) = e^u - 1$.\footnote{In numerical software packages such as NumPy, Torch, and JAX, $\mrm{expm1}$ is available as a function because the straightforward computation is not very accurate and numerically stable.} See the proof at Proposition~\ref{thm:backward-process-simplification} in Appendix~\ref{sec:model-properties}.
\end{itemize}

\subsection{Noise Prediction Model}

\begin{itemize}
  \item Following Ho \etal~\cite{Ho:2020}, the paper trains the noise prediction model $\hat{\ves{\xi}}_{\ves{\theta}}(\cdot; \cdot)$.
  
  \item The relationship between $\hat{\ves{\xi}}_{\ves{\theta}}(\cdot; \cdot)$ and $\hat{\ve{x}}_{\ves{\theta}}(\cdot; \cdot)$ is as follows:
  \begin{align*}
    \hat{\ve{x}}_{\ves{\theta}}(\ve{z}_t; t)
    &= \frac{\ve{z}_t - \sigma_t \hat{\ves{\xi}}_{\ves{\theta}}(\ve{z}_t; t) }{\alpha_t}.
  \end{align*}

  \item The paper uses an architecture similar to that of Ho \etal with a number of modifications. The modification that they want to highlight the most is the use of Fourier features \cite{Tancik:2020}.
  \begin{itemize}
    \item The paper optimizes the network for likelihood, which is sensitive to the exact pixel values. So, it needs to capture all the fine details in the data.
    
    \item To do so, the authors propose adding a set of Fourier features to the input of the noise prediction model.
    \begin{itemize}
      \item Let $\ve{x}$ be the original data, scaled to the range $[-1,1]$, and let $\ve{z}_t$ be a latent code. 
      
      \item They concatenate to $\ve{z}_t$ channels $\sin(2^n \pi \ve{z}_t)$ and $\cos(2^n \pi \ve{z}_t)$ where $n$ runs over a range of integers from $n_{\min}$ to $n_{\max}$, and then they feed the concatenated tensor to the noise prediction model.
    \end{itemize}
    
    \item Including the features led to large improvements in log-likelihood, especially when combined with learned noise schedule. In particular, it allows the network to learn with much higher value of $\SNR_{\max}$ (i.e., much lower value for $\sigma_0^2$) than without.
    
    \item The authors got the best results with $n_{\min} = 7$ and $n_{\max} = 8$.
    \begin{itemize}
      \item This is quite surprising because it's just only 4 more channels.
      
      \item The author says lower frequencies can be learned from $\ve{z}$ itself, and high frequencies are simply not present or irrelevant for likelihood.    
    \end{itemize}    
  \end{itemize}
  
  \item Other modifications include:
  \begin{itemize}
    \item The paper's network does not perform nay downsampling or upsampling. The tensors remain at the original input resolution.
    
    \item The network is deeper than ones used by Ho \etal\ in \cite{Ho:2020}.
    \begin{itemize}
      \item For the CIFAR10 and the $32\times32$ ImageNet datasets, the authors uses U-Nets with depth of $32$ in the downsampling and upsampling (which are not actually performed).
    
      \item For the $64\times64$  ImageNet dataset, they double the depth!
    \end{itemize}
    
    \item Intead of taking time $t$ as input to the noise prediction model, they feed a scaled version of $\gamma_{\ves{\eta}}(t)$ as input to the network. The scaling is done in such a way that the value is in the range $[0,1]$.
    
    \item Apart from the attention block that connects the upward and downard branches of the U-Net in \cite{Ho:2020}, the authors remove all attention blocks from the model.
    
    \item The model use dropout of rate $0.1$.
    
    \item The authors optimized the model with the AdamW algorithm \cite{Loshchilov:2017}. The settings are as follows.
    \begin{itemize}
      \item Learning rate of $2 \times 10^{-4}$.
      \item $\beta_1 = 0.9, \beta_2 = 0.99$.
      \item Weight decay coefficient of $0.01$.
    \end{itemize}    
    
    \item The model weights are accumulated with exponential moving average with decay rate of $0.9999$.
  \end{itemize}
\end{itemize}

\subsection{Variational Lower Bound}

\begin{itemize}
  \item We train the model by trying to minimize the variational lower bound of the log likelihood. This is given by
  \begin{align*}
    -\log p(\ve{x})
    \leq \mrm{VLB}(\ve{x})
    = \underbrace{D_{KL}(q(\ve{x}_1|x) \| p(\ve{z_1}))}_{\mbox{prior loss}}
    + \underbrace{E_{\ve{z}_0 \sim q(\ve{z}_0|x)}[-\log p(\ve{x}|\ve{z}_0)]}_{\mbox{reconstruction loss}}
    + \underbrace{\mcal{L}_T(\ve{x})}_{\mbox{diffusion loss}}.
  \end{align*}
  You can find how to derive the above expression in another note of mine \cite{Khungurn:2022b}.

  \item The prior loss and the reconstruction loss can be estimated using standard techniques.
  
  \item The diffusion loss depends on the number of time steps $T$, and we will discuss it in the next sections.
\end{itemize}

\section{Discrete-Time Model}

\begin{itemize}
  \item In case of finite $T$, the diffusion loss is
  \begin{align*}
    \mcal{L}_T(\ve{x})
    &= \sum_{i=1}^T E_{\ve{z}_{t(i)} \sim q(\ve{z}_{t(i)}|\ve{x})} \big[
    D_{KL}
    (
      q( \ve{z}_{s(i)} | \ve{z}_{t(i)}, \ve{x} ) 
      \| 
      p(\ve{z}_{s(i)} | \ve{z}_{t(i)})
    )
    \big].
  \end{align*}

  \item We can simplify the diffusion loss to
  \begin{align*}
    \mcal{L}_T(\ve{x})
    &= \frac{T}{2} E_{\ves{\xi} \sim \mcal{N}(\ves{0},I), i \sim \mcal{U}\{1:T\}} \Big[ \big(\SNR(s(i)) - \SNR(t(i))\big) \| \ve{x} - \hat{\ve{x}}_{\ves{\theta}}(\ve{z}_{t(i)}; t(i)) \|^2 \Big].
  \end{align*}
  where $\ve{z}_{t(i)} = \alpha_{t(i)} \ve{x} + \sigma_{t(i)}\ves{\xi}$. See the proof in Proposition~\ref{thm:diffusion-loss-snr-finite} of Appendix~\ref{sec:model-properties}. 
  
  \item Another expression is for the loss is given by
  \begin{align*}
    \mcal{L}_T(\ve{x})
    &= \frac{T}{2} 
    E_{\ves{\xi} \sim \mcal{N}(\ves{0},I), i \sim \mcal{U}\{1:T\}} \Big[ 
      \mrm{expm1}\big(\gamma_{\ves{\eta}}(t(i)) - \gamma_{\ves{\eta}}(s(i)) \big) \| \ve{\xi} - \hat{\ve{\xi}}_{\ves{\theta}}(\ve{z}_{t(i)}; t(i)) \|^2 \Big].
  \end{align*}
  See Proposition~\ref{thm:diffusion-loss-expm1-finite} in Appendix~\ref{sec:model-properties} for the proof.

  \item Note that the rewritten loss contains explicit dependencies on $\ves{\theta}$ and $\ves{\eta}$. So, if we optimize it, we optimize both the noise prediction model and the noise schedule.
  \begin{itemize}
    \item This is different from the simplified loss in \cite{Ho:2020}, which can only be used to optimize the noise prediction model.
    
    \item It is also much simpler than the loss in Nichol and Dhariwal \cite{Nichol:2021}, which treats the loss for the noise prediction model and the loss for the noise schedule differently.
  \end{itemize}

  \item The paper also observes that more timesteps are always better in terms of minimizing the loss value.
  \begin{itemize}
    \item Imagine you graph $\SNR(t)$ versus $\| \ve{x} - \hat{\ve{x}}_{\ves{\theta}}(\ve{z}_t;t) \|^2$ where $\SNR(t)$ goes from $0$ (much noise) to $1$ (no noise).
    
    \item You would have that, when $\hat{\ve{x}}_{\ves{\theta}}$ is good enough, the graph would be descreasing as you go from $\SNR(t) = 0$ to $\SNR(t) = 1$. This is simply because it is easier to denoise an image when there is less noise in the image.
    
    \item Now, we can interpret $\SNR(s) - \SNR(t)$ as the with of an interval, and $\| \ve{x} - \hat{\ve{x}}_{\ves{\theta}}(\ve{z}_t;t) \|^2$ as the height of the graph at the beginning of the interval in the graph above.
    
    \item So, the discrete time diffusion loss is an upper Riemann sum approximation of an integral of a strictly decreasing function.
    
    \item This implies that more time steps leads to a more accurate upper bound, which is lower.
    
    \item See Figure 2 in the paper for an illustation.
  \end{itemize} 
\end{itemize}

\section{Continuous-Time Model}

\begin{itemize}
  \item We now take $T \rightarrow \infty$. The limit of $\mcal{L}_T(\ve{x})$ is given by
  \begin{align*}
    \mcal{L}_{\infty}(\ve{x}) 
    &= -\frac{1}{2} E_{\ves{\xi} \sim \mcal{N}(\ve{0},I)} \bigg[ \int_0^1 \SNR'(t) \| \ve{x} - \hat{\ve{x}}_{\ves{\theta}}(\ve{z}_t; t) \|^2 \, \dee t \bigg] \\
    &= -\frac{1}{2} E_{\ves{\xi} \sim \mcal{N}(\ve{0},I), t \sim \mcal{U}(0,1)} \big[ \SNR'(t) \| \ve{x} - \hat{\ve{x}}_{\ves{\theta}}(\ve{z}_t; t) \|^2 \big] \\
    &= \frac{1}{2} E_{\ves{\xi} \sim \mcal{N}(\ve{0},I), t \sim \mcal{U}(0,1)} \big[ \gamma_{\ves{\eta}}'(t) \| \ves{\xi} - \hat{\ves{\xi}}_{\ves{\theta}}(\ve{z}_t; t) \|^2 \big]
  \end{align*}
  where $\SNR'(t) = \dee \SNR(t) / \dee t$ and $\gamma_{\ves{\eta}}'(t) = \dee \gamma_{\ves{\theta}}(t) / \dee t$. Note that the last equality is not trivial, and its proof can be found in Appendix~\ref{sec:model-properties}.

  \item The signal-to-noise function $\SNR(t)$ is invertible because it is monotonically decreasing. So, we can perform a change of variable with $v = \SNR(t)$. This gives
  \begin{align*}
    \mcal{L}_{\infty}(\ve{x}) 
    &= -\frac{1}{2} E_{\ves{\xi} \sim \mcal{N}(\ve{0},I)} \int_0^1 \SNR'(t) \| \ve{x} - \hat{\ve{x}}_{\ves{\theta}}(\ve{z}_t; t) \|^2\, \dee t \\
    &= -\frac{1}{2} E_{\ves{\xi} \sim \mcal{N}(\ve{0},I)} \bigg[ \int_{\SNR_{\max}}^{\SNR_{\min}} \| \ve{x} - \hat{\ve{x}}_{\ves{\theta}}(\ve{z}_{\SNR^{-1}(v)}; \SNR^{-1}(v)) \|^2\, \dee v \bigg] \\
    &= \frac{1}{2} E_{\ves{\xi} \sim \mcal{N}(\ve{0},I)} \bigg[ \int_{\SNR_{\min}}^{\SNR_{\max}} \| \ve{x} - \hat{\ve{x}}_{\ves{\theta}}(\ve{z}_{\SNR^{-1}(v)}; \SNR^{-1}(v)) \|^2\, \dee v \bigg]
  \end{align*}
  where $\SNR_{\min} = \SNR(1)$ and $\SNR_{\max} = \SNR(0)$.
  
  \item The above equation shows us that the only effect the function $\alpha(t)$ and $\sigma(t)$ have on the diffusion loss is the value of $\SNR(t)$ at endpoints $t = 0$ and $t = 1$. The loss value is invarient to the shape of the function $\SNR(t)$ between $t = 0$ and $t = 1$.
  
  \item Moreover, the distribution $p(\ve{x})$ defined by the generative model is also invariant to the specification of the diffusion model.
  \begin{itemize}
    \item Let $p^A(\ve{x})$ denote the distribution defined by the combination of $\alpha_t^A$, $\hat{\sigma}_t^A$, and $\ve{x}_{\ves{\theta}}^A$. \\
    Let $p^B(\ve{x})$ be defined similarly for $\alpha_t^B$, $\sigma_t^B$, and $\hat{\ve{x}}_{\ves{\theta}}^B$.\\
    We require that both distributions have the same values of $\SNR_{\min}$ and $\SNR_{\max}$.

    \item Then, we can show that $p^A(\ve{x}) = p^B(\ve{x})$ if $\hat{\ve{x}}_{\ves{\theta}}^A(\ve{z}_t, t) = \hat{\ve{x}}^A_{\ves{\theta}}((\alpha^A_t/\alpha_t^B)\ve{z}_t, t)$. Moreover, the distribution of all latents $\ve{z}_t$ is the same up to scaling.

    \item Hence, all models that satisfies the following mild conditions are equivalent (up to scaling).
    \begin{itemize}
      \item $\alpha_t$ and $\sigma_t$ are positive scalar value functions.
      \item $\SNR(t) = \alpha_t^2 / \sigma_t^2$ is monotonically decreasing in $t$.
      \item $q(\ve{z}_t|\ve{x}) = \N(\ve{z}_t; \alpha_t \ve{x}, \sigma_t^2 I)$.
      \item For all $0 \leq s < t \leq 1$, it is true that $q(\ve{z}_t|\ve{z}_s) = \N(\ve{z}_t; \alpha_{t|s}\ve{z}_s, \sigma^2_{t|s} I)$.
      \item The forward process is Markov. That is, for any $0 \leq s < t < u \leq 1$, it follows that $q(\ve{z}_u|\ve{z}_t, \ve{z}_s) = q(\ve{z}_u|\ve{z}_t)$.
      \item $\SNR(0)$ and $\SNR(1)$ are fixed constants that agree with other models.
    \end{itemize}
    
    \item This means that the models based on the variance-exploding SDE and variance-preserving SDE in \cite{Song:2020:SDE} are equivalent in continuous time up to time-dependent scaling factors.
  \end{itemize}  

  \item The equivalence between diffusion models continues to hold even if the loss is weighted and of the form:
  \begin{align*}
  \mcal{L}_{\infty}(\ve{x}, w) &= \frac{1}{2} E_{\ves{\xi} \sim \mcal{N}(\ve{0},I)} \bigg[ \int_{\SNR_{\min}}^{\SNR_{\max}} w(v) \| \ve{x} - \hat{\ve{x}}_{\ves{\theta}}(\ve{z}_{\SNR^{-1}(v)}; \SNR^{-1}(v)) \|^2\, \dee v \bigg]
  \end{align*}

  \item Optimization of $\mcal{L}_{\infty}$ requires a lot of care. The paper has the details on how to compute the gradient of the loss in its appendix.
  \begin{itemize}
    \item We will not cover it now because I've become tired of reading.
  \end{itemize}
\end{itemize}

\section{Summary}

\begin{itemize}
  \item The paper gives a new formulation of the DDPM that deals with the noise schedule in a systematic way.
  \begin{itemize}
    \item It yields a loss function that can be used to optimize both the noise prediction model and the noise schedule in one go.
    
    \item It also shows that diffusion models that can be formulated in the paper's framework are equivalent up to scaling if the $\SNR_{\min}$ and $\SNR_{\max}$ match.
  \end{itemize}

  \item While the theoretical component of the paper is certainly valuable, I double whether the proposed new model architecture and losses are practical.
  \begin{itemize}
    \item The paper's model is very deep and hard to train.
    \item The loss is still quite complicated and require a lot of care, especially in the continuous-time setting.
    \item In the end, the architecture and the loss are designed to get better likelihood, not image quality as measured by FID scores.
  \end{itemize}
\end{itemize}

\appendix

\section{Gaussian Identities}

\begin{itemize}
  \item Many of these identities come from a lecture note by Marc Toussaint \cite{Toussaint:2011}.
  
  \item A multivariate Gaussian with mean $\ves{\mu}$ and covariance matrix $\Sigma$, denoted by $\N(\ves{\mu},\Sigma)$ is the distribution:
  \begin{align*}
  \N(\ve{x}; \ves{\mu}, \Sigma) = \frac{1}{(\det 2\pi\Sigma)^{1/2}} \exp\bigg(-\frac{1}{2} (\ve{x}-\ves{\mu})^T \Sigma^{-1} (\ve{x}-\ves{\mu}) \bigg).
  \end{align*}
  It is defined only if the covariance matrix is positive definite.  
  
  \item \begin{proposition} \label{thm:gaussian-linear-transform}
    For any invertible matrix $A$ and any vector $\ve{b}$, we have that
    \begin{align*}
    \N(A\ve{x} + \ve{b}; \ves{\mu}, \Sigma) = \frac{1}{|\det A|} \N(\ve{x}, A^{-1}(\ves{\mu} - \ve{b}), A^{-1}\Sigma A^{-T}).
    \end{align*}
  \end{proposition}

  \begin{proof}
    \begin{align*}
    \N(\ve{x}; \ves{\mu}, \Sigma) 
    &= \frac{1}{(\det 2\pi\Sigma)^{1/2}} \exp\bigg(-\frac{1}{2} (A\ve{x} + \ve{b} -\ves{\mu})^T \Sigma^{-1} (A\ve{x} + \ve{b} -\ves{\mu}) \bigg) \\
    &= \frac{1}{(\det 2\pi\Sigma)^{1/2}} \exp\bigg(-\frac{1}{2} (A\ve{x} + \ve{b} -\ves{\mu})^T A^{-T} A^T \Sigma^{-1} A A^{-1} (A\ve{x} + \ve{b} -\ves{\mu}) \bigg) \\
    &= \frac{1}{(\det 2\pi\Sigma)^{1/2}} \exp\bigg(-\frac{1}{2} (\ve{x} - A^{-1}(\ves{\mu} - \ve{b}))^T (A^{-1} \Sigma A^{-T})^{-1} (\ve{x} + A^{-1}(\ves{\mu} - \ve{b})) \bigg) \\
    &= \frac{1}{(\det AA^T)^{1/2}} \frac{1}{ (\det A^{-1}A^{-T})^{1/2} (\det 2\pi\Sigma)^{1/2}} \\
    & \qquad \exp\bigg(-\frac{1}{2} (\ve{x} - A^{-1}(\ves{\mu} - \ve{b}))^T (A^{-1} \Sigma A^{-T})^{-1} (\ve{x} + A^{-1}(\ves{\mu} - \ve{b})) \bigg) \\
    &= \frac{1}{|\det A|} \frac{1}{ (\det 2\pi A^{-1}\Sigma A^{-T})^{1/2}} \\
    & \qquad \exp\bigg(-\frac{1}{2} (\ve{x} - A^{-1}(\ves{\mu} - \ve{b}))^T (A^{-1} \Sigma A^{-T})^{-1} (\ve{x} + A^{-1}(\ves{\mu} - \ve{b})) \bigg) \\
    &= \frac{1}{|\det A|} \N(\ve{x}, A^{-1}(\ves{\mu} - \ve{b}), A^{-1}\Sigma A^{-T})
    \end{align*}
    as required.
  \end{proof}  

  \item \begin{corollary} \label{thm:gaussian-linear-transformation-scalar-ver}
    if $a \in \Real$ and $\ve{b} \in \Real^d$ is a vector, then
    \begin{align*}
      \N(a\ve{x} + \ve{b}; \ves{\mu}, \Sigma)
      &= \frac{1}{|a|^d} \N\bigg(\ve{x}; \frac{\ves{\mu} - \ve{b}}{a}, \frac{\Sigma}{a^2}\bigg).
    \end{align*}
  \end{corollary}

  \item \begin{proposition} \label{thm:gaussian-product}
    \begin{align*}
      \N(\ve{x};\mu_1, \Sigma_1) \N(\ve{x};\mu_2, \Sigma_2) = \N(\mu_1; \mu_2, \Sigma_1 + \Sigma_2) \N(\ve{x}; \mu_3, \Sigma_3)
  \end{align*}
  where
  \begin{align*}
  \mu_3 &= \Sigma_2(\Sigma_1 + \Sigma_2)^{-1} \mu_1 + \Sigma_1(\Sigma_1 + \Sigma_2)^{-1} \mu_2, \\
  \Sigma_3 &= \Sigma_1 (\Sigma_1 + \Sigma_2)^{-1} \Sigma_2.
  \end{align*}
  \end{proposition}

  We will not prove this proposition. It looks painful.

  \item \begin{proposition} \label{thm:gaussian-kl}
    Let $\ves{\mu}_1, \ves{\mu}_2 \in \Real^d$ and $\Sigma_1, \Sigma_2 \in \Real^{d\times d}$ be positive definite matrices. We have that
    \begin{align*}
      D_{KL}(\mathcal{N}(\ves{\mu}_1, \Sigma_1)\, \|\, \mathcal{N}(\ves{\mu}_2, \Sigma_2))
      &= \frac{1}{2} \bigg( \log \frac{\det \Sigma_2}{\det \Sigma_1} + \tr(\Sigma_2^{-1}\Sigma_1) + (\ves{\mu}_2 - \ves{\mu}_1)^T \Sigma_2^{-1} (\ves{\mu}_2 - \ves{\mu}_1) - d \bigg).
    \end{align*}
  \end{proposition}
  \begin{proof}
    See other sources. We will not prove this.
  \end{proof}

  \item \begin{corollary} \label{thm:gaussian-kl-spherical}
    \begin{align*}
      D_{KL}(\mathcal{N}(\ves{\mu}_1, \sigma_1^2 I)\, \|\, \mathcal{N}(\ves{\mu}_2, \sigma_2^2 I))
      &= \frac{1}{2} \bigg( 
        \frac{\| \ves{\mu}_2 - \ves{\mu}_1 \|^2}{\sigma_2^2}          
        + 2d(\log |\sigma_2| - \log |\sigma_1|)
        + d \frac{\sigma_1^2}{\sigma_2^2} 
        - d
        \bigg).
    \end{align*}
  \end{corollary}

  \begin{proof}
    Applying Proposition~\ref{thm:gaussian-kl}, we have that
    \begin{align*}
      & D_{KL}(\mathcal{N}(\ves{\mu}_1, \sigma_1^2 I)\, \|\, \mathcal{N}(\ves{\mu}_2, \sigma_2^2 I)) \\
      &= \frac{1}{2} \bigg(  \log \frac{\det (\sigma_2^2 I)}{\det (\sigma_1^2 I)}  + \tr( (\sigma_2^2 I)^{-1}\sigma_1^2 I ) + (\ves{\mu}_2 - \ves{\mu}_1)^T (\sigma_2^2 I)^{-1} (\ves{\mu}_2 - \ves{\mu}_1) - d  \bigg) \\
      &= \frac{1}{2} \bigg(  
        \log \frac{\sigma_2^{2d}}{\sigma_1^{2d}}  
        + \tr\Big( \frac{\sigma_1^2}{\sigma_2^2} I \Big) 
        + \frac{\| \ves{\mu}_2 - \ves{\mu}_1 \|^2}{\sigma_2^2}
        - d  \bigg) \\
      &= \frac{1}{2} \bigg( 
        2d(\log \sigma_2 - \log \sigma_1)
        + d \frac{\sigma_1^2}{\sigma_2^2} 
        + \frac{\| \ves{\mu}_2 - \ves{\mu}_1 \|^2}{\sigma_2^2}
        - d
        \bigg) \\
        &= \frac{1}{2} \bigg( 
          \frac{\| \ves{\mu}_2 - \ves{\mu}_1 \|^2}{\sigma_2^2}          
          + 2d(\log |\sigma_2| - \log |\sigma_1|)
          + d \frac{\sigma_1^2}{\sigma_2^2} 
          - d
          \bigg)
    \end{align*}
    as required.
  \end{proof}
\end{itemize}

\pagebreak

\section{Proofs of Model Properties} \label{sec:model-properties}

\begin{itemize}
\item \begin{proposition} \label{thm:zt-x-consistency}
  The property in Equation \eqref{eq:zt-given-x} is consistent with the property in Equation \eqref{eq:zt-given-zs}. In other words, for any $0 \leq s < t \leq 1$, it holds that
  \begin{align*}
    q(\ve{z}_t | \ve{x}) = \int q(\ve{z}_t|\ve{z}_s) q(\ve{z}_s|\ve{x})\, \dee \ve{z}_s.
  \end{align*}
\end{proposition}

\begin{proof}
\begin{align*}
  &\int q(\ve{z}_t|\ve{z}_s) q(\ve{z}_s|\ve{x})\, \dee \ve{z}_s \\
  &= \int \N(\ve{z}_t ; \alpha_{t|s}\ve{z}_s, \sigma^2_{t|s} I) \N(\ve{z}_s ; \alpha_s \ve{x}, \sigma_s^2 I) \, \dee \ve{z}_s\\
  &= \int \N(\alpha_{t|s}\ve{z}_s ; \ve{z}_t, \sigma^2_{t|s} I) \N(\ve{z}_s ; \alpha_s \ve{x}, \sigma_s^2 I) \, \dee \ve{z}_s\\
  &= \int \frac{1}{\alpha_{t|s}^d} \N\bigg(\ve{z}_s ; \frac{\ve{z}_t}{\alpha_{t|s}}, \frac{\sigma^2_{t|s}}{\alpha_{t|s}^2} I\bigg) \N(\ve{z}_s ; \alpha_s \ve{x}, \sigma_s^2 I) \, \dee \ve{z}_s& (\mbox{Corollary }\ref{thm:gaussian-linear-transformation-scalar-ver}) \\
  &= \int \frac{1}{\alpha_{t|s}^d} \N\bigg( \frac{\ve{z}_t}{\alpha_{t|s}} ; \alpha_s \ve{x}, \Big( \frac{\sigma_{t|s}^2}{\alpha_{t|s}^2} + \sigma_s^2 \Big) I \bigg) \N(\ve{z}_s ; \ves{\mu}_3, \Sigma_3) \, \dee \ve{z}_s& (\mbox{Proposition }\ref{thm:gaussian-product}) \\
  &= \frac{1}{\alpha_{t|s}^d} \N\bigg( \frac{\ve{z}_t}{\alpha_{t|s}} ; \alpha_s \ve{x}, \Big( \frac{\sigma_{t|s}^2}{\alpha_{t|s}^2} + \sigma_s^2 \Big) I \bigg) \int \N(\ve{z}_s ; \ves{\mu}_3, \Sigma_3) \, \dee \ve{z}_s\\
  &= \frac{1}{\alpha_{t|s}^d} \N\bigg( \frac{\ve{z}_t}{\alpha_{t|s}} ; \alpha_s \ve{x}, \Big( \frac{\sigma_{t|s}^2}{\alpha_{t|s}^2} + \sigma_s^2 \Big) I \bigg) \\
  &= \N\bigg( \ve{z}_t ; \alpha_{t|s} \alpha_s \ve{x}, ( \sigma_{t|s}^2 + \alpha_{t|s}^2 \sigma_s^2 ) I \bigg) & (\mbox{Corollary }\ref{thm:gaussian-linear-transformation-scalar-ver}) \\
  &= \N\bigg( \ve{z}_t ; \frac{\alpha_t}{\alpha_s} \alpha_s \ve{x}, ( \sigma_t^2 - \alpha_{t|s}^2 \sigma_s^2 + \alpha_{t|s}^2 \sigma_s^2 ) I \bigg) \\
  &= \N\bigg( \ve{z}_t ; \alpha_t \ve{x}, \sigma_t^2  I \bigg) \\
  &= q(\ve{z}_t | \ve{x})
\end{align*}
as required.
\end{proof}

\pagebreak 
\item \begin{proposition}
  The property in Equation~\ref{eq:zt-given-zs} is consistent with itself. In other words, for any $0 \leq s < t < u \leq 1$, it holds that
  \begin{align*}
    q(\ve{z}_u|\ve{z}_s) = \int q(\ve{z}_u|\ve{z}_t) q(\ve{z}_t|\ve{z}_s)\, \dee \ve{z}_t.
  \end{align*}
\end{proposition}
\begin{proof}
\begin{align*}
  &\int q(\ve{z}_u|\ve{z}_t) q(\ve{z}_t|\ve{z}_s)\, \dee \ve{z}_t \\
  &= \int q(\ve{z}_u; \alpha_{u|t}\ve{z}_t, \sigma_{u|t}^2 I ) q(\ve{z}_t; \alpha_{t|s}\ve{z}_s, \sigma_{t|s}^2 I )\, \dee\ve{z}_t \\
  &= \int q(\alpha_{u|t}\ve{z}_t; \ve{z}_u, \sigma_{u|t}^2 I ) q(\ve{z}_t; \alpha_{t|s}\ve{z}_s, \sigma_{t|s}^2 I )\, \dee\ve{z}_t \\
  &= \int \frac{1}{\alpha_{u|t}^2} q\bigg(\ve{z}_t; \frac{\ve{z}_u}{\alpha_{u|t}}, \frac{\sigma_{u|t}^2}{\alpha_{u|t}^2} I \bigg) q(\ve{z}_t; \alpha_{t|s}\ve{z}_s, \sigma_{t|s}^2 I )\, \dee\ve{z}_t & (\mbox{Corollary~\ref{thm:gaussian-linear-transformation-scalar-ver}})\\
  &= \frac{1}{\alpha_{u|t}^2} q\bigg(\frac{\ve{z}_u}{\alpha_{u|t}}; \alpha_{t|s}\ve{z}_s, \Big( \frac{\sigma_{u|t}^2}{\alpha_{u|t}^2} + \sigma_{t|s}^2 \Big) I \bigg) & (\mbox{same reasoning as Proposition~\ref{thm:zt-x-consistency}})\\
  &= q\big(\ve{z}_u; \alpha_{u|t}\alpha_{t|s}\ve{z}_s, ( \sigma_{u|t}^2 + \alpha_{u|t}^2 \sigma_{t|s}^2 ) I \big)\\
  &= q\big(\ve{z}_u; \alpha_{u|s}\ve{z}_s, \big( (\sigma_u^2 - \alpha_{u|t}^2 \sigma_t^2) + \alpha_{u|t}^2 (\sigma_t^2 - \alpha_{t|s}^2 \sigma_s^2) \big) I \big) \\
  &= q\big(\ve{z}_u; \alpha_{u|s}\ve{z}_s, ( \sigma_u^2 - \alpha_{u|t}^2 \sigma_t^2 + \alpha_{u|t}^2 \sigma_t^2 - \alpha_{u|t}^2 \alpha_{t|s}^2 \sigma_s^2) I \big) \\
  &= q\big(\ve{z}_u; \alpha_{u|s}\ve{z}_s, ( \sigma_u^2 - \alpha_{u|s}^2 \sigma_s^2) I \big) \\
  &= q\big(\ve{z}_u; \alpha_{u|s}\ve{z}_s, \sigma_{u|s}^2 I \big) \\
  &= q(\ve{z}_u|\ve{z}_s)
\end{align*}
as required.
\end{proof}

\item \begin{proposition}
  For any $0 \leq s < t \leq 1$, we have that
  \begin{align*}
    q(\ve{z}_s|\ve{z}_t, \ve{x}) = \N(\ve{z}_s; \ves{\mu}_Q(\ve{z}_t, \ve{x}; s, t), \sigma^2_Q(s,t) I )
  \end{align*}
  where
  \begin{align*}
    \sigma_Q^{2}(s,t) &= \sigma_{t|s}^2 \sigma_s^2 / \sigma_t^2, \\
    \ves{\mu}_Q(\ve{z}_t, \ve{x}; s, t) &= \frac{\alpha_{t|s}\sigma_s^2}{\sigma_t^2} \ve{z}_t + \frac{\alpha_s \sigma_{t|s}^2}{\sigma_t^2} \ve{x}.
  \end{align*}
\end{proposition}

\begin{proof}
  By Baye's rule,
  \begin{align*}
    q(\ve{z}_s|\ve{z}_t, \ve{x}) 
    &= \frac{q(\ve{z}_t | \ve{z}_s, \ve{x}) q(\ve{z}_s|\ve{x})}{q(\ve{z}_t|\ve{x})}
    = \frac{q(\ve{z}_t | \ve{z}_s) q(\ve{z}_s|\ve{x})}{q(\ve{z}_t|\ve{x})}.
  \end{align*}
  The last equality follows from the fact that we require $q$ to be Markov: $q(\ve{z}_t|\ve{z}_s,\ve{x}) = q(\ve{z}_t|\ve{z}_s)$. Now, we apply Proposition~\ref{thm:gaussian-product} to get
  \begin{align*}
    q(\ve{z}_t | \ve{z}_s) q(\ve{z}_s|\ve{x})
    &= \N(\ve{z}_t; \alpha_{t|s} \ve{z}_s, \sigma^2_{t|s} I) \N(\ve{z}_s; \alpha_s \ve{x},  \sigma_s^2 I) \\
    &= \N(\alpha_{t|s} \ve{z}_s; \ve{z}_t, \sigma^2_{t|s} I) \N(\ve{z}_s; \alpha_s \ve{x},  \sigma_s^2 I) \\
    &= \frac{1}{\alpha_{t|s}^d} \N\bigg(\ve{z}_s ; \frac{\ve{z}_t}{\alpha_{t|s}}, \frac{\sigma^2_{t|s}}{\alpha_{t|s}^2} I\bigg) \N(\ve{z}_s ; \alpha_s \ve{x}, \sigma_s^2 I) \\
    &= q(\ve{z}_t|\ve{x}) \N(\ve{z}_s; \ves{\mu}_3, \Sigma_3).
  \end{align*}
  where $\ves{\mu}_3$ and $\Sigma_3$ are as described in the statement of Proposition~\ref{thm:gaussian-product}. The $q(\ve{z}_t|\ve{x})$ comes from the reasoning we used in the proof of Proposition~\ref{thm:zt-x-consistency}. So, it turns out that
  \begin{align*}
    q(\ve{z}_s|\ve{z}_t, \ve{x}) &= \frac{q(\ve{z}_t|\ve{x}) \N(\ve{x}_s; \ves{\mu}_3, \Sigma_3)}{q(\ve{z}_t|\ve{x})} = \N(\ve{z}_s; \ves{\mu}_3, \Sigma_3).
  \end{align*}
  So, what is left for us to do is to compute $\ves{\mu}_3$ and $\Sigma_3$ and see if the results agree with $\ves{\mu}_Q(\ve{z}_t,\ve{x}; s, t)$ and $\sigma_Q^2(s,t) I$. 
  
  We have that $\ves{\mu}_1 = \ve{z}_t / \alpha_{t|s}$, $\Sigma_1 = (\sigma_{t|s}^2 / \alpha_{t|s}^2) I$, $\ves{\mu}_2 = \alpha_s \ve{x}$, and $\Sigma_2 = \sigma_s^2 I$, so
  \begin{align*}
    \ves{\mu_3} 
    &= \Sigma_2(\Sigma_1 + \Sigma_2)^{-1} \ves{\mu}_1 + \Sigma_1 (\Sigma_1 + \Sigma_2)^{-1} \ves{\mu}_2 \\ 
    &= \frac{\sigma_s^2}{\sigma_{t|s}^2/\alpha_{t|s}^2 + \sigma_s^2} \frac{\ve{z}_t}{\alpha_{t|s}} + \frac{\sigma_{t|s}^2 / \alpha^2_{t|s}}{\sigma_{t|s}^2 / \alpha^2_{t|s} + \sigma_s^2} \alpha_s \ve{x} \\
    &= \frac{\alpha_{t|s} \sigma_s^2}{\sigma_{t|s}^2/\alpha_{t|s} + \alpha_{t|s}\sigma_s^2} \frac{\ve{z}_t}{\alpha_{t|s}} + \frac{\sigma_{t|s}^2 }{\sigma_{t|s}^2  + \alpha_{t|s}^2 \sigma_s^2} \alpha_s \ve{x} \\
    &= \frac{\alpha_{t|s} \sigma_s^2}{\sigma_{t|s}^2 + \alpha_{t|s}^2 \sigma_s^2} \ve{z}_t + \frac{\sigma_{t|s}^2 }{\alpha_s \sigma_{t|s}^2  + \alpha_{t|s}^2 \sigma_s^2} \ve{x} \\
    &= \frac{\alpha_{t|s} \sigma_s^2}{\sigma_t^2 - \alpha_{t|s}^2 \sigma_s^2 + \alpha_{t|s}^2 \sigma_s^2} \ve{z}_t + \frac{\alpha_s \sigma_{t|s}^2 }{ \sigma_t^2 - \alpha_{t|s}^2 \sigma_s^2 + \alpha_{t|s}^2 \sigma_s^2} \ve{x} \\
    &= \frac{\alpha_{t|s} \sigma_s^2}{\sigma_t^2} \ve{z}_t + \frac{\alpha_s \sigma_{t|s}^2 }{ \sigma_t^2 } \ve{x} \\
    &= \ves{\mu}_Q(\ve{z}_t, \ve{x}; s, t).
  \end{align*}
  Moreover,
  \begin{align*}
    \Sigma_3 
    &= \Sigma_1 (\Sigma_1 + \Sigma_2)^{-1} \Sigma_2 
    = \frac{\sigma_{t|s}^2}{\alpha_{t|s}^2} \bigg( \frac{\sigma_{t|s}^2}{\alpha_{t|s}^2} + \sigma_s^2 \bigg)^{-1} \sigma_s^2 I 
    = \frac{\sigma_{t|s}^2 \sigma_s^2}{\alpha^2_{t|s}(\sigma_{t|s}^2 / \alpha_{t|s}^2 + \sigma_s^2)} I 
    = \frac{\sigma_{t|s}^2 \sigma_s^2}{\sigma_{t|s}^2  + \alpha_{t|s}^2 \sigma_s^2} I 
    = \frac{\sigma_{t|s}^2 \sigma_s^2}{\sigma_t^2} I \\
    &= \sigma^2_Q(s,t) I
  \end{align*}
  as required.
\end{proof}

\pagebreak 

\item \begin{proposition}
\begin{align*}
  \ves{\mu}_{\ves{\theta}}(\ve{z}_t; s, t)
  &= \frac{1}{\alpha_{t|s}} \ve{z}_t - \frac{\sigma_{t|s}^2}{\alpha_{t|s}\sigma_t} \hat{\ves{\xi}}_{\ves{\theta}}(\ve{z}_t; t)
\end{align*}
\end{proposition}

\begin{proof}
We have that
\begin{align*}
  \ves{\mu}_{\ves{\theta}}(\ve{z}_t; s, t)
  &= \frac{\alpha_{t|s} \sigma_s^2}{\sigma_t^2} \ve{z}_t + \frac{\alpha_s \sigma_{t|s}^2}{\sigma_t^2} \hat{\ve{x}}(\ve{z}_t; t) \\
  &= \frac{\alpha_{t|s} \sigma_s^2}{\sigma_t^2} \ve{z}_t + \frac{\alpha_s \sigma_{t|s}^2}{\sigma_t^2} \bigg( \frac{\ve{z}_t - \sigma_t \hat{\ves{\xi}}_{\ves{\theta}}(\ve{z}_t; t)}{\alpha_t} \bigg) \\
  &= \bigg( \frac{\alpha_{t|s} \sigma_s^2}{\sigma_t^2} + \frac{\alpha_s \sigma_{t|s}^2}{\alpha_t \sigma_t^2} \bigg) \ve{z}_t - \frac{\alpha_s \sigma_t \sigma_{t|s}^2}{\alpha_t \sigma_t^2} \hat{\ves{\xi}}_{\ves{\theta}}(\ve{z}_t; t) \\
  &= \frac{1}{\sigma_t^2} \bigg( \frac{\alpha_t \sigma_s^2}{\alpha_s} + \frac{\alpha_s \sigma_{t|s}^2}{\alpha_t} \bigg) \ve{z}_t - \frac{\alpha_s \sigma_{t|s}^2}{\alpha_t \sigma_t} \hat{\ves{\xi}}_{\ves{\theta}}(\ve{z}_t; t) \\
  &= \frac{1}{\sigma_t^2} \bigg( \frac{\alpha_t^2 \sigma_s^2 + \alpha_s^2 \sigma_{t|s}^2}{\alpha_s \alpha_t} \bigg) \ve{z}_t - \frac{\sigma_{t|s}^2}{\alpha_{t|s} \sigma_t} \hat{\ves{\xi}}_{\ves{\theta}}(\ve{z}_t; t) \\
  &= \frac{1}{\sigma_t^2} \bigg( \frac{\alpha_t^2 \sigma_s^2 + \alpha_s^2 (\sigma_t^2 - \alpha_{t|s}^2 \sigma_s^2)}{\alpha_s \alpha_t} \bigg) \ve{z}_t - \frac{\sigma_{t|s}^2}{\alpha_{t|s} \sigma_t} \hat{\ves{\xi}}_{\ves{\theta}}(\ve{z}_t; t) \\
  &= \frac{1}{\sigma_t^2} \bigg( \frac{\alpha_t^2 \sigma_s^2 + \alpha_s^2 \sigma_t^2 - \alpha_{t}^2 \sigma_s^2}{\alpha_s \alpha_t} \bigg) \ve{z}_t - \frac{\sigma_{t|s}^2}{\alpha_{t|s} \sigma_t} \hat{\ves{\xi}}_{\ves{\theta}}(\ve{z}_t; t) \\
  &= \frac{1}{\sigma_t^2} \frac{\alpha_s^2 \sigma_t^2}{\alpha_s \alpha_t} \ve{z}_t - \frac{\sigma_{t|s}^2}{\alpha_{t|s} \sigma_t} \hat{\ves{\xi}}_{\ves{\theta}}(\ve{z}_t; t) \\
  &= \frac{\alpha_s }{\alpha_t} \ve{z}_t - \frac{\sigma_{t|s}^2}{\alpha_{t|s} \sigma_t} \hat{\ves{\xi}}_{\ves{\theta}}(\ve{z}_t; t) \\
  &= \frac{1}{\alpha_{t|s}} \ve{z}_t - \frac{\sigma_{t|s}^2}{\alpha_{t|s} \sigma_t} \hat{\ves{\xi}}_{\ves{\theta}}(\ve{z}_t; t)
\end{align*}
as required.
\end{proof}

\item \begin{proposition} \label{thm:backward-process-simplification}
\begin{align*}
  \ves{\mu}_{\ves{\theta}}(\ve{z}_t; s, t) 
  &= \frac{\ve{z}_t + \sigma_t \mrm{expm1}( \gamma_{\ves{\eta}}(s) - \gamma_{\ves{\eta}}(t)) \hat{\ves{\xi}}_{\ves{\theta}}(\ve{z}_t; t)}{\alpha_{t|s}} \\
  \sigma_Q^2(s,t) &= -\sigma_s^2 \mrm{expm1}(\gamma_{\ves{\eta}}(s) - \gamma_{\ves{\eta}}(t))
\end{align*}
where $\mrm{expm1}(u) = e^u - 1$.
\end{proposition}

\begin{proof}
First, we have that
\begin{align*}
  \ves{\mu}_{\ves{\theta}}(\ve{z}_t; s, t)
  &= \frac{1}{\alpha_{t|s}} \ve{z}_t - \frac{\sigma_{t|s}^2}{\alpha_{t|s}\sigma_t} \hat{\ves{\xi}}_{\ves{\theta}}(\ve{z}_t; t)
  = \frac{1}{\alpha_{t|s}} \bigg( \ve{z}_t - \frac{\sigma_{t|s}^2}{\sigma_t} \hat{\ves{\xi}}_{\ves{\theta}}(\ve{z}_t; t) \bigg).
\end{align*}
Now,
\begin{align*}
  \frac{\sigma_{t|s}^2}{\sigma_t} 
  &= \frac{\sigma_t^2 - \alpha^2_{t|s}\sigma^2_s}{\sigma_t}
  = \sigma_t - \frac{\alpha_t^2 \sigma_s^2}{\alpha_s^2 \sigma_t}
  = \sigma_t \bigg( 1 - \frac{\alpha_t^2 \sigma_s^2}{\alpha_s^2 \sigma_t^2 } \bigg)  
  = \sigma_t \bigg( 1 - \frac{(1 - \sigma_t^2) \sigma_s^2}{(1 - \sigma_s^2) \sigma_t^2 } \bigg)
  = \sigma_t \bigg( 1 - \frac{(1 - \sigma_t^2) \sigma_s^2}{(1 - \sigma_s^2) \sigma_t^2 } \bigg) \\
  &= \sigma_t \bigg( 1 - \frac{\sigma_t^{-2} - 1}{\sigma_s^{-2} -1} \bigg)
  = \sigma_t \bigg( 1 - \frac{1 + \exp(-\gamma_{\ves{\eta}}(t)) - 1}{1 + \exp(-\gamma_{\ves{\eta}}(s)) -1} \bigg)
  = \sigma_t \big( 1 - \exp(\gamma_{\ves{\eta}}(s) - \gamma_{\ves{\eta}}(t)) \big) \\
  &= -\sigma_t \mrm{expm1}(\gamma_{\ves{\eta}}(s) - \gamma_{\ves{\eta}}(t)).
\end{align*}
So,
\begin{align*}
  \ves{\mu}_{\ves{\theta}}(\ve{z}_t; s, t)
  &= \frac{\ve{z}_t + \sigma_t \mrm{expm1}( \gamma_{\ves{\eta}}(s) - \gamma_{\ves{\eta}}(t)) \hat{\ves{\xi}}_{\ves{\theta}}(\ve{z}_t; t)}{\alpha_{t|s}},
\end{align*}
and we are done with $\ves{\mu}_{\ves{\theta}}(\ve{z}_t; s, t)$.

For $\sigma_Q^2(s,t)$, we have that
\begin{align*}
  \sigma_Q^2(s,t) 
  &= \frac{\sigma_{t|s}^2\sigma_s^2}{\sigma_t^2}
  = \frac{\sigma_s^2}{\sigma_t} \frac{\sigma_{t|s}^2}{\sigma_t}
  = \frac{\sigma_s^2}{\sigma_t} \big( -\sigma_t \mrm{expm1}(\gamma_{\ves{\eta}}(s) - \gamma_{\ves{\eta}}(t)) \big) \\
  &= -\sigma_s^2 \mrm{expm1}(\gamma_{\ves{\eta}}(s) - \gamma_{\ves{\eta}}(t))
\end{align*}
as required.
\end{proof}

\item \begin{proposition} \label{thm:diffusion-loss-snr-finite}
  In the case of finite $T$, the diffusion loss $\mcal{L}_T(\ve{x})$ can be expressed as
  \begin{align*}
    \mcal{L}_T(\ve{x})
    &= \frac{T}{2} E_{\ves{\xi} \sim \mcal{N}(\ves{0},I), i \sim \mcal{U}\{1:T\}} \Big[ \big(\SNR(s(i)) - \SNR(t(i))\big) \| \ve{x} - \hat{\ve{x}}_{\ves{\theta}}(\ve{z}_{t(i)}; t(i)) \|^2 \Big].
  \end{align*}
\end{proposition}

\begin{proof}
  Let us use $s$ and $t$ as shorthands for $s(i)$ and $t(i)$. We have that
  \begin{align*}
    q(\ve{z}_s | \ve{z}_t, \ve{x})
    &= \mcal{N}(\ve{z}_s; \ves{\mu}_Q(\ve{z}_t, \ve{x}; s,t), \sigma^2_Q(s,t)),\\
    p(\ve{z}_s | \ve{z}_t)
    &= \mcal{N}(\ve{z}_s, \ves{\mu}_{\ves{\theta}}(\ve{z}_t; s, t), \sigma^2_Q(s,t)), \\
    \ves{\mu}_Q(\ve{z}_t, \ve{x}; s,t) 
    &= \frac{\alpha_{t|s}\sigma_s^2}{\sigma_t^2} \ve{z}_t + \frac{\alpha_s \sigma_{t|s}^2}{\sigma_t^2} \ve{x} \\
    \ves{\mu}_{\ves{\theta}}(\ve{z}_t; s, t) 
    &= \frac{\alpha_{t|s}\sigma_s^2}{\sigma_t^2} \ve{z}_t + \frac{\alpha_s \sigma_{t|s}^2}{\sigma_t^2} \hat{\ve{x}}_{\ves{\theta}}(\ve{z}_t; s, t), \\
    \sigma_Q^2 
    &= \sigma_{t|s}^2 \sigma_s^2 / \sigma_t^2.
  \end{align*}
  Applying Proposition~\ref{thm:gaussian-kl-spherical}, we have that
  \begin{align*}
    D_{KL}(
      q(\ve{z}_s | \ve{z}_t, \ve{x})
      \| p(\ve{z}_s | \ve{z}_t) )
    &= \frac{1}{2 \sigma^2_Q(s,t)} \| \ves{\mu}_Q - \ves{\mu}_{\ves{\theta}} \|^2 
    = \frac{\sigma_t^2}{2 \sigma_{t|s}^2 \sigma_s^2} \frac{\alpha_s^2 \sigma_{t|s}^4}{\sigma_t^4} 
    \| \ve{x} - \hat{\ve{x}}_{\ves{\theta}}(\ve{z}_t; t) \|^2.
  \end{align*}
  Now,
  \begin{align*}
    \frac{\sigma_t^2}{2 \sigma_{t|s}^2 \sigma_s^2} \frac{\alpha_s^2 \sigma_{t|s}^4}{\sigma_t^4}
    &= \frac{1}{2 \sigma_s^2} \frac{\alpha_s^2 \sigma_{t|s}^2}{\sigma_t^2}
    = \frac{1}{2 \sigma_s^2} \frac{\alpha_s^2 (\sigma_t^2 - \alpha_{t|s}^2 \sigma_s^2)}{\sigma_t^2}
    = \frac{1}{2} \frac{\alpha_s^2 \sigma_t^2 - \alpha_t^2 \sigma_s^2}{ \sigma_s^2 \sigma_t^2}
    = \frac{1}{2} \bigg( \frac{\alpha_s^2}{\sigma_s^2} - \frac{\alpha_t^2}{\sigma_t^2} \bigg) \\
    &= \frac{1}{2} \big( \SNR(s) - \SNR(t) \big).
  \end{align*}
  As a result,
  \begin{align*}
  D_{KL}(
      q(\ve{z}_s | \ve{z}_t, \ve{x})
      \| p(\ve{z}_s | \ve{z}_t) )
  = \frac{1}{2} \big( \SNR(s) - \SNR(t) \big) \| \ve{x} - \hat{\ve{x}}_{\ves{\theta}}(\ve{z}_t; t) \|^2,
  \end{align*}
  and
  \begin{align*}
    \mcal{L}_T(\ve{x})
    &= \sum_{i=1}^T E_{\ve{z}_t \sim q(\ve{z}_t|\ve{x})} [
      D_{KL}(
      q(\ve{z}_s | \ve{z}_t, \ve{x})
      \| p(\ve{z}_s | \ve{z}_t) )
    ] \\
    &= \frac{1}{2} \sum_{i=1}^T E_{\ve{z}_t \sim q(\ve{z}_t|\ve{x})} \Big[ \big( \SNR(s) - \SNR(t) \big) \| \ve{x} - \hat{\ve{x}}_{\ves{\theta}}(\ve{z}_t; t) \|^2 \Big] \\
    &= \frac{1}{2} \sum_{i=1}^T E_{\ves{\xi} \sim \mcal{N}(\ve{0}, I)} \Big[ \big( \SNR(s) - \SNR(t) \big) \| \ve{x} - \hat{\ve{x}}_{\ves{\theta}}(\ve{z}_t; t) \|^2 \Big] \\
    &= \frac{T}{2} \sum_{i=1}^T E_{\ves{\xi} \sim \mcal{N}(\ve{0}, I), i \sim \mcal{U}\{1:T\}} \Big[ \big( \SNR(s) - \SNR(t) \big) \| \ve{x} - \hat{\ve{x}}_{\ves{\theta}}(\ve{z}_t; t) \|^2 \Big]
  \end{align*}
  as required.
\end{proof}

\item \begin{proposition} \label{thm:diffusion-loss-expm1-finite}
  In the case of finite $T$, the diffusion loss $\mcal{L}_T(\ve{x})$ can be expressed as
  \begin{align*}
    \mcal{L}_T(\ve{x})
    &= \frac{T}{2} 
    E_{\ves{\xi} \sim \mcal{N}(\ves{0},I), i \sim \mcal{U}\{1:T\}} \Big[ 
      \mrm{expm1}\big(\gamma_{\ves{\eta}}(t(i)) - \gamma_{\ves{\eta}}(s(i)) \big) \| \ve{\xi} - \hat{\ve{\xi}}_{\ves{\theta}}(\ve{z}_{t(i)}; t(i)) \|^2 \Big].
  \end{align*}
\end{proposition}

\begin{proof}
  The reasoning of this proposition is similar to the last one. We start with
  \begin{align*}
    D_{KL}(
      q(\ve{z}_s | \ve{z}_t, \ve{x})
      \| p(\ve{z}_s | \ve{z}_t) )
    &= \frac{1}{2 \sigma^2_Q(s,t)} \| \ves{\mu}_Q - \ves{\mu}_{\ves{\theta}} \|^2 
    = \frac{\sigma_t^2}{2 \sigma_{t|s}^2 \sigma_s^2} \frac{\sigma_{t|s}^4}{\alpha^2_{t|s} \sigma_t^2} 
    \| \ves{\xi} - \hat{\ves{\xi}}_{\ves{\theta}}(\ve{z}_t; t) \|.
  \end{align*}
  We have that
  \begin{align*}
    \frac{\sigma_t^2}{2 \sigma_{t|s}^2 \sigma_s^2} 
    \frac{\sigma_{t|s}^4}{\alpha^2_{t|s} \sigma_t^2}
    &= \frac{1}{2 \sigma_s^2} 
    \frac{\sigma_{t|s}^2}{\alpha^2_{t|s}}
    = \frac{\alpha_s^2 (\sigma_t^2 - \alpha_{t|s}^2 \sigma_s^2)}{2\sigma_s^2 \alpha_t^2}
    = \frac{\alpha_s^2 \sigma_t^2 - \alpha_t^2 \sigma_s^2}{2\alpha_t^2 \sigma_s^2} 
    = \frac{1}{2} \bigg( \frac{\alpha_s^2 \sigma_t^2}{\alpha_t^2 \sigma_s^2} - 1 \bigg).
  \end{align*}
  In the proof of Proposition~\ref{thm:backward-process-simplification}, we showed that
  \begin{align*}
    \frac{\alpha_t^2 \sigma_s^2}{\alpha_s^2 \sigma_t^2}
    &= \exp(\gamma_{\ves{\eta}}(s) - \gamma_{\ves{\eta}}(t)).
  \end{align*}
  As a result,
  \begin{align*}
    \frac{\alpha_s^2 \sigma_t^2}{\alpha_t^2 \sigma_s^2}
    &= \exp(\gamma_{\ves{\eta}}(t) - \gamma_{\ves{\eta}}(s)).
  \end{align*}
  Hence,
  \begin{align*}
    \frac{\sigma_t^2}{2 \sigma_{t|s}^2 \sigma_s^2} 
    \frac{\sigma_{t|s}^4}{\alpha^2_{t|s} \sigma_t^2}
    &= \frac{1}{2} \big( \exp(\gamma_{\ves{\eta}}(t) - \gamma_{\ves{\eta}}(s)) - 1\big)
    = \mrm{expm1}(\gamma_{\ves{\eta}}(t) - \gamma_{\ves{\eta}}(s)).
  \end{align*}
  We are done.
\end{proof}

\item \begin{proposition}
\begin{align*}
  \mcal{L}_{\infty}(\ve{x}) 
  = \frac{1}{2} E_{\xi \sim \N(\ve{0},I), t \sim \mcal{U}(0,1)} \big[ \gamma_{\ve{\eta}}'(t) \| \ves{\xi} - \hat{\ves{\xi}}(\ve{z}_t; t) \|^2 \big]
\end{align*}
\end{proposition}

\begin{proof}
  First, we have that
  \begin{align*}
    \mcal{L}_{\infty}(\ve{x}) 
    = -\frac{1}{2} E_{\xi \sim \N(\ve{0},I), t \sim \mcal{U}(0,1)} \big[ \SNR'(t) \| \ve{x} - \hat{\ve{x}}(\ve{z}_t; t) \|^2 \big].
  \end{align*}
  Because $\ve{z}_t = \alpha_t \ve{x} + \sigma_t \ves{\xi}$, we have that $\ve{x} = (\ve{z}_t - \sigma_t \ves{\xi}) / \alpha_t$. It follows that
  \begin{align*}
    \| \ve{x} - \hat{\ve{x}}_{\ves{\theta}}(\ve{z}_t ; t) \|^2
    &= \bigg\| \frac{\ve{z}_t - \sigma_t \ves{\xi}}{\alpha_t} - \frac{\ve{z}_t - \sigma_t \hat{\ves{\xi}}_{\ves{\theta}}(\ve{z}_t; t)}{\alpha_t} \bigg\|^2
    = \frac{\sigma_t^2}{\alpha_t^2} \|\ves{\xi} - \hat{\ves{\xi}}_{\ves{\theta}}(\ve{z}_t; t) \|^2
    = \frac{1}{\SNR(t)} \|\ves{\xi} - \hat{\ves{\xi}}_{\ves{\theta}}(\ve{z}_t; t) \|^2.
  \end{align*}
  Now,
  \begin{align*}
    \frac{\SNR'(t)}{\SNR(t)}
    &= \frac{\{ \exp(-\gamma_{\ves{\eta}}(t)) \}'}{\exp(-\gamma_{\ves{\eta}}(t))}
    = - \frac{ \exp(-\gamma_{\ves{\eta}}(t)) }{\exp(-\gamma_{\ves{\eta}}(t))} \gamma_{\ves{\eta}}'(t)
    = - \gamma_{\ves{\eta}}'(t).
  \end{align*}
  As a result,
  \begin{align*}
    \mcal{L}_{\infty}(\ve{x}) 
    &= -\frac{1}{2} E_{\xi \sim \N(\ve{0},I), t \sim \mcal{U}(0,1)} \big[ \SNR'(t) \| \ve{x} - \hat{\ve{x}}(\ve{z}_t; t) \|^2 \big] \\
    &= -\frac{1}{2} E_{\xi \sim \N(\ve{0},I), t \sim \mcal{U}(0,1)} \bigg[ \frac{\SNR'(t)}{\SNR(t)} \| \ves{\xi} - \hat{\ves{\xi}}(\ve{z}_t; t) \|^2 \bigg] \\
    &= \frac{1}{2} E_{\xi \sim \N(\ve{0},I), t \sim \mcal{U}(0,1)} \big[ \gamma'_{\ves{\eta}}(t) \| \ves{\xi} - \hat{\ves{\xi}}(\ve{z}_t; t) \|^2 \big]
  \end{align*}
  as required.
\end{proof}

\end{itemize}


\bibliographystyle{alpha}
\bibliography{variational-diffusion-models}  
\end{document}