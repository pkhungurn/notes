\documentclass[10pt]{article}
\usepackage{fullpage}
\usepackage{amsmath}
\usepackage[amsthm, thmmarks]{ntheorem}
\usepackage{amssymb}
\usepackage{graphicx}
\usepackage{enumerate}
\usepackage{verse}
\usepackage{tikz}
\usepackage{verbatim}
\usepackage{hyperref}

\newtheorem{lemma}{Lemma}
\newtheorem{theorem}[lemma]{Theorem}
\newtheorem{definition}[lemma]{Definition}
\newtheorem{proposition}[lemma]{Proposition}
\newtheorem{corollary}[lemma]{Corollary}
\newtheorem{claim}[lemma]{Claim}
\newtheorem{example}[lemma]{Example}

\newcommand{\dee}{\mathrm{d}}
\newcommand{\Dee}{\mathrm{D}}
\newcommand{\In}{\mathrm{in}}
\newcommand{\Out}{\mathrm{out}}
\newcommand{\pdf}{\mathrm{pdf}}
\newcommand{\Cov}{\mathrm{Cov}}
\newcommand{\Var}{\mathrm{Var}}

\newcommand{\ve}[1]{\mathbf{#1}}
\newcommand{\mrm}[1]{\mathrm{#1}}
\newcommand{\mtt}[1]{\mathtt{#1}}
\newcommand{\ves}[1]{\boldsymbol{#1}}
\newcommand{\etal}{{et~al.}}
\newcommand{\sphere}{\mathbb{S}^2}
\newcommand{\modeint}{\mathcal{M}}
\newcommand{\azimint}{\mathcal{N}}
\newcommand{\ra}{\rightarrow}
\newcommand{\mcal}[1]{\mathcal{#1}}
\newcommand{\X}{\mathcal{X}}
\newcommand{\Y}{\mathcal{Y}}
\newcommand{\Z}{\mathcal{Z}}
\newcommand{\x}{\mathbf{x}}
\newcommand{\y}{\mathbf{y}}
\newcommand{\z}{\mathbf{z}}
\newcommand{\tr}{\mathrm{tr}}
\newcommand{\sgn}{\mathrm{sgn}}
\newcommand{\diag}{\mathrm{diag}}
\newcommand{\Real}{\mathbb{R}}
\newcommand{\sseq}{\subseteq}
\newcommand{\ov}[1]{\overline{#1}}
\DeclareMathOperator*{\argmax}{arg\,max}
\DeclareMathOperator*{\argmin}{arg\,min}

\title{Neural Ordinary Differential Equations}
\author{Pramook Khungurn}

\begin{document}
\maketitle

This is a note on the paper ``Neural Ordinary Differential Equations'' by Chen \etal \cite{Chen:2018}.

\section{Introduction}

\begin{itemize}
  \item Many existing neural networks models creates a sequence of hidden states $\ve{h}_0$, $\ve{h}_1$, $\ve{h}_2$, $\dotsc$ $\ve{h}_T$ by adding something to the previous state:
  \begin{align*}
    \ve{h}_{t+1} = \ve{h}_t + \ve{f}(\ve{h}_t, t, \ves{\theta})
  \end{align*}
  Such models include such as residual networks \cite{He:2015}, recurrent neural networks, and normalizing flows \cite{Rezende:2015,Dinh:2014}.

  \item What if we take the limit as the number of time step goes to infinity? We will have a differential equation:
  \begin{align*}
    \frac{\dee\ve{h}(t)}{\dee t} = \ve{f}(\ve{h}(t), t, \ves{\theta}).
  \end{align*}

  \item To use the network, we simply say that $\ve{h}(0)$ is the input layer, and the output is $\ve{h}(T)$ at some time $T$. The output can be found by solving the initial value problem, and this can be done by any black-box differential equation solver. 
\end{itemize}

\section{How to train a neural ODE model}

\begin{itemize}
  \item The problem with the above approach is that it is unclear how to train such a neural ODE model.
  \begin{itemize}
    \item The computation of the solution can require a lot of time steps. Differentiating through these time steps to compute the gradient would requires saving a lot of information in memory.
  \end{itemize}

  \item The good news is that there is a method to compute the gradient using constant memory (i.e., does not depend on the number of time steps). This is called the {\bf adjoint sensitivity method}. It requires, however, an ODE solve, which can be done, again, by any ODE solver.
\end{itemize}  
  
\subsection{Problem Setup}

\begin{itemize}
  \item Let the hidden state be a vector in $\Real^n$. We typically denote it by $\ve{z}$.
  
  \item Let the neural network's parameters be a vector in $\Real^m$, and we typically denote it by $\ves{\theta}$.
  
  \item We will work on a state space vector $\ve{r} = (\ve{z}, t, \ves{\theta}) \in \Real^{n+1+m}$.
  
  \item We will want to see how $\ve{r}$ evolves through time. We denote the $\ve{r}$ at time $t$ with $\ve{r}_t = (\ve{z}_t, t, \ves{\theta})$. Note that $\ves{\theta}$ does not vary with $t$.
  
  \item It also makes sense to talk about the function that sends $t$ to $\ve{r}_t$. We denote this by $\ve{R}: \Real \rightarrow \Real^{n+1+m}$, and we can write
  \begin{align*}
    \ve{r}_t = \ve{R}(t) = (\ve{Z}(t), T(t), \ves{\Theta}(t)) = (\ve{z}_t, t, \ves{\theta}).
  \end{align*}
  Note that $T$ is the identity function, and $\ves{\Theta}$ is a constant function.

  \item The act of solving the neural ODE is a function that maps $\ve{r}_{t}$ to some $\ve{r}_{t + \Delta t}$ for some $\Delta t \geq 0$. Let us denote this function by $\ve{s}_{\Delta t}^+: \Real^{n+1+m} \rightarrow \Real^{n+1+m}.$ (The letter $\ve{s}$ stands for ``solve.'') We have that
  \begin{align*}
    \ve{s}^+_{\Delta t}(\ve{z}_t, t, \ves{\theta})     
    = (\ve{z}_{t+\Delta}, t, \ves{\theta})
    = \begin{bmatrix}
      \ve{z}_{t + \Delta t} \\
      t + \Delta t \\
      \ves{\theta}
    \end{bmatrix}
    = \begin{bmatrix}
      \ve{z}_t + \int_{t}^{t+\Delta t} \ve{f}(\ve{z}_u, u, \ves{\theta})\, \dee u \\
      t + \Delta t \\
      \ves{\theta}
    \end{bmatrix}.
  \end{align*}

  \item The above function runs the ODE for a fixed time internal $\Delta t$. However, we can also talk about running the ODE until a fixed time $t_1$. We denote this by
  \begin{align*}
    \ve{s}^+_{\rightarrow t_1}(\ve{z}_t, t, \ves{\theta}) 
    = \ve{s}^+_{t_1 - t}(\ve{z}_t, t, \ves{\theta})
    = \begin{bmatrix}
      \ve{z}_t + \int_{t}^{t_1} \ve{f}(\ve{z}_u, u, \ves{\theta})\, \dee u \\
      t + \Delta t \\
      \ves{\theta}
    \end{bmatrix}.
  \end{align*}

  \item When optimizing a neural network, we need a loss function. In our case, the loss function is given by $L: \Real^{n+1+m} \rightarrow \Real$ that maps a state vector to a real number. When we write $L(\ve{r}) = L(\ve{z}, t, \ves{\theta})$, it is typical to say that the function only depends on $\ve{z}$, the produced hidden state. So, $$L(\ve{r}) = L(\ve{z},t,\ves{\theta}) = L(\ve{z}).$$ 
  
  \item When training a neural ODE, we start with the input state vector $\ve{r}_{t}$. We then solve the ODE to get the state $\ve{r}_{t_1}$. We then evaluate $L(\ve{r}_{t_1})$ to compute the loss. Let $\mcal{L}: \Real^{n+1+m} \rightarrow \Real$ be the function that maps the input state to the final loss. This function is thus given by
  \begin{align*}
    \mcal{L}(\ve{z}_t, t, \ves{\theta}) = L(\ve{s}^+_{\rightarrow t_1}(\ve{z}_t, t, \ves{\theta})).
  \end{align*}

  \item To train the neural network, we need the gradient
  \begin{align*}
    \nabla_{\S3} \mcal{L}(\ve{z}_{t_0}, t_0, \ves{\theta})
  \end{align*}
  where $t_0$ is the time we designate for the input, typically $0$. Here, we use the notations for multivariable derivatives from \cite{KhungurnDeriv} to avoid confusion. $\nabla_{\S3}\mcal{L}$ denotes the gradient with respect to the third block of arguments of $\mcal{L}$, which is the network parameters $\ves{\theta}$.  
\end{itemize}

\subsection{Adjoint Sensitivity Method}

\begin{itemize}
  \item Define the {\bf adjoint} to be the function $\ve{a}: \Real \rightarrow \Real^{1 \times (n+1+m)}$ such that
  \begin{align*}
    \ve{a}: t \mapsto \nabla \mcal{L}(\ve{z}_t, t, \ves{\theta}).
  \end{align*}
  In other words,
  \begin{align*}
    \ve{a}(t) = \mcal{L}(\ve{R}(t)) = L(\ve{s}^+_{\rightarrow t_1}(\ve{R}(t)))
  \end{align*}
  or $\ve{a} = \mcal{L} \circ \ve{R} = L \circ s_{\rightarrow t_1}^+ \circ \ve{R}$.

  \item With the adjoint function, our end goal is to evaluate $$\ve{a}_{\S3}(t_0) 
  = \ve{a}(t_0)[:,\S3] 
  = \nabla \mcal{L}(\ve{z}_{t_0}, t_0, \ves{\theta})[:, \S3] 
  = \nabla_{\S3}\mcal{L}(\ve{z}_{t_0}, t_0, \ves{\theta}).$$

  \item The adjoint sensivity method relies on the fact that we can express $\dee\ve{a} / \dee t$ in terms for $\ve{a}$ and $\ve{f}$.
  \begin{theorem} \label{thm:adjoint-deriv}
  We have that
  \begin{align*}
    \frac{\dee \ve{a}(t)}{\dee t}
    = -\ve{a}(t)
    \begin{bmatrix}
      \nabla_{\S1}\ve{f}(\ve{z}_t, t, \ves{\theta})
      & \nabla_{\S2}\ve{f}(\ve{z}_t, t, \ves{\theta})
      & \nabla_{\S3}\ve{f}(\ve{z}_t, t, \ves{\theta}) \\
      \ve{0} & 0 & \ve{0} \\
      \ve{0} & \ve{0} & \ve{0}
    \end{bmatrix}
  \end{align*}
  In particular,
  \begin{align*}
    \frac{ \dee \ve{a}_{\S 1}(t)}{\dee t} &= -\ve{a}_{\S1}(t) \nabla_{\S1}\ve{f}(\ve{z}_t, t, \ves{\theta}), \\
    \frac{ \dee \ve{a}_{\S 3}(t)}{\dee t} &= -\ve{a}_{\S1}(t) \nabla_{\S3}\ve{f}(\ve{z}_t, t, \ves{\theta}). \\
  \end{align*}
  \end{theorem}

  \begin{proof}
    We have that 
    \begin{align*}
      \frac{\dee \ve{a}(t)}{\dee t} = \lim_{\varepsilon \rightarrow 0} \frac{\ve{a}(t + \varepsilon) + \ve{a}(t)}{\varepsilon}.
    \end{align*}
    To prove the theorem, we shall write $\ve{a}(t)$ in terms of $\ve{a}(t+\varepsilon)$.

    Consider the function $\mcal{L}$. We have that, for any $\varepsilon > 0$ such that $t + \varepsilon < t_1$, 
    \begin{align*}
      \mcal{L}(\ve{z}_t, t, \ves{\theta}) = \mcal{L}(\ve{z}_{t+\varepsilon}, t+\varepsilon, \ves{\theta}).
    \end{align*}
    This is because both $(\ve{z}_t, t, \ves{\theta})$ and $(\ve{z}_{t+\varepsilon}, t+\varepsilon, \ves{\theta})$ are on the trajectory to the final state vector $(\ve{z}_{t_1}, t_1, \ves{\theta})$. So, starting running the ODE from either points would lead to the same result. As a result, we may say that
    $$\mcal{L} = \mcal{L} \circ \ve{s}^+_\varepsilon$$
    if $\varepsilon$ is small enough. Applying the chain rule, we have that
    \begin{align*}
      \nabla \mcal{L}(\ve{z}_t, t, \ves{\theta})
      &= \nabla \mcal{L}(\ve{s}^+_\varepsilon(\ve{z}_t, t, \ves{\theta})) \nabla \ve{s}_{\varepsilon}^+(\ve{z}_t, t, \ves{\theta}) \\
      \nabla \mcal{L}(\ve{z}_t, t, \ves{\theta})
      &= \nabla \mcal{L}(\ve{z}_{t+\varepsilon}, t+\varepsilon, \ves{\theta}) \nabla \ve{s}_{\varepsilon}^+(\ve{z}_t, t, \ves{\theta}) \\
      \ve{a}(t) &= \ve{a}(t+\varepsilon) \nabla \ve{s}_{\varepsilon}^+(\ve{z}_t, t, \ves{\theta}).
    \end{align*}
    Now,
    \begin{align*}
      \ve{s}^+_\varepsilon(\ve{z}_t, t, \ves{\theta})
      &= \begin{bmatrix}
        \ve{z}_t + \int_{t}^{t+\varepsilon} \ve{f}(\ve{z}_u, u, \ves{\theta}) \, \dee u \\
        t + \varepsilon \\
        \ves{\theta}
      \end{bmatrix} 
      = \begin{bmatrix}
        \ve{z}_t + \varepsilon \ve{f}(\ve{z}_t, t, \ves{\theta}) + O(\varepsilon^2) \\
        t + \varepsilon \\
        \ves{\theta}
      \end{bmatrix} \\
      &= \begin{bmatrix}
        \ve{z}_t \\ t \\ \ves{\theta}
      \end{bmatrix} 
      + \varepsilon
      \begin{bmatrix}
        \ve{f}(\ve{z}_t, t, \ves{\theta}) \\
        1 \\
        \ve{0}
      \end{bmatrix} 
      + O(\varepsilon^2).      
    \end{align*}
    So,
    \begin{align*}
      \nabla \ve{s}^+_\varepsilon(\ve{z}_t, t, \ves{\theta})
      &= I + \varepsilon \begin{bmatrix}
        \nabla_{\S1}\ve{f}(\ve{z}_t, t, \ves{\theta})
        & \nabla_{\S2}\ve{f}(\ve{z}_t, t, \ves{\theta})
        & \nabla_{\S3}\ve{f}(\ve{z}_t, t, \ves{\theta}) \\
        \ve{0} & 0 & \ve{0} \\
        \ve{0} & \ve{0} & \ve{0}
      \end{bmatrix} + O(\varepsilon^2).
    \end{align*}
    This gives
    \begin{align*}
      \ve{a}(t) 
      &= \ve{a}(t + \varepsilon) 
      + \varepsilon \ve{a}(t + \varepsilon) \begin{bmatrix}
        \nabla_{\S1}\ve{f}(\ve{z}_t, t, \ves{\theta})
        & \nabla_{\S2}\ve{f}(\ve{z}_t, t, \ves{\theta})
        & \nabla_{\S3}\ve{f}(\ve{z}_t, t, \ves{\theta}) \\
        \ve{0} & 0 & \ve{0} \\
        \ve{0} & \ve{0} & \ve{0}
      \end{bmatrix} + O(\varepsilon^2),
    \end{align*}
    and so
    \begin{align*}
      \frac{\ve{a}(t+\varepsilon) - \ve{a}(t)}{\varepsilon} 
      &= -\ve{a}(t + \varepsilon) \begin{bmatrix}
        \nabla_{\S1}\ve{f}(\ve{z}_t, t, \ves{\theta})
        & \nabla_{\S2}\ve{f}(\ve{z}_t, t, \ves{\theta})
        & \nabla_{\S3}\ve{f}(\ve{z}_t, t, \ves{\theta}) \\
        \ve{0} & 0 & \ve{0} \\
        \ve{0} & \ve{0} & \ve{0}
      \end{bmatrix} + O(\varepsilon).
    \end{align*}
    Taking the limit as $\varepsilon \rightarrow 0$, we have that
    \begin{align*}
      \frac{\dee \ve{a}(t)}{\dee t}
      &= -\ve{a}(t) \begin{bmatrix}
        \nabla_{\S1}\ve{f}(\ve{z}_t, t, \ves{\theta})
        & \nabla_{\S2}\ve{f}(\ve{z}_t, t, \ves{\theta})
        & \nabla_{\S3}\ve{f}(\ve{z}_t, t, \ves{\theta}) \\
        \ve{0} & 0 & \ve{0} \\
        \ve{0} & \ve{0} & \ve{0}
      \end{bmatrix}
    \end{align*}
    as required.
  \end{proof}
  
  \item In a typical traning process, we start from $\ve{r}_{t_0} = (\ve{z}_{t_0}, t_0, \ves{\theta})$, and we solve the neural SDE forward in time to obtain $\ve{r}_{t_1} = (\ve{z}_{t_1}, t_1, \ves{\theta})$. We assume that we do not save any intermediate information in the forward solving process. Now, we need to compute the gradient $\ve{a}_{\S3}(t_0) = \nabla_{\S3}\mcal{L}(\ve{z}_{t_0}, t_0, \ves{\theta}).$
    
  \item The idea is then to start at time $t_1$ and jointly solve the following differential equations backward in time to $t_0$:
  \begin{align*}
    \frac{\dee \ve{z}_t}{\dee t} &= \ve{f}(\ve{z}_t, t, \ves{\theta}), \\
    \frac{\dee \ve{a}_{\S1}(t)}{\dee t} &= -\ve{a}_{\S 1}(t) \nabla_{\S 1}\ve{f}(\ve{z}_t, t, \ves{\theta}), \\
    \frac{\dee \ve{a}_{\S3}(t)}{\dee t} &= -\ve{a}_{\S 1}(t) \nabla_{\S 3}\ve{f}(\ve{z}_t, t, \ves{\theta}).
  \end{align*}
  In other words, we would like to compute the following integrals:
  \begin{align*}
    \ve{z}_{t_0} &= \ve{z}_{t_1} + \int_{t_1}^{t_0} \ve{f}(\ve{z}_t, t, \ves{\theta})\, \dee t, \\
    \ve{a}_{\S1}(t_0) &= \ve{a}_{\S1}(t_1) - \int_{t_1}^{t_0} \ve{a}_{\S 1}(\ve{z}_t, t, \ves{\theta}) \nabla_{\S 1}\ve{f}(\ve{z}_t, t, \ves{\theta})\, \dee t, \\
    \ve{a}_{\S3}(t_0) &= \ve{a}_{\S3}(t_1) - \int_{t_1}^{t_0} \ve{a}_{\S 1}(\ve{z}_t, t, \ves{\theta}) \nabla_{\S 3}\ve{f}(\ve{z}_t, t, \ves{\theta})\, \dee t.
  \end{align*}
  The initial conditions include $\ve{z}_{t_1}$, which we just computed using the forward process. The other initial conditions are:
  \begin{align*}
    a_{\S1}(t_1) 
    &= \nabla_{\S1}\mcal{L}(\ve{z}_{t_1},t_1,\ves{\theta}) = \nabla_{\S1}L(\ve{z}_{t_1},t_1,\ves{\theta}) = \nabla L(\ve{z}_{t_1}), \\
    a_{\S3}(t_1) 
    &= \nabla_{\S3}\mcal{L}(\ve{z}_{t_1},t_1,\ves{\theta})
    = \nabla_{\S3}L(\ve{z}_{t_1},t_1,\ves{\theta}) 
    = \ve{0}.
  \end{align*} 
  The last line follows from the fact that we assumed that $L$ does not depend on $\ves{\theta}$. All of these values are easy to compute.

  \item To solve the ODEs, we can use any black-box ODE solver. The interface for such a solver requires us to provide (1) an initial state vector, and (2) a function that computes the time derivative of the state vector given the time and the state vector. 
  
  Here, our state vector would be $\ve{q}^{(t)} \in \Real^{n+n+m}$. It would be divided into three blocks $\ve{q}^{(t)} = (\ve{q}^{(t)}_{\S 1}, \ve{q}^{(t)}_{\S 2}, \ve{q}^{(t)}_{\S 3})$, and the blocks would correspond to $\ve{z}_t$, $\ve{a}_{\S 1}(t)^T$, and $\ve{a}_{\S 3}(t)^T$, respectively. The initial state vector would be
  \begin{align*}
    \ve{q}^{(t_1)} = \begin{bmatrix}
      \ve{z}_{t_1} \\
      \nabla \big( L(\ve{z}_{t_1}) \big)^T \\
      \ve{0}
    \end{bmatrix}.
  \end{align*}
  The derivative would be given by
  \begin{align*}
    \frac{\dee \ve{q}^{(t)}}{\dee t}
    &= \begin{bmatrix}
      \ve{f}(\ve{q}_{\S 1}^{(t)}, t, \ves{\theta}) \\
      -\big( \ve{q}^{(t)}_{\S 2}\big)^T \nabla_{\S 1}\ve{f}(\ve{q}_{\S 1}^{(t)}, t, \ves{\theta}) \\
      -\big( \ve{q}^{(t)}_{\S 2} \big)^T \nabla_{\S 3}\ve{f}(\ve{q}_{\S 1}^{(t)}, t, \ves{\theta}) 
    \end{bmatrix}.
  \end{align*}
  Note that both $\big( \ve{q}^{(t)}_{\S 2}\big)^T \nabla_{\S 1}\ve{f}(\ve{q}_{\S 1}^{(t)}, t, \ves{\theta})$ and $\big( \ve{q}^{(t)}_{\S 2} \big)^T \nabla_{\S 3}\ve{f}(\ve{q}_{\S 1}^{(t)}, t, \ves{\theta})$ are both vector-Jacobian products (i.e., they are directional derivatives). They can thus be evaluated efficiently using automatic differentiation at the cost proportational to the evaluation of $\ve{f}(\ve{q}_{\S 1}^{(t)}, t, \ves{\theta})$.

  \item All in all, the adjoint sensitivity method allows us to compute the gradient without backpropagating through the operations of the forward solver. If we use forward-mode automatic differentiation, then the required memory is proportional to the size of the intermediate tensor vectors. There's no dependence on the network's depth at all. Hence, neural ODE is a very memory efficient architecture.
\end{itemize}

\section{Continuous Normalizing Flows}

\subsection{Introduction to (Discrete) Normalizing Flows}

\begin{itemize}
  \item {\bf Normalizing flows} refer to a body of techniques for modeling probability distributions that work by transforming a simple probability distribution (such as an isotropic Gaussian) to a more complicated one by compositing multiple simple transformations \cite{Kobyzev:2021}.
  
  \item More concretely, we may start with $\ve{z}_0 \sim p(\ve{z}_0)$ where $p(\ve{z}_0)$ is simple. We can now make the probability distribution more complex by applying a bijective function $\ve{g}_1$ to get
  \begin{align*}
    \ve{z}_1 = \ve{g}_1(\ve{z}_0).
  \end{align*}
  We have that
  \begin{align*}
    p(\ve{z}_1) = p(\ve{z}_0) |\det \nabla \ve{g}_1(\ve{z}_0) |^{-1}
  \end{align*}
  or
  \begin{align*}
    \log p(\ve{z}_1) = \log p(\ve{z}_0) - \log | \det \nabla \ve{g}_1(\ve{z}_0) |.
  \end{align*}

  \item In most normalizing flow techniques, multiple transformations are used:
  \begin{align*}
    \ve{z}_k = (\ve{g}_k \circ \ve{g}_{k-1} \circ \dotsb \circ \ve{g}_2 \circ \ve{g}_1)(\ve{z}_0)
    = \ve{g}_k(\ve{g}_{k-1}(\dotsm\ve{g}_2(\ve{g}_1(\ve{z}_0)))),
  \end{align*}
  which implies
  \begin{align} \label{eqn:normalizing-flow-log-p}
    \log p(\ve{z}_k) = \log p(\ve{z_0}) - \sum_{j=1}^k |\det \nabla \ve{g}_j(\ve{z}_{j-1}) |.
  \end{align}  
  
  \item To use normalizing flows for generative modeling, we just approximate the data distribution $p_{\mrm{data}}(\cdot)$ with $p_k(\cdot)$.
  \begin{itemize}    
    \item Model parameters can be obtained by maximum likelihood estimation. In other words, given a collection of data points $\{ \ve{z}^{(1)}_k, \ve{z}^{(2)}_k, \dotsc, \ve{z}^{(N)}_k \}$, we maximize
    \begin{align*}
      \frac{1}{N}\sum_{i=1}^N \log p(\ve{z}_k^{(i)})
      &= \frac{1}{N} \log p(\ve{z_0}^{(i)}) - \frac{1}{N} \sum_{i=1}^N \sum_{j=1}^k |\det \nabla \ve{g}_j(\ve{z}^{(i)}_{j-1}) |
    \end{align*}
    Here, for each data point $\ve{z}_k^{(i)}$, the hidden states $\ve{z}_{k-1}^{(i)}$, $\ve{z}_{k-1}^{(i)}$, $\dotsc$, $\ve{z}_{0}^{(i)}$ can be obtained by applying the inverse functions $\ve{g}^{-1}_k$, $\ve{g}^{-1}_{k-1}$, $\dotsc$, $\ve{g}^{-1}_1$ in order.

    \item Once the parameters are estimated, we can compute the probability of data point $\ve{p}_k$ by first computing the hidden states $\ve{z}_{k-1}$, $\dotsc$, $\ve{z}_0$ by applying the inverse transformations and then applying \eqref{eqn:normalizing-flow-log-p}.
    
    \item Also, we can sample a data point by first sampling $\ve{z}_0 \sim p_0$, which should be simple. We then apply $\ve{g}_1$, $\ve{g}_2$, $\dotsc$, $\ve{g}_k$ in order to obtain $\ve{z}_k$, which would be distributed according to $p_k \approx p_{\mrm{data}}$.    
  \end{itemize}

  \item In order to make normalizing flows work efficiently, we require transformations $\ve{g}_i$'s that are (1) easy to invert and (2) have Jacobians whose determinants are easy to compute and find gradients of. The survey article \cite{Kobyzev:2021} catalogs such transformations.
\end{itemize}

\subsection{Continuous Normalizing Flows and Its Distribution}
\begin{itemize}
  \item Normalizing flows can be casted into the neural ODE framework if we require that all transformations have the same form
  \begin{align*}
    \ve{z}_{t+1} = \ve{g}_{t+1}(\ve{z}_t) = \ve{z}_t + \ve{f}(\ve{z}_t, t, \ves{\theta}).
  \end{align*}
  As usual, we take the limit as $t \gets \infty$ to obtain
  \begin{align*}
    \frac{\dee \ve{z}(t)}{\dee t} = \ve{f}(\ve{z}, t, \ves{\theta}),
  \end{align*}
  which gives us a continuous normalizing flow.

  \item To compute probability and to train our neural ODE model, we need an expression like \eqref{eqn:normalizing-flow-log-p}. This is given by the following theorem.
  
  \begin{theorem}[Instantataneous change of variables]
    Let $\ve{z}_t$ be a finite continuous random variable with probability $p(\ve{z}_t)$ dependent on time. Let $\dee \ve{z}_t / \dee t = \ve{f}(\ve{z}_t, t, \ves{\theta})$ be a differential equation governing the value of $\ve{z}_t$. Assuming that $\ve{f}$ is uninformly Lipschitz continuous in $\ve{z}$ and continuous in $t$. Then,
    \begin{align*}
      \frac{\dee \log p(\ve{z}_t)}{\dee t} = -\tr(\nabla_{\S1} \ve{f}(\ve{z}_t, t, \ves{\theta})).
    \end{align*}
  \end{theorem}

  \begin{proof}
    Because we assume that $\ve{f}$ is Lipschitz continuous in $\ve{z}_t$ and continuous in $t$, we have that every initial value problem has a unique solution by Picard's existence theorem. Because we assume that $\ve{z}_t$ is bounded, it implies that $\ve{f}$, $\ve{s}^+_\varepsilon$, and $\nabla_{\S1} \ve{s}^+_\varepsilon$ are all bounded.

    Suppose that $\epsilon$ is small enough that $\ve{s}_{\varepsilon}^+$ is bijective. (It is in the limit as $\varepsilon \rightarrow 0$.) We have that
    \begin{align*}
      \ve{z}_{t+\varepsilon}  = \ve{s}_{\varepsilon}^+(\ve{z}_t, t, \ves{\theta})[\S1].
    \end{align*}
    So,
    \begin{align*}
      \log p(\ve{z}_{t+\varepsilon})  = \log p(\ve{z}_t) - \log |\det \nabla_{\S1} \ve{s}_\varepsilon^+(\ve{z}_t, t, \ves{\theta})|.
    \end{align*}
    Hence,
    \begin{align*}
      \frac{\dee \log p(\ve{z})}{\dee t}
      &= \lim_{\varepsilon \rightarrow 0^+} \frac{\log p(\ve{z}_{t+\varepsilon}) - \log p(\ve{z}_t)}{\varepsilon} \\
      &= \lim_{\varepsilon \rightarrow 0^+} \frac{\log p(\ve{z}_t) - \log |\det \nabla_{\S1} \ve{s}_\varepsilon^+(\ve{z}_t, t, \ves{\theta})| - \log p(\ve{z}_t)}{\varepsilon} \\
      &= - \lim_{\varepsilon \rightarrow 0^+} \frac{\log |\det \nabla_{\S1} \ve{s}_\varepsilon^+(\ve{z}_t, t, \ves{\theta})|}{\varepsilon}.
    \end{align*}
    Applying L'Hospital's rule, we have
    \begin{align*}
      \frac{\dee \log p(\ve{z})}{\dee t}
      &= -\lim_{\varepsilon \rightarrow 0^+} \frac{ \frac{\partial}{\partial \varepsilon} \log |\det \nabla_{\S1} \ve{s}_\varepsilon^+(\ve{z}_t, t, \ves{\theta})|}{ \frac{\partial}{\partial \varepsilon} \varepsilon} \\
      &= -\lim_{\varepsilon \rightarrow 0^+} \frac{ \frac{\partial}{\partial \varepsilon} |\det \nabla_{\S1} \ve{s}_\varepsilon^+(\ve{z}_t, t, \ves{\theta})|}{ |\det \nabla_{\S1} \ve{s}_\varepsilon^+(\ve{z}_t, t, \ves{\theta})| }
    \end{align*}
    As $\varepsilon \rightarrow 0^+$, $\ve{s}_\varepsilon^+(\cdot)$ approachs the identity function. So, $\lim_{\varepsilon \rightarrow 0^+}|\det \nabla_{\S1} \ve{s}_\varepsilon^+(\ve{z}_t, t, \ves{\theta})| = 1 \neq 0.$ As result,
    \begin{align*}
      \frac{\dee \log p(\ve{z})}{\dee t} &= - \frac{ \lim_{\varepsilon \rightarrow 0^+} \frac{\partial}{\partial \varepsilon} |\det \nabla_{\S1} \ve{s}_\varepsilon^+(\ve{z}_t, t, \ves{\theta})|}{ \lim_{\varepsilon \rightarrow 0^+} |\det \nabla_{\S1} \ve{s}_\varepsilon^+(\ve{z}_t, t, \ves{\theta})| }
      = -\lim_{\varepsilon \rightarrow 0^+} \frac{\partial}{\partial \varepsilon} |\det \nabla_{\S1} \ve{s}_\varepsilon^+(\ve{z}_t, t, \ves{\theta})|.
    \end{align*}
  Again, we note that, as $\varepsilon \rightarrow 0^+$, $\ve{s}_\varepsilon^+(\cdot)$ approachs the identity function. As a result, it cannot change orientation of the local frame. So, $\det \nabla_{\S1} \ve{s}_\varepsilon^+(\ve{z}_t, t, \ves{\theta})$ must be positive. As a result, we can drop the absolute function and write
  \begin{align*}
    \frac{\dee \log p(\ve{z})}{\dee t} 
    &= -\lim_{\varepsilon \rightarrow 0^+} \frac{\partial}{\partial \varepsilon} \det \nabla_{\S1} \ve{s}_\varepsilon^+(\ve{z}_t, t, \ves{\theta}).
  \end{align*}
  Applying Jacobi's formula \cite{JacobisFormula}, we have that
  \begin{align*}
    \frac{\dee \log p(\ve{z})}{\dee t} 
    &= - \lim_{\varepsilon \rightarrow 0^+} 
    \tr \bigg( \mrm{adj}\big( \nabla_{\S1} \ve{s}_\varepsilon^+(\ve{z}_t, t, \ves{\theta}) \big) \frac{\partial \nabla_{\S1} \ve{s}_\varepsilon^+(\ve{z}_t, t, \ves{\theta}) }{\partial \varepsilon} \bigg) \\
    &= - \tr \Bigg( 
      \Big( \lim_{\varepsilon \rightarrow 0^+} \mrm{adj}\big( \nabla_{\S1} \ve{s}_\varepsilon^+(\ve{z}_t, t, \ves{\theta}) \big) \Big)
      \bigg( \lim_{\varepsilon \rightarrow 0^+} \frac{\partial \nabla_{\S1} \ve{s}_\varepsilon^+(\ve{z}_t, t, \ves{\theta}) }{\partial \varepsilon} \bigg)
    \Bigg)
  \end{align*}
  As $\varepsilon \rightarrow 0^+$, $\ve{s}_\varepsilon^+(\cdot)$ approachs the identity function, and so $\mrm{adj}\big( \nabla_{\S1} \ve{s}_\varepsilon^+(\ve{z}_t, t, \ves{\theta}) \big)$ approaches the identity matrix. Hence, 
  \begin{align*}
    \frac{\dee \log p(\ve{z})}{\dee t} 
    &= - \tr \bigg( 
      \lim_{\varepsilon \rightarrow 0^+} \frac{\partial \nabla_{\S1} \ve{s}_\varepsilon^+(\ve{z}_t, t, \ves{\theta}) }{\partial \varepsilon} 
    \bigg) \\
    &= - \tr \bigg( 
      \lim_{\varepsilon \rightarrow 0^+} \frac{\partial }{\partial \varepsilon} \frac{\partial}{\partial \ve{z}_t}
      \Big( 
        \ve{z}_t + \varepsilon \ve{f}(\ve{z}_t, t, \ves{\theta}) + O(\varepsilon^2)
      \Big) 
    \bigg) \\
    &= - \tr \bigg( 
      \lim_{\varepsilon \rightarrow 0^+} \frac{\partial }{\partial \varepsilon} 
      \Big( 
        I + \varepsilon \frac{\partial}{\partial \ve{z}_t} \ve{f}(\ve{z}_t, t, \ves{\theta}) + O(\varepsilon^2)
      \Big) 
    \bigg) \\
    &= - \tr \bigg( 
      \lim_{\varepsilon \rightarrow 0^+} 
      \Big( 
        \frac{\partial}{\partial \ve{z}_t} \ve{f}(\ve{z}_t, t, \ves{\theta}) + O(\varepsilon)
      \Big) 
    \bigg) \\
    &= - \tr \bigg( 
        \frac{\partial}{\partial \ve{z}_t} \ve{f}(\ve{z}_t, t, \ves{\theta})
    \bigg) \\
    &= - \tr(\nabla_{\S1}  \ve{f}(\ve{z}_t, t, \ves{\theta}) )
  \end{align*}  
  as required.
  \end{proof}

  \item Note that computing $\tr(\nabla_{\S1}  \ve{f}(\ve{z}_t, t, \ves{\theta}) )$ exactly is expensive if we do not restrict the form of $\ve{f}$. The best we can do is to evaluate
  \begin{align*}
    \nabla_{\S1} f_1(\ve{z}_t, t, \ves{\theta}), 
    \quad \nabla_{\S1} f_1(\ve{z}_t, t, \ves{\theta}), 
    \quad \dotsc, 
    \quad \nabla_{\S1} f_n(\ve{z}_t, t, \ves{\theta}), 
  \end{align*}
  and then add up the right components. This is equivalent to $n$ evaluations of $\ve{f}$ with automatic differentiation.

  \item A follow-up work by pretty the same group of authors proposes an algorithm that can generate an unbiased estimate of $\tr(\nabla_{\S1}  \ve{f}(\ve{z}_t, t, \ves{\theta}) )$ with just one evaluation of $\ve{f}$ with automatic differentiation \cite{Grathwohl:2018}. It uses something called the Hutchinson's trace estimator \cite{Hutchinson:1989}. However, we will not discuss this technique here in this note.  
\end{itemize}

\subsection{Generative Modeling with Continuous Normalizing Flows}

\begin{itemize}
  \item We fix $p(\ve{z}_{t_0})$ to be a simple probability distribution such as the isotropic Gaussian. Then, we would approximate $p_{\mrm{data}}$ with $p(\ve{z}_{t_1})$.
  
  \item Sampling is easy. We sample a $\ve{z}_{t_0}$ according to $p(\ve{z}_{t_0})$. Then, we compute
  \begin{align*}
    \ve{p}_{t_1} = \ve{p}_{t_0} + \int_{t_0}^{t_1} \ve{f}(\ve{z}_t, t, \ves{\theta})\, \dee t
  \end{align*}
  with an ODE solver.

  \item Given a data point $\ve{z}_{t_1}$, we can compute its probability by first noting that
  \begin{align*}
    \begin{bmatrix}
      \ve{z}_{t_1} \\
      \log p(\ve{z}_{t_1})
    \end{bmatrix}
    = \begin{bmatrix}
      \ve{z}_{t_0} \\
      \log p(\ve{z}_{t_0})
    \end{bmatrix}
    + \int_{t_0}^{t_1}
    \begin{bmatrix}
      \ve{f}(\ve{z}_t, t, \ves{\theta}) \\
      -\tr(\nabla_{\S1}\ve{f}(\ve{z}_t, t, \ves{\theta}))
    \end{bmatrix}\, \dee t.
  \end{align*}
  Rearranging, we have that
  \begin{align*}
    \begin{bmatrix}
      \ve{z}_{t_0} \\
      \log p(\ve{z}_{t_0}) - \log p(\ve{z}_{t_1})
    \end{bmatrix}
    = \begin{bmatrix}
      \ve{z}_{t_1} \\
      0
    \end{bmatrix}
    + \int_{t_1}^{t_0}
    \begin{bmatrix}
      \ve{f}(\ve{z}_t, t, \ves{\theta}) \\
      -\tr(\nabla_{\S1}\ve{f}(\ve{z}_t, t, \ves{\theta}))
    \end{bmatrix}\, \dee t.
  \end{align*}
  Hence, we can solve the reverse-time ODE
  \begin{align*}
    \frac{\dee}{\dee t} \begin{bmatrix}
      \ve{z}_t \\
      \log p(\ve{z}_t) - \log p(\ve{z}_{t_1})
    \end{bmatrix}
    &= \begin{bmatrix}
      \ve{f}(\ve{z}_t, t, \ves{\theta}) \\
      -\tr(\nabla_{\S1}\ve{f}(\ve{z}_t, t, \ves{\theta}))
    \end{bmatrix}
  \end{align*}
  from $t_1$ to $t_0$ with the initial value $(\ve{z}_{t_1}, 0)$. With the solution, we can easily compute $\log p(\ve{z}_{t_0})$ and then derive $\log p(\ve{z}_{t_1})$.

  \item Surprisingly, it is easier to find the gradient $\nabla_{\ve{z}_t} \log p(\ve{z}_{t_1})$ and $\nabla_{\ves{\theta}} \log p(\ve{z}_{t_1})$ and than to find $\log p(\ve{z}_{t_1})$. 
  
  First, though, let us redefine the probability function so that we can reuse consistent notation. Define the function $p^*$ as \begin{align*}
    p^*(\ve{z}_t, t, \ves{\theta}) = p(\ve{z}_t).
  \end{align*}
  So, we can now write $\nabla_{\S1} \log p^*(\ve{z}_t, t, \ves{\theta})$ and $\nabla_{\S3} \log p^*(\ve{z}_t, t, \ves{\theta})$ instead of $\nabla_{\ve{z}_t} \log p(\ve{z}_{t})$ and $\nabla_{\ves{\theta}} \log p(\ve{z}_{t})$.

  To find $\nabla_{\S1} \log p^*(\ve{z}_{t_1}, t_1, \ves{\theta})$ and $\nabla_{\S3} \log p^*(\ve{z}_{t_1}, t_1, \ves{\theta})$, we do the following. After being given $\ve{z}_{t_1}$, we can solve the neural ODE backward in time to find $\ve{z}_{t_0}$. Because $p(\ve{z}_{t_0})$ is fixed to a simple distribution, it should be simple to compute $p(\ve{z}_0)$ and
  \begin{align*}
    \nabla_{\S1}p^*(\ve{z}_{t_0}, t_0, \ves{\theta}) = \nabla p(\ve{z}_{t_0}).
  \end{align*}
  So, it is also easy toc compute $\nabla_{\S1} \log p^*(\ve{z}_{t_0}, t_0, \ves{\theta})$ because
  \begin{align*}
    \nabla_{\S1} \log p^*(\ve{z}_t, t, \ves{\theta}) 
    = \frac{\nabla_{\S1} p^*(\ve{z}_t, t, \ves{\theta})}{p^*(\ve{z}_t, t, \ves{\theta})}
    = \frac{\nabla p(\ve{z}_{t_0})}{p(\ve{z}_{t_0})}.
  \end{align*}

  Theorem~\ref{thm:adjoint-deriv} gives us the differential equation
  \begin{align*}
    \frac{\dee}{\dee t} \begin{bmatrix}
      \ve{z}_t \\
      \nabla_{\S1} \log p^*(\ve{z}_t, t, \ves{\theta}) \\
      \nabla_{\S3} \log p^*(\ve{z}_t, t, \ves{\theta})
    \end{bmatrix}
    = \begin{bmatrix}
      \ve{f}(\ve{z}_t, t, \ves{\theta}) \\
      \nabla_{\S1} \log p^*(\ve{z}_t, t, \ves{\theta}) \nabla_{\S1} \ve{f}(\ve{z}_t, t, \ves{\theta}) \\
      \nabla_{\S1} \log p^*(\ve{z}_t, t, \ves{\theta}) \nabla_{\S3} \ve{f}(\ve{z}_t, t, \ves{\theta})
    \end{bmatrix},
  \end{align*}  
  which we can now solve with the initial condition
  \begin{align*}
    \begin{bmatrix}
      \ve{z}_{t_0} \\
      \nabla p(\ve{z}_{t_0}) / p(\ve{z}_{t_0}) \\
      \ve{0}
    \end{bmatrix}
  \end{align*}
  from time $t_0$ to $t_1$. The second and the third block of the solution would give us $\nabla_{\S1} \log p^*(\ve{z}_{t_1}, t_1, \ves{\theta})$ and $\nabla_{\S3} \log p^*(\ve{z}_{t_1}, t_1, \ves{\theta})$.
\end{itemize}

\subsection{Continuous Planar Flow}

\begin{itemize}
  \item 
\end{itemize}

\bibliographystyle{alpha}
\bibliography{neural-ode}  
\end{document}