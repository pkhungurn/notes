\documentclass[10pt]{article}
\usepackage{fullpage}
\usepackage{amsmath}
\usepackage[amsthm, thmmarks]{ntheorem}
\usepackage{amssymb}
\usepackage{graphicx}
\usepackage{enumerate}
\usepackage{verse}
\usepackage{tikz}
\usepackage{verbatim}
\usepackage{hyperref}

\newtheorem{lemma}{Lemma}
\newtheorem{theorem}[lemma]{Theorem}
\newtheorem{definition}[lemma]{Definition}
\newtheorem{proposition}[lemma]{Proposition}
\newtheorem{corollary}[lemma]{Corollary}
\newtheorem{claim}[lemma]{Claim}
\newtheorem{example}[lemma]{Example}

\newcommand{\dee}{\mathrm{d}}
\newcommand{\Dee}{\mathrm{D}}
\newcommand{\In}{\mathrm{in}}
\newcommand{\Out}{\mathrm{out}}
\newcommand{\pdf}{\mathrm{pdf}}
\newcommand{\Cov}{\mathrm{Cov}}
\newcommand{\Var}{\mathrm{Var}}

\newcommand{\ve}[1]{\mathbf{#1}}
\newcommand{\mrm}[1]{\mathrm{#1}}
\newcommand{\ves}[1]{\boldsymbol{#1}}
\newcommand{\etal}{{et~al.}}
\newcommand{\sphere}{\mathbb{S}^2}
\newcommand{\modeint}{\mathcal{M}}
\newcommand{\azimint}{\mathcal{N}}
\newcommand{\ra}{\rightarrow}
\newcommand{\mcal}[1]{\mathcal{#1}}
\newcommand{\X}{\mathcal{X}}
\newcommand{\Y}{\mathcal{Y}}
\newcommand{\Z}{\mathcal{Z}}
\newcommand{\x}{\mathbf{x}}
\newcommand{\y}{\mathbf{y}}
\newcommand{\z}{\mathbf{z}}
\newcommand{\tr}{\mathrm{tr}}
\newcommand{\sgn}{\mathrm{sgn}}
\newcommand{\diag}{\mathrm{diag}}
\newcommand{\Real}{\mathbb{R}}
\newcommand{\sseq}{\subseteq}
\newcommand{\ov}[1]{\overline{#1}}
\DeclareMathOperator*{\argmax}{arg\,max}
\DeclareMathOperator*{\argmin}{arg\,min}

\title{Neural Ordinary Differential Equations}
\author{Pramook Khungurn}

\begin{document}
\maketitle

This is a note on the paper ``Neural Ordinary Differential Equations'' by Chen \etal \cite{Chen:2018}.

\section{Introduction}

\begin{itemize}
  \item Many existing neural networks models creates a sequence of hidden states $\ve{h}_0$, $\ve{h}_1$, $\ve{h}_2$, $\dotsc$ $\ve{h}_T$ by adding something to the previous state:
  \begin{align*}
    \ve{h}_{t+1} = \ve{h}_t + f(\ve{h}_t, t, \ves{\theta})
  \end{align*}
  Such models include such as residual networks \cite{He:2015}, recurrent neural networks, and normalizing flows \cite{Rezende:2015,Dinh:2014}.

  \item What if we take the limit as the number of time step goes to infinity? We will have a differential equation:
  \begin{align*}
    \frac{\dee\ve{h}(t))}{\dee t} = f(\ve{h}(t), t, \ves{\theta}).
  \end{align*}

  \item To use the network, we simply say that $\ve{h}(0)$ is the input layer, and the output is $\ve{h}(T)$ at some time $T$. The output can be found by solving the initial value problem, and this can be done by any black-box differential equation solver. 
\end{itemize}

\section{How to train a neural ODE model}

\begin{itemize}
  \item The problem with the above approach is that it is unclear how to train such a neural ODE model.
  \begin{itemize}
    \item The computation of the solution can require a lot of time steps. Differentiating through these time steps to compute the gradient would requires saving a lot of information in memory.
  \end{itemize}

  \item The good news is that there is a method to compute the gradient using constant memory (i.e., does not depend on the number of time steps). This is called the {\bf adjoint sensitivity method}. It requires, however, an ODE solve, which can be done, again, by any ODE solver.
  
  \item To derive the method, we first change the notations.
  \begin{itemize}
    \item The hidden state at time $t$ is now denoted by $\ve{z}(t)$.
    \item The start and end time is now $t_0$ and $t_1$ instead of $0$ and $T$, respectively.
  \end{itemize}
\end{itemize}


\bibliographystyle{alpha}
\bibliography{neural-ode}  
\end{document}