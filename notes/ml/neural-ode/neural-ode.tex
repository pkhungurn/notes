\documentclass[10pt]{article}
\usepackage{fullpage}
\usepackage{amsmath}
\usepackage[amsthm, thmmarks]{ntheorem}
\usepackage{amssymb}
\usepackage{graphicx}
\usepackage{enumerate}
\usepackage{verse}
\usepackage{tikz}
\usepackage{verbatim}
\usepackage{hyperref}

\newtheorem{lemma}{Lemma}
\newtheorem{theorem}[lemma]{Theorem}
\newtheorem{definition}[lemma]{Definition}
\newtheorem{proposition}[lemma]{Proposition}
\newtheorem{corollary}[lemma]{Corollary}
\newtheorem{claim}[lemma]{Claim}
\newtheorem{example}[lemma]{Example}

\newcommand{\dee}{\mathrm{d}}
\newcommand{\Dee}{\mathrm{D}}
\newcommand{\In}{\mathrm{in}}
\newcommand{\Out}{\mathrm{out}}
\newcommand{\pdf}{\mathrm{pdf}}
\newcommand{\Cov}{\mathrm{Cov}}
\newcommand{\Var}{\mathrm{Var}}

\newcommand{\ve}[1]{\mathbf{#1}}
\newcommand{\mrm}[1]{\mathrm{#1}}
\newcommand{\ves}[1]{\boldsymbol{#1}}
\newcommand{\etal}{{et~al.}}
\newcommand{\sphere}{\mathbb{S}^2}
\newcommand{\modeint}{\mathcal{M}}
\newcommand{\azimint}{\mathcal{N}}
\newcommand{\ra}{\rightarrow}
\newcommand{\mcal}[1]{\mathcal{#1}}
\newcommand{\X}{\mathcal{X}}
\newcommand{\Y}{\mathcal{Y}}
\newcommand{\Z}{\mathcal{Z}}
\newcommand{\x}{\mathbf{x}}
\newcommand{\y}{\mathbf{y}}
\newcommand{\z}{\mathbf{z}}
\newcommand{\tr}{\mathrm{tr}}
\newcommand{\sgn}{\mathrm{sgn}}
\newcommand{\diag}{\mathrm{diag}}
\newcommand{\Real}{\mathbb{R}}
\newcommand{\sseq}{\subseteq}
\newcommand{\ov}[1]{\overline{#1}}
\DeclareMathOperator*{\argmax}{arg\,max}
\DeclareMathOperator*{\argmin}{arg\,min}

\title{Neural Ordinary Differential Equations}
\author{Pramook Khungurn}

\begin{document}
\maketitle

This is a note on the paper ``Neural Ordinary Differential Equations'' by Chen \etal \cite{Chen:2018}.

\section{Introduction}

\begin{itemize}
  \item Many existing neural networks models creates a sequence of hidden states $\ve{h}_0$, $\ve{h}_1$, $\ve{h}_2$, $\dotsc$ $\ve{h}_T$ by adding something to the previous state:
  \begin{align*}
    \ve{h}_{t+1} = \ve{h}_t + \ve{f}(\ve{h}_t, t, \ves{\theta})
  \end{align*}
  Such models include such as residual networks \cite{He:2015}, recurrent neural networks, and normalizing flows \cite{Rezende:2015,Dinh:2014}.

  \item What if we take the limit as the number of time step goes to infinity? We will have a differential equation:
  \begin{align*}
    \frac{\dee\ve{h}(t))}{\dee t} = \ve{f}(\ve{h}(t), t, \ves{\theta}).
  \end{align*}

  \item To use the network, we simply say that $\ve{h}(0)$ is the input layer, and the output is $\ve{h}(T)$ at some time $T$. The output can be found by solving the initial value problem, and this can be done by any black-box differential equation solver. 
\end{itemize}

\section{How to train a neural ODE model}

\begin{itemize}
  \item The problem with the above approach is that it is unclear how to train such a neural ODE model.
  \begin{itemize}
    \item The computation of the solution can require a lot of time steps. Differentiating through these time steps to compute the gradient would requires saving a lot of information in memory.
  \end{itemize}

  \item The good news is that there is a method to compute the gradient using constant memory (i.e., does not depend on the number of time steps). This is called the {\bf adjoint sensitivity method}. It requires, however, an ODE solve, which can be done, again, by any ODE solver.
  
  \item To derive the method, we first change the notations.
  \begin{itemize}
    \item The hidden state at time $t$ is now denoted by $\ve{z}(t)$.
    
    \item The start and end time is now $t_0$ and $t_1$ instead of $0$ and $T$, respectively.
    
    \item With this notation, we have that, for any $t \geq t_0$,
    \begin{align*}
      \ve{z}(t) = \ve{z}(t_0) + \int_{t_0}^t \ve{f}(\ve{z}(u), u, \ves{\theta})\, \dee u.
    \end{align*}
    In this case, we view $\ve{z}(t)$ as a result of simulating the ODE forward in time from $t_0$ to $t$. It is thus probably better to write 
    \begin{align*}
      \ve{z}(t) = \ve{z}_0 + \int_{t_0}^t \ve{f}(\ve{z}(u), u, \ves{\theta})\, \dee u
    \end{align*}
    to emphasize the fact that $\ve{z}_0 := \ve{z}(t_0)$ is a part of the initial condition.

    \item We also have that
    \begin{align*}
      \ve{z}(t) = \ve{z}(t_1) + \int_{t_1}^{t_0} \ve{f}(\ve{z}(u), u, t)\, \dee u.
    \end{align*}
    Here, we view $\ve{z}(t)$ as a resulf simulating the ODE {\it backward} in time from $t_1$ to $t$. This view will become important later on. Again, when taking this view, it is better to write
    \begin{align*}
      \ve{z}(t) = \ve{z}_1 + \int_{t_1}^{t_0} \ve{f}(\ve{z}(u), u, t)\, \dee u.
    \end{align*}
    instead.
  \end{itemize}

  \item The goodness of the output $\ve{z}(t_1)$ is computed by a loss function $L(\cdot)$ as $L(\ve{z}(t_1))$.
  
  \item Because $\ve{z}(t_1)$ is a function of $\ve{z}(t_0)$, $t_0$, and $\ves{\theta}$, we may define a function $$\mcal{L}: (\ve{z}(t_0), t_0, \ves{\theta}) \mapsto L(\ve{z}(t_1))$$
  that maps the initial condition to the loss.

  \item Now, for any $t \leq t_1$, it makes sense to talk about $$\mcal{L}(\ve{z}, t, \ves{\theta}).$$ This expression gives the loss if we simulate the ODE starting with the hidden state $\ve{z}$ at time $t$ until time $t_1$.
  
  \item It also makes sense to talk about three gradient functions
  \begin{align*}
    \nabla_1\mcal{L}(\ve{z},t,\ves{\theta})
    & = \nabla_{\ve{z}}\mcal{L}(\ve{z},t,\ves{\theta})
    = 
    \frac{\partial \mcal{L}}{\partial \ve{z}} \bigg|_{(\ve{z}, t, \ves{\theta})}, \\
    \nabla_2\mcal{L}(\ve{z},t,\ves{\theta})
    & = \nabla_t \mcal{L}(\ve{z}, t, \ves{\theta})
    = \frac{\partial \mcal{L}}{\partial t} \bigg|_{(\ve{z}, t, \ves{\theta})}, \\
    \nabla_3\mcal{L}(\ve{z},t,\ves{\theta})
    &= \nabla_{\ves{\theta}} \mcal{L}(\ve{z}, t, \ves{\theta})
    = \frac{\partial \mcal{L}}{\partial \ves{\theta}} \bigg|_{(\ve{z}, t, \ves{\theta})}.
  \end{align*}
  These functions output linear transformations that can turn a perturbation on their associated variables into perturbation of the value of $\mcal{L}$. In particular, we have that
  \begin{align*}
    \mcal{L}(\ve{z} + \Delta\ve{z}, t, \ves{\theta})
    - \mcal{L}(\ve{z}, t, \ves{\theta})
    \approx \nabla_1 \mcal{L}(\ve{z}, t, \ves{\theta}) \Delta\ve{z}.
  \end{align*}
  Similar equations can written for $\nabla_2 \mcal{L}$ and $\nabla_3 \mcal{L}$. 

  \item We use the convention that, for a vector function $\ve{f}(\ve{x}): \Real^n \rightarrow \Real^m$, the gradient  $\nabla_{\ve{x}} \ve{f} = \frac{\partial \ve{f}}{\partial \ve{x}}$ would be an $m \times n$ matrix.
  \begin{itemize}
    \item As a result, $\nabla_\ve{z}\mcal{L} = \frac{\partial \mcal{L}}{\partial \ve{z}}$ is a row vector in $\Real^{1 \times d}$.
  \end{itemize}

  \item Note that our end goal is to compute
  \begin{align*}
    \nabla_3\mcal{L}(\ve{z}_0, t_0, \ves{\theta}) = \frac{\partial \mcal{L}}{\partial \ves{\theta}}\bigg|_{(\ve{z}_0, t_0, \ves{\theta})}.
  \end{align*}
  This is the gradient that we will use to update $\ves{\theta}$ in any SGD algorithm.
  

  \item To derive the adjoint sensitivity method, we focus on a trajectory of the hidden variable $\ve{z}$ during the time interval $[t_0, t_1]$. To avoid confusion, let us denote this fixed trajectory by $\ve{z}^*(t)$, and we say that it is determined by simulating the neural ODE backward in time from $t_1$. In other words,
  \begin{align*}
    \ve{z}^*(t) = \ve{z}^*_1 + \int_{t_1}^t \ve{f}(\ve{z}^*(u), u, \ves{\theta})\, \dee u.
  \end{align*}
  We note that $\ve{z}^*(t)$ is a function of $t$, $\ve{z}^*_1$, $t_1$, and $\ves{\theta}$.
   
  \item We define the {\bf adjoint function}
  \begin{align*}
    \ve{a}: (t,\ves{\theta}) \mapsto \nabla_1 \mcal{L} (\ve{z}^*(t), t, \ves{\theta}).
  \end{align*}
  In other words,
  \begin{align*}
    \ve{a}(t,\ves{\theta}) = \frac{\partial \mcal{L}}{\partial \ve{z}}\bigg|_{(\ve{z}^*(t), t, \ves{\theta})}.
  \end{align*}

  \item We adjoint function has the following property.
  \begin{theorem}
  \begin{align*}
    \frac{\partial \ve{a}}{\partial t}\bigg|_{(t,\ves{\theta})} = -\ve{a}(t,\ves{\theta}) \frac{\partial \ve{f}}{\partial \ve{z}}\bigg|_{(\ve{z}^*(t), t, \ves{\theta})}
  \end{align*}
  \end{theorem}

  \begin{proof}
    By definition,
    \begin{align*}
      \frac{\partial \ve{a}}{\partial t}\bigg|_{(t,\ves{\theta})}
      &= \lim_{\varepsilon \rightarrow 0^+} \frac{\ve{a}(t+ \varepsilon, \ves{\theta}) - \ve{a}(t, \ves{\theta})}{\varepsilon}.
    \end{align*}
    To evalute the above expression, we shall rewrite $\ve{a}(t,\ves{\theta})$ in terms of $\ve{a}(t+\varepsilon, \ves{\theta})$.

    First, we note that
    \begin{align*}
      \mcal{L}(\ve{z}^*(t), t, \ves{\theta}) 
      = \mcal{L}(\ve{z}^*(t+\varepsilon), t + \varepsilon, \ves{\theta})
    \end{align*}
    for any $0 \leq \varepsilon \leq t_1 - t$. This is because the loss should be the same if we start at any point on the trajectory $\{ \ve{z}^*(t) : t_0 \leq t \leq t_1 \}$. Hence, we can consider the process of computing $\mcal{L}(\ve{z}^*(t), t, \ves{\theta})$ by first sending $(\ve{z}^*(t), t, \ves{\theta})$ to $(\ve{z}^*(t + \varepsilon), t + \varepsilon, \ves{\theta})$ and then slapping $\mcal{L}$ on $(\ve{z}^*(t + \varepsilon), t + \varepsilon, \ves{\theta})$. Let $\ve{e}_\varepsilon$ be the function $(\ve{z}^*(t), t, \ves{\theta})$ to $(\ve{z}^*(t + \varepsilon), t + \varepsilon, \ves{\theta})$. It follows that
    \begin{align*}
      \ve{e}_\varepsilon\left( \begin{bmatrix} \ve{z} \\ t \\ \ves{\theta} \end{bmatrix} \right)
      &= \begin{bmatrix} \ve{z} \\ t \\ \ves{\theta} \end{bmatrix} 
      + \varepsilon  \begin{bmatrix} \ve{f}(\ve{z}, t, \ves{\theta}) \\ 1 \\ \ve{0} \end{bmatrix} 
      + O(\varepsilon^2).
    \end{align*}
    So,
    \begin{align*}
      \frac{\partial \ve{e}_\varepsilon}{\partial (\ve{z}, t, \ves{\theta})}\bigg|_{(\ve{z}^*(t), t, \ves{\theta})}
      &= I + \varepsilon \begin{bmatrix}
        \frac{\partial \ve{f}}{\partial\ve{z}}\big|_{(\ve{z}^*(t), t, \ves{\theta})}
        & \frac{\partial \ve{f}}{\partial t}\big|_{(\ve{z}^*(t), t, \ves{\theta})}
        & \frac{\partial \ve{f}}{\partial \ves{\theta}}\big|_{(\ve{z}^*(t), t, \ves{\theta})} \\
        \ve{0} & 0 & \ve{0} \\
        \ve{0} & \ve{0} & \ve{0}
      \end{bmatrix}
      + O(\varepsilon^2).
    \end{align*}
    Now, by the chain rule, we have that
    \begin{align*}
      \frac{\partial \mcal{L}}{\partial (\ve{z}, t, \ves{\theta})}\bigg|_{(\ve{z}^*(t),t,\ves{\theta})}
      &= \frac{\partial \mcal{L}}{\partial \ve{e}_\varepsilon(\ve{z}(t), t, \ves{\theta})} \frac{\partial \ve{e}_\varepsilon (\ve{z}(t), t, \ves{\theta})}{\partial (\ve{z}(t), t, \ves{\theta})} \\
      \begin{bmatrix}
        \frac{\partial \mcal{L}}{\partial \ve{z}(t)} &
        \frac{\partial \mcal{L}}{\partial t} &
        \frac{\partial \mcal{L}}{\partial \ves{\theta}}
      \end{bmatrix}
      &= \begin{bmatrix}
        \frac{\partial \mcal{L}}{\partial \ve{z}(t + \epsilon)} &
        \frac{\partial \mcal{L}}{\partial (t + \epsilon)} &
        \frac{\partial \mcal{L}}{\partial \ves{\theta}}
      \end{bmatrix}
      \bigg( I + \varepsilon \begin{bmatrix}
        \frac{\partial \ve{f}}{\partial\ve{z}(t)}
        & \frac{\partial \ve{f}}{\partial t}
        & \frac{\partial \ve{f}}{\partial \ves{\theta}} \\
        \ve{0} & 0 & \ve{0} \\
        \ve{0} & \ve{0} & \ve{0}
      \end{bmatrix}
      + O(\varepsilon^2) \bigg).
    \end{align*}
    By multipling the matrices out, it follows that
    \begin{align*}
      \frac{\partial \mcal{L}}{\partial \ve{z}(t)}
      &= \frac{\partial \mcal{L}}{\partial \ve{z}(t+\varepsilon)}
      + \varepsilon \frac{\partial \mcal{L}}{\partial \ve{z}(t+\varepsilon)} \frac{\partial \ve{f}}{\partial \ve{z}(t)} + O(\varepsilon^2).
    \end{align*}
    In other words,
    \begin{align*}
      \frac{\partial \mcal{L}}{\partial \ve{z}}\bigg|_{(\ve{z},t,\ves{\theta})}
      &= \frac{\partial \mcal{L}}{\partial \ve{z}}\bigg|_{(\ve{z+\varepsilon},t+\varepsilon,\ves{\theta})}
      \frac{\partial \ve{f}}{\partial \ve{z}}\bigg|_{(\ve{z}(t), t, \ves{\theta})}
    \end{align*}
  \end{proof}  
    
\end{itemize}


\bibliographystyle{alpha}
\bibliography{neural-ode}  
\end{document}