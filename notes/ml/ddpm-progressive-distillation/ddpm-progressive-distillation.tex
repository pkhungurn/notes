\documentclass[10pt]{article}
\usepackage{fullpage}
\usepackage{amsmath}
\usepackage[amsthm, thmmarks]{ntheorem}
\usepackage{amssymb}
\usepackage{graphicx}
\usepackage{enumerate}
\usepackage{verse}
\usepackage{tikz}
\usepackage{verbatim}
\usepackage{hyperref}
\usepackage{algorithm}
\usepackage{clrscode3e}

\newtheorem{lemma}{Lemma}
\newtheorem{theorem}[lemma]{Theorem}
\newtheorem{definition}[lemma]{Definition}
\newtheorem{proposition}[lemma]{Proposition}
\newtheorem{corollary}[lemma]{Corollary}
\newtheorem{claim}[lemma]{Claim}
\newtheorem{example}[lemma]{Example}

\newcommand{\dee}{\mathrm{d}}
\newcommand{\Dee}{\mathrm{D}}
\newcommand{\In}{\mathrm{in}}
\newcommand{\Out}{\mathrm{out}}
\newcommand{\pdf}{\mathrm{pdf}}
\newcommand{\Cov}{\mathrm{Cov}}
\newcommand{\Var}{\mathrm{Var}}

\newcommand{\ve}[1]{\mathbf{#1}}
\newcommand{\ves}[1]{\boldsymbol{#1}}
\newcommand{\mrm}[1]{\mathrm{#1}}
\newcommand{\etal}{{et~al.}}
\newcommand{\sphere}{\mathbb{S}^2}
\newcommand{\modeint}{\mathcal{M}}
\newcommand{\azimint}{\mathcal{N}}
\newcommand{\ra}{\rightarrow}
\newcommand{\mcal}[1]{\mathcal{#1}}
\newcommand{\X}{\mathcal{X}}
\newcommand{\Y}{\mathcal{Y}}
\newcommand{\Z}{\mathcal{Z}}
\newcommand{\x}{\mathbf{x}}
\newcommand{\y}{\mathbf{y}}
\newcommand{\z}{\mathbf{z}}
\newcommand{\tr}{\mathrm{tr}}
\newcommand{\sgn}{\mathrm{sgn}}
\newcommand{\diag}{\mathrm{diag}}
\newcommand{\Real}{\mathbb{R}}
\newcommand{\sseq}{\subseteq}
\newcommand{\ov}[1]{\overline{#1}}
\newcommand{\N}{\mathcal{N}}
\newcommand{\data}{\mathrm{data}}
\newcommand{\SNR}{\mathrm{SNR}}

\DeclareMathOperator*{\argmin}{arg\,min}

\title{Progressive Distillation of Diffusion Models}
\author{Pramook Khungurn}

\begin{document}
\maketitle

This note is written as I read ``Progressive Distillation for Fast Sampling of Diffusion Models'' by Salimans and Ho~\cite{Salimans:2022}.

\section{Introduction}

\begin{itemize}
  \item DDPMs can generate high quality samples. However, they are slow at sampling. Vanilla DDPMs (circa 2021) takes hundreds and thousands of models evaluations to sample one data item.
  \begin{itemize}
    \item Note, however, that improvements by Karras \etal~\cite{Karras:2022}, for example, reduces the number of sampling steps to tens (20 to 50) in practice.
  \end{itemize}

  \item The paper mentions that sampling a DDPM can be fast in ``strongly conditioned settings.''
  \begin{itemize}
    \item Text-to-speech.
    \item Image super-resolution.
    \item Classifier-guided sampling \cite{Dhariwal:2021}.
  \end{itemize}
  
  \item It is the slowest when less conditioning information is available.
  \begin{itemize}
    \item Unconditional sampling.
    \item Class conditional sampling.
  \end{itemize}

  \item The paper claims to make three contributions.
  \begin{itemize}
    \item A new parametermizations of DDPMs that make them more stable when sampling with a few steps.
    \item A method to distill a deterministic sampler that use many steps to one that uses a few steps.
    \item Showing that the distillation process does not take more time than the process to train the original model.
  \end{itemize}

  \item The core algorithm's specification.
  \begin{itemize}
    \item {\bf Input.} A pre-trained DDPM that takes $N$ steps to sample.
    \item {\bf Output.} A new DDPM that takes $N/2$ steps to sample.
  \end{itemize}

  \item Progressive distillation $=$ keep applying the core algorithm until it takes a few, say $4$, steps to sample.
  
  \item Summary of results.
  \begin{itemize}
    \item Start out with SOTA DDPM that takes 8192 steps to sample.
    \item Distill it down to 4 steps.
    \item Perceptual quality does not suffer much: FID of 3.0 on CIFAR-10 in 4 steps.
  \end{itemize}
\end{itemize}

\section{Background}

\begin{itemize}
  \item The paper uses the notations in introduced in Kingma \etal's ``Variational Diffusion Models'' paper \cite{Kingma:2021}.
  
  \item A data item is denoted by $\ve{x}$. The data distribution is denoted by $p_{\data}$.
  
  \item Diffusion models work on {\bf latent variables} $\{ \ve{z}_t : t \in [0,1] \}$. The latent variables form a Gaussian process that evolves over time according to the {\bf forward process} $q(\ve{z}_t|\ve{x})$ where
  \begin{align*}
    q(\ve{z}_t | \ve{x}) &= \mcal{N}(\ve{z}_t; \alpha_t \ve{x}, \sigma^2_t I)    
  \end{align*}
  where $\alpha_t$ and $\sigma_t$ are differentiable functions of $t$, collectively known as the {\bf noise schedule}.

  \item Let $\lambda_t = \log (\alpha_t^2 / \sigma_t^2)$. We call it the {\bf signal-to-noise ratio (SNR)}. We require that the SNR decreases monotonically with $t$ and that  
  \begin{align*}
    q(\ve{z}_t | \ve{z}_s) = \mcal{N}\bigg(\ve{z}_t ; \frac{\alpha_t}{\alpha_s}\ve{z}_s; \sigma_{t|s}^2 I\bigg)
  \end{align*}
  where $0 \leq s < t \leq 1$ and $$\sigma_{t|s}^2 = (1 - e^{\lambda_t - \lambda_s})\sigma_t^2.$$

  \item We train a neural network $\hat{\ve{x}}_{\ves{\theta}}$ so that $\hat{\ve{x}}_{\ves{\theta}}(\ve{z}_t, \lambda_t) \approx \ve{x}$. In other words, $\hat{\ve{x}}(\ve{z}_t, \lambda_t)$ denoise $\ve{z}_t$ back to $\ve{x}$. The network is trained with the following loss:
  \begin{align*}
    E_{\ve{x} \sim p_{\data}, t \sim \mcal{U}([0,1]), \ve{z}_t \sim \mcal{N}(\alpha_t \ve{x}, \sigma^2_t I)}[w(\lambda_t)\| \hat{\ve{x}}_{\ves{\theta}}(\ve{z}_t, \lambda_t) - \ve{x} \|^2].
  \end{align*}
  Here, $\mcal{U}([0,1])$ is the uniform distribution over the interval $[0,1]$. The weighting function $w(\lambda_t)$ will be discussed later.

  \item We can deduce that, for any $0 \leq s < t \leq 1$,
  \begin{align*}
    q(\ve{z}_s | \ve{z}_t, \ve{x}) = \mcal{N}(\ve{z}_s; \tilde{\ves{\mu}}_{s|t}(\ve{z}_t, \ve{x}), \tilde{\sigma}_{s|t}^2 I)
  \end{align*}
  where
  \begin{align*}
    \tilde{\ves{\mu}}_{s|t}(\ve{z}_t, \ve{x}) &= e^{\lambda_t - \lambda_s} \frac{\alpha_s}{\alpha_t} \ve{z}_t + (1 - e^{\lambda_t - \lambda_s})\alpha_s \ve{x}, \\
    \tilde{\sigma}_{s|t}^2 &= (1 - e^{\lambda_t - \lambda_s}) \sigma_s^2.
  \end{align*}

  \item We can use the above equation as a basis for the {\bf ancestral sampling algortihm}.
  \begin{enumerate}
    \item We have a sequence of times $0 = t_0 < t_1 < t_2 < \dotsc < t_K = 1$.    
    \item We start with $\ve{z}_1 \sim \mcal{N}(0,I)$.
    \item Given that we have a sample at time $t$ and we want sample at another time $s < t$, then,
    \begin{align*}
      \ve{z}_s = \tilde{\ves{\mu}}_{s|t}(\ve{z}_t, \hat{\ve{x}}_{\ves{\theta}}(\ve{z}_t, \lambda_t)) + \sqrt{(\tilde{\sigma}_{s|t}^2)^{1-\gamma}(\sigma_{t|s}^2)^\gamma} \, \ves{\xi}
    \end{align*}
    where $\xi \sim \mcal{N}(\ve{0},I)$, and $\gamma$ is a hyperparameter that controls how much noise is added during sampling \cite{Nichol:2021}.
  \end{enumerate}

  \item Alternatively, $\ve{z}_t$ satisfies the following {\bf probability flow ODE}:
  \begin{align*}
    \dee\ve{z}_t = \bigg( f(\ve{z}_t, t) - \frac{1}{2}g(t)^2 \nabla_{\ve{z}}\log \hat{p}_{\ves{\theta}}(\ve{z}_t) \bigg)\, \dee t
  \end{align*}
  where
  \begin{align*}
    \nabla_{\ve{z}}\log \hat{p}_{\ves{\theta}}(\ve{z}_t) &= \frac{\alpha_t \hat{\ve{x}}_{\ves{\theta}}(\ve{z}_t, \lambda_t) - \ve{z}_t}{\sigma_t^2}, \\
    f(\ve{z}_t, t) &= \frac{\dee \log \alpha_t}{\dee t} \ve{z}_t, \\
    g(t)^2 &= \frac{\dee \sigma_t^2}{\dee t} - 2 \frac{\dee \log \alpha_t}{\dee t} \sigma_t^2.
  \end{align*}
  Sampling can then be done by simulating the ODE backward in time from $t = 1$ to $t = 0$.

  \item The DDIM sampler proposed by Song \etal~\cite{Song:DDIM:2020} is given by
  \begin{align*}
    \ve{z}_s 
    &= \alpha_s \hat{\ve{x}}_{\ves{\theta}}(\ve{z}_t) + \sigma_s \frac{\ve{z}_t - \alpha_t \hat{\ve{x}}_{\ves{\theta}}(\ve{z}_t)}{\sigma_t} \\
    &= e^{(\lambda_t - \lambda_s)/2}(\alpha_s / \alpha_t) \ve{z}_t + (1 - e^{(\lambda_t - \lambda_s)/2}) \alpha_s \hat{\ve{x}}_{\ves{\theta}}(\ve{z}_t).
  \end{align*}
  The paper claims that this can be understood in terms of integrating the probability flow ODE.
\end{itemize}

\section{Progressive Distillation}

\begin{itemize}
  \item The core algorithm receives as input a {\bf teacher model} that is trained in the standard way.
  
  \begin{codebox}
    \Procname{$\proc{Standard-Diffusion-Training}$}
    \li \While not converged
    \li \Do
          $\ve{x} \sim p_{\data}$
    \li   $t \sim \mcal{U}([0,1])$
    \li   $\ves{\xi} \sim \mcal{N}(\ve{0}, I)$
    \li   $\ve{z}_t \leftarrow \alpha_t \ve{x} + \sigma_t \ves{\theta} $
    \li   $\lambda_t \leftarrow \log (\alpha_t^2 / \sigma_t^2)$
    \li   $L_{\ves{\theta}} \leftarrow w(\lambda_t) \| \ve{x} - \hat{\ve{x}}_{\ves{\theta}}(\ve{z}_t, \lambda_t) \|_2^2 $
    \li   Update $\ves{\theta}$ according to $\nabla_{\ves{\theta}} L_{\ves{\theta}}$.
        \End
  \end{codebox}

  \item The result of the distillation process is a {\bf student model}, which has the same architecture as the teacher model.
  
  \item The distillation process proceeds in iterations.
  \begin{itemize}
    \item Each iteration halves the number of sampling steps required.
    \item At the start of each iteration, the student model is initialized with the paremeters of the teacher model. 
    \item At the end of each iteration, the student model becomes the teacher model of the next round.
    \item The training process in each iteration is similar to the standard training. However, the main difference is the target that $\hat{\ve{x}}_{\ves{\theta}}$ is asked to approximate from $\ve{z}_t$.
    \begin{itemize}
      \item For the standard training, the target is $\ve{x}$.
      \item For the distillation process, the target is the denoised data that would have been predicted by the teacher in two distillation steps.
    \end{itemize}
    \item Another difference is that the distillation process always work in discrete time instead of continuous time like the standard training.
  \end{itemize}

  \item Here's the distillation algorithm.
  
  \begin{codebox}
    \Procname{$\proc{Progressive-Distillation}(\ves{\eta}, N)$}
    \zi \Comment $\ves{\eta}$ denotes the parameters of the trained teacher model.
    \zi \Comment $N$ is the number of iterations that the teacher model takes to sample.
    \li \For $K$ iterations
    \li \Do    
          $N \leftarrow N/2$
    \li   $\ves{\theta} \leftarrow \ves{\eta}$
    \li   \While not converged
    \li   \Do
            $\ve{x} \sim p_{\data}$
    \li     $i \sim \mcal{U}(\{ 1, 2, \dotsc, N \})$
    \li     $t \leftarrow i / N$
    \li     $\ves{\xi} \sim \mcal{N}(\ve{0},I)$
    \zi     \Comment 2 steps of DDIM sampling with teacher model
    \li     $t' \leftarrow t - 0.5/N;\quad t'' \leftarrow t - 1/N $
    \li     $\lambda_{t} \leftarrow \log(\alpha_{t}^2/\sigma_{t}^2);\quad \lambda_{t'} \leftarrow \log(\alpha_{t'}^2 / \sigma_{t'}^2)$
    \li     $\ve{z}_{t'} \leftarrow \alpha_{t'} \hat{\ve{x}}_{\ves{\eta}}(\ve{z}_t, \lambda_t) + \frac{\sigma_{t'}}{\sigma_t}\big(\ve{z}_t - \alpha_t \hat{\ve{x}}_{\ves{\eta}}(\ve{z}_t, \lambda_t)\big)$
    \li     $\ve{z}_{t''} \leftarrow \alpha_{t''} \hat{\ve{x}}_{\ves{\eta}}(\ve{z}_{t'}, \lambda_{t'}) + \frac{\sigma_{t''}}{\sigma_{t'}}\big(\ve{z}_{t'} - \alpha_{t'} \hat{\ve{x}}_{\ves{\eta}}(\ve{z}_{t'}, \lambda_{t'})\big)$
    \li     $\tilde{\ve{x}} \leftarrow \frac{\ve{z}_{t''} - (\sigma_{t''}/\sigma_t)\ve{z}_t}{\alpha_{t''} - (\sigma_{t''}/\sigma_t)\alpha_t}$
    \li     $L_{\ves{\theta}} \leftarrow w(\lambda_t) \| \tilde{\ve{x}} - \hat{\ve{x}}_{\ves{\theta}}(\ve{z}_t, \lambda_t) \|^2$
    \li     Update $\ves{\theta}$ according to $\nabla_{\ves{\theta}} L_{\ves{\theta}}$
          \End
    \zi   \Comment Make the converged student the teacher of the next round.
    \li   $\ves{\eta} \leftarrow \ves{\theta}$
        \End  
  \end{codebox}

  \item The paper notes that using the value predicted by the teacher in two steps as a target makes the prediction task for the student model much easier.
  \begin{itemize}
    \item If using $\ve{x}$ as the target, the model must predict the average of possible $\ve{x}$ value, which produces blurry predictions.
    \item However, the value predicted by the teacher model is completely determined by the teacher model and $\ve{z}_t$. So, the prediction is sharp.
  \end{itemize}
  As a result, the student model can make progress much faster than the vanilla teacher model can during sampling.
\end{itemize}

\section{Parameterization and Training Loss}

\begin{itemize}
  \item This section is about the details of the models: $\alpha_t$, $\sigma_t$, $w(\lambda_t)$, and what $\hat{\ve{x}}_{\ves{\theta}}$ is.
  
  \item The paper assumes a variance preserving model: $\sigma_t^2 = 1 - \alpha_t^2$.
  
  \item Cosine schedule is used: $\alpha_t = \cos(0.5 \pi t)$.  
\end{itemize}

\subsection{Model Parameterization}

\begin{itemize}
  \item Many works on DDPM parameterize $\hat{\ve{x}}_{\ves{\theta}}$ by requiring it to predict the noise $\ves{\xi}$ that is used to make $\ve{z}_t$ from $\ve{x}$. In other words, our network is $\hat{\ves{\xi}}_{\ves{\theta}}$, and we take
  \begin{align*}
    \hat{\ve{x}}_{\ves{\theta}}(\ve{z}_t, \lambda_t) = \frac{1}{\alpha_t}\big(\ve{z}_t - \sigma_t \hat{\ves{\xi}}_{\ves{\theta}}(\ve{z}_t, \lambda_t)\big).
  \end{align*}
  
  \item While the above parametermization might work well for training the original model, the paper argues that this does not work well with distillation.
  \begin{itemize}
    \item As distillation progresses, we increasingly evaluate at lower and lower signal-to-noise ratios.
    
    \item As $\alpha_t \rightarrow 0$, the changes made by $\hat{\ves{\xi}}_{\ves{\theta}}$ gets amplified drastically. This is because
    \begin{align*}
      \hat{\ve{x}}_{\ves{\theta}}(\ve{z}_t, \lambda_t) = \frac{1}{\alpha_t}\big(\ve{z}_t - \sigma_t \hat{\ves{\xi}}_{\ves{\theta}}(\ve{z}_t, \lambda_t)\big).
    \end{align*}
    So, changes are multipled by $1 / \alpha_t$.

    \item If we distill down to a single sampling step, the input to the model is pure noise $\ves{\xi}$, and the distilled $\hat{\ves{\xi}}_{\ves{\theta}}$ would have to predict $\ves{\xi}$. The link between $\ves{\xi}$-prediction and $\ve{x}$-prediction breaks down completely.
  \end{itemize}

  \item So, for distillation to work, we need to parametermize the model so that the prediction $\hat{\ve{x}}_{\ves{\theta}}$ remains stable as the SNR varies. The paper tried a number of combinations and found all of them to work well.
  \begin{itemize}
    \item Predicting $\ve{x}$ directly.
    \item Predicting both $\ve{x}$ and $\ves{\xi}$, resulting in $\tilde{\ve{x}}$ and $\tilde{\ves{\xi}}$, and then merging via
    \begin{align*}
      \hat{\ve{x}} = \sigma_t^2 \tilde{\ve{x}} + \alpha_t(\ve{z}_t - \sigma_t \tilde{\ves{\xi}}).
    \end{align*}
    Note that $\sigma_t^2 \tilde{\ve{x}} \approx \sigma_t^2 \ve{x}$, and
    \begin{align*}
      \alpha_t (\ve{z}_t - \sigma_t \tilde{\ves{\xi}}) 
      = \alpha_t (\alpha_t \ve{x} + \sigma_t \ves{\theta} - \sigma_t \tilde{\ves{\xi}}) \approx \alpha_t^2 \ve{x}.
    \end{align*}
    Because $\alpha_t^2 + \sigma_t^2 = 1$, we have that $\sigma_t^2 \tilde{\ve{x}} + \alpha_t(\ve{z}_t - \sigma_t \tilde{\ves{\xi}}) \approx \sigma_t^2 \ve{x} + \alpha_t^2\ve{x}$ is an interpolation between the $\ve{x}$-prediction and the implied $\ve{x}$ from $\ves{\xi}$-prediction.
    \item Predicting $\ve{v} := \alpha_t \ves{\xi} - \sigma_t\ve{x}$ and setting $\hat{\ve{x}} := \alpha_t \ve{z}_t - \sigma_t \hat{\ve{v}}_{\ves{\theta}}(\ve{z}_t, \lambda_t)$.
  \end{itemize}

  \item The paper also tried all the above combinations when training the original models, and it found that they worked well too.
\end{itemize}

\subsection{Weighting Function}

\begin{itemize}
  \item In the 2020 work by Ho \etal~\cite{Ho:2020}, they sample the time uniformly and compute the squared L2-distance between $\ves{\xi}$ and $\hat{\ves{\xi}}_{\ves{\theta}}(\ve{z}_t, \lambda_t)$. One can deduce that
  \begin{align*}
    L_{\ves{\theta}} = \| \ves{\xi} - \ves{\xi}_{\ves{\theta}}(\ve{z}_t,\lambda_t) \|^2 = \frac{\alpha_t^2}{\sigma_t^2} \| \ve{x} - \hat{\ve{x}}_{\ves{\theta}}(\ve{z}_t, \lambda_t) \|^2.
  \end{align*}
  This is called the {\bf SNR} weighting scheme. This means that the loss gives zero weight to data with zero SNR. This is not a suitable loss for distillation.

  \item The paper tried the following choices of weighting scheme.
  \begin{itemize}
    \item {\bf Truncated SNR}: $L_{\ves{\theta}} = \max(\|\ve{x} - \hat{\ve{x}} \|^2, \| \ves{\xi} - \hat{\ves{\xi}} \|^2 ) = \max(\frac{\alpha_t^2}{\sigma_t^2}, 1) \| \ve{x} - \hat{\ve{x}} \|^2$.
    \item {\bf SNR+1}: $L_{\ves{\theta}} = \| \ve{v} - \hat{\ve{v}} \| = (1 + \frac{\alpha_t^2}{\sigma_t^2} ) \| \ve{x} - \hat{\ve{x}} \|^2. $
  \end{itemize}
  It found that both worked well.

  \item The authors did not put much attention on how the time is sampled. They just sampled it uniformly randomly from $[0,1]$.
\end{itemize}

\section{Experiments}

\subsection{Model Parameterization and Training Loss}

\begin{itemize}
  \item The paper tried the 4 combinations of model parameterization and 3 combination of weighting schemes.
  \begin{itemize}
    \item One case lead to training divergence: predicting $\ves{\xi}$ and using the truncated SNR weighting scheme.
    \item In all other cases, the numbers are similar.
  \end{itemize}

  \item The paper observed that predicting $\ve{v}$ is the most stable method because it makes the DDIM step-sizes independent of the SNR. However, predicting $\ve{x}$ gives the best empirical results.
\end{itemize}

\subsection{Progressive Distillation}

\begin{itemize}
  \item The authors started with a teacher model trained on continous time. They then started distilling from $N = 8192$ (or $N = 1024$ for bigger models) down to $1$.
  
  \item In each distillation iteration, they optimize the model for $50\,000$ parameter updates. The exceptions are when distilling to $2$ or $1$ step(s), which they took $100\,000$ updates. They claimed that the total time used to distill does not exceed that used to train the original model.
  
  \item FID scores of distilled models increases very slowly until $N = 4$, after which they rises very rapidly. So, $N = 4$ seems to be a sweet spot between speed and quality.
  
  \item Undistilled models' performance degrades very fast after $N = 128$.
\end{itemize}

\bibliographystyle{alpha}
\bibliography{ddpm-progressive-distillation}  
\end{document}