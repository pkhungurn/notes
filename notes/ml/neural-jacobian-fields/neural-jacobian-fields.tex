\documentclass[10pt]{article}
\usepackage{fullpage}
\usepackage{amsmath}
\usepackage[amsthm, thmmarks]{ntheorem}
\usepackage{amssymb}
\usepackage{graphicx}
\usepackage{enumerate}
\usepackage{verse}
\usepackage{tikz}
\usepackage{verbatim}
\usepackage{hyperref}
\usepackage{bbm}

\newtheorem{lemma}{Lemma}
\newtheorem{theorem}[lemma]{Theorem}
\newtheorem{definition}[lemma]{Definition}
\newtheorem{proposition}[lemma]{Proposition}
\newtheorem{corollary}[lemma]{Corollary}
\newtheorem{claim}[lemma]{Claim}
\newtheorem{example}[lemma]{Example}

\newcommand{\dee}{\mathrm{d}}
\newcommand{\Dee}{\mathrm{D}}
\newcommand{\In}{\mathrm{in}}
\newcommand{\Out}{\mathrm{out}}
\newcommand{\pdf}{\mathrm{pdf}}
\newcommand{\Cov}{\mathrm{Cov}}
\newcommand{\Var}{\mathrm{Var}}

\newcommand{\ve}[1]{\mathbf{#1}}
\newcommand{\ves}[1]{\boldsymbol{#1}}
\newcommand{\mrm}[1]{\mathrm{#1}}
\newcommand{\etal}{{et~al.}}
\newcommand{\sphere}{\mathbb{S}^2}
\newcommand{\modeint}{\mathcal{M}}
\newcommand{\azimint}{\mathcal{N}}
\newcommand{\ra}{\rightarrow}
\newcommand{\mcal}[1]{\mathcal{#1}}
\newcommand{\X}{\mathcal{X}}
\newcommand{\Y}{\mathcal{Y}}
\newcommand{\Z}{\mathcal{Z}}
\newcommand{\x}{\mathbf{x}}
\newcommand{\y}{\mathbf{y}}
\newcommand{\z}{\mathbf{z}}
\newcommand{\tr}{\mathrm{tr}}
\newcommand{\sgn}{\mathrm{sgn}}
\newcommand{\diag}{\mathrm{diag}}
\newcommand{\Real}{\mathbb{R}}
\newcommand{\sseq}{\subseteq}
\newcommand{\ov}[1]{\overline{#1}}
\newcommand{\data}{\mathrm{data}}

\DeclareMathOperator*{\argmin}{arg\,min}


\title{Neural Jacobian Fields}
\author{Pramook Khungurn}

\begin{document}
\maketitle


This note is written as I read the paper ``Neural Jacobian Fields: Learning Intrinsic Mappings of Arbitrary Meshes'' by Aigerman \etal~\cite{Aigerman:2022}.

\section{Problem Setup}

\begin{itemize}
    \item This paper presents a framework to generate ``piecewise linear mappings'' of meshes with a neural network.
    
    \item In this note, we treat points in $\Real^d$ as \emph{row vectors}.
    \begin{itemize}
        \item So, if $\ve{x} \in \Real^3$, then
        \begin{align*}
            \ve{x} = \begin{bmatrix}
                x_1 & x_2 & x_3
            \end{bmatrix}.
        \end{align*}

        \item This means that a linear transformation from $\Real^3$ to $\Real^2$ is encoded with a $3 \times 2$ matrix instead of $2 \times 3$. This matrix is multiplied to the right instead of to the left.
    \end{itemize}

    \item This makes it quite convenient to represent a batch of $N$ points and matrices.
    \begin{itemize}
        \item A batch of $N$ points in $\Real^3$ is represented by a matrix $\ve{P}$ of size $N \times 3$.
        \item A linear transformations from $\Real^3$ to $\Real^2$ is represented by a matrix $M$ fo size $3 \times 2$.
        \item If we wish to apply this linear transformation to all points in the batch, we just compute $\ve{P} M$.
    \end{itemize}    
\end{itemize}

\subsection{Linear Piecewise Mapping}

\begin{itemize}
    \item Let us formally define what a mesh is.
    \begin{itemize}
        \item We are given a triangular mesh $\mcal{S}$ in $\Real^3$ with vertices $\ve{V}$ and triangles $\ve{T}$.
        \item The $i$th triangle in $\ve{T}$ is denoted by $\ve{t}_i$.
        \item The {\bf tangent space} of $\ve{t}_i$ is the linear space orthogonal to its normal. It is denoted by $T_i$.
        \begin{itemize}
            \item Bascially, this is the plane in $\Real^3$ that the triangle lies in.
        \end{itemize}
        \item A {\bf frame} of triangle $\ve{t}_i$ is an oriented orthonomal basis fo its tangent space. It is denoted by $\mcal{B}_i$.
        \begin{itemize}
            \item We don't actually know in which direction the frame points to.
            \item I guess the natural direction is the direction of the normal vector.            
        \end{itemize}
    \end{itemize}

    \item Let's talk about how the tangent space is represented numerically.
    \begin{itemize}
        \item Let us say that the three vertices of $\ve{t}_i$ have indices $j$, $k$, and $l$.
        \item So, the triangle vertices are $\ve{v}_j$, $\ve{v}_k$, and $\ve{v}_l$.
        \item Let us also say that we designate $\ve{v}_j$ is the origin of tangent space $T_i$.
        \item So, we have that $T_i = \{ a(\ve{v}_k - \ve{v}_j) + b(\ve{v}_l - \ve{v}_j) : a, b \in \Real \}$.
        \begin{itemize}
            \item Here, we see that $T_i$ is a set of vectors because $\ve{v}_k - \ve{v}_j$ and $\ve{v}_l - \ve{v}_j$ are vectors.            
        \end{itemize}
        \item $\mcal{B}_i$ is an orthonomal basis of $T_i$. So, we may say that it is a $2 \times 3$ matrix:
        \begin{align*}
            \mcal{B}_i = \begin{bmatrix}
                \ves{\beta}_{i,1} \\ 
                \ves{\beta}_{i,2}
            \end{bmatrix}            
        \end{align*}
        where $\ves{\beta}_{i,1}, \ves{\beta}_{i,2} \in \Real^{3}$ such that $\| \ves{\beta}_{i,1} \| = \| \ves{\beta}_{i,2} \| = 1$ and $\ves{\beta}_{i,1} \cdot \ves{\beta}_{i,2} = 1$. Last but not least, $T_i = \{ a \ves{\beta}_{i,1} + b \ves{\beta}_{i,2} : a, b \in \Real \}.$
        \item Because $\mcal{B}_i$ is an orthonormal basis of $T_i$, we have that, for any vector $\ve{v} \in T_i$, we can find the coordinates of $\ve{v}$ with respect to $\mcal{B}_i$ as follows:
        \begin{align*}
            \mbox{$\ve{v}$'s coordinate} = \ve{v} \mcal{B}_i^T.
        \end{align*}
    \end{itemize}
    
    \item A {\bf map} or a {\bf mapping}, in our context, is a function $\Phi: \Real^{3} \ra \Real^{3}$.
    
    \item A map $\Phi$ is a {\bf piecewise linear mapping} with respect to a mesh $\mcal{S}$ if the restriction of $\Phi$ to any triangle $\ve{t}_i$, denote $\Phi|_{\ve{t}_i}$ is affine.
    \begin{itemize}
        \item This is the most common family of maps used when considering mappings of meshes.
        \item It is uniquely defined by assigning a new position to one of the vertices.
        \begin{itemize}
            \item In other words, let $\ve{v}_j$ denote the position of the $j$th vertex. Suppose that we know $\Phi_j$, the image of $\ve{v}_j$ under the mapping, for all $j$.
            
            \item Then, any point inside the triangle is sent to the interpolation of the $\Phi_j$s with barycentric coordinates.            
            
            \item As a result, we may encode $\Phi$ as a matrix of the same dimension as $\ve{V}$.
            
            \item In this case, $\Phi$ is a matrix of size $|\ve{V}| \times 3$.            
        \end{itemize}        
    \end{itemize}
    
    \item There's another way to encoder $\Phi$ using matrices we call ``Jacobians.''
    
    \item The {\bf Jacobian} $J_i$ of triangle $\ve{t}_i$ is a linear transformation of dimension $2 \times 3$ from $\ve{t}_i$'s tangent space to $\Real^{1 \times 3}$.
    \begin{itemize}
        \item So, it's signature is $J_i: T_i \ra \Real^{3}$.
        \item We can compute it as  $2 \times 3$ matrix in the coordinates of the frame $\mcal{B}_i$ by solving
        \begin{align}
            \begin{bmatrix}
                \ve{v}_k - \ve{v}_j \\ 
                \ve{v}_l - \ve{v}_j
            \end{bmatrix}
            \mcal{B}_i^T
            J_i
            = \begin{bmatrix}
                \Phi_k - \Phi_j \\ \Phi_l - \Phi_j
            \end{bmatrix}. \label{eqn:jacobian-definition}
        \end{align}
        \item We can write $J_i$ in terms of $\Phi \in \Real^{|\ve{V}| \times 3}$ as follows.
        \begin{itemize}
            \item We first rewrite the RHS of Equation~\eqref{eqn:jacobian-definition} in terms of $\Phi$.
            \item There's a matrix $S_{jkl} \in \Real^{3 \times |\ve{V}|}$ whose entries are $0$s and $1$s such that
            \begin{align*}
                S_{jkl} \Phi = \begin{bmatrix} \Phi_j \\ \Phi_k \\ \Phi_l \end{bmatrix}.
            \end{align*}
            \item As a resulit, we have that
            \begin{align*}
                \begin{bmatrix} -1 & 1 & 0 \\ -1 & 0 & 1 \end{bmatrix}  S_{jkl} \Phi  
                = \begin{bmatrix} -1 & 1 & 0 \\ -1 & 0 & 1 \end{bmatrix} \begin{bmatrix} \Phi_j \\ \Phi_k \\ \Phi_l \end{bmatrix}
                = \begin{bmatrix} \Phi_k - \Phi_j \\ \Phi_l - \Phi_j \end{bmatrix}.
            \end{align*}
            
            \item Because $\mcal{B}_j \in \Real^{2 \times 3}$ and $\begin{bmatrix} \ve{v}_k-\ve{v}_j \\ \ve{v}_l-\ve{v}_j \end{bmatrix} \in \Real^{2\times 3}$, we have that $\begin{bmatrix} \ve{v}_k-\ve{v}_j \\ \ve{v}_l-\ve{v}_j \end{bmatrix} \mcal{B}_j^T$ is a $2 \times 2$ matrix. Morever, it better has full rank because we don't want $\ve{v}_k - \ve{v}_j$ and $\ve{v}_l - \ve{v}_j$ to be linearly dependent. Now,
            \begin{align*}
                \begin{bmatrix}
                    \ve{v}_k - \ve{v}_j \\ 
                    \ve{v}_l - \ve{v}_j
                \end{bmatrix}
                \mcal{B}_i^T
                J_i
                = \begin{bmatrix} -1 & 1 & 0 \\ -1 & 0 & 1 \end{bmatrix}  S_{jkl} \Phi.
            \end{align*}
            So,
            \begin{align*}
                J_i = \bigg( \begin{bmatrix}
                    \ve{v}_k - \ve{v}_j \\ 
                    \ve{v}_l - \ve{v}_j
                \end{bmatrix}
                \mcal{B}_i^T \bigg)^{-1} \begin{bmatrix} -1 & 1 & 0 \\ -1 & 0 & 1 \end{bmatrix}  S_{jkl} \Phi.
            \end{align*}
            
            \item The paper defines
            \begin{align*}
                \nabla_i = J_i = \bigg( \begin{bmatrix}
                    \ve{v}_k - \ve{v}_j \\ 
                    \ve{v}_l - \ve{v}_j
                \end{bmatrix}
                \mcal{B}_i^T \bigg)^{-1} \begin{bmatrix} -1 & 1 & 0 \\ -1 & 0 & 1 \end{bmatrix}  S_{jkl},
            \end{align*}
            so that we may write
            \begin{align*}
                J_i = \nabla \Phi.
            \end{align*}
            The ``operator'' $\nabla_i$ maps $\Phi$ to the Jacobian $J_i$ in the basis $\mcal{B}_i$. It is a matrix of size $2 \times |\mathbf{V}|$.
        \end{itemize}
        \item We can also write the whole stack of Jacobians as follows:
        \begin{align*}
            J = \begin{bmatrix}
                J_1 \\ J_2 \\ \cdots \\ J_{|\ve{T}|}                
            \end{bmatrix}
            &= \nabla \Phi 
        \end{align*}
        where
        \begin{align*}
            \nabla = \begin{bmatrix}
                \nabla_1 \\ \nabla_2 \\ \vdots \\ \nabla_{|\ve{T}|}
            \end{bmatrix}.
        \end{align*}
        Here, $J$ is a matrix of size $2|\ve{T}| \times 3$, and $\nabla$ is a matrix of size $2|\ve{T}| \times |\ve{V}|$.
    \end{itemize}

    \item If we have $J$, the stack of every Jacobian matrices, then it determines $\Phi$ up to a translation if the mesh is connected.
    \begin{itemize}
        \item If the mesh is connected, then there is a sequence of triangle edges that connect $\ve{v}_0$ to any particular vertex $\ve{v}_m$.
        \item For each of these triangle edges, say $(\ve{v}_j, \ve{v}_k)$ that belongs to Triangle $i$, then we can use $J_i$ to compute $\Phi_k - \Phi_j$.
        \item We can then follow this chain to compute $\Phi_m - \Phi_0$ for all indices $m \in \{1, \dotsc, |\ve{V}|-1\}$.
        \item So, $\Phi$ is determined up to the position of $\Phi_0$.
    \end{itemize}
    As a result, computing the Jacobians is tantamount to computing a piecewise linear mapping (up to a translation).
\end{itemize}

\subsection{Predicting Piecewise Linear Mapping}

\begin{itemize}
    \item The goal of the paper is to create a neural network that consumes a mesh and produces a piecewise linear mapping of the mesh.
    
    \item We know what the output looks like. It's a stack of Jacobian $J$, which determines the mapping $\Phi$ up to a translation.
    
    \item So, let's talk about the missing part: what ``consuming a mesh'' means.
    
    \item Before that, let's mention the main goal of the paper. {\bf How the prediction gets done should be agnostic to the triangulation of the mesh.}
    \begin{itemize}
        \item The network should not need to know how many triangles the input mesh has. What triangles are in the mesh, and how are the triangles are connected.
    \end{itemize}

    \item So, the input that the network takes is as follow.
    \begin{itemize}
        \item The network takes a {\bf global code} $\ve{z}$ of the mesh. 
        \begin{itemize}
            \item The paper uses the encoding by PointNet \cite{Qi:PointNet:2017}.
            \begin{itemize}
                \item We sample $1024$ points on the mesh.
                \item We create a tensor of point information. Each entry corresponds to a point and has the following fields: (1) the point's 3D position, (2) the point's normal vector, and (3) the point's Wave-Kernel signature \cite{Aubry:WKS:2011} of size 50.
                \item We feed the above tensor to PointNet to get an encoding.            
            \end{itemize}                    
        \end{itemize}

        \item It can also takes other conditioning information $\ve{y}$ such as the desired pose of the output mesh, the joint angles of the skeleton, desired positon to which mesh should bend to, and so on.
        
        \item To produce a Jacobian matrix for each triangle $\ve{t}_i$, the network takes in information of the centroid $\ve{c}_i$ of the triangle as input.
        \begin{itemize}
            \item The centroid's position.
            \item The centroid's normal vector, which is the triangle's normal vectors.
            \item The centroid's Wave-Kernel signature of size 50.
        \end{itemize}
    \end{itemize}

    \item The paper mentions that it trains the main network alongside the PointNet encoder. So, in effect, the paper wants to create the system of two networks.
    \begin{itemize}
        \item A {\bf mesh encoder} that maps the shape of a mesh to a {\bf global code} $\ve{z}$.
        \item A {\bf mapper} that takes in the global code $\ve{z}$, the conditioning information $\ve{y}$, and centroid features $\ve{c}_i$ and output the Jacobian of that triangle $J_i$.
    \end{itemize}

    \item Note that the above two networks do not have to know how many triangles there are in the mesh or how the triangles are connected.
\end{itemize}

\section{Method}

\begin{itemize}
    \item The training data is a collection of tuples $\{ \mcal{S}^\ell, \Psi^\ell, \ve{y}^\ell \}$ where
    \begin{itemize}
        \item $\mcal{S}^\ell$ is the a mesh.
        \item $\Psi^\ell$ is the groundtruth mapping, which is basically an assignment of each vertex in $\mcal{S}^\ell$ to a new position.
        \item $\ve{y}^\ell$ is a conditioning information.
    \end{itemize}

    \item In order to for the prediction to be triangulation agnostic, it is done in two phases.
    \begin{itemize}
        \item First, we predict a matrix $P_i \in \Real^{3 \times 3}$ from $(\ve{c}_i, \ve{z}, \ve{y})$.
        \item Then, we restrict the matrices $P_i$ to the tangent space of $\ve{t}_i$ to get a new matrix $R_i$, which we pass to the next phase.
        \item $R_i$ is computed as follows.
        \begin{align*}
            R_i = \mathcal{B}_i P_i
        \end{align*}
    \end{itemize}

    \item The 2nd phase is based on the following logic.
    \begin{itemize}
        \item We are given an arbitrary assignment of a matrix $R_i \in \Real^{2 \times 3}$ to each triangle $\ve{t}_i$.
        \begin{itemize}
            \item When stacked up, this becomes $R = \begin{bmatrix}
                R_1 \\ R_2 \\ \cdots \\ R_{|\ve{T}|}
            \end{bmatrix} \in \Real^{2|\ve{T}| \times 3}$
        \end{itemize}
        
        \item We would like to retrive a map $\Phi^*$ such that $\nabla \Phi$ is the closest to $P$ in the least square sense.
        
        \item This can be done by solving the following optimization problem.
        \begin{align*}
            \Phi^* = \argmin_{\Phi} \sum_{i} |\ve{t}_i| \big\| \nabla \Phi - R_i \big\|^2
        \end{align*}
        where $|\ve{t}_i|$ is the area of the triangle $\ve{t}_i$ on the source mesh $\mcal{S}$.
        
        \item We then return $\Phi^*$ is the prediction.
    \end{itemize}

    \item Now, here's the confusing thing about this paper.
    
    \begin{itemize}
        \item It calls 
        \begin{align*}
            \Phi^* = \argmin_{\Phi} \sum_{i} |\ve{t}_i| \big\| \nabla \Phi - P_i \big\|^2
        \end{align*}
        the ``Poisson's equation.''
        \begin{itemize}
            \item This is not Poisson's equation.
            \item {\bf Poisson's equation} looks like this:
            \begin{align*}
                \Delta u = f
            \end{align*}
            where $\Delta$ is the Laplacian operator.
            \item The equation above has no raplacian.
        \end{itemize}

        \item However, what it tries to do resembles the {\bf Poisson's problem.}
        \begin{itemize}
            \item We are given a domain $\Omega \subseteq \Real^3$.
            \item We are also given a vector field $\ve{g}: \Real^3 \ra \Real^3$.
            \item Find a scalar function $u: \Omega \ra \Real$ such that it minimizes the Dirichlet energy
            \begin{align*}
                \int_\Omega \| \nabla u(\ve{x}) - \ve{g}(\ve{x}) \|^2\, \dee\ve{x}.
            \end{align*}            
        \end{itemize}

        \item The solution to the Poisson's problem above can be found by solving the following Poisson's equation.
        \begin{align*}
            \Delta u = \nabla \cdot \ve{g}.
        \end{align*}

        \item The paper proceeds to find matrices that look like they reprsent the $\Delta$ and the $\nabla \cdot$ operators.
        \begin{itemize}
            \item Unfortunately, the matrix that they use to represent $\Delta$ is of size $|\ve{V}| \times |\ve{V}|$.
            \item Moreover, the matrix that they use to represent $\nabla \cdot$ is of size $|\ve{V}| \times 3|\ve{T}|$.
        \end{itemize}
        
        \item However, the $\Delta$ operator that is used in the equation $J = \nabla \Phi$ has size $2|\ve{T}| \times |V|$, which doesn't actually fit anything.
        
        \item The main problem is that \emph{Jacobians are not exactly gradients}, so using solving the Poisson's equation for it does not make sense!
    \end{itemize}

\end{itemize}


\bibliographystyle{arxivalpha}
\bibliography{neural-jacobian-fields}  
\end{document}