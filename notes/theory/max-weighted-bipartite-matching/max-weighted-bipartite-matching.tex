\documentclass[10pt]{article}
\usepackage{fullpage}
\usepackage{amsmath}
\usepackage[amsthm, thmmarks]{ntheorem}
\usepackage{amssymb}
\usepackage{graphicx}
\usepackage{epstopdf}
\usepackage{enumerate}
\usepackage{verse}
\usepackage{tikz}

\newtheorem{lemma}{Lemma}[section]
\newtheorem{theorem}[lemma]{Theorem}
\newtheorem{definition}[lemma]{Definition}
\newtheorem{proposition}[lemma]{Proposition}
\newtheorem{corollary}[lemma]{Corollary}
\newtheorem{claim}[lemma]{Claim}
\newtheorem{example}[lemma]{Example}

\newcommand{\dee}{\mathrm{d}}
\newcommand{\Dee}{\mathrm{D}}
\newcommand{\In}{\mathrm{in}}
\newcommand{\Out}{\mathrm{out}}
\newcommand{\pdf}{\mathrm{pdf}}

\newcommand{\ve}[1]{\mathbf{#1}}
\newcommand{\mrm}[1]{\mathrm{#1}}
\newcommand{\etal}{{et~al.}}
\newcommand{\sphere}{\mathbb{S}^2}
\newcommand{\modeint}{\mathcal{M}}
\newcommand{\azimint}{\mathcal{N}}
\newcommand{\ra}{\rightarrow}
\newcommand{\mcal}[1]{\mathcal{#1}}
\newcommand{\likelihood}{\mathcal{L}}
\newcommand{\X}{\mathcal{X}}
\newcommand{\Y}{\mathcal{Y}}
\newcommand{\Z}{\mathcal{Z}}
\newcommand{\x}{\mathbf{x}}
\newcommand{\y}{\mathbf{y}}
\newcommand{\z}{\mathbf{z}}
\newcommand{\tr}{\mathrm{tr}}
\newcommand{\sgn}{\mathrm{sgn}}
\newcommand{\diag}{\mathrm{diag}}
\newcommand{\new}{\mathrm{new}}
\newcommand{\Arg}{\mathrm{Arg\,}}
\newcommand{\Log}{\mathrm{Log\,}}
\newcommand{\RE}{\mathrm{Re\,}}
\newcommand{\IM}{\mathrm{Im\,}}
\newcommand{\Res}{\mathrm{Res}}
\newcommand{\pv}{\mathrm{p.v.}}
\newcommand{\Real}{\mathbb{R}}
\newcommand{\sseq}{\subseteq}
\newcommand{\II}{\mathrm{II}}
\DeclareMathOperator{\Bd}{Bd}

\newcommand{\IF}{\mathbf{if}}
\newcommand{\FI}{\mathbf{fi}}
\newcommand{\THEN}{\mathbf{then}}
\newcommand{\ELSE}{\mathbf{else}}
\newcommand{\WHILE}{\mathbf{while}}
\newcommand{\DO}{\mathbf{do}}
\newcommand{\OD}{\mathbf{od}}
\newcommand{\PROC}{\mathbf{proc}}
\newcommand{\END}{\mathbf{end}}
\newcommand{\CALL}{\mathbf{call}}

\title{Maximum Weighted Bipartite Matching}
\author{Pramook Khungurn}

\begin{document}
	\maketitle

  \section{Maximum Weight Bipartite Matching Problem}

  \begin{itemize}
    \item We consider the problem of {\bf maximum weight bipartite matching} (MWBM).

    \item As input, we are given a weighted bipartite graph $G = (V,E)$ where
    \begin{itemize}
      \item $V = X \cup Y$,
      \item $X \cap Y = \emptyset$, and
      \item $E \sseq X \times Y$.
    \end{itemize}
    Also, there's a function $w: E \ra \Real^{+} \cup \{0\}$.

    \item A {\bf matching} is a subset $M \sseq E$ such that, for every vertex $v \in V$, at most one edge in $M$ is incident upon $v$.

    \item The {\bf size of matching $M$}, denoted by $|M|$, is the number of edges in $M$.

    \item The {\bf weight of matching $M$}, denoted by $w(M)$, is the sum of the weights of the edges in $M$. That is,
    \begin{align*}
      w(M) = \sum_{e\in M} w(e).
    \end{align*}

    \item The MWBM problem wants to find a matching $M$ whose weight is the maximum among all possible matchings.
  \end{itemize}

  \section{The Assignment Problem}
  \begin{itemize}
    \item In the {\bf assignement problem}, we are given a complete weighted bipartite graph, and we want to find the maximum weight matching.

    \item The MWBM problem can be reduced to the assignment problem.

    This can be done by:
    \begin{itemize}
       \item introducing dummy nodes so that $|X| = |Y|$, and
       \item for every pair of vertices $(x,y)$ such that $(x,y) \not\in E$, creating a new edge $(x,y)$ with weight $0$.
     \end{itemize} 

     \item A maximum weight matching in a complete bipartite graph can be made {\it perfect}. (That is, every vertex is incident to an edge.)

     \item So, the assignment problem is to find a perfect matching with maximum weight.
  \end{itemize}

  \section{Feasible Labeling}
  \begin{itemize}
    \item A {\bf vertex labeling} is a function $\ell: V \ra \Real$.

    \item A {\bf feasible labeling} is one such that
    \begin{align*}
      \ell(x) + \ell(y) \geq w(x,y)
    \end{align*}
    for all $x \in X$ and $y \in Y$.

    \item An edge $(x,y)$ is called {\bf tight} if $\ell(x) = \ell(y) = w(x,y)$.

    \item The {\bf equality graph} with respect to a labeling $\ell$ is $G_\ell = (V, E_\ell)$ where $E_\ell$ is the set of tight edges.

    \item \begin{theorem}
      If $\ell$ is feasible and $M$ is a perfect matching in $G_\ell$, then $M$ is a maximum weight matching.
    \end{theorem}
    \begin{proof}
      Denote edge $e \in E$ by $e = (e_x, e_y)$.

      Let $M'$ be any perfect matching in $G$ (not necessarily in $E_\ell$). Since every vertex $v \in V$ is incident to exactly one edge in $M'$, we have that
      \begin{align*}
        w(M') = \sum_{e \in M'} w(e) \leq \sum_{e \in M'} (\ell(e_x) + \ell(e_y)) = \sum_{v\in V} \ell(v).
      \end{align*}
      Hence, $\sum_{v\in V} \ell(v)$ is an upper bound on the cost of any perfect matching.

      Now, let $M$ be a perfect matching in $E_\ell$. Then, $w(M) = \sum_{e\in M} w(e) = \sum_{v\in V} \ell(v)$. So, $w(M') \leq w(M)$ and $M$ is optimal.
    \end{proof}

    \item If you wonder where the heck the above theorem comes from, it comes from writing the matching problem as a linear programming and take the dual.
  \end{itemize}

  \section{The Hungarian Algorithm}

  \begin{itemize}
    \item With respect to a graph $G$ (not necessarily complete) a matching $M$ in $G$,
    \begin{itemize}
      \item a vertex is {\bf free} is it is incident to no edges in $M$,
      \item a vertex is {\bf matched} if it is not free,
      \item a path in $G$ is {\bf alternating} if its edges alternate between $M$ and $E - M$,
      \item a path is {\bf augmenting} if both end points are free, and
      \item the {\bf residual graph} of $G$ with respect to $M$ is a directed graph $G' = (V', E')$ where      
      \begin{itemize}
        \item $V' = V$, and
        \item for each edge $(x,y) \in E$,
        \begin{itemize}
          \item if $(x,y) \not\in M$, then $(x,y) \in E'$, and
          \item if $(x,y) \in M$, then $(y,x) \in E'$
        \end{itemize}
      \end{itemize}
    \end{itemize}

    \item The sketch of the algorithm is as follows:
    \begin{enumerate}
      \item Start with a feasible labeling $\ell$, and a maximum size matching $M$ in $G_\ell$.

      \item If $M$ is perfect, we are done. 

      \item If not, then then we find another feasible labeling $\ell'$ such that $E_\ell \subset E_{\ell'}$.\\
      Then, we set $\ell$ to $\ell'$, recompute $M$, and go back to Step 2.      
    \end{enumerate}

    \item After Step 3, either $M$ or $E_\ell$ increases in size. Hence, the algorithm must terminate.

    \item An initial feasible labeling is given by:
    \begin{itemize}
      \item $\ell(y) = 0$ for all $y \in Y$, and
      \item $\ell(x) = \max{y\in Y} \{ w(x,y) \}$ for all $x \in X$.
    \end{itemize}

    \item Before we go on to find how to find labeling $\ell'$ such that $E_\ell \subset E_{\ell'}$, we need to define one more set of terminology.

    \item Let $\ell$ be a feasible labeling.

    The {\bf neighbor} of a vertex $u \in V$ is the set $N_\ell(u) = \{v : (u,v) \in E_\ell\}$.

    The {\bf neighbof} of the set $S \sseq V$ is the set $N_\ell(V) = \bigcup_{u \in S} N_\ell(u)$.

    \item The process of finding $\ell'$ where $E_{\ell} \subset E_{\ell'}$ uses the following lemma:

    \begin{lemma}
      Let $S \sseq X$ and $T = N_\ell(S) \neq Y$. Let
      \begin{align*}
        \alpha_\ell = \min_{x\in S, y \neq T} \{ \ell(x) + \ell(y) - w(x,y) \}.
      \end{align*}
      Define $\ell'$ as follows:
      \begin{align*}
        \ell'(v) = \begin{cases}
          \ell(v) - \alpha_\ell, & \mbox{if } v \in S,\\
          \ell(v) + \alpha_\ell, & \mbox{if } v \in T,\\
          \ell(v), & \mbox{otherwise.}
        \end{cases}
      \end{align*}
      Then, $\ell'$ is a feasible labeling, and
      \begin{itemize}
        \item if $(x,y) \in E_\ell$ for $x \in S$ and $y \in T$, then $(x,y) \in E_{\ell'}$,
        \item if $(x,y) \in E_\ell$ for $x \not\in S$ and $y \not\in T$, then $(x,y) \in E_{\ell'}$, and
        \item there exists some edge $(x,y) \in E_{\ell'}$ for $x \in S$ and $y \not\in T$.
      \end{itemize}
    \end{lemma}

    \begin{proof}
      We first show that $E_{\ell'}$ is a feasible labeling. Let $x \in X$ and $y \in Y$. There four cases three cases:
      \begin{enumerate}
        \item $x \in S$ and $y \in T$. In this case, $\ell'(x) + \ell'(y) = \ell(x) - \alpha_\ell + \ell(y) + \alpha_\ell = \ell(x) + \ell(y) \geq w(x,y).$
        \item $x \not\in S$ and $y \in T$. In this case, $\ell'(x) + \ell'(y) = \ell(x) + \ell(y) + \alpha_\ell \geq \ell(x) + \ell(y) \geq w(x,y).$ This is simply because $\alpha_\ell \geq 0$.
        \item $x \in S$ and $y \not\in T$. In this case,
        \begin{align*}
          \ell'(x) + \ell'(y) 
          &= \ell(x) - \alpha_\ell + \ell(y) \\
          &= w(x,y) + (\ell(x) + \ell(y) - w(x,y)) - \min_{x\in S, y \not\in T} \{ \ell(x) + \ell(y) - w(x,y) \}\\
          &\geq w(x,y).
        \end{align*}
        \item $x \not\in S$ and $y \not\in T$. In this case, $\ell'(x) + \ell'(y) = \ell(x) + \ell(y) \geq w(x,y).$
      \end{enumerate}
      So, $\ell'$ is a feasible labeling.

      From the above analysis, in Case 1 and Case 4, we have that $\ell'(x) + \ell'(y) = \ell(x) + \ell(y)$. Thus, an edge $(x,y)$ remains in $E_{\ell'}$ if it is (1) already in $E_\ell$, and (2) either $x \in S$ and $y \in T$ or $x \neq S$ and $y \neq T$.

      Also, there exists an edge $(x,y)$ with $x \in S$ and $y \not\in T$ where $\ell(x) + \ell(y) - w(x,y)$ achieve its minimum. This edge cannot already be in $E_\ell$; otherwise, it would already be included in the neighborhood $N_\ell(S)$. After the update, we see that $\ell(x) + \ell(y) - w(x,y)$ goes to $0$. Hence, this is a new edge in $E_{\ell'}$ which is not in $E_\ell$ before.
    \end{proof}  

    \item It remains to find a set $S \sseq X$ such that $N_S(X) \neq Y$.

     \begin{lemma}
      Let $M$ be a maximum size matching in a bipartite graph $G$. Suppose there exists some vertices in $X$ that are free. Let $L$ be the vertices reachable from any free vertex in $X$ in the residual graph of $G$ with respect to $M$. Then, $C = (X - L) \cup (B \cap L)$ is a vertex cover and $|C| = |M|.$
    \end{lemma}

    \begin{proof}
      If $C$ is not a vertex cover, then there exists an edge $(x,y) \in E$ such that $x \in X \cap L$ and $y \in Y - L$.

      First, we claim that $(x,y) \not\in E - M$. Otherwise, we have that $x \in X \cap L$, which means it is reachable from a free vertex in the residual graph. Moreover, $(x,y) \in E'$, so this we can follow a path from a free vertex to $x$ and then to $y$.

      Next, we claim that $(x,y) \not\in M$. Otherwise, the fact that $(x,y) \in M$ means that $x$ is matched. Now, if a matched vertex $x$ is reachable from a free vertex, it means that it must be reached to the directed edge $(y,x)$. This implies that $y$ is reachable from a free vertex, but this contradicts the fact that $y \in B - L$.

      So, such an edge $(x,y)$ does not exist, and $C$ is a vertex cover.

      We now show that $|C| \leq |M|$. 

      First, we have that no vertices in $X - L$ are free because free vertices in $X$ are included in $L$ by definition. 

      Also, no vertices in $Y \cap L$ are free. Otherwise, a path from a free node to a free vertex in $Y$ exists and is an augmenting path. This contradicts the fact that $M$ is not a maximum matching.

      Moreover, there cannot be any edge $(x,y) \in M$ where $x \in X-L$ and $y \in Y \cap L$. Othewise, $x$ would be included in $L$. So, every vertex in $X-L$ and $Y \cap L$ is incident to an edge in $M$, but no two vertices in $(X-L) \cup (Y \cap L)$ can share an edge. This means that $|M| \geq |(X - L) \cup (Y \cap L)| = |C|.$

      Now, the size of a maximum matching is a lower bound on the size of a vertex cover. Hence, $|M| \leq |C|$. It follows that $|C| = |M|$.
    \end{proof}

    \item Now, let $M$ be a maximum matching in $G_\ell$. If $M$ is not perfect, then there exists some vertices that are free. We let $L$ be the set of vertices reachable from these free vertices in the residual graph of $G_\ell$ with respect to $M$. We can then set $S = X \cap L$.

    Observe that $N_\ell(S) = Y \cap L$.

    It follows that $|N_\ell(S)| = |Y \cap L| \leq |(X - L) \cup (Y \cap L)| = |C| = |M| < |Y|$. So, there exists some vertex $y \in Y$ such that $y \not\in N_\ell(S)$.
    
    \item The Hungarian algorithm:
    \begin{enumerate}
      \item Generate initial labeling $\ell$ and maximum cardinality matching $M$ in $E_\ell$.

      \item If $M$ is perfect, stop.
      
      \item Let $v$ be a free vertex with respect to $M$.\\
      Construct an alternating tree in $G_\ell$ with respect to $M$, eninating from $v$.\\
      Let $L$ be the set of vertices reachable from any free vertex in $X$, including the free vertices themselves.

      \item Set $S = X \cap L$ and $T = Y \cap L$.\\
      Compute the new weight $\ell'$ according to the process in Lemma 4.1.

      \item Add the new edge created by this process to $G_\ell$ and recompute $M$.

      \item Go to Step 2.    
    \end{enumerate}

    \item We will find at most $|V|/2 = O(|V|)$ augmenting paths. Finding an augmenting path requires a breadth first search, which takes $O(|V|^2)$ because we have a complete graph. So, augmenting the paths take $O(|V|^3)$ time.

    Updating the weight takes $O(|V|^2)$ time. However, we only need up update the weight $O(|V|)$ time because, each time we update, there will always be a new augmenting path. So, updating the weights take $O(|V|^3)$ time as well.

    All in all, the algorithm takes $O(|V|^3)$ time.
  \end{itemize}

\end{document}