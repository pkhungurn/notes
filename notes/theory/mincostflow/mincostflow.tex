\documentclass[10pt]{article}
\usepackage{fullpage}
\usepackage{amsmath}
\usepackage[amsthm, thmmarks]{ntheorem}
\usepackage{amssymb}
\usepackage{clrscode3e}

\newtheorem{lemma}{Lemma}[section]
\newtheorem{theorem}[lemma]{Theorem}
\newtheorem{definition}[lemma]{Definition}
\newtheorem{proposition}[lemma]{Proposition}
\newtheorem{claim}[lemma]{Claim}

\title{Notes on Minimum-Cost Flow Algorithms}
\author{Pramook Khungurn}

\begin{document}
  \maketitle
  
  \section{Problem Definitions} % (fold)
  \label{sec:problem-definition}
  
    \begin{itemize}
      \item In the \emph{minimum-cost flow problem}, you are given:
        \begin{itemize}
          \item a graph $G = (V,E)$;
          \item a \emph{capacity function} $u: E \rightarrow \mathbb{R}^{+} \cup \{ 0 \}$;
          \item a \emph{demand function} (defined on the vertices) $b: V \rightarrow \mathbb{R}$;
          \item a \emph{cost function} $c: E \rightarrow \mathbb{R}$.
        \end{itemize}
        You are to find a \emph{flow} $f: E \rightarrow \mathbb{R}^{+} \cup \{ 0 \}$ such that
        \begin{itemize}
          \item for all edge $e \in E$, we have $0 \leq f(e) \leq u(e)$;
          \item for all vertex $v \in V$, $$\sum_{(v,w) \in E} f(v,w) - \sum_{(w,v) \in E} f(w,v) = b(v);$$
          \item the \emph{the cost of the flow} $\sum_{e \in E} c(e)f(e)$ is minimal.
        \end{itemize}
      
      \item The min-cost flow problem is a generalization of many graphs problems. For examples:
        \begin{itemize}
          \item The {\bf shortest path} problem is a min-cost flow where you set
            (1) the capacity of every edge to 1, (2) the cost to its length,
            and (3) $b(s) = 1$ and $b(t) = -1.$
          
          \item The {\bf max flow} problem sets the cost of all edges to zero,
            and tries to find the maximum possible flow. (We shall see
            that this is actually equivalent to min-cost flow.)
          
          \item The {\bf disjoint path} problem asks to connect $s$ and $t$
            with $k$ disjoint paths using the smallest number of edges
            possible. This can be casted as a min-cost flow problem
            where (1) all the edges have capacity 1 and cost 1, and 
            (2) $b(s) = k$ and $b(t) = -k.$
        \end{itemize}
      
      \item In the \emph{minimum-cost circulation problem}, you are given:
        \begin{itemize}
          \item a graph $G = (V,E);$
          \item a \emph{capacity function} $u: E \rightarrow \mathbb{R}^{+} \cup \{ 0 \}$;
          \item a \emph{demand function} $l: E \rightarrow \mathbb{R}^{+} \cup \{ 0 \}$ such that $l(e) \leq u(e)$ for all $e \in E$;
          \item a \emph{cost function} $c: E \rightarrow \mathbb{R}$.
        \end{itemize}
        You are to find a \emph{circulation} $f: E \rightarrow \mathbb{R}^+ \cup \{ 0 \}$ such that
        \begin{itemize}
          \item for all edge $e \in E$, we have $l(e) \leq f(e) \leq u(e)$;
          \item for all vertex $v \in V$, $$\sum_{(v,w) \in E} f(v,w) - \sum_{(w,v) \in E} f(w,v) = 0;$$
          \item the \emph{the cost of the circulation} $\sum_{e \in E} c(e)f(e)$ is minimal.
        \end{itemize}
        
      \item
        \begin{lemma}
          The min-cost flow problem and the min-cost circulation problem are equivalent.
        \end{lemma}
        
        \begin{proof}
          ($circulation \implies flow$) Let $(G, u, b, c)$ be an instance of
            min-cost flow problem. Construct instance $(G', u', l', c')$ of
            the min-cost circulation problem as follows:
            \begin{itemize}
              \item Construct $G'$ by adding a new vertex $s$ to $G$.
              \item For each $v \in V$ where $b(v) > 0$, add an edge $(s,v)$
                with $u'(s,v) = l'(s,v) = b(v)$, and $c(s,v) = 0$.
              \item For each $v \in V$ where $b(v) < 0$, add an edge $(v,s)$
                with $u'(v,s) = l'(v,s) = |b(v)|$ and $c(v,s) = 0$.
              \item For any other edge $e$, set $l'(e) = 0$ and
                $u'(e) = u(e)$ and $c'(e) = c(e)$.
            \end{itemize}
            Then, there is a bijection between a feasible flow in $G$ and 
            a feasible circulation in $G'$. Moreover, since the new edges
            added to $G'$ have cost 0, the cost of the flow equals to
            the cost of the circulation. Hence, a min-cost circulation
            gives a min-cost flow.
            
          ($flow \implies circulation$) Let $(G, u, l, c)$ be an instance 
            of min-cost circulation problem. Construct instance $(G', u', b', c')$
            of the min-cost flow problem as follows:
            \begin{itemize}
              \item $G'$ is the same as $G$.
              \item For all $e \in E$, set $u'(e) = u(e) - l(e)$.
              \item For all $v \in V$, set $b'(v) = \sum_{(v,w) \in E} l(v,w) - \sum_{(w,v) \in E} l(w,v)$.
              \item For all $e \in E$, set $c'(e) = c(e).$
            \end{itemize}
            There is a bijection between a feasible circulation in $G$
            and a feasible flow in $G'$. The cost of the flow in $G'$
            is not equal to the cost of circulation in $G$, but they
            differ by a constant. Hence, a min-cost flow gives a
            min-cost circulation.
        \end{proof}
        
        Since the two problems are equivalent, we will work only with
        the min-cost circulation problem.
        
      \item The circulation problem's formulation can be simplified to
        make proofs and algorithm descriptions easier as follows:
        \begin{itemize}
          \item We replace each edge by two edges of opposite direction.
          \item For each newly created opposite $(w,v)$ of edge $(v,w)$,
            we set:
            \begin{itemize}
              \item $u(w,v) = -l(v,w)$.
              \item $c(w,v) = -c(v,w)$.
            \end{itemize}
          \item We eliminate the lower bound function altogether.
        \end{itemize}
        
        Now, we define a new notion of circulation in the above graph
        as follows: a function $f: E \rightarrow \mathbb{R}$ is a
        \emph{circulation} if the following conditions are satisfied:
        \begin{itemize}
          \item $f(v,w) = -f(w,v)$ for all $(v,w) \in E$,
          \item $f(e) \leq u(e)$ for all $e \in E$, and
          \item $\sum_{(v,w) \in E} f(v,w) = 0$.
        \end{itemize}
        
        Let us note that the new formulation is equivalent to the old
        one. For each old edge $(v,w)$, we have that $f(v,w) \geq l(v,w)$
        iff $-f(v,w) \leq -l(v,w)$. So the new opposite edges enforces
        the lower bound of the flow. 
        
        If $(v,w)$ is one of the edge in the graph of the previous
        formulation. The cost this edge incurs is $c(v,w)f(v,w)$.
        In the new formulation, there is also a flow of value $-f(v,w)$ 
        going across $(w,v)$. So the cost incurs by the edge $(v,w)$
        is actually $$c(v,w)f(v,w) + c(w,v)f(w,v)
        = c(v,w)f(v,w) + (-c(v,w)(-f(v,w)) = 2c(v,w)f(v,w).$$
        Hence, the cost of the circulation in the new formulation
        is two times that in the old formulation.

      \item 
        \begin{lemma}
          Given one instance of the min-cost circulation
          problem, we can find whether there is a \emph{feasible}
          solution by running a single max flow.
        \end{lemma}
        
        \begin{proof}
          We revert the problem back to the version with demand.
          Let $(G, u, l, c)$ be an instance of the min-cost circulation
          problem. For vertex $v$, define its \emph{demand} $b(v)$
          as $\sum_{(v,w) \in E} l(v,w) - \sum_{(w,v) \in E} l(w,v).$
          We construct a new instance $(G', u')$ for max flow problem 
          as follows:
          \begin{itemize}
            \item $G'$ is obtained by adding a source vertex $s$ and
              a sink vertex $t$ to $G$.
            \item For any edge $(v,w)$ in $G$ with positive capacity,
              we set $u'(v,w) = u(v,w) - l(v,w)$.
            \item For any vertex $v$ in $G'$ with $b(v) > 0$,
              we create an edge $(v,t)$ with $u'(v,t) = b(v)$.
            \item For any vertex $v$ in $G'$ with $b(v) < 0$,
              we create an edge $(s,v)$ with $u'(s,v) = -b(v)$.
          \end{itemize}
          
          Note that it is the case that $\sum_{v \in V} b(v) = 0.$
          Let $B = \sum_{b(v) > 0} b(v) = -\sum_{b(v) < 0} b(v).$
          
          It should be clear a max flow of value $B$ yields a 
          feasible circulation, and vice versa.
        \end{proof}
    \end{itemize}
  % section  (end)
  
  \section{Optimality Conditions} % (fold)
  \label{sec:optimality_conditions}
  
  \begin{itemize}
    \item From now on, let $G = (V,E)$ be a graph such that each directed
      edge has a backward edge. Let $(G, u, c)$ be a flow network, and $f$ be 
      a circulation in it.
    
    \item The \emph{residual network} $(G_f, u_f, c)$ is given by:
      \begin{itemize}
        \item $G_f = (V, E_f)$ where $E_f$ is the set of edges such
          that $u(e) - f(e) > 0$.
        \item $u_f(e) = u(e) - f(e)$ for all $e \in E_f$, and
      \end{itemize}
      
    \item A function $p: V \rightarrow \mathbb{R}$ is called a \emph{potential function}
      or a \emph{price function}.
      
    \item Let $p$ be a price function, and $c$ be a cost function in network.
      The \emph{reduced cost function} $c^p$ is defined as: $c^p(v,w) = c(v,w) + c(v) - c(w).$
      
    \item If $\Gamma$ is a cycle in $G$ and $c$ is a cost function, 
      let $c(\Gamma) = \sum_{e \in \Gamma} c(e).$
      
    \item 
      \begin{claim}
        For any cycle $\Gamma$, we have $c(\Gamma) = c^p(\Gamma).$
      \end{claim}
      \begin{proof}
        Let $\Gamma = (v_1, v_2), (v_2, v_3), \dotsc, (v_k, v_1).$
        We have
        \begin{align*}
          c^p(\Gamma) &= c^p(v_1, v_2) + c^p(v_2, v_3) + \dotsb c(v_k, v_1)\\
          &= [c(v_1, v_2) + p(v_1) - p(v_2)] + [c(v_2, v_3) + p(v_2) - p(v_3)] + \dotsb + [c(v_k, v_1) + p(v_k) - p(v_1)]\\
          &= [c(v_1, v_2) + c(v_2, v_3) + \dotsb + c(v_k, v_1)] + [p(v_1) - p(v_2) + p(v_2) - p(v_3) + \dotsb + p(v_k) - p(v_1)]\\
          &= c(v_1, v_2) + c(v_2, v_3) + \dotsb + c(v_k, v_1) = c(\Gamma).
        \end{align*}
      \end{proof}
      
    \item If $c$ is a cost function and $f$ a circulation, let $c \cdot f = \sum_{e \in E} c(e) f(e).$
    
    \item
      \begin{claim} \label{cycle-invariant}
        For any cost function $c$ and price function $p$ and circulation $f$, we have
        $$c \cdot f = c^p \cdot f.$$
      \end{claim}
      \begin{proof}
        \begin{align*}
          c^p \cdot f 
          &= \sum_{(v,w) \in E} c^p(v,w) f(v,w) \\
          &= \sum_{(v,w) \in E} (c(v,w) + p(v) - p(w))f(v,w) \\
          &= \sum_{(v,w) \in E} c(v,w) + \sum_{(v,w) \in E} p(v) f(v,w) - \sum_{(v,w) \in E} p(w) f(v,w)\\
          &= c \cdot f + \sum_{(v,w) \in E} p(v) f(v,w) - \sum_{(v,w) \in E} p(v) f(w,v) \\
          &= c \cdot f + \sum_{(v,w) \in E} p(v) (f(v,w) - f(w,v))\\
          &= c \cdot f + \sum_{v \in V} p(v) \Big( \sum_{(v,w) \in E} f(v,w) - \sum_{(w,v) \in E} f(w,v) \Big)\\
          &= c \cdot f.
        \end{align*}
        The last equality is true because $\sum_{(v,w) \in E} f(v,w) = \sum_{(w,v) \in E} f(w,v) = 0$
        because of flow conservation.
      \end{proof}
    
    \item
      \begin{theorem} \label{optimality}
        The following statements are equivalent:
        \renewcommand{\labelenumi}{(\alph{enumi})}
        \begin{enumerate}
          \item $f$ is a minimum-cost circulation.
          \item There is no negative-cost cycle in the residual network $G_f$.
          \item There exists a price function $p$ such that $c^p(e) \geq 0$ for all $e \in E_f$
        \end{enumerate}
      \end{theorem}
      \begin{proof}
        ($\neg(b) \rightarrow \neg(a)$) Augment along the negative-cost cycle gives
          a circulation with the lower cost. \medskip
        
        ($\neg(a) \rightarrow \neg(b)$) Let $f$ be a feasible circulation that is not of minimum-cost.
          Let $f^*$ be feasible circulation with minimum-cost. We have that $f* - f$ is a circulation
          that is feasible in $G_f$ (because $f*(v,w) - f(v,w) \leq u(v,w) - f(v,w)$).
          Since $f^*$ has lower cost than $f$, we have that $f^* -f$ has negative cost.
          We can decompose $f^* - f$ into cycles, and at least one cycle must be of negative cost.
          \medskip
          
        ($(b) \rightarrow (c)$) Start with $G_f$. Construct a new vertex $s$ 
          and connect $s$ to every vertex $v$ with $c(s,v) = 0.$ 
          Since there is no negative cycle, the shortest path distance is well-defined.
          Now, define $p(s) = 0$, and $p(v) =$ shortest path distance from $s$ to $v$, 
          taking $c$ is the length of each edge.
          
          By property of shortest path distance, we have that $p(w) \leq p(v) + c(v,w)$
          for all edge $(v,w) \in E_f$. Hence, $c^p(v,w) = c(v,w) + p(v) - p(w) \geq 0.$ \medskip
          
        ($(c) \rightarrow (b)$) By Claim~\ref{cycle-invariant}, $c(\Gamma) = c^p(\Gamma)$
         for any cycle $\Gamma$. Since $c^p(e) \geq 0$ for all $e \in E$, we have
         that all cycle has positive cost.
      \end{proof}
  \end{itemize}
  % section optimality_criteria (end)
  
  \section{Cycle Canceling Algorithm} % (fold)
  \label{sec:cycle_canceling_algorithm}
    \begin{itemize}
      \item The above theorem gives a simple algorithm for finding min-cost flow.
        Just find a negative-cost cycle and augment along it. Repeat until
        there are no negative-cost cycle.
        
        This algorithm is called Klein's algorithm.
      
      \item
        \begin{theorem}
          Let $(G, u, c)$ be a network with integer capacity and integer cost.
          Let $U = \max_{e \in E} \{ u(e) \}$ and $C = \max_{e \in E} \{ |c(e)| \}.$
          Then, Klein's algorithm runs in $O(m^2nUC).$
        \end{theorem}
        
        \begin{proof}
          We can find a negative cycle in $O(mn)$ using Bellman--Ford algorithm.
          The minimum-cost is bounded below by $-mUC.$ Each negative cycle
          decreases the cost of the circulation by at most $-1$. So $mUC$ cycles
          suffice.
        \end{proof}
    \end{itemize}
  % section cycle_canceling_algorithm (end)
  
  \section{Minimum Mean-Cost Cycle Canceling Algorithm} % (fold)
  \label{sec:minimum_mean_cost_cycle_canceling_algorithm}
    \begin{itemize}
      \item For any cycle $\Gamma$, we define its \emph{mean cost}
        $\mu(\Gamma) = c(\Gamma) / |\Gamma|.$
        
      \item Let $\mu^*$ be the minimum mean cost of all cycles.
        In other words, $\mu^* = \min_{\Gamma} \mu(\Gamma)$
        
      \item Instead of picking any negative-cost cycle,
        this algorithm picks the one with cost $\mu^*$ and augment
        along that cycle.
        
      Let $C = \max_{e \in E} \{ |c(E)| \}$. It can be shown
        that only $O(m^2 n\log n)$ augmentation suffices. We will
        not be showing why this is true.
        
      \item An interesting problem is how to find the minimum mean cost cycle.
        We will present two algorithms: one with complexity $O(mn \log (n^2C))$
        and the other with complexity $O(mn).$
        
      \item In the $O(mn \log(n^2C))$ algorithm, we assume that the edges
        have integer costs. The idea is to binary search for $\mu^*$.
        We know that $\mu^* \in [-C, C]$ so we can set the search interval
        accordingly.
        
        Suppose that we guess $\mu^* = a.$ We will subtract $a$ from
        all the edge cost of the graph. Notice that for any cycle,
        its mean cost is reduced by $a.$ We have the following case.
        \begin{itemize}
          \item If $\mu^* \geq a$, then all the cycles have positive mean
            cost, and thus they have positive cost.
          \item If $\mu^* = a$, then there exists some cycle with
            zero cost, but none of the cycles have negative cost.
          \item If $\mu^* < a$, then some cycles have negative cost.
        \end{itemize}
        We can check whether there is a negative-cost cycle by running
        Bellman--Ford algorithm, which takes $O(mn)$ time.
        
        How many iterations do we need? Since the denominator
        of $\mu^*$ is an integer from $1$ to $n$, we have that
        two candidate values for $\mu^*$ cannot differ by
        more than $1/n^2$. So, when the interval is smaller
        than $1/n^2$, we can be sure that only one candidate
        is inside. Hence, we need at most $O(\log(n^2C))$
        iterations to shrink the interval to this size.
        
        Once the interval is small enough, we can find the value by 
        searching though all possible denominators, which takes 
        $O(n)$ time. Overall, the algorithm takes $O(mn \log (n^2C))$
        time.
      
      \item Once we find $\mu^*$, how can we find a cycle
        with minimum mean cost?
        
        We first subtract $\mu^*$ form all edge cost.
        Since there is no negative-cost cycle,
        by Theorem~\ref{optimality}
        there exists a price function $p$ such that
        $c^p(e) \geq 0$ for all $e \in E.$
        This function can be founded by taking
        $p(v) = $ the shortest path distance
        from a new vertex $s$ with an edge of cost $0$
        pointing to every vertex.
        
        Now, the cycle with minimum mean cost 
        turns into a cycle with zero cost.
        Since all edges have non-negative cost,
        the cycle has to consists only of edges
        with zero cost. We can locate all those edges
        and find a cycle formed by them.
        
      \item The $O(mn)$ algorithm is rather involved. We start
        with a definition.
        \begin{definition}
          Let $v$ be a vertex and $k$ be a non-negative integer.
          Define $d_k(v)$ to be the cost of a walk
          containing exactly $k$ edges ending at $v$
          with the least possible cost.
        \end{definition}
        Then, $\mu^*$ can be characterized as follows:
        \begin{lemma} \label{min-mean-cost-characterization}
          $$\mu^* = \min_{v \in V} \max_{0 \leq k \leq n-1} \Bigg\{ \frac{d_n(v) - d_k(v)}{n - k} \Bigg\}$$
        \end{lemma}
        \begin{proof}
          Let $\alpha$ denote the RHS of the equation. 
          Notice that if we subtract $a$ from all the edge cost
          then $\alpha$ is reduced by $a$ as well.
          Hence, we  can show that $\mu^* = \alpha$ by working 
          in a graph where $\mu^*$ is subtracted from the cost of 
          all edge and show that $\alpha = 0$ in this graph.
          Note that, in this graph, all cycles have positive cost,
          and there is at least one cycle with zero cost. \medskip
          
          ($\alpha \geq 0$) Let $$\alpha(v) = 
            \max_{0 \leq k \leq n-1} \frac{d_n(v) - d_k(v)}{n-k}.$$
            Let $v$ be the vertex where $\alpha(v)$ is minimum.
            Let $p$ be the walk of length $n$ ending at $v$ with
            minimum cost, i.e., $c(p) = d_n(v).$ Since $p$ has
            length $n$, it must contain a cycle. Thus, we
            can compose $p$ into a cycle $\pi$ and a path $\tau$
            leading to $v$. Let $j$ be the number of edges in $\tau$.
            We must have that $c(\tau) = d_j(v),$ otherwise $p$
            would not have been the shortest walk. So,
            $$\alpha(v) = \max_{0 \leq k \leq n-1} \frac{d_n(v) - d_k(v)}{n-k} 
              \geq \frac{d_n(v) - d_j(v)}{n-j} = \frac{c(\tau)}{n-j} \geq 0.$$
            \medskip
            
        ($\alpha \leq 0$) Let $\Gamma$ be a cycle with cost $0$,
          and let $v$ be a vertex in the cycle. 
          
          Consider the sequence $d_0(v), d_1(v), d_2(v), \dotsc$. 
          We claim that there exists $r$ such that $d_r(v)$ is minimum 
          and $r < n$. That is, Suppose that $r \geq n.$
          Since the walk that achieves $d_r(v)$ has at least $n$ edges,
          it must contain a cycle. We can take the cycle out without 
          increasing $d_r(v)$ until there $r < n$.
          
          Let $\eta$ be a walk from $v$ that proceeds along the cycle
          for $n-r$ hops, and let the last vertex in the hop be $w$.
          Let $\tau$ be the walk from $w$ along the cycle to $v$.
          We have that $c(\tau) + c(\eta) = 0.$ So, for any $k$,
          we have that $$d_k(w) = d_v(w) + c(\tau) + c(\eta) \geq d_r(v) + c(\eta)
           \geq d_n(w).$$
          The inequality $d_v(w) + d(\tau) \geq d_r(v)$ comes from
          the fact that $d_v(w) + d(\eta)$ is the length of a path
          to $v$. Moreover, $d_r(v) + c(\eta) \geq d_n(v)$ because
          $d_r(v) + c(\eta)$ is the length of a path of length $n$
          to $w$. Hence, $d_n(w) - d_k(w) \leq 0.$ So, $\alpha \leq 0.$ 
        \end{proof}
      
      \item The $O(mn)$ algorithm computes $d_k(v)$ for all $0 \leq k \leq n$
        and $v \in V$, which can be done by dynamic programming in $O(mn).$
        It then compute $\mu^*$ according to the formula given 
        by Lemma~\ref{min-mean-cost-characterization}, and
        then uses $\mu^*$ to find the minimum mean cost cycle.
    \end{itemize}
  % section minimum_mean_cost_cycle_canceling_algorithm (end)
  
  \section{Cost Scaling Algorithm} % (fold)
  \label{sec:cost_scaling_algorithm}
    \begin{itemize}
      \item A potential function is said to be \emph{$\epsilon$-optimal}
        if $c^p(e) \geq -\epsilon$ for all $e \in E.$
        
      \item Let $f$ be a circulation. Define $\epsilon(f)$ to
        be the minimal $\epsilon$ such that there exists a potential
        function that is $\epsilon$-optimal in $G_f$.
        
      \item Let $\mu^*(f)$ be the mean cost of the cycle in $G_f$
        with the minimum cost.
        
      \item
        \begin{lemma}
          $$\mu^*(f) = -\epsilon(f)$$
        \end{lemma}
        \begin{proof}
          ($\mu^*(f) \leq - \epsilon(f)$) Subtract $\mu^*(f)$ from
            all edge cost in the residual graph. Add a new vertex $s$
            and add edge $(s,v)$ with cost $0$ to all $v \in V.$ 
            Define $p(v)$ to be the shortest path distance from $s$ to $v$.
            We know that $p(w) \leq p(v) + c(v,w) - \mu^*(f)$ for all edges
            in $G_f.$ Hence, $c(u,v) + p(v) - p(w) \geq \mu^*(f).$
            Therefore, $p$ is $-\mu^*(f)$-optimal, which means that
            $\epsilon(f) \leq -\mu^*(f)$ or $\mu^*(f) \leq -\epsilon(f).$
            \medskip

          ($\mu^*(f) \geq -\epsilon(f)$) Let $p$ be a position function
            that is $\epsilon(f)$-optimal. Take any cycle $\Gamma$ 
            with the minimum negative mean cost in $G_f$. 
            We have that $|\Gamma| \mu^*(f) = c(\Gamma) = c^p(\Gamma) \geq -|\Gamma|\epsilon(f)$.
            Therefore, $\mu^*(f) \geq -\epsilon(f).$
        \end{proof}
      \item 
        \begin{lemma}
          In a network with integer cost, if $\epsilon(f) < 1/n$, then
          $f$ is the min-cost circulation. 
        \end{lemma}
        \begin{proof}
          Let $p$ be a potential function that is $\epsilon(f)$ optimal.
          Take any cycle $\Gamma$ of length at most $n$ in $G_f.$ We have that
          $c(\Gamma) = c^p(\Gamma) \geq |\Gamma|\mu^*(f) = -|\Gamma|\epsilon(f) > -|\Gamma|/n = 1.$
          Since the cost of the cycle is integral, we have that $c(\Gamma) \geq 0.$
          
          Now, for any cycle of length more than $n$, we can break it to
          a number of cycles of length at most $n$. So, its cost is greater
          than or equal to 0 as well. In conclusion, there is no negative-cost
          cycle, and $f$ is optimal.
        \end{proof}
      
      \item The idea of cost-scaling algorithm is that, when we start with $f = 0$,
        we have that $\epsilon(f) \leq C.$ We will then do something to the flow
        so that $\epsilon(f)$ is reduced by a half until it is less than $1/n$,
        which at thas point we have an optimal flow. Hence, we will need $\log(nC)$
        iterations.
        
      \item A function $f: E \rightarrow \mathbb{R}$ is said to be a \emph{preflow}
        if it satisfies the following condition:
        \begin{itemize}
          \item $f(v,w) = -f(w,v)$, and
          \item $f(v,w) \leq u(v,w)$
        \end{itemize}
        for all $(v,w) \in E.$
        
      \item Let $f$ be a preflow. We define the \emph{excess} of vertex $v$, denoted
        by $e^f(v)$ as $$e^f(v) = \sum_{(v,w) \in E} f(v,w).$$
        Intuitively, a vertex with positive excess has left-over flow to send out. 
        One with negative excess needs flow to come in.
        
      \item Obviously, if all vertices' excesses are zero, then the preflow
        is a circulation.
        
      \item We now describe an algorithm that takes in 
        \begin{itemize}
          \item a circulation $f'$, and
          \item a potential function $p'$ that is $2\epsilon$-optimal in $G_{f'}$
        \end{itemize}
        and produces a
        \begin{itemize}
          \item a circulation $f$, and
          \item a potential function $p$ that is $\epsilon$-optimal in $G_{f}$.
        \end{itemize}
        
        We first set $f = f'$ and $p = p'.$ We then make $p$ $0$-optimal
        by saturating any edges with $c^p(e) < 0$, thereby removing
        them from $G_f.$ However, this makes $f$ a preflow, not a circulation.
        
        The rest of the process is to make $f$ a circulation again,
        while maintaining $\epsilon$-optimality of $p$. This is done
        by a push/relabel type of algorithm as follows:
        \begin{itemize}
          \item {\bf Push:} If there is an edge $(v,w)$ such that
            $e^f(v) > 0$ and $u^f(v,w) > 0$ and $c^p(v,w) < 0$,
            we push $\min\{ e^f(v), u^f(v,w) \}$ through the edge.
          \item {\bf Relabel:} For any vertex $v$ with no edges
            that flow can be pushed trough, we set 
            $$p(v) = \max_{(v,w) \in E} \{ p(w) - c(v,w) - \epsilon. \}$$
            Note that setting $p(v)$ to this value makes $c^p(v,w) = p(v) - p(w) + c(v,w) \geq -\epsilon$
            for all edges going out from $v$.
        \end{itemize}
        
        The above description can be summarized into the following pseudocode.
        \begin{codebox}
          \Procname{$\proc{Find-$\epsilon$-Opt-Circ}(f', p', \epsilon)$}
          \li Set $f \gets f'$ and $p \gets p'.$
          \li For all $e \in E_f$ such that $c^p(e) < 0$, set $f(e) = u(e).$
          \li \While there exists $v$ such that $e^f(v) > 0$
          \li   \Do
                  \If there exists $(v,w)$ such that $u^f(u,v) > 0$ and $c^p(u,v) < 0$
          \li       \Then
                      Push flow $\min\{ u^f(u,v), e^f(v) \}$ through $(u,v).$
          \li       \Else
                      Set $p(v) = \max_{(v,w) \in E} \{ p(w) - c(v,w) - \epsilon\}$.
                    \End
                \End
        \end{codebox}
        
        Using the FIFO implementation of Push/Relabel algorithm, the routine 
        $\proc{Find-$\epsilon$-Opt-Circ}(f', p', \epsilon)$ runs in $O(n^3)$ time.
      
      \item Hence, the cost scaling algorithm takes $O(n^3 \log(nC))$ time.
    \end{itemize}
  % section cost_scaling_algorithm (end)
  
  \section{Capacity Scaling Algorithm} % (fold)
  \label{sec:capacity_scaling_algorithm}
    \begin{itemize}
      \item The capacity scaling algorithm tries to maintain a preflow
        that is $0$-optimal on subgraphs with residual capacity more than
        $\Delta$. It then divides $\Delta$ until $\Delta < 1$, at which point
        the preflow becomes the optimal circulation.
        
        Initially, we set $f = 0$, $p = 0$, and $\Delta = U = \max_{e \in E} u(e).$
       
        Let $$A^f(\Delta) = \{ e \in E : u^f(e) \geq \Delta \}.$$
        The algorithm looks for any edge $e \in A^f(\Delta)$ such 
        that $c^p(e) < 0.$ It then pushes $\Delta$ amount of flow through
        the edge. This creates a preflow where some nodes have positive
        excesses and some have negative excesses. Let
        \begin{itemize}
          \item $S^f(\Delta) = \{ v \in V : e^f(v) \geq \Delta \}$, and
          \item $T^f(\Delta) = \{ v \in V : e^f(v) \leq -\Delta \}.$
        \end{itemize}
        The algorithm then tries to move $\Delta$ unit of flow from
        a vertex in $S^f(\Delta)$ to one in $T^f(\Delta)$ until
        $S^f(\Delta) = \emptyset$ or $T^f(\Delta) = \emptyset,$
        while maintaining $0$-optimality of $p$. It then divides $\Delta$
        by $2$ and proceed with the above process again.
        
        Note that it must be possible to move flow from a 
        vertex to any other vertex in other for the algorithm to
        work. This can be made possible by adding edges with infinite
        capacity but with high cost to the graph.
      \item 
        \begin{lemma}
          In a network with integer capacity, when $\Delta < 1$, 
          then $f$ is an optimal circulation.
        \end{lemma}
        \begin{proof}
          After the algorithm finishes $S^f(\Delta) = \{ v \in V : e^f(v) \geq 1 \} = \emptyset$
          and $T^f(\Delta) = \{ v \in V : e^f(v) \leq -1 \} = \emptyset$.
          This implies that all $e^f(v) = 0$. So $f$ is a circulation.
          
          When $\Delta < 1$, then $A^f(\Delta)$ is the whole graph $G_f$.
          Since $p$ is $0$-optimal, we have that $G_f$ has no negative-cost
          cycle.
        \end{proof}
      
      \item How do we send $\Delta$ unit of flow from a vertex in $S^f(\Delta)$
        to a vertex in $T^f(\Delta)$ while maintaining $0$-optimality of $p$?
        We do this as follows:
        \begin{itemize}
          \item Find a vertex $s \in S^f(\Delta)$.
          \item Compute $\hat{p}(v) = $ shortest path distance from $s$ to $v$
            using $c^p(e)$ as length of edge $e$.
            This can be done because $p$ is $0$-optimal.
          \item Compute a shortest path from $s$ to some vertex in $T^f(e)$.
            Push $\Delta$ unit of flow along the shortest path.
          \item Update $p(v) = p(v) + \hat{p}(v).$
        \end{itemize}
        This procedure amounts to running Dijkstra's algorithm one time. 
        So it takes $O(m + n\log n)$ time.
      
      \item It can be shown that the total amount of excess after saturating
        edges is at most $2\Delta(n+m).$ Hence, $O(m+n)$ flow pushing is enough
        before we halve $\Delta.$ Therefore, the whole algorithm takes
        $O((m+n)(m + n\log n)\log U) = O(m^2 \log U)$ time.
    \end{itemize}
  % section capacity_scaling_algorithm (end)
\end{document}