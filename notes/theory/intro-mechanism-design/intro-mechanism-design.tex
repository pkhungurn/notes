\documentclass[10pt]{article}
\usepackage{fullpage}
\usepackage{amsmath}
\usepackage[amsthm, thmmarks]{ntheorem}
\usepackage{amssymb}

\newtheorem{lemma}{Lemma}[section]
\newtheorem{theorem}[lemma]{Theorem}
\newtheorem{definition}[lemma]{Definition}
\newtheorem{proposition}[lemma]{Proposition}
\newtheorem{claim}[lemma]{Claim}

\title{Introduction to Mechanism Design}
\author{Pramook Khungurn}

\begin{document}
	\maketitle
	
	\section{Social Choice Theory}
	
		\begin{itemize}
			
			\item Let $A$ be a finite set of \emph{alternatives}. 
			
			\item A \emph{preference} $\prec$ on $A$ is an anti-symmetric, transitive,
			and complete binary relation on $A$. That is, a preference
			is a \emph{total order} on $A.$
			
			\item Let $L$ denote the set of all preferences on $A$.
			
			\item We can think of a set of $n$ voters, each with his own preferences
			over the alternatives. We typically write $\prec_i$ as the preference
			of Voter $i$. 
			
			A \emph{preference profile} is a tuple containing preferences of all voters.
			We typically use $\pi$ to denote a profile, and we write $\pi = (\prec_1, \prec_2, \dotsc, \prec_n)$
			to show the components of $\pi$.
			
			\item The set of preferences of all voters is precisely $L^n.$
			
			\item A function $F: L^n \rightarrow L$ is called a \emph{social welfare} function.
			
			A function $f: L^n \rightarrow A$ is called a \emph{social choice} function.
			
		\end{itemize}
	
	\subsection{Arrow's Impossibility Theorem}
	
		\begin{itemize}
			
			\item A social welfare function $F$ is said to satisfies \emph{unanimity} if it
			satisfies the following condition:
			
			\begin{quote}
				Let $a$ and $b$ be any two alternatives in $A$. \\
				Let $\pi = (\prec_1, \prec_2, \dotsc, \prec_n)$ be any profile. \\
				Let $\prec = F(\pi).$ \\
				\\
				If $a \prec_i b$ for all $i$, then $a \prec b$.
			\end{quote}
			
			That is, if all voters prefer $b$ over $a$, then the group prefers $b$ over $a$.
			
			\item A social welfare function $F$ is said to satisfies \emph{independence of irrelevant alternatives} (IIA)
			if it satisfies the following condition:
			
			\begin{quote}
				Let $a$ and $b$ be any two alternatives in $A.$ \\
				Let $\pi = (\prec_1, \prec_2, \dotsc, \prec_n)$ and $\pi' = (\prec_1', \prec_2', \dotsc, \prec_n')$ be any two profiles. \\
				Let $\prec = F(\pi)$ and $\prec' = F(\pi').$ \\
				\\
				If $a \prec_i b \iff a \prec_i' b$, then $a \prec b \iff a \prec' b.$
			\end{quote}
			
			That is, the preference between $a$ and $b$ of the group depends only on the preferences between $a$ and $b$ of
			the voters. Information regarding other alternatives are irrelevant.
			
			\item Voter $i$ is said to be a \emph{dictator} of social welfare function $F$
			if $F(\prec_1, \prec_2, \dotsc, \prec_n) = \prec_i$ for all profiles.
			
			If $F$ has a dictator, we say that $F$ is a \emph{dictatorship}.
			
			\item 
			\begin{theorem}[Arrow's Impossibility Theorem]
				Every social welfare function on set $A$ with more than two elements that satisfies unanimity and
				IIA is a dictatorship.
			\end{theorem}
			
			We first prove the following claim.
			
			\begin{claim}
				Let $F$ be a social welfare function on set $A$ with at least 3 elements that satisfies unanimity and IIA. 
				Then, $F$ satisfies the following \emph{neutrality} condition:
				
				\begin{quote}
					Let $a, b, \alpha, \beta$ be any four alternatives in $A$ such that $a \neq b$ and $\alpha \neq \beta$ \\
					Let $\pi = (\prec_1, \prec_2, \dotsc, \prec_n)$ and $\pi' = (\prec_1', \prec_2', \dotsc, \prec_n')$ be any two profiles. \\
					Let $\prec = F(\pi)$ and $\prec' = F(\pi').$ \\
					\\
					If $a \prec_i b \iff \alpha \prec_i' \beta$, then $a \prec b \iff \alpha \prec' \beta.$
				\end{quote}
				
				That is, the decision is made the same way for every two alternatives.
			\end{claim}

			\begin{proof} (Claim) WLOG, we can assume that, in profile $\pi$, the group prefers $b$ over $a$. 
				That is, $a \prec b$.
				
				Our strategy is to construct a new profile $\pi^* = (\prec^*_1, \prec^*_2, \dotsc, \prec^*_n)$
				with the following properties:
				\begin{enumerate}
					\item the preferences between $a$ and $b$ are the same as those in $\pi$, 
					\item the preferences between $\alpha$ and $\beta$ are the same as those in $\pi'$, and
					\item for all $i$, $\alpha \prec^*_i a$ and $b \prec^*_I \beta$.
				\end{enumerate}
				
				Assuming that we are successful, let $\prec^* = F(\pi^*).$ We have that:

				\begin{itemize}
					\item By Property 1 and IIA, it is the case that $a \prec^* b$.
					\item By Property 3 and unanimity, it is the case that $\alpha \prec^* a$ and $b \prec^* \beta$.
					\item Thus, $\alpha \prec^* a \prec^* b \prec^* \beta$. Therefore, $\alpha \prec^* \beta$.
					\item Lastly, by Property 2 and IIA, we have that $\alpha \prec' \beta$ as well.
				\end{itemize}  
				
				Hence, the claim is true if we can establishes the existence of $\pi^*.$
				
				To construct $\pi^*$, we first assume that $\alpha \neq b$ and $\beta \neq a$, and we will deal with
				other cases later. We now specify each $\prec^*_i$. We construct $\prec^*_i$ by first setting it to $\prec_i$.
				Then, we move $\alpha$ and $\beta$ so that $\alpha$ is just before $a$ (if $\alpha \neq a$) and 
				$\beta$ is just after $b$ (if $\beta \neq b$). This is done in such a way that preserves the relative
				order between $a$ and $b$ and that between $\alpha$ and $\beta$. More precisely,
				
				\begin{itemize}
					\item if $a \prec_i b$ and $\alpha \prec'_i \beta$, then we move $\alpha$ and $\beta$ so that
						$\alpha \prec^*_i a \prec^*_i b \prec^*_i \beta$;
					\item if $b \prec_i a$ and $\beta \prec'_i \alpha$, then we move $\alpha$ and $\beta$ so that
						$b \prec^*_i \beta \prec^*_i \alpha \prec^*_i a$.
				\end{itemize}
				
				It can be seen that the new profile $\pi^*$ satisfies all the three properties.
				
				Now, we turn to cases where $\alpha = b$ or $\beta = a$. Let $c$ be an alternative
				which is different from $a$ and $b$. From what we have proved so far, we know that decisions 
				about $(a,b)$ are made in the same way as those about $(a,c)$. These
				decisions, in turn, are made the same way as those about $(b,c)$,
				and then $(b,a)$, and then $(c,a)$. We have covered all the cases.
			\end{proof}
			
			\begin{proof} (Arrow's impossibility theorem)
				Let $a$ and $b$ be two alternatives. We define a set of profiles
				$\pi^0$, $\pi^1$, $\pi^2$, $\dotsc$, $\pi^n$ as follows.
				In $\pi^i$, the first $i$ voters prefers $b$ over $a$, but
				the remaining voters prefer $a$ over $b.$ (That is, $a \prec_j b$
				if $j \leq i$, and $b \prec_j a$ if $j > i$.)
				
				Let $\prec^i = F(\pi^i)$ for all $i$. By unanimity, we have
				that $b \prec^0 a$, but $a \prec^n b$. Hence, there must be
				a profile $i^*$ such that $b \prec^{i^*-1} a$, but $a \prec^{i^*} b$.
				We call voter $i^*$ the \emph{pivotal voter}.
				
				We now show that $i^*$ is a dictator. Let $\pi = (\prec_1, \prec_2, \dotsc, \prec_n)$
				be an arbitrary profile, and let $\prec = F(\pi).$ We shall show that,
				for any $c, d \in A$, if $c \prec_{i^*} d$, then $c \prec d$.
				
				To do so, let $e$ be an alternative different from $c$ and $d$.
				By IIA, moving $e$ around the preferences of any voter without
				changing the relative order between $c$ and $d$ does not change
				the group's preference between $c$ and $d$. We then conduct the
				following moves:
				
				\begin{itemize}
					\item for $j < i^*$, we move $e$ so that it is the least preferred alternative in
					$\prec_j$.
					\item for $j > i^*$, we move $e$ so that it is the most preferred alternative in
					$\prec_j$.
					\item we move $e$ so that $c \prec_i e \prec_i d$.
				\end{itemize}
				
				Note that the relative orders of $c$ and $e$ are the same as those
				of $a$ and $b$ in $\pi^{i^*-1}$. By neutrality, we can conclude that
				$c \prec e$. Note also that the relative orders of $e$ and $d$ are
				the same as those of $b$ and $a$ in $\pi^{i^*}$. Hence, $e \prec d$.
				So, we have that $c \prec e \prec d$.
			\end{proof}
			
		\end{itemize}
		
\end{document}