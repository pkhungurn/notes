\documentclass[10pt]{article}
\usepackage{fullpage}
\usepackage{amsmath}
\usepackage{amsthm}
\usepackage{amssymb}

\newtheorem{lemma}{Lemma}[section]
\newtheorem{theorem}[lemma]{Theorem}
\newtheorem{definition}[lemma]{Definition}

\title{Notes on Maximum Flow Algorithms}
\author{Pramook Khungurn}

\begin{document}
  \maketitle
  \section{Networks, Flows, and Cuts} % (fold)
  \label{sec:networks_and_flows}
    \begin{itemize}
      \item A \emph{network} is composed of
        \begin{itemize}
          \item a set of vertices $V$;
          \item a vertex $s$ called the \emph{source};
          \item a vertex $t$ called the \emph{sink};
          \item a function $c: V^2 \rightarrow \mathbb{R}^{+} \cup \{ 0 \}$ called
            the \emph{capacity}.
        \end{itemize}
      \item We can draw a network as a directed graph by drawing a
        directed edge from vertex $u$ to $v$ if $c(u,v) > 0$.
      \item A \emph{flow} in a network $G$ is a function $f: E \rightarrow \mathbb{R}$ 
        with the following properties:
        \begin{itemize}
          \item {\bf Skew Symmetry:} for all $u,v \in V$, we have $$f(u,v) = -f(v,u).$$
          \item {\bf Conservation:} for all vertex $u$ not equal to $s$ or $t$, we have $$\sum_{v \in V} f(u,v) = 0.$$
          \item {\bf Capacity Constraint:} for all $u, v \in V$, we have $$f(u,v) \leq c(u,v).$$
        \end{itemize}
      \item If $f(u,v) > 0$, we say that there is a flow out of $u$ by the value $f(u,v)$ to $v$.
        if $f(u,v) < 0$, we say that there is a flow of value $f(u,v)$ from $v$ to $u$.
      \item An \emph{$s,t$-cut} of a network $G$ is a partition of $V$ 
        into two sets $A$ and $B$ where $s \in A$ and $b \in B$.
      \item The \emph{capacity} of the cut $A,B$ is $$c(A,B) = \sum_{u \in A, v \in B} c(u,v).$$
      \item Let $f$ be a flow. The \emph{value of the flow across the cut $A,B$} is given by
        $$f(A,B) = \sum_{u \in A, v \in B} f(u,v).$$
      \item Obviously, $f(A,B) \leq c(A,B).$
      \item The \emph{value} of the flow $f$, denoted by $|f|$ is 
        defined as $|f| = f( \{ s \}, V - \{ s \} ).$
        In other words, it is the flow out of vertex $s$.
      \item We can prove by induction that $|f| = f(A,B)$ for any cut $A, B$.
      \item As a result, $|f| \leq c(A,B)$ for any cut $A, B$.
      \item Let $G$ be a network and $f$ is a flow. 
        Define the \emph{residual capacity} $r: V^2 \rightarrow \mathbb{R}^+ \cup \{ 0 \}$ as follows:
        for all $u,v \in V$, $$r(u,v) = c(u,v) - f(u,v).$$
        We also define the \emph{residual network} $G_f$, which is $G$
        with its capacity function replaced by $r$.
      \item The following lemma says that a flow in the residual network
        gives a new flow in the original network.
        
        \begin{lemma}
          Let $G$ be a network, $f$ a flow in $G$, and
          $G_f$ the residual network.
          \renewcommand{\labelenumi}{(\alph{enumi})}
          \begin{enumerate}
            \item The function $f'$ is a flow in $G_f$ if and only if $f + f'$ is a flow in $G$.
            \item The value function is additive: $|f + f'| = |f| + |f'|$ and $|f - f'| = |f| - |f'|.$
          \end{enumerate}
        \end{lemma}
        
        \begin{proof}
          Let $f'$ be a flow in $G_f$. We have that $f'$ satisfies skew symmetry and conservation
          if and only if $f + f'$ satisfies both. Now, we have that
          \begin{align*}
            f'(u,v) \leq c(u,v) - f(u,v) \iff f(u,v) + f'(u,v) \leq c(u,v)
          \end{align*}
          So, $f'$ satisfies the capacity constraint if and only if $f + f'$ satisfies it too.
          We are done with (a).
          
          Now, $$|f \pm f'| = \sum_{v \in V} (f \pm f')(s, v) = \sum_{v \in V} (f(u,v) \pm f'(u,v))
          \sum_{v \in V} f(u,v) \pm \sum_{v \in V} f'(u,v) = |f| \pm |f'|.$$
          We are done with (b).
        \end{proof}
      \item A flow of maximum value is called a \emph{max flow}.
      \item An \emph{augmenting path} is a path in $G$ from $s$ to $t$ such
        that all edges on the path have positive capacity.
      \item We now have the famous \emph{Max Flow-Min Cut} theorem.
        \begin{theorem}
          The following statements are equivalent:
          \renewcommand{\labelenumi}{(\alph{enumi})}
          \begin{enumerate}
            \item $f$ is a max flow in $G$;
            \item there is an $s,t$-cut $A, B$ with $c(A,B) = |f|$;
            \item there is no augmenting path in $G_f$.
          \end{enumerate}
        \end{theorem}
    \end{itemize}
  % section networks_and_flows (end)
  
  \section{Ford--Fulkerson Algorithm} % (fold)
  \label{sec:ford-fulkerson}
    \begin{itemize}
      \item The Max Flow-Min Cut gives a simple algorithm for
        finding the maximum flow:
        \begin{enumerate}
          \item Start with the zero flow $f(u,v) = 0$ for all $u, v \in V$.
          \item Find an augmenting path $(v_1, v_2), \dotsc, (v_{k-1}, v_k)$
            where $v_1 = s$ and $v_k = t$ in $G_f$.
          \item If there is no augmenting path, terminate.
          \item If there is an augmenting path, let $d$ be the capacity
             of the edge in the path with the lowest capacity.
             (This capacity will now be referred to as the \emph{bottleneck capacity}.)
             Define flow $g$ as follows:
             \begin{itemize}
              \item $g(v_i, v_{i+1}) = d$ for all $i$;
              \item $g(v_{i+1}, v_i) = -d$ for all $i$;
              \item $g(u,v) = 0$ for all other pairs of vertices $u, v \in V.$
             \end{itemize}
          \item Up date $f$ to be $f + g$.
          \item Go back to Step 2.
        \end{enumerate}
        This algorithm is called the \emph{Ford--Fulkerson} algorithm.
        
      \item The Ford--Fulkerson algorithm is, unfortunately, not polynomial time.
        
        The problem is that it does not specify how to find an augmenting path.
        
        As such, there are graphs where if the augmenting path is found in
        a specified way, the algorithm runs forever.
        
        
        \item One tool used to analyze network flow algorithms is 
          the following \emph{flow decomposition lemma.}

          \begin{lemma} \label{flow-decomposition}
            Any flow in $G$ can be expressed as a sum of 
            at most $m$ \emph{path flows} and a zero-valued flow.
          \end{lemma}

          By a \emph{path flow}, we mean the flow that we generate
          from an augmenting path as seen in the description of
          the Ford--Fulkerson algorithm.

          \begin{proof}
            Let $f$ be a flow with $|f| > 0$. 
            Consider the new capacity function
            $c'(u,v) = \max\{ f(u,v), 0 \}$, and a new network
            $G'$ with this capacity function.

            Since the value of max flow in $G$ is $|f| > 0$,
            there must be an augmenting path. Let $p$
            be the path flow of this augmenting path.
            We have that $f - p$ is a flow in $G'$.
            Consider the network $G''$ generated by the 
            capacity function $c''(u,v) = \max \{ f(u,v) - p(u,v), 0 \}.$
            We have that $G''$ has one fewer edges than $G'$
            because the edge with the bottleneck capacity
            of $p$ gets its capacity reduced to zero.

            We can repeat the same process to $G''$.
            This process can continue at most $m$ times
            because one edge disappear each time.
            Also, each time, we take out a path flow
            from the original flow.
            Hence, the flow can be decomposed into
            at most $m$ path flows.
          \end{proof}
    \end{itemize}
        
  \section{Heuristics of Edmonds and Karp}
    \begin{itemize}
      \item Edmonds and Karp gave two heuristics for choosing augmenting paths.
        \begin{enumerate}
          \item {\bf Use the path with maximum bottleneck capacity first.} 
            If the capacity function takes on integer values and $f^*$ is a
            maximum flow, then this heuristics will result in at most $O(m \log |f*|)$
            augmenting steps, where $m$ is the number of edges.
            
            We can find the path with maximum bottleneck capacity
            with a modified Dijkstra's algorithm. Hence, the heuristics
            gives an $O(m^2 \log n \log |f^*| )$ time, where $n$
            is the number of vertices.
            
          \item {\bf Use the path with the minimum hops first.} This
            results in an $O(m^2 n)$ algorithm for maximum flow.
            We will discuss this heuristic in details.
        \end{enumerate}
      
      \item The reason why the second heuristic works is that
        \emph{the shortest augumenting path never gets shorter}.
        In fact, we have the following lemma.
        
        \begin{lemma} \label{augmenting-path-no-shorter}
          Let $p$ be a shortest augmenting path in $G$.
          Let $G'$ be the residual network obtained by using the 
          path flow along $p$. Let $q$ be a shortest augmenting
          path in $G'$. Then, $q$ is not shorter than $p$.
        \end{lemma}
        
        To prove the lemma, we need the following definition.
        
        \begin{definition}
          A \emph{level graph} $L_G$ of network $G$ is a graph formed
          by: 
          \begin{enumerate}
            \item doing a BFS starting at $s$, thereby partitioning the graph into layers, and
            \item removing any edges pointing backwards or pointing to vertices in the same layer.
          \end{enumerate}
        \end{definition}
        
        \begin{proof} (Lemma~\ref{augmenting-path-no-shorter})
          Let $p$ be the shortest augmenting path. Note
          that all edges of $p$ are present in the level graph.
          Consider the residual graph after augmenting along $p$.
          We have that one edge of the level graph will not
          appear in the residual graphs. Moreover, there may 
          be at most new $|p|$ edges appearing in the residual graph,
          but they all point backwards in the level graph.
          Hence, these new edges cannot make the sink closer
          to the source. Therefore, the shortest path from 
          sink to source cannot be shorter than the one in 
          the original graph.
        \end{proof}
        
      \item We now show that the distance between $s$ and $t$
        must increase after augmenting along the shortest
        augmenting paths $m$ times.
        
        \begin{lemma}
          We can augment along the shortest augmenting paths
          with the same length at most $m$ times.
        \end{lemma}
        
        \begin{proof}
          Consider the level graph. A shortest path must consists
          of only edges in the level graph. Since an edge in
          the level graph disappears each time we augment,
          we can augment at most $m$ times.
        \end{proof}
        
      \item Since the shortest distance from $s$ to $t$ can
        be at most $n-1$. There are at most $mn$ augmentation.
        Since finding the shortest augmenting path takes $O(m)$
        time, the second heuristics of Edmonds and Karp gives
        an $O(m^2n)$ algorithm for finding the maximum flow.

    \end{itemize}
    
  \section{Blocking Flows and Dinic's Algorithm} % (fold)
  \label{sec:blocking_flows}  
    \begin{itemize}
      \item An edge in a network $G$ is called \emph{admissible}
        if it is in the level graph $L_G$.
        
      \item  An \emph{admissible path} is an $s,t$-path consist entirely
        of admissible edges.
      
      \item A \emph{blocking flow} is a flow where at least
        on edge in every admissible path is saturated.
      
      \item We have the following lemma.
      
        \begin{lemma}
          Let $G$ be a network where $s$ and $t$ are $d$ hops
          apart. Let $f$ be a blocking flow. Then, $s$ and $t$
          are at least $d+1$ hops apart.
        \end{lemma}
        
        \begin{proof}
          Since $s$ and $t$ are $d$ hops apart in $G$,
          and augmenting a flow only add backward edges 
          to $G$, we must have that $s$ and $t$ are at least
          $d$ hops apart in $G_f$ as well.
          
          For $s$ and $t$ to be exactly $d$ hops
          apart, there must be an admissible path from $s$
          to $t$. However, by the definition of blocking flow,
          any $s,t$-path in $G_f$ must 
          contain an edge that is not admissible.
          Therefore, $s$ and $t$ must be at least $d+1$
          hops apart. 
        \end{proof}
      
      \item We now have a framework for solving the max flow problem.
      
        \begin{enumerate}
          \item Find a blocking flow $f$ in $G$
          \item Push $f$ through the network.
          \item Update $G \gets G_f$.
          \item Go back to Step 1 until we cannot find more blocking flows.
        \end{enumerate}
        
        Note that We have to keep doing this at most $n$ times
        because $s$ and $t$ can only be at most $n-1$ hops apart.
        
      \item How to find a blocking flow? We present the following
        algorithm by Dinic.
        
        \begin{enumerate}
          \item Compute the layer graph $L_G.$
          
          \item Start at vertex $v = s$ and keep carrying out the following operations in $L_G$:
            \begin{itemize}
              \item {\bf Advance:} If $v$ has an outgoing edge $(v,w)$ with $c(v,w) > 0$,
                go to $w$.
              
              \item {\bf Augment:} If $v = t$, then we have found a path from $s$ to $t$.
                Augment the path with the bottleneck capacity, and remove all the
                saturated edges from the level graph. Then start at $s$ again.
                
              \item {\bf Retreat:} If $v$ does not contain any outgoing edges,
                we remove $v$ and all its incident edges from the graph, 
                and backtrack to the vertex that lead to $v$.
            \end{itemize}
            
            We continue doing these operations until $s$ does not have any 
            out going edges.
        \end{enumerate}
        
      \item The above algorithm runs in $O(mn)$ time. The analysis goes 
        as follows.
      
        Observe that each time we perform Augment, at least an edge in
        the level graph disappear. Therefore, there can be at most $m$
        Augment. So all the Augment takes $O(mn)$ time.
        
        We can Retreat at most $n$ time, and it takes $O(m)$ time
        to remove all the edges. So all the Retreat takes $O(m+n)$ time.
        
        An Advance that leads to the release can be charged with
        the Retreat. A series of Advances that lead to an Augment
        can also be charged with the Augment. So, there can be
        no more than $O(mn)$ advances.
      
      \item Since finding a blocking flow takes $O(mn)$ time,
        we have a max flow algorithm that runs in $O(mn^2)$ time.
        
    \end{itemize}
  % section blocking_flows (end)
  
  \section{Dinic's Algorithm in Unit Capacity Network} % (fold)
  \label{sec:dinic_s_algorithm_in_unit_capacity_graph}
    \begin{itemize}
      \item A \emph{unit capacity network} is a network whose
        edges have capacity $0$ or $1$.
      
      \item Dinic's algorithm runs faster in a unit capacity network.
      
        \begin{lemma}
          In a unit capacity network, Dinic's algorithm to
          find a blocking flow takes $O(m)$ time.
        \end{lemma}
        \begin{proof}
          An Augment in a unit capacity graph makes
          all edges in the augmenting path disappear.
          Therefore, all Augments take at most $O(m)$ time.
        \end{proof}
        
      \item Moreover, fewer blocking flows are needed to
        complete the max flow.
        
        \begin{lemma}
          Let $d$ be a positive integer. In a unit capacity
          network, after pushing $d$ blocking flows,
          pushing only $m/d$ more blocking flows completes
          the max flow.
        \end{lemma}
        
        \begin{proof}
          After pushing $d$ blocking flows, the distance
          between $s$ and $t$ is at least $d$. Consider the level 
          graph of the residual graph. Let $V_i$ be the set of 
          vertices in the $i$th layer.
          
          Define $A_i = V_0 \cup V_1 \cup \dotsb \cup V_i$,
          and $B_i = V - A_i$. We have that
          $(A_0, B_0)$, $(A_1, B_1)$, $(A_2, B_2)$, 
          and $(A_{d-1}, B_{d-1})$ are cuts.
          The capacity of $(A_i, B_i)$ is equal 
          to the number of edges going from Layer $i$
          to Layer $i+1$. Consider the the cut
          among these $d$ cuts with the lowest capacity.
          We must have that this cut's capacity
          is at most $m/d$. Therefore, the 
          max flow in the residual graph has value
          at most $m/d$. Since each blocking flow increases
          the value of the overall flow by $1$,
          only $m/d$ more blocking flows are needed.
        \end{proof}
        
      \item In conclusion, Dinic's algorithm is very fast
        in unit capacity networks.
        
        \begin{theorem}
          The running time of Dinic's algorithm in a unit capacity
          network is $O(m^{3/2}).$
        \end{theorem}
        
        \begin{proof}
          The last lemma tells us that Dinic's algorithm runs
          in $O(m(d + m/d))$ for any integer $m$. We can minimize
          the expression $d + m/d$ by choosing $d = \sqrt m$.
        \end{proof}
    \end{itemize}
  % section dinic_s_algorithm_in_unit_capacity_graph (end)
    
  \section{Scaling Algorithm} % (fold)
  \label{sec:scaling_algorithm}

    \begin{itemize}
      \item We can also find max flow faster than the
        general case when we assume that the capacity
        are integers. 
        
      \item Let $U$ be an integer upper
        bound on all the capacity of all edges.
        
      \item We decompose the network
        $G$ into $k = \log U$ networks: $G_0$,
        $G_1$, $\dotsc$, $G_{k-1}$. Here, $G_i$
        is the network whose capacity is given by
        $$c_i(u,v) = \Big\lfloor \frac{c(u,v)}{2^i} \Big\rfloor.$$
        In other words, $c_i(u,v)$ is obtained by $c(u,v)$
        truncating $i$ least significant digits off.
        
      \item Now, we start by finding a maximum flow $f_0$
        in $G_0$. Since this is a unit capacity network,
        it can be done in $O(m^2)$ time. (We know that
        it's actually $O(m^{3/2})$, but let's say 
        it's $O(m^2)$ for now.)
        
        Next, we will find the maximum flow $f_1$ in $G_1$.
        We do so by observing that the flow $2f_0$ is
        a flow in $G_1$. We push the $2f_0$ to the network
        first, and then try to find the max flow in the
        residual graph.
        
        The nice thing is that $|f_1| \leq 2|f_0| + m$.
        The reason is that there is a cut in $G_0$
        which has capacity $|f_0|$. Looking at the same
        cut in $G_1$, the value of each edge is doubled
        and may be incremented by 1. So the capacity
        of that cut does not exceed $2|f_0| + m.$
        
        This means that we only need at most $m$ augmentations.
        Thus, finding $|f_1|$ can be accomplished in $O(m^2).$
        
        Of course, this line of reasoning genaralizes to all $G_i$.
        So, we can use the same trick to find the max flow in $G_2$, $G_3$, 
        and so on until we reach $G_{k-1}$, which is $G$ itself. 
        Hence, we have a max flow algorithm which runs in $O(m^2 \log U).$
        
    \end{itemize}
  % section scaling_algorithm (end)
\end{document}