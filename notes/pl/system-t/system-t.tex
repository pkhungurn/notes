\documentclass[10pt]{article}
\usepackage{fullpage}
\usepackage{amsmath}
\usepackage[amsthm, thmmarks]{ntheorem}
\usepackage{amssymb}
\usepackage{graphicx}
\usepackage{epstopdf}
\usepackage{enumerate}

\newtheorem{lemma}{Lemma}[section]
\newtheorem{theorem}[lemma]{Theorem}
\newtheorem{definition}[lemma]{Definition}
\newtheorem{proposition}[lemma]{Proposition}
\newtheorem{corollary}[lemma]{Corollary}
\newtheorem{claim}[lemma]{Claim}

\newcommand{\dee}{\mathrm{d}}
\newcommand{\In}{\mathrm{in}}
\newcommand{\Out}{\mathrm{out}}
\newcommand{\pdf}{\mathrm{pdf}}
\newcommand{\Ei}{\mathrm{Ei}}
\newcommand{\Tri}{\mathbf{tri}}
\newcommand{\Th}{\mathbf{th}}
\newcommand{\V}{\mathbf{V}}
\newcommand{\Const}{\mathbf{const}}

\newcommand{\ve}[1]{\mathbf{#1}}

\title{G\"odel's System T}
\author{Pramook Khungurn}

\begin{document}
	\maketitle
	
	G\"odel's System T is the first logical system of functions such that, if you can determine a type of a function,
	the function terminates. The system is a restrictive subset of extended lambda calculus.
	
	\section{Definitions}
	
	\subsection{Types}
	\begin{itemize}
	  \item We define {\bf types} as follows:
	    \begin{itemize}
	      \item $\mathbb{N}$ is a type.
	      \item If $\alpha$ and $\beta$ are types, then $\alpha \rightarrow \beta$ is a type.
	    \end{itemize}
	  
	  \item $\mathbb{N}$ is called the {\bf atomic type},\\
	    and types of the form $\alpha \rightarrow \beta$ are called {\bf higher types}.
	  
	  \item Types are typically denoted by $\sigma$ and $\tau$.
	  
	  \item $\sigma_1 \rightarrow \sigma_2 \rightarrow \sigma_3 \rightarrow \dotsb \rightarrow \sigma_n$
	    means $\sigma_1 \rightarrow (\sigma_2 \rightarrow (\sigma_3 \rightarrow \dotsb \rightarrow \sigma_n ) ) )$
	    
	  \item Stenlund called all the objects we deal with in System T as {\bf computable functionals}.
	  
	  \item An object of type $\mathbb{N}$ is a natural number.\\
	   Natural numbers are computable functionals.
	  
	  \item An object of type $\alpha \rightarrow \beta$ is a function\\ 
	  which takes a computable functional of type $\alpha$ and\\ 
	  spits out a computable functional of type $\beta$.
	\end{itemize}
	
	\subsection{Constants}
	
	\begin{itemize}
	  \item $0$ is a constant of type $\mathbb{N}$.
	  \item $s$ is a constant of type $\mathbb{N} \rightarrow \mathbb{N}$,\\
	    representing the successor function.
	  \item For each type $\tau$, there's a constant $R^{\tau}$ of type $\tau \rightarrow (\mathbb{N} \rightarrow \tau \rightarrow \tau) \rightarrow \mathbb{N} \rightarrow \tau.$\\	    
	    The $R^\tau$ represents the primitive recursive arithmetic combinator.\\
	    (When $R^\tau$ is used, however, we will only write it as $R$.)
	    
	  \item The {\bf numerals} in System T as defined as $0$, $s(0)$, $s(s(0))$, $\dotsc$.
	\end{itemize}
	
	\subsection{Terms}
	
	\begin{itemize}
	  \item Each constant of type $\tau$ is a term of type $\tau$.
	  
	  \item For each type $\tau$, there exists a countably finite list of variables of type $\tau$,\\
	    which we will denote by $x^\tau$, $y^\tau$, $z^\tau$, $\dotsc$.
	    
	  \item If $a$ is a term of type $\tau$ and $x$ is a variable of type $\sigma$,\\
	    then $\lambda x.\ a$ is a term of type $\sigma \rightarrow \tau$\\
	    (Abtraction)
	    
	  \item If $a$ is a term of type $\sigma \rightarrow \tau$ and $b$ is term of type $\sigma$,\\
	    then $a\ b$ is a term of type $\tau.$\\
	    (Application)
	\end{itemize}
	
	\section{Equality}

  \begin{itemize}	
    \item There are two equality relations involved.
      \begin{itemize}
        \item ``$=$'' is the equality between natural numbers.
        \item ``$=_i$'' is the intensional equality between terms.
      \end{itemize}
      
    \item Of course, ``$=$'' implies ``$=_i$''.
    
    \item Equality can be inferred from a number of axioms:
      \begin{itemize}
        \item ``$=_i$'' is reflexive, symmetric, and transitive.
        \item If $a =_i b$ and $c = _i d$, then $a\ b =_i b\ d$.
        \item If $a =_i b$, then $\lambda x.\ a =_i \lambda x.\ b$.
        \item $(\lambda x.\ (a\ x))\ b =_i a\ b$
        \item $\lambda x.\ (a\ x) =_i a$
        \item $R\ a\ b\ 0 =_i a$
        \item $R\ a\ b\ s(t) =_i b\ t\ (R\ a\ b\ t)$
        \item ``$=$'' is reflexive, symmetric, and transitive.
        \item If $c = d$ and $a \in \mathbb{N} \rightarrow \mathbb{N}$, then $a\ c = a\ d$.
        \item If $a = b$, then $(\lambda x.\ a) c = (\lambda x.\ b) c$ where $c \in \mathbb{N}$.
        \item If $a, b \in \mathbb{N}$ and $a =_i b$, then $a = b$.
        \item If $f\ 0 = a$ and $f\ s(t) = b\ t\ (f\ t)$, then $f\ t = R\ a\ b\ t$.\\
          Here, $a \in \mathbb{N}$, $b \in \mathbb{N} \rightarrow \mathbb{N} \rightarrow \mathbb{N}$, and $t \rightarrow \mathbb{N}$.
      \end{itemize}
  \end{itemize}
  
  \section{Reduction}

    \begin{itemize}
      \item As with other lambda calculus, System T has its rule for reduction of terms.
      
      \item Here are the small-step reduction rules for System T.
        \begin{itemize}
          \item $R\ a\ b\ 0 \rightarrow_1 a$
          \item $R\ a\ b\ s(t) \rightarrow_1 b\ t\ (R\ a\ b\ t)$
          \item $(\lambda x.\ a)\ b \rightarrow_1 a[b/x]$
          \item $\lambda x. (a\ x) \rightarrow_1 a$
        \end{itemize}
        
      \item A term is called {\bf normal} if it does not contain a subterm which can be reduced by one of the above rules.          
      
      \item We write $a \rightarrow_* b$ if there's a finite series of reduction
        $a \rightarrow_1 a_1 \rightarrow_1 a_2 \rightarrow_1 a_3 \rightarrow_1 \dotsb \rightarrow_1 b$.\\
        Here, we say that $a$ {\bf reduces to} $b$.
      
      \item A term $a$ is {\bf normalizable} if it reduces to a normal term.\\
        The normal term is said to be the {\bf normal form} of $a$.
              
      \item We say that two terms $a$ and $b$ are {\bf definitionally equal} if reduces to the same normal term.
        
      \item \begin{theorem}[Church--Rosser]
        If $a \rightarrow_* b$ and $a \rightarrow_* c$, then there exists $d$ such that $b \rightarrow_* d$ and $c \rightarrow_* d$.
      \end{theorem}            
      
      \item \begin{corollary}
        Definitional equality is an equivalence relation.
      \end{corollary}
      
      \item \begin{theorem} Two terms $a$ and $b$ are definitionally equal if and only if $a =_i b$. \end{theorem}
      \begin{proof}
        ($\rightarrow$) Observe that the two sides of the reduction rules are intensionally equal to one another.
        So, definitional equality implies intensional equality.
        
        ($\leftarrow$) This is done by the induction on the rules of intensional equality.
      \end{proof}
    \end{itemize}
    
  \section{Computability and Normal Form}
  
    \begin{itemize}
      \item A term $a$ is {\bf strongly normalizable} (SN) if all reduction sequences
        $$a \rightarrow_1 a_1 \rightarrow_1 a_2 \rightarrow_1 a_3 \rightarrow_1 \dotsb$$
        starting from $a$ are finite.
        
      \item In other words, $a$ is SN if $a$ is normalizable and each reduction sequence starting
      with $a$ ends up in the normal form of $a$.
      
      \item There are terms which are normalizable but not SN. For example:
        $$(\lambda x. y)\ ((\lambda x.\ x\ x)\ (\lambda x.\ x\ x)) $$
        
      \item The main theorem is that all typed terms in System T are SN. 
      
      \item The proof has three steps.
        \begin{itemize}
          \item Define what it means for a term of the \emph{computable}.
          \item Show that computable terms are SN.
          \item Show that all typed terms in System T are computable.
        \end{itemize}
        
      \item We call a term $a$ {\bf computable} if it satisfies one of the following rules:
        \begin{itemize}
          \item If $a \in \mathbb{N}$, then $a$ is computable if it is SN.
          \item If $a$ has type $\tau \rightarrow \sigma$, then $a$ is computable\\ 
            if $(a\ b)$ is computable for all comptuble terms $b$ of type $\tau$.
        \end{itemize}
        
      \item We can state the second rule in another way:
        \begin{quote}
          A term $a$ is computable if $a\ a_1\ a_2\ \dotsb\ a_n$ is computable
          for all computable terms $a_1$, $a_2$, $\dotsc$, $a_n$ such
          that $a\ a_1\ a_2\ \dotsb\ a_n \in \mathbb{N}$.
        \end{quote}
        
      \item \begin{lemma}
        If $a \rightarrow_* b$ and $a$ is computable, then $b$ is computable.
      \end{lemma}
      \begin{proof}
        The lemma follows immediately if $a \in \mathbb{N}$. If $a$ is of higher type,
        then it follows from the other form of the second rule above.
      \end{proof}
      
      \item \begin{lemma}
        For any type $\tau$, a computable term of type $\tau$ is SN.
      \end{lemma}
      
      \begin{proof}
        The proof is by structural induction on the type $\tau$.
        
        The first case is when $\tau$ is $\mathbb{N}$. We have all computable terms are SN by definition.
        
        The second case is when $\tau$ is $\alpha \rightarrow \beta$. Let $a$ be a computable term
        of type $\tau$. Then, we have that for all computable term $b$ of type $\alpha$,
        we have that $a\ b$ is a computable term of type $\beta$. By induction hypothesis,
        we have that $a$ and $a\ b$ are SN. Consider any reduction sequence
        $$a \rightarrow_1 a_1 \rightarrow a_2 \rightarrow a_3 \rightarrow \dotsb .$$
        We have that it generates the corresponding sequence
        $$a\ b \rightarrow_1 a_1\ b \rightarrow a_2\ b \rightarrow a_3\ b \rightarrow \dotsb .$$
        Since the sequence sequence terminates, the first one must terminate as well.
        Therefore, $a$ is normalizable, and since all reduction sequence terminates
        $a$ is SN.
      \end{proof}
      
      \item \begin{lemma}
        The constants $0$ and $s$ are computable.
      \end{lemma}
      \begin{proof}
        $0$ is a normal term, so it is computable.
        
        Let $a$ be a computable term of type $\mathbb{N}$.
        We have that $a$ is SN. Then, $s(a)$ also SN because all
        the reduction done to $s(a)$ must be done to $a$.
        Therefore, $s(a)$ is computable. Hence, we have that
        $s$ is also computable.        
      \end{proof}
      
      \item \begin{lemma}
        If $a[b/x]$ is computable, then $(\lambda x.\ a)\ b$
        is also computable.
      \end{lemma}
      
      \begin{proof}
        Let $a_1$, $a_2$, $\dotsc$, $a_n$ be computable terms
        such that $a[b/x]\ a_1\ a_2\ \dotsb\ a_n$
        is a term of type $\mathbb{N}$. It follows that
        $a[b/x]\ a_1\ a_2\ \dotsb\ a_n$ is computable and SN.
        We can prove the lemma by showing that 
        $(\lambda x.\ a)\ b\ a_1\ a_2\ \dotsb\ a_n$ is SN.
        
        Suppose by way of proof by contradiction that there is a infinite 
        sequence of reduction in $$(\lambda x.\ a)\ b\ a_1\ a_2\ \dotsb\ a_n.$$
        We know that thisreudction must be an infinite reduction of
        $(\lambda x.\ a)$. Suppose such a sequence is
        \begin{align*}
          \lambda x.\ a \rightarrow_1 \lambda x.\ a' \rightarrow_1 \lambda x.\ a'' \rightarrow_1 \dotsb.
        \end{align*}
        This sequence would yield
        \begin{align*}
          a[b/x] \rightarrow_1 a'[b/x] \rightarrow_1 a''[b/x] \rightarrow_1 \dotsb
        \end{align*}
        which is an infinite reduction of $a[b/x]$, which is impossible. We have a contradiction.
      \end{proof}
      
      \item \begin{lemma}
        The constant $R^\tau$ for all $\tau$ is computable.
      \end{lemma}
      \begin{proof}
        It is sufficient to prove that $R\ a\ b\ c$ is computable for all computable
        $a$, $b$, and $c$ of the appropriate types. We do so by the number of $s$ in $c$.
        
        Let $a_1$, $a_2$, $\dotsc$, $a_n$ be computable terms such that
        $R\ a\ b\ c\ a_1\ a_2\ \dotsb\ a_n$ is of type $\mathbb{N}$.
        
        If $c$ does not reduce to $0$ or $s(t)$ for some $t \in \mathbb{N}$, then
        we cannot use the axiom for $R$ to reduce $R\ a\ b\ c$.
        Thus, all the reduction done to $R\ a\ b\ c\ a_1\ a_2\ \dotsb\ a_n$
        must be done in $a$, $b$, $c$, $a_1$, $a_2$, $\dotsc$, $a_n$. Since 
        all these terms are computable and thus $SN$, then 
        $R\ a\ b\ c\ a_1\ a_2\ \dotsb\ a_n$ is SN and thus computable.
        
        If $c$ reduces to $0$, then we have that
        $$R\ a\ b\ c\ a_1\ a_2\ \dotsb\ a_n \rightarrow_* R\ a'\ b'\ 0\ a_1'\ a_2'\ \dotsb\ a_n' \rightarrow_1 a'\ a'_1\ a'_2\ \dotsb\ a'_n.$$
        For some $a', b', a_1', a_2', \dotsc a'_n$ that $a, b, a_1, a_2, \dotsc, a_n$ reduce to, respectively.
        Since $a\ a_1\ a_2\ \dotsb\ a_n \rightarrow a'\ a_1'\ a_2'\ \dotsb\ a_n'$, it must
        be the case that $a'\ a_1'\ a_2'\ \dotsb\ a_n'$ is SN. Therefore, $R\ a\ b\ c\ a_1\ a_2\ \dotsb\ a_n$
        is SN as well.
        
        If $c$ reduces to $s(t)$, we have that $s(t)$ is computable, thus SN. This means
        that $t$ is SN, thus computable. It follows by induction hypothesis that $(R\ a'\ b'\ t)$
        is computable for any computable $a'$ and $b'$. Consider the reduction of
        $R\ a\ b\ c\ a_1\ a_2\ \dotsb\ a_n$, we have that it must be of the form
        $$R\ a\ b\ c\ a_1\ a_2\ \dotsb\ a_n 
        \rightarrow_* R\ a'\ b'\ s(t)\ a_1'\ a_2'\ \dotsb\ a_n'
        \rightarrow_1 b'\ t\ (R\ a'\ b'\ t)\ a_1'\ a_2'\ \dotsb\ a_n'.$$
        Since all the terms in the last expression is computable,
        we have that the term is SN. Therefore, 
        $R\ a\ b\ c\ a_1\ a_2\ \dotsb\ a_n$ is SN, that thus computable too.
      \end{proof}
      
      \item \begin{lemma}
        If $a$ is a term with free variable $x_1$, $x_2$, $\dotsb$, $x_n$
        and $b_1, b_2, \dotsc, b_n$ are computable terms with the same
        types as $x_1$, $x_2$, $\dotsc$, $x_n$, respectively, then
        $a[b_1/x_1][b_2/x_2]\dotsb[b_n/x_n]$ is computable.
      \end{lemma}
      
      \begin{proof}
        We prove this by induction on the structure of $a$.
        For convenience let $$a' := a[b_1/x_1][b_2/x_2]\dotsb[b_n/x_n].$$
        
        If $a$ is one of the free variables, say $x_i$, then $a' = b_i$,
        which is computable.
        
        If $a$ is $a_1\ a_2$, then $a'$ is $a'_1\ a'_2$.
        $a'_1$ and $a'_2$ are computable by the induction hypothesis.
        Therefore, $a'_1\ a'_2$ is computable too.
        
        If $a$ is $\lambda x.\ a_1$, then by induction hypothesis
        $a'_1[b/x]$ is computable for all computable $b$.
        This implies that $(\lambda x. a_1)\ b$ is computable
        for all computable $b$, which means that $\lambda x.\ a_1$
        is computable.
      \end{proof}
        
      
      \item \begin{theorem} 
        All terms are computable, and thus SN.
      \end{theorem}
      \begin{proof}
        Use the last four lemmas with structural induction on the terms.
      \end{proof}
    \end{itemize}      
\end{document}