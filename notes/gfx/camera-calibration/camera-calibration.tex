\documentclass[10pt]{article}
\usepackage{fullpage}
\usepackage{amsmath}
\usepackage[amsthm, thmmarks]{ntheorem}
\usepackage{amssymb}

\newtheorem{lemma}{Lemma}[section]
\newtheorem{theorem}[lemma]{Theorem}
\newtheorem{definition}[lemma]{Definition}
\newtheorem{proposition}[lemma]{Proposition}
\newtheorem{claim}[lemma]{Claim}
\newtheorem{corollary}[lemma]{Corollary}

\newcommand{\dee}{\mathrm{d}}
\newcommand{\ve}[1]{\mathbf{#1}}
\newcommand{\msf}[1]{\mathsf{#1}}

\title{Camera Calibration}
\author{Pramook Khungurn}

\begin{document}
	\maketitle
	
	The material in this note is lifted from relevant sections of 
	Szeliski's \emph{Computer Vision} book. A note on notation
	is that if $\ve{v}$ is a vector, then $\bar{\ve{v}}$ is its homogeneous
	coordinate. Therefore, if $\ve{v} \in \mathbb{R}^d$,
	then $\bar{\ve{v}} \in \mathbb{R}^{d+1}.$
	
	\section{Camera Intrinsics} % (fold)
	
	\begin{itemize}
	  \item Pixels are indexed by {\bf pixel coordinates} $\ve{x}_s = (x_s, y_s)$.
	    We let $\bar{\ve{x}}_s$ denote the homogeneous coordinate of $(x_s, y_s)$.
	    That is, $$\bar{\ve{x}}_s \sim \begin{bmatrix}x_s \\ y_s \\ 1\end{bmatrix}$$
	    where $\sim$ denotes ``equals modulo a scaling factor.''	  	  
	  
	  \item Most coordinate systems starts at the top left of the image plane. We let $\ve{c}_s$ denote
	    3D coordinate of this top left point, which we call the {\bf origin}.
	    
    \item $\ve{O}_c$ denote the camera's center of projection.	    
    	  
	  \item We let $\ve{p}_c$ denote a \emph{camera-centered} point, which is a point
	    written in a coordinate system where ${\bf O}_c$ is the origin.  Let $\ve{p}$ be the 
	    projection of $\ve{p}_c$ on to the image plane.
	  
	  \item To map $\bar{\ve{x}}_s$ to a 3D point on the image place, we first scale it by
	    the pixel spacing $(s_x, s_y)$, and then rotate it with a 3D rotation $\ve{R}_s$ and
	    then add the resulting vector to the origin $\ve{c}_s$:
	    \begin{align*}
	      \ve{p} = \left[ \begin{array}{c|c} \ve{R}_s & \ve{c}_s \end{array} \right]
	        \begin{bmatrix}
	          s_x & 0 & 0\\
	          0 & s_y & 0\\
	          0 & 0 & 0\\
	          0 & 0 & 1
	        \end{bmatrix}
	        \begin{bmatrix}
	          x_s \\ y_s \\ 1
	        \end{bmatrix}
	        = \ve{M}_s \bar{\ve{x}}_s.
	    \end{align*}
	    Here, $\ve{M}_s$ is called the {\bf sensor homography}.
	    
	  \item The matrix $\ve{M}_s$ has eight parameters:
	    \begin{itemize}
	      \item 3 parameters to describe the rotation $\ve{R}_s$.
	      \item 3 parameters to describe the center $\ve{c}_s$.
	      \item 2 parameters to describe the scale $(s_x, s_y)$.
	    \end{itemize}
	    
	  \item In practice, unless we have accurate information about
	    sensor spacing or sensor orientation, there are only 
	    7 degrees of freedom.
	    
	    The reason is that the distance from the sensor to the origin
	    cannot be separated from sensor spacing, based on 
	    external image measurement alone.
	    	  
	  \item The relationship between the pixel center $\ve{p}$ and
	    the camera-centered point $\ve{p}_c$ is given by an unknown 
	    scaling factor $\alpha$, where $\ve{p} = \alpha \ve{p}_c$.
	    Hence,
	    $$\bar{\ve{x}}_s = \ve{M}_s^{-1} \ve{p} = \alpha \ve{M}_s^{-1} \ve{p}_c $$
	    
	  \item The {\bf calibration matrix} $\ve{K}$ is defined as
	    $$\ve{K} = \alpha \ve{M}_c^{-1}.$$
	    
	  \item Literature treats $\ve{K}$ as being an upper triangular matrix 
	    with having only 5 degrees of freedom instead
	    of the full 7 or 8. This is because we cannot retrieve all the
	    degrees of freedom using external measurements alone.	  
  \end{itemize}		
  % section section_name (end)	
  
  \section{Camera Pose Estimation}\label{sec:camera_pose_estimation} % (fold)
  
  \begin{itemize}
    \item The input a number of known 3D locations 
      $\ve{p}_1$, $\ve{p}_2$, $\ve{p}_3$, $\dotsc$, $\ve{p}_n$
      and its corresponding screen coordinates 
      $\ve{x}_1$, $\ve{x}_2$, $\dotsc$, $\ve{x}_n$.
      
    \item We would like to find a $3 \times 4$ matrix $\ve{C}$ such that      
      $\bar{\ve{x}}_i = \ve{C} \bar{\ve{p}}_i$. This is called
      the {\bf camera matrix}.
      
    \item After we have $\ve{C}$, we can retrieve the calibration matrix $\ve{K}$
      by the following equation:
      \begin{align*}
        \ve{C} = \ve{K} \left[ \begin{array}{c|c} \ve{R} & \ve{t} \end{array} \right]
      \end{align*}
      where $\ve{K}$ is upper triangular and $\ve{R}$ is a $3 \times 3$ matrix.
      
    \item $\ve{K}$ in the above equation can be found by QR factorization. That is,
      we know that
      \begin{align*}
        \ve{C}^T = QR = 
        \begin{bmatrix} 
          * & * & * \\
          * & * & * \\
          * & * & * \\
          * & * & *          
        \end{bmatrix}
        \begin{bmatrix} * & * & * \\ & * & * \\ &   & * \end{bmatrix}
      \end{align*}
      Note that, $P^T = Q E E R$ where $$E = \begin{bmatrix} 0 & 0 & 1 \\ 0 & 1 & 0 \\ 1 & 0 & 0 \end{bmatrix}.$$
      Hence, we can write $P = R^T E E Q^T$. Note that $R^T E$ is upper triangular because
      \begin{align*}
        R^T E 
        &= 
        \left( \begin{bmatrix} * & * & * \\ & * & * \\ &   & * \end{bmatrix} \right)^T
        \begin{bmatrix} 0 & 0 & 1 \\ 0 & 1 & 0 \\ 1 & 0 & 0 \end{bmatrix}\\
        &=
        \begin{bmatrix} * &  &  \\ * & * & \\ * & * & * \end{bmatrix}
        \begin{bmatrix} 0 & 0 & 1 \\ 0 & 1 & 0 \\ 1 & 0 & 0 \end{bmatrix}\\
        &= \begin{bmatrix} * & * & * \\ & * & * \\ &   & * \end{bmatrix}.
      \end{align*}
      Therefore, we can set $\ve{K} = R^T E$ and $[\ve{R}|\ve{t}] = E Q^T$.          
  \end{itemize}
  
  \subsection{A Simple Analytic Algorithm}
  
  \begin{itemize}
    \item Consider a point $\ve{p}_i = (X_i, Y_i, Z_i)$ and its screen coordinate
      $\ve{x}_i = (x_i, y_i)$. We know that
      \begin{align*}
        x_i &= \frac{ c_{00} X_i + c_{01} Y_i + c_{02} Z_i + c_{03} }{ c_{20} X_i + c_{21} Y_i + c_{22} Z_i + c_{23} }\\
        y_i &= \frac{ c_{10} X_i + c_{11} Y_i + c_{12} Z_i + c_{13} }{ c_{20} X_i + c_{21} Y_i + c_{22} Z_i + c_{23} }
      \end{align*}
    \item The above equations give us the following system of linear equations:      
      \begin{align*}
        \left[
        \begin{array}{cccccccccccc}
          X_1 & Y_1 & Z_1 & 1 &     &     &        &   & -x_1 X_1 & -x_1 Y_1 & -x_1 Z_1  & -x_1\\
              &     &     &   & X_1 & Y_1 & Z_1    & 1 & -y_1 X_1 & -y_1 Y_1 & -y_1 Z_1  & -y_1\\
          X_2 & Y_2 & Z_2 & 1 &     &     &        &   & -x_2 X_2 & -x_2 Y_2 & -x_2 Z_2  & -x_2\\
              &     &     &   & X_2 & Y_2 & Z_2    & 1 & -y_2 X_2 & -y_2 Y_2 & -y_2 Z_2  & -y_2\\
              &     &     &   &     &     &   & \vdots &          &          &           &    \\
          X_n & Y_n & Z_n & 1 &     &     &        &   & -x_n X_n & -x_2 Y_n & -x_n Z_n  & -x_n\\
              &     &     &   & X_n & Y_n & Z_n    & 1 & -y_n X_n & -y_2 Y_n & -y_n Z_n  & -y_n              
        \end{array}
        \right]
        \begin{bmatrix}
          c_{00} \\
          c_{01} \\
          \vdots \\
          c_{23}
        \end{bmatrix}
        =
        \begin{bmatrix}
          0\\
          0\\
          \vdots\\
          0          
        \end{bmatrix}        
      \end{align*}
      In other words, if we let 
      \begin{align*}
        \msf{c} = \begin{bmatrix}
          c_{00} \\
          c_{01} \\
          \vdots \\
          c_{23}
        \end{bmatrix}
      \end{align*}
      and we let $\msf{A}$ denote the big matrix on the right, then we want to solve the equation
      \begin{align*}
        \msf{A} \msf{c} = \ve{0}.
      \end{align*}
      
    \item Note that if we find $\msf{c} \neq \ve{0}$ that is a solution to the above system, then any
      matrix of the form $k\msf{c}$ should also be a solution to the system. This implies that $\msf{A}$ should be rank deficient. 
                
      Because $\msf{c}$ has dimention 12 and each 3D-2D correspondence gives us 2 equations,
      we need at least 6 points in order to be able to solve the equation. However, we need more
      than $6$ points to ensure that $\msf{A}$ is actually rank deficient.
      
    \item To actually find $\msf{c}$, we may set one component of $\msf{c}$, say $c_{23}$, to 1, and get a smaller
      system of linear equations to solve. 
      
    \item Alternatively, since $\ve{A}$ is rank deficient, its smallest singular value is 0 (or should be close to 0).
      We can take $\ve{c}$ to be the singular vector corresponding to this singular value.
      
    \item This method is called the {\bf direct linear transform} (DLT).    
  \end{itemize}
  
  \subsection{Iterative Algorithms}
  
  \begin{itemize}
    \item Instead of solving for the entries of $\ve{C}$, we try to minimize the reprojection
      error of 2D points as a function of unknown pose parameters in $(\ve{R}, \ve{t})$ and $\ve{K}$.
      
    \item We first write the projection equation as:
      \begin{align*}
        \ve{x}_i = \ve{f}(\ve{p}_i; \ve{R}, \ve{t}, \ve{K})
      \end{align*}      
  \end{itemize}
  
  
  % section section_name (end)
\end{document}