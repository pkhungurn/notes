\documentclass[10pt]{article}
\usepackage{fullpage}
\usepackage{amsmath}
\usepackage[amsthm, thmmarks]{ntheorem}
\usepackage{amssymb}
\usepackage{graphicx}
\usepackage{epstopdf}
\usepackage{enumerate}

\newtheorem{lemma}{Lemma}[section]
\newtheorem{theorem}[lemma]{Theorem}
\newtheorem{definition}[lemma]{Definition}
\newtheorem{proposition}[lemma]{Proposition}
\newtheorem{claim}[lemma]{Claim}

\newcommand{\dee}{\mathrm{d}}
\newcommand{\In}{\mathrm{in}}
\newcommand{\Out}{\mathrm{out}}
\newcommand{\pdf}{\mathrm{pdf}}
\newcommand{\Ei}{\mathrm{Ei}}
\newcommand{\Tri}{\mathbf{tri}}
\newcommand{\Th}{\mathbf{th}}
\newcommand{\V}{\mathbf{V}}
\newcommand{\Const}{\mathbf{const}}

\newcommand{\ve}[1]{\mathbf{#1}}

\title{Approximating Volume Line Integral\\
by Low-Rank Matrix Approximation}
\author{Pramook Khungurn}

\begin{document}
	\maketitle
		
	Given a 3D function $V$ represented as a trilinear interpolation
	of a 3D grid, we would like to construct a data structure that 
	approximates/upper bounds/lower bounds the answer to 
	the following query:
	
	\begin{quote}
	    Given two points $\mathbf{p}_0$ and $\mathbf{p}_1$ in
	    space, compute the line integral
	    \begin{align} \label{line-integral}
	        \| \mathbf{p}_1 - \mathbf{p}_0 \| \int_0^1 V(\mathbf{p}_0 + (\mathbf{p}_1 - \mathbf{p}_0)t ) \ \dee t.
	    \end{align}
	\end{quote}
	
\section{Motivation and Settings}

\begin{itemize}
    \item Suppose a volume of participating media at point $\mathbf{p}$
    has extinction coefficient $\sigma_t(\mathbf{p})$. Then, radiance
    originating at point $\mathbf{p}_0$ going to point $\mathbf{p}_1$
    in the direction of $\mathbf{p}_1 = \mathbf{p}_0$ would be
    attenuated by a factor of 
    \begin{align} \label{extinction-attenuation}
        \exp \bigg( - \| \mathbf{p}_1 - \mathbf{p}_0 \| \int_0^1 \sigma_t(\mathbf{p}_0 + (\mathbf{p}_1 - \mathbf{p}_0)t ) \ \dee t \bigg).
    \end{align}
    
    \item The attenuation factor (\ref{extinction-attenuation})
    is useful for multiple importance sampling and computation
    of shadow rays.
    
    \item Ray marching is a popular technique used in computer graphics
    to compute the line integral. However, the evaluation 
    is not accurate
    if the step is too large, and it becomes slow if the step is too 
    small. Moreover, when used in a renderer, the resulting image 
    is biased. 
    
    \item Simpling techniques such as Woodcock tracking gives
    a random variable whose expectation is equal to 
    (\ref{extinction-attenuation}), but it cannot evaluate
    or give an approximation to (\ref{extinction-attenuation}).
    Hence, the technique cannot be used with multiple importance
    sampling.
    
    \item As such, we seek a data structure or a precomputation
    scheme that approximates the line integral (\ref{line-integral})
    while incurring small computational cost.
    
    \item We focus on cube volumes $V$ that are represented by
        a 3D grid of size $(n+1)\times(n+1)\times(n+1)$
        where $n = 2^m$. The grid partitions the volume
        into $n^3$ cells. For each cell, the value of the
        volume function at a point in the cell is defined
        to be the trilinear interpolation of the values
        of the grid points that are the cell's corners.                        
    \item In this seeting, we can calculate the line integral
        through $V$ exactly using an algorithm similar to
        the 3D differential analyzer. The complexity of evaluating
        this integral is $O(n)$. We aim to do better than this
        baseline by sacrificing some quality of the result.    
\end{itemize}

\section{Exact Integral for One Cell}

\begin{itemize}
    \item Before we go on to talk about approximations we make,
    we need to discuss how to compute the exact result.
    
    \item Imagine a cube whose bottom corner is at $(0,0,0)$
        and whose top corner is at $(1,1,1).$ The volume contained
        in this cube is $[0,1] \times [0,1] \times [0,1].$
        
        We view the corners of the grid as a $2\times 2\times 2$ grid.
        Let $c_{ijk}$, where $i, j, k \in \{0, 1\}$ denote the 
        value stored at the corner $(i,j,k).$
        
    \item Let $V_{i'j'k'}(x,y,z)$ denote a volume function defined
        on $[0,1] \times [0,1] \times [0,1]$ as the trilinear
        interpolation of the values stored in following grid:
        \begin{align*}
            c_{ijk} = \begin{cases}
                1, & i = i', j=j',\mbox{ and } k=k',\\
                0, & \mbox{otherwise}
            \end{cases}.
        \end{align*}
        We have that
        \begin{align*}
            V_{000}(x,y,z) &= (1-x)(1-y)(1-z),\\
            V_{001}(x,y,z) &= (1-x)(1-y)z,\\
            V_{010}(x,y,z) &= (1-x)y(1-z),\\
            V_{011}(x,y,z) &= (1-x)yz,\\
            V_{100}(x,y,z) &= x(1-y)(1-z),\\
            V_{101}(x,y,z) &= x(1-y)z,\\
            V_{110}(x,y,z) &= xy(1-z), \mbox{ and}\\
            V_{111}(x,y,z) &= xyz
        \end{align*}    
        for all $(x,y,z) \in [0,1] \times [0,1] \times [0,1].$
    \item So, for a general cube where values stored at corners
        can be arbitrary, we let $V(x,y,z)$ denote the volume function
        defined as the trilinear interpolation of the values stored
        at the corners. We have that
        \begin{align*}
            V(x,y,z) = c_{000}V_{000}(x,y,z) + c_{001}V_{001}(x,y,z) +
            \dotsb + c_{111}V_{111}(x,y,z).
        \end{align*}
        
    \item Given two points $\mathbf{p}_0 = (x_0, y_0, z_0)$ and 
        $\mathbf{p}_1 = (x_1, y_1, z_1)$ inside the volume 
        $[0,1] \times [0,1] \times [0,1]$, the line connecting
        them is given by the function
        $\mathbf{p}(t) = \mathbf{p}_0 + (\mathbf{p}_1 - \mathbf{p}_1)t$.        
    
    \item Again, we are interested in computing the line integral.
        \begin{align*}
            \mathcal{L}_V(\mathbf{p}_0, \mathbf{p}_1) 
            &= \| \mathbf{p}_1 - \mathbf{p}_0 \| \int_0^1 
                V(\mathbf{p}_0 - (\mathbf{p_1} - \mathbf{p_0})t)\ 
                \dee t.
        \end{align*}
        We have that
        \begin{align*}
            \mathcal{L}_V(\mathbf{p}_0, \mathbf{p}_1) 
            &= \| \mathbf{p}_1 - \mathbf{p}_0 \| \int_0^1 
                V(\mathbf{p}(t))\ 
                \dee t \\
            &= \| \mathbf{p}_1 - \mathbf{p}_0 \|
                \int_0^1 \bigg( 
                     c_{000}V_{000}(\mathbf{p}(t)) +
                    \dotsb
                    c_{111}V_{111}(\mathbf{p}(t))
                \bigg)\ \dee t\\
            &= c_{000} \| \mathbf{p}_1 - \mathbf{p}_0 \|
                \int_0^1 V_{000}(\mathbf{p}(t)) \dee t +
                \dotsb +
                c_{111} \| \mathbf{p}_1 - \mathbf{p}_0 \|
                \int_0^1 V_{111}(\mathbf{p}(t)) \dee t\\
            &= c_{000} \mathcal{L}_{V_{000}}(\mathbf{p}_0, \mathbf{p}_1)
                + \dotsb +
                c_{111} \mathcal{L}_{V_{111}}(\mathbf{p}_0, \mathbf{p}_1).
        \end{align*}    
        Hence, $\mathcal{L}_{V_{000}}$, $\mathcal{L}_{V_{001}}$,
        $\dotsc$, and $\mathcal{L}_{V_{111}}$ are basis functions
        of trilinearly interpolated volume.
        
    \item We can evaluate the integrals using the following identity:
        \begin{align*}
            \int_0^1 (at + b)(ct + d)(et + f)\ \dee t
            &= \int_0^1 \big( ace t^3 + (acf + ade + bce)t^2  
            + (adf + bcf + bde)t + bdf \big) \ \dee t \\
            &= \frac{ace}{4} + \frac{acf + ade + bce}{3}
                + \frac{adf + bcf + bde}{2} + bdf.
        \end{align*}
        
    \item Trying the identity out with $V_{000}$,
        we have
        \begin{align*}
            \int_0^1 V_{000}(\mathbf{p}(t))\ \dee t
            &= \int_0^1 \big( (x_1-x_0)t + x_0 \big)
                \big( (y_1-y_0)t + y_0 \big)
                \big( (z_1-z_0)t + z_0 \big)\ \dee t\\
            &= \frac{(x_1-x_0)(y_1-y_0)(z_1-z_0)}{4} +
                \dotsb + x_0 y_0 z_0.
        \end{align*}
        Using Mathematica to simplify the terms,
        we have
        \begin{align} \label{v000-close-form}
            \int_0^1 V_{000}(\mathbf{p}(t))\ \dee t
            &= \frac{x_0 y_0 z_0}{4} +
                \frac{x_1 y_0 z_0 + x_0 y_1 z_0 + x_1 y_1 z_0 +
                    x_0 y_0 z_1 + x_1 y_0 z_1 + x_0 y_1 z_1}{12} +
                \frac{x_1 y_1 z_1}{4} \notag\\
            &= \frac{x_0 y_0 z_0}{6} 
                + \frac{(x_0+x_1)(y_0+y_1)(z_0+z_1)}{12}
                + \frac{x_1 y_1 z_1}{6} \notag\\
            &= \frac{2x_0y_0z_0 +
                 (x_0+x_1)(y_0+y_1)(z_0+z_1) + 
                 2x_1 y_1 z_1}{12}.
        \end{align}
        
    \item For other $V_{ijk}$, we can use \eqref{v000-close-form}
        to evaluate it. For example, for $V_{111}$, the integrand
        is
        \begin{align*}
            & \big(1 - (x_1-x_0)t - x_0 \big)
                \big(1 - (y_1-y_0)t - y_0 \big)
                \big(1 - (z_1-z_0)t - z_0 \big)\\
            &= \big((x_0-x_1)t + 1 - x_0 \big)
                \big((y_0-y_1)t + 1 - y_0 \big)
                \big((z_0-z_1)t + 1 - z_0 \big) \\
            &= ((x'_1 - x'_0)t + x'_0)
                ((y'_1 - y'_0)t + y'_0)
                ((z'_1 - z'_0)t + z'_0)
        \end{align*}
        where $x'_0 = 1 - x_0$, $x'_1 = 1 - x_1$, and so on.
        Thus, using \eqref{v000-close-form}
        \begin{align*}
            \int_0^1 V_{111}(\mathbf{p}(t))\ \dee t
            &= \frac{2x'_0y'_0z'_0 +
                 (x'_0+x'_1)(y'_0+y'_1)(z'_0+z'_1) + 
                 2x'_1 y'_1 z'_1}{12}\\
            &= \frac{1}{12}\bigg( 2(1-x_0)(1-y_0)(1-z_0)
                + (2-x_0-x_1)(2-y_0-y_1)(2-z_0-z_1)\\
            & \qquad + 2(1-x_0)(1-y_0)(1-z_0) \bigg).
        \end{align*}
\end{itemize}

\section{A Framework for Multiscale Precomputed Line Integrals}

\subsection{Notations}

\begin{itemize}
	\item For $k \in \mathbb{Z}^+$, let $\mathbb{Z}_k$ denote the set $\{ 0, 1, \dotsc, k-1 \}.$ 
	
	\item We call a function $c: \mathbb{Z}_n \times \mathbb{Z}_n \times \mathbb{Z}_n \rightarrow \mathbb{R}$ 
		a \emph{grid of size $n$}. For notational convenience, we denote the value of $c$ as 
		$c_{ijk}$ instead of $c(i,j,k).$
		
	\item Let $c$ be a grid of size $n$. We will discuss 
	    \emph{subgrids} of $c$, and we shall define a notion 
	    for it. 
	    
	    \begin{definition}
	    Using Python's array slicing notation, 
	    we let $$c^{i_0:i_1, j_0:j_1, k_0:k_1}$$
	    where $i_0 < i_1$, $j_0 < j_1$, and $k_0 < k_1$ to be a grid 
	    of dimension $(i_1-i_0) \times (j_1-j_0) \times (k_1-k_0)$
		where
		\begin{align*}
		    c^{i_0:i_1, j_0:j_1, k_0:k_1}_{ijk} = c_{i_0+i, j_0+j, k_0+k}
		\end{align*}
		for all $0 \leq i < i_1-i_0$, $0 \leq j < j_1-j_0$,
		and $k < k_1-k_0$.
	    \end{definition}
	
	\item Notice that $c = c^{0:n,0:n,0:n}$.
	
	\item We will also discuss \emph{subvolumes} of a volume.
	    \begin{definition}
	        Let $V: [x_0,x_1]\times[y_0,y_1] \times [z_0,z_1] 
	    \rightarrow \mathbb{R}$ be a volume function. We define
	    \begin{align*}
	        V^{[x'_0, x'_1] \times [y'_0,y'_1] \times [z'_0,z'_1]},
	    \end{align*}
	    where $x_0 \leq x'_0 \leq x'_1 \leq x_1$, 
	    $y_0 \leq y'_0 \leq y'_1 \leq y_1$,
	    and $z_0 \leq z'_0 \leq z'_1 \leq z_1$,
	    to be another volume function from 
	    $[x_0, x_1] \times [y_0, y_1] \times [z_0, z_1]$
	    to $\mathbb{R}$ such that
        \begin{align*}
            V^{[x'_0, x'_1] \times [y'_0,y'_1] \times [z'_0,z'_1]}(x,y,z) =
            \begin{cases}
                V(x,y,z), &\mbox{if }(x,y,z) \in 
                [x'_0, x'_1] \times [y'_0, y'_1] \times [z'_0, z'_1]\\
                0, &\mbox{otherwise}
            \end{cases}
        \end{align*}
        \end{definition}
		
    \item We will also discuss functions that takes a grid and turns
    it into a volume. 
    \begin{definition}
        An function $\V$ is called a \emph{grid-to-volume operator}
        if it satisfies the following properties:
        \begin{itemize}        
        \item $\V$ takes as input a grid $c$. Its output is
            denoted by $\V(c)$ or just $\V c.$
        \item $\V(c)$ is a functiom from 
            $[0,1] \times [0,1] \times [0,1]$
            to $\mathbb{R}$.
        \item $\V$ is a linear function in $c$.
        \item $\V$ ``stretches'' $c$ so that it fills         
        the cube $[0,1] \times [0,1] \times [0,1]$. That is, 
	    the point $(0,0,0)$ coincides with $c_{000}$ and 
	    the point $(1,1,1)$ coincides with $c_{nnn}$.
	    \item The value of a subvolume of $\V$ that coincides
	        with a subgrid of $c$ depends only on that subgrid.
	        
	        More precisely, let $c$ be a grid of size 
	        $n+1$. Then, the values of 
	        $$\V(c)^{[i_0/n,i_1/n]\times[j_0/n,j_1/n]
	        \times[k_0/n,k_1/n]}$$ depends only on
	        $$c^{i_0:i_1+1,j_0:j_1+1,k_0:k_1+1}.$$
    \end{itemize}
    \end{definition}
    
    In general, we denote such a function 
    with a boldfaced name such as $\V$.    
    
    \item An example of such an operator is the one
    that takes a grid c of size $n+1$,
    subdivides the unit cube into $n \times n \times n$
    smaller cubes, and let the value of the $(i,j,k)$-cube
    be $c_{ijk}.$ We denote this operator by $\Const$.
    
    \item The main grid-to-volume operator we will work on
        is the trilinear interpolation, which we shall
        denote by $\Tri$. Note that $\Tri$ satisfies all the
        properties above. Moreover, the last section
        gave the closed form formula for line integrals
        through $\Tri(c)$ where $c$ is a grid of size 2.
		
	\item A volume function can be scaled and translated.
	    We are particularly interested in the following
	    operator.
	    	    
	    \begin{definition}
	    For $i,j,k \in \{0,1\}$, define the operator 
	    $\Th_{ijk}$ (standing for ``half and then translate'') 
	    as follows. If 
	    $V: [0,1]\times[0,1]\times[0,1] \rightarrow \mathbb{R}$
	    is a volume function, then $\Th_{ijk}V$ 
	    is another volume function from $[0,1]\times[0,1]\times[0,1]$
	    to $\mathbb{R}$ such that
	    \begin{align*}
	        (\Th_{ijk}V)(x,y,z) = \begin{cases}
	            V(2x - i, 2y-j, 2z-k), & \mbox{if}(x,y,z)
	            \in [\frac{i}{2},\frac{i+1}{2}] 
	            \times [\frac{j}{2},\frac{j+1}{2}] 
	            \times [\frac{k}{2},\frac{k+1}{2}]\\
	            0, & \mbox{otherwise}
            \end{cases}.
	    \end{align*}
        \end{definition}
	    
	    The idea of $\Th_{ijk}$ is to scale the volume
	    down by a factor of 2 and translate it so that
	    the lower-left-front corner is at $(i/2,j/2,k/2)$.
	
	\item The following lemma gives a way to decompose a
	    volume function constructed from a grid
	    to the volume functions constructed from
	    the grid's octants. We state it without proof.
	    
	    \begin{lemma} \label{volume-octant}
	    If $c$ is a grid of size $2m+1$,
	    and let $\V$ be a grid-to-volume operator 
	    discussed above, then
	    \begin{align*}
	        \V(c)^{[0,0.5]\times[0,0.5]\times[0,0.5]} &=
	        \Th_{000}\V(c^{0:m+1,0:m+1,0:m+1}),\\
	        \V(c)^{[0.5,1]\times[0,0.5]\times[0,0.5]} &=
	        \Th_{100}\V(c^{m:2m+1,0:m+1,0:m+1}),\\
	        \V(c)^{[0,0.5]\times[0.5,1]\times[0,0.5]} &=
	        \Th_{010}\V(c^{0:m+1,m:2m+1,0:m+1}),\\
	        \V(c)^{[0.5,1]\times[0.5,1]\times[0,0.5]} &=
	        \Th_{110}\V(c^{0:m+1,m:2m+1,0:m+1}),\\	        
	        \V(c)^{[0,0.5]\times[0,0.5]\times[0.5,1]} &=
	        \Th_{001}\V(c^{0:m+1,0:m+1,m:2m+1}),\\
	        \V(c)^{[0.5,1]\times[0,0.5]\times[0.5,1]} &=
	        \Th_{101}\V(c^{m:2m+1,0:m+1,m:2m+1}),\\
	        \V(c)^{[0,0.5]\times[0.5,1]\times[0.5,1]} &=
	        \Th_{011}\V(c^{0:m+1,m:2m+1,m:2m+1}),\mbox{ and}\\
	        \V(c)^{[0.5,1]\times[0.5,1]\times[0.5,1]} &=
	        \Th_{111}\V(c^{m:2m+1,m:2m+1,m:2m+1}).
	    \end{align*}
        \end{lemma}
    	    
\end{itemize}

\subsection{The Framework} \label{framework}

\begin{itemize}
    \item Let $c$ be a grid of size $2^k+1$ for some $k \in \mathbb{Z}^+\cup \{0\}$, and 
    let $\V$ be a grid-to-volume operator.
    We are interested in constructing a data structure that
    can answer the following query:
    \begin{quote} 
        Given two points $\mathbf{p}_0$ and $\mathbf{p}_1$ in
	    $[0,1] \times [0,1] \times [0,1]$, compute the line
	    integral:
	    \begin{align} \label{grid-volume-line-integral}
	        \| \mathbf{p}_1 - \mathbf{p}_0 \| \int_0^1 
	        (\V(c))(\mathbf{p}_0 + (\mathbf{p}_1 - \mathbf{p}_0)t ) \ \dee t.
	    \end{align}
    \end{quote}
    
    Let us calls this a \emph{volume line integral data structure}
    (VLIDS).
    
    \item We start by showing that a data that computes
    \eqref{grid-volume-line-integral} can be obtained
    from ones that answers a slightly different but easier
    query:
    \begin{quote}
        Given two points $\mathbf{p}_0$ and $\mathbf{p}_1$
        \emph{on the boundary} of $[0,1] \times [0,1] \times [0,1]$,
        compute the line integral:
        \begin{align} \label{grid-surface-line-integral}
	        \| \mathbf{p}_1 - \mathbf{p}_0 \| \int_0^1 
	        (\V(c))(\mathbf{p}_0 + (\mathbf{p}_1 - \mathbf{p}_0)t ) \ \dee t.
	    \end{align}
    \end{quote}
    Let us call this data structure a \emph{surface point 
    line integral data structure} (SPLIDS).
    
    \item A SPLIDS represents a 4D function 
        $s(\mathbf{p_0}, \mathbf{p_1})$
        that maps two points on the boundary of the unit
        cube to the line integral of the segment between
        them. We shall refer to this kind of
        functions as the \emph{surface point line integral 
        functions} (SPLIF).
    
    \item \begin{lemma} \label{hierarchical-splids}
        Let all the input sizes in this lemma be powers of 2.
        If there is a SPLIDS that takes $f(n)$ time to answer 
        a query and takes $g(n)$ total space for 
        a grid of size $n+1$, then there is a VLIDS that 
        takes $$O(f(1) + f(2) + \dotsb + f(2^r) + \dotsb + f(n))$$ 
        time to answer a query and
        $$O(n^3g(1) + n^3g(2)/8 + \dotsb + n^3 g(2^r) / 2^{3r}
        + \dotsb + g(n))$$ total space.        
    \end{lemma}
    
    \begin{proof}
        Note that a grid of size $n+1$ partitions the unit
        cube to $n^3$ cells. We build an octree on these $n^3$
        cells such that each leaf corresponds to a cell.
        For each node in the octree, we build a SPLIDS for
        the volume corresponding to the node. This yields 
        a data structure with 
        $$O(n^3g(1) + n^3g(2)/8 + \dotsb + n^3 g(2^r) / 2^{3r}
        + \dotsb + g(n))$$ space.
        
        Given two points $\mathbf{p}_0$ and $\mathbf{p}_1$
        inside the unit cube, we answer the query as follows.
        From the data structure, we can find a sequence of 
        non-intersecting volumes with the following properties:
        \begin{itemize}
            \item The volumes in the sequence are non-intersecting.
            \item The volumes together covers the line segment
                from $\mathbf{p}_0$ to $\mathbf{p}_1$.
            \item Each volume corresponds to a node in the octree.
            \item There are at most 4 volumes in the sequence
                belonging to each level of the octree.
        \end{itemize}
        The procedure of finding such a sequence of volumes
        is pretty much like tracing a ray from $\mathbf{p}_0$
        to $\mathbf{p}_1$ in an octree. We can then use the 
        SPLIDS for each volume to compute the line integral
        of the part of the segment that goes through the volume,
        and then add all the result together. This process
        thus takes $O(f(1) + f(2) + \dotsb + f(2^r) + \dotsb + f(n))$
        as desired.
    \end{proof}
    
    \item We still need to determine the time needed to create
        all the SPLIDSs according to the data structure
        outlined in the last Lemma. We would prefer not to
        create the SPLID for an octree node from the underlying
        grid data. Instead, we would like to build it from
        the SPLIDSs of the node's children. If we can
        do so, we can characterize the running time of
        the preprocessing phase as follows:
        
        \begin{lemma}
            If the SPLIDS for an octree node corresponding
            to a subvolume of size $n+1$ can be built from
            the 8 SPLIDS of the node's children in time $h(n)$,
            then the whole data structure in 
            Lemma~\ref{hierarchical-splids} can be built
            in $$O(n^3 + n^3h(2)/8 + \dotsb +  
            n^3 h(2^r) / 2^{3r} +\dotsb + h(n)),$$
            assuming that SPLIDS for a grid
            of size $2$ (the leaf node) can be built 
            in constant time.
        \end{lemma}
        
        \begin{proof}
            Just count the number of SPLIDS to be built
            in each level.
        \end{proof}
    
    \item We want an implementation of SPLIDSs allow us
        to accomplish the following tasks:        
        \begin{enumerate}[(i)]
            \item Given a SPLIDS of a volume function
            $V$, compute the SPLIDS of $\Th_{ijk}V$
            for any $i,j,k \in \{0,1\}.$
            
            \item Given SPLIDSs for volume functions
            $V_1$ and $V_2$, compute the SPLIDS of
            $V_1 + V_2$.
        \end{enumerate}        
        
        If we can do so, then we create create the SPLIDS of
        a node from the SPLIDSs of its children. This is
        the result of Lemma~\ref{volume-octant}
        and the fact that the SPLIF is linear in its underlying
        volume. 
    
    \item We aim to make $f(n) = O(1)$ so that
        the lookup time is $O(\log n)$, and $g(n) = h(n) = O(n^2)$
        so that the space used and the precessing time
        is
        $$O(n^3 \cdot 1 + n^3 \cdot 4 / 8 + n^3 \cdot 16 / 64
        + \dotsb + n^3 \cdot 2^{2r} / 2^{3r} + \dotsb + n^2)
        = O(n^3 + n^3/2 + \dotsc + n^3/2^r + \dotsc + n^2)
        = O(n^3)$$
        so that the precomputed data do not take more space
        than the original data.
\end{itemize}

\section{Possible SPLIDS Implementations}

\subsection{Lookup Tables}

\begin{itemize}
    \item We can implement a SPLID as a table of equally
        spaced samples of the SPLIF it represents. 
        Reconstruction involves doing a quadrilinear interpolation
        of the stored samples. Hence, a query can
        be answered in constant time.
    
    \item The implementation would be to define a resolution
        parameter $m$. Then, we put an $m \times m$ grid of 
        sampling points on each face of the cube. This results
        result in $6m^2$ points, and we can make a lookup
        table of size $6m^2 \times 6m^2 = 36m^4$.
        
    \item However, the space requirement can be reduced by
        exploiting basic properties of the SPLIF, which we
        shall denote with $s$.
        
        First, if $\mathbf{p_0}$ and $\mathbf{p}_1$ lie
        on the same face of the cube, then 
        $s(\mathbf{p}_0, \mathbf{p}_1) = 0.$ Thus, there
        is no need to store the sample whose points lie
        on the same face.
        
        Second, the SPLIF is a symmetric function. That is,
        $s(\mathbf{p}_0,\mathbf{p}_1) = s(\mathbf{p}_1,\mathbf{p}_0).$
        Hence, we can totally order the faces and store
        only those samples that go from a ``lower'' face to 
        ``higher'' faces.
        
        Exploting the above properties, we store the sample
        in ${6 \choose 2} = 15$ tables of size $m^2 \times m^2$.
        The space requirement is then $15m^4$.
        
    \item To facilitate interpolation, we recommend positioning
        the grid of sample points so that the grid's edge coincide 
        with the cube's face edge. In this way, we have that for 
        any other points on the face, we can always find 4 
        sampling points forming a rectangle around that point.                    
    \item In conclusion, the loopup table takes $O(1)$ to answer
        a query and $O(m^4)$ space. The space requirement makes
        it impractical. That is, for a volume grid of size $n$,
        it would be reasonable to set $m = \Theta(n)$, 
        but $O(n^4)$ space for the SPLIDS is not reasonable at all.
\end{itemize}

\subsection{Factored Matrices}

\begin{itemize}
    \item In the last section, a SPLIDS is represented by
        $15$ lookup tables of size $m^2 \times m^2$. We
        can think of each lookup table as a matrix $A$ and
        approximate it with a rank-$r$ approximation
        $A \approx UV^T$ where $U,V \in \mathbb{R}^{m^2 \times p}$
        for some $r << m^2.$
        
    \item Using this representation, a lookup takes $O(r)$
    and the space requirement is $O(m^2r)$, which is a huge
    saving from $O(m^4)$.
    
    \item One drawback is that this representation is that
        the reconstruction would not be accurate if the underlying
        SPLIF is not a low rank signal. However, we believe
        that, if the data in the volume grid is low in frequency,
        the SPLIF should be represented well by a low rank
        approximation.
        
    \item In order to use this SPLIDS implementation in the
    framework outlined in Section~\ref{framework}, we
    must be able to
    \begin{enumerate}[(a)]
        \item Construct a SPLIDS for a grid of size 2.
        \item Combine 8 SPLIDSs for octants of a volume
            into the SPLIDS for the volume itself
            (which involves task (i) and (ii) in
            Section~\ref{framework}).            
    \end{enumerate}
    
    \item Requirement (a) is easy to satisfy. We may precompute
        the full lookup tables for $V_{000}$, $V_{001}$, $\dotsc$,
        $V_{111}$ for some small resolution parameter $m$, and
        then compute a linear combination of them to get the full
        lookup table for any grid of size 2. We can then factor
        the resulting matrix.
        
    \item Requirement (b) is harder to satisfy as factored matrices
        cannot be directly added together. Moreover, we don't know 
        how to compute the factored matrix of $\Th_{ijk}V$
        from the factored matrix of $V$ without expanding
        the matrix and multiplying it with the matrix 
        associated with $\Th_{ijk}$, which is much bigger 
        (its size is $O(m^8)$).
        
    \item Rather, we propose satisfying Requirement (b) by
        a data-driven approach. We view computing the SPLIDS as 
        a matrix approximation problem. In literature, 
        there are already a number of algorithms that can 
        approximate a matrix by sampling a few of its rows
        and columns. \cite{kernel-nystrom, optical-computing}.
        
    \item More specifically, if we have 8 SPLIDS of children of 
        an octree node, then we can evaluate the SPLIF of
        the parent by breaking the line segment connecting
        the end points into parts, each of which intersects
        only one child. We can then use the SPLIDS of the 
        children to evalute the line integral for each
        segment and add the results up to get the parent's
        SPLIF value. This evaluation can be done in $O(r)$ time.        
        Hence, it is possible to sample a row or a column
        of the parent's SPLIF matrix in $O(m^2r)$ time.
        
    \item The Nystrom method \cite{kernel-nystrom}, for example,
        samples $r$ rows and $r$ columns of the matrix
        to get a rank-$r$ approximation. In our settings,
        these sampling can be done in $O(m^2 r^2)$ time.
        Additional $O(m^2 r^2 + r^2)$ time is required to
        produced the low rank approximation. (This includes
        multiplying an $r \times m^2$ matrix to a $m^2 \times r$ 
        matrix, and finding the pseudo-inverse of an $r \times r$
        matrix.)
        
    \item Therefore, if we set the resolution parameter $m$
        to be $O(n)$, then we have $f(n) = O(r)$,
        $g(n) = O(n^2 r)$, and $h(n) = O(n^2 r^2 + r^3).$
        Since $r$ is a constant, we have an implement
        of SPLIDS that satisfies all the goals we were
        aiming for in Section~\ref{framework}.
        
    \item However, one worrying aspect of this implementation
        is that we do not know how to assess the accuracy of
        the implementation.        
\end{itemize}

\subsection{Spherical Harmonic Reflection Map}

\begin{itemize}
    \item The spherical harmonic reflection map (SHRM)
        was invented by Ramamoorthi and Hanrahan to
        represent isotropic BRDFs \cite{shrm}. 
        An SHRM is a cube map whose entries are 
        spherical harmonoics. In other words, 
        each sample point on the surface of the
        cube stores a 2D function. Since the SHRM
        is capable of representing any 4D function, it
        can be used to represent SPLIF as well.
        
    \item We use the SHRM to store the SPLIF as follows.
        Let the cube map have resolution $m \times m$
        where $m$ is the resolution parameter. For each
        sample point $\mathbf{p}$ on a cube's face,
        we store the spherical function
        $$Y_{\mathbf{p}}(\omega) = s(\mathbf{p}, l(\mathbf{p}, \omega))$$
        where $l(\mathbf{p},\omega)$ is the first point
        that the ray $\mathbf{p} + t\omega$ intersects
        the unit cube. 
    
    \item Evaluating $s(\mathbf{p}_0, \mathbf{p}_1)$ involves
        finding 4 sample points $\mathbf{p}_{BL}$,
        $\mathbf{p}_{BR}$, $\mathbf{p}_{TL}$, $\mathbf{p}_{TR}$
        forming a rectangle around $\mathbf{p}_0$, and then
        doing a bilinear interpolation between
        \begin{align*}
            &Y_{\mathbf{p}_{BL}}(\mathbf{p}_{BL}, (\mathbf{p}_1 - \mathbf{p}_{BL})/\| \mathbf{p}_1 - \mathbf{p}_{BL} \|),\\
            &Y_{\mathbf{p}_{BR}}(\mathbf{p}_{BR}, (\mathbf{p}_1 - \mathbf{p}_{BR})/\| \mathbf{p}_1 - \mathbf{p}_{BR} \|),\\
            &Y_{\mathbf{p}_{TL}}(\mathbf{p}_{TL}, (\mathbf{p}_1 - \mathbf{p}_{TL})/\| \mathbf{p}_1 - \mathbf{p}_{TL} \|), \mbox{ and}\\
            &Y_{\mathbf{p}_{TR}}(\mathbf{p}_{TR}, (\mathbf{p}_1 - \mathbf{p}_{TR})/\| \mathbf{p}_1 - \mathbf{p}_{TR} \|).
        \end{align*}
        
    \item If we use a spherical harmonic representation of order
        $q$, then the representation uses $6m^2 q^2$ floating points,
        and a query can be answered in $O(q^2)$ time.
        
    \item One nice thing about the SHRM is that, if have
        an SHRM that represents the SPLIF of $V_1$ and
        another that represents the SPLIF of $V_2$,
        we can compute the SHRM that represents the 
        SPLIF of $V_1 + V_2$ by just adding the two SHRMs together.
        This makes conbining children's SPLIDS when creating
        parent's SPLIDS very easy.
        
    \item Moreover, suppose we have an SHRM representation
        of the SPLIF of $V$, and this representation is at
        resolution $m'$. Then, there is a linear 
        operator that transforms it to the SHRM representation 
        of the SPLIF of $\Th_{ijk} V$ at resolution $m$.
        This linear operator is a matrix of dimension 
        $m^2 q \times m'^2 q$. Let us denote this matrix
        by $\mathbf{TH}_{ijk}^{m' \rightarrow m}$. 
        We can precompute them for all $i,j,k$ and various
        $m'$ and $m$.
        
    \item Naively storing $\mathbf{TH}_{ijk}^{m' \rightarrow m}$
         uses $O(m^2 m'^2 q^4)$ space, which is not scalable
         if $m$ and $m'$ are large. To alleviate this problem
         by storing and using its rank-$r$ approximation. 
         This reduces the space requirement to $O((m^2 + m'^2)q^2 r)$.
         It also reduces the time requirement of 
         transforming the SHRM to $O((m^2 + m'^2)q^2 r)$ as well.
         
    \item All in all, setting $m = O(n)$, 
        we have that $f(n) = O(q^2)$, $g(n) = O(n^2 q^2)$,
        and $h(n) = O(n^2 q^2 r)$. We thus have another
        representation that satisfies all the requirements
        of the framework in Section~\ref{framework}.
        
    \item However, this SPLIDS implementation can only represent low frequency SPLIF, and we don’t know
how to quantify the error it introduces.
\end{itemize}


\bibliography{precomputed-line-integrals}{}
\bibliographystyle{plain}

\end{document}