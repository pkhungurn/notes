\documentclass[10pt]{article}
\usepackage{fullpage}
\usepackage{amsmath}
\usepackage[amsthm, thmmarks]{ntheorem}
\usepackage{amssymb}
\usepackage{graphicx}
\usepackage{epstopdf}
\usepackage{enumerate}
\usepackage{verse}

\newtheorem{lemma}{Lemma}[section]
\newtheorem{theorem}[lemma]{Theorem}
\newtheorem{definition}[lemma]{Definition}
\newtheorem{proposition}[lemma]{Proposition}
\newtheorem{claim}[lemma]{Claim}

\newcommand{\dee}{\mathrm{d}}
\newcommand{\In}{\mathrm{in}}
\newcommand{\Out}{\mathrm{out}}
\newcommand{\pdf}{\mathrm{pdf}}
\newcommand{\Ei}{\mathrm{Ei}}
\newcommand{\Tri}{\mathbf{tri}}
\newcommand{\Th}{\mathbf{th}}
\newcommand{\V}{\mathbf{V}}
\newcommand{\Const}{\mathbf{const}}
\newcommand{\exit}{\mathrm{exit}}
\newcommand{\solve}{\mathrm{solve}}

\newcommand{\IF}{\mathbf{if}}
\newcommand{\FI}{\mathbf{fi}}
\newcommand{\THEN}{\mathbf{then}}
\newcommand{\ELSE}{\mathbf{else}}
\newcommand{\WHILE}{\mathbf{while}}
\newcommand{\DO}{\mathbf{do}}
\newcommand{\OD}{\mathbf{od}}
\newcommand{\PROC}{\mathbf{proc}}
\newcommand{\END}{\mathbf{end}}
\newcommand{\CALL}{\mathbf{call}}

\newcommand{\ve}[1]{\mathbf{#1}}

\title{Implementing the Marschner's Hair Scattering Model}
\author{Pramook Khungurn}

\begin{document}
	\maketitle

  The Marshner's hair scattering model \cite{Marschner:2003} 
  is a complex piece of 
  mathematical modeling that has a lot of details.
  This document attempts to iron out all the components of 
  the model, so that the reader can easily implement it
  in a program.
  
  \section{Azimuthal Scattering}
  
  \subsection{Bravais Index of Refraction}
  
  \begin{itemize}
    \item The {\bf Bravais index of refraction} determines the
      effective index of refraction when refraction in
      3D is projected on the normal plane of the hair fiber.
      
    \item Let $\eta$ be the index of refraction of the hair
      material. Suppose that the light strikes the hair at 
      incoming direction, which makes an angle
      $\theta_d$ with the normal plane.
      Then, the Bravias index of refraction 
      is given by:
      \begin{align*}
        \eta' = \frac{\sqrt{\eta^2 - \sin^2 \theta_d}}{\cos \theta_d}
      \end{align*}
      
    \item When computing Fresnel reflectance, we need another
      index of refraction, which we shall call the 
      {\bf anti-Bravais index of refraction}. It is given by:
      \begin{align*}
        \eta'' = \frac{\eta \cos \theta_d}{\sqrt{1 - \eta^{-2} \sin^2 \theta_d}}
      \end{align*}
    
    \item Using the two indices, the Fresnel reflectance, resulting
      from light coming in at angle $\gamma_i$ (made with the normal)
      and refracted to the angle $\gamma_t$ (again, made with the 
      normal), is given by
      \begin{align*} 
        F = \frac{1}{2}(F_p^2 + F_s^2)      
      \end{align*}
      where
      \begin{align*}
        F_p &= \frac{\eta'' \cos \gamma_i - \cos \gamma_t}{\eta'' \cos \gamma_i + \cos \gamma_t}, \mbox{ and}\\
        F_s &= \frac{\cos \gamma_i - \eta' \cos \gamma_t}{\cos \gamma_i + \eta' \cos \gamma_t}.
      \end{align*}
  \end{itemize}
  
  \subsection{Height Parameterization, Transmitted Angle, 
    and Exit Angles}
    
  \begin{itemize}
    \item Marschner models light arriving at a hair fiber's cross
      section as being distributed through the ``height'' of the
      cross section. The height is denoted by the variable $h$,
      and it has value ranging from $-1$ to $1$.
      
    \item The angle $\gamma_i$ that the light's direction make
      with the normal is given by:
      $$\gamma_i = \sin^{-1}(h).$$
      
    \item The transmitted angle $\gamma_t$ that the light direction
      make with the normal is given by:
      $$\gamma_t = \sin^{-1}(h / \eta').$$
      This is derived from Snell's law:
      \begin{align*}
        \sin \gamma_i = \eta' \sin \gamma_t.
      \end{align*}
      
    \item Given the scattering mode $p = 0, 1, 2, \ldots$
      ($p = 0$ is the R mode, $p = 1$ is the TT mode,
      and $p = 2$ is the TRT mode) and height $h$, 
      the azimuthal angle that the light particle exits 
      the hair fiber is given by:
      \begin{align*}
        \phi_\exit(\theta_d, p, h) 
        &= 2\gamma_i - 2p\gamma_t + p\pi\\
        &= 2\sin^{-1}(h) - 2p\sin^{-1}(h/\eta') + p\pi.
      \end{align*}
      We put $\theta_d$ in the function because $\eta'$
      depends on $\theta_d$.
      
      \item The model requires us to compute the derivative
        of the exit angle with respect to $h$. The derivatives
        for $p = 0$, $1$, and $2$ are given below:
        \begin{align*}
          \frac{\dee \phi_\exit(\theta_d, 0,h)}{\dee h}
          &= \frac{2}{\sqrt{1-h^2}},\\
          \frac{\dee \phi_\exit(\theta_d, 1,h)}{\dee h}
          &= \frac{2}{\sqrt{1-h^2}} - \frac{2}{\eta' \sqrt{1-(h/\eta')^2}},\\
          \frac{\dee \phi_\exit(\theta_d, 2,h)}{\dee h}
          &= \frac{2}{\sqrt{1-h^2}} - \frac{4}{\eta' \sqrt{1-(h/\eta')^2}}.\\
        \end{align*}
  \end{itemize}
  
  \subsection{Azimuthal Scattering Function}
  
  \begin{itemize}
    \item The azimuthal scattering function is given in 
      terms of the longitudinal angle difference 
      $\theta_d = |\theta_i - \theta_r|/2$,
      the azimuthal difference $\phi_d = \phi_r - \phi_i$:
      $$N(\theta_d, \phi_d).$$
      It is the sum of the azimuthal scattering functions
      of different modes:
      \begin{align*}
        N(\theta_d, \phi_d) = 
          N_0(\theta_d, \phi_d) + N_1(\theta_d, \phi_d) + N_2(\theta_d, \phi_d).
      \end{align*}
  \end{itemize}
  
  \subsection{Approximation for R and TT Mode}
  
  \begin{itemize}    
    \item For $p = 0$ and $p = 1$, the azimuthal scattering
      function is found using the following procedure.
      Given $\phi_i$ and $\phi_r$, we find $h$ such that
      $\phi_\exit(\theta_d, p, h) = \phi_d.$ Let us 
      call this $h_\solve(\theta_d, p, \phi_d)$.
      
    \item For $p = 0$, we have that
      \begin{align*}
        h_\solve(\theta_d, 0, \phi)d) 
        = \sin \bigg( \frac{\phi_d}{2} \bigg).
      \end{align*}
      
    \item For other $p$, we have to solve the equation:
      \begin{align*}
        \phi_d
        &= 2 \sin^{-1}(h) - 2p\sin^{-1}(h/\eta') + p\pi\\
        &= 2 \gamma_i - 2p\gamma_t + p\pi.
      \end{align*}
      Marschner makes the following approximation:
      \begin{align*}
        \gamma_t = \frac{3\sin^{-1}(1/\eta')}{\pi} \gamma_i
          - \frac{4\sin^{-1}(1/\eta')}{\pi^3} \gamma_i^3.
      \end{align*}
      The equation then becomes:
      \begin{align*}
        \phi_d
        &= 2\gamma_i 
        - \frac{6 p\sin^{-1}(1/\eta')}{\pi} \gamma_i 
        + \frac{8 p\sin^{-1}(1/\eta')}{\pi^3} \gamma_i^3 + p\pi\\
        &= \frac{8 p \sin^{-1}(1/\eta')}{\pi^3} \gamma_i^3
        + \bigg( 2 - \frac{6 p\sin^{-1}(1/\eta')}{\pi} \bigg) \gamma_i
        + p\pi
      \end{align*}
      which we solve for the root.
      
    \item Once we have $h_\solve(\theta_d, p, \phi_d)$,
      we have that the azimuthal scattering function is given by
      \begin{align*}
        N_p(\theta_d, \phi_d) =
          A(\theta_d, p, h_\solve(\theta_d, p, \phi_d))
          \bigg| 2 \frac{\dee}{\dee h } 
            \phi_\exit(\theta_d, p, h_\solve(\theta_d, p, \phi_d) ) \bigg|^{-1}.
      \end{align*}
      
    \item The $A(\theta_d, p, h)$ function gives the attenuation
      due to absorption by the material of the hair. It is given
      by
      \begin{align*}
        A(\theta_d, p, h) = \begin{cases}
          F(\eta', \eta'', \gamma_i), & \mbox{if }p = 0\\
          (1-F(\eta', \eta'', \gamma_i))^2 F( \frac{1}{\eta'}, \frac{1}{\eta''}, \gamma_t )^{p-1} ( \exp(2\sigma_a \cos \gamma_t ))^p, 
          & \mbox{if }p > 0
        \end{cases}
      \end{align*}
      where
      \begin{itemize}
        \item $\sigma_a$ is the absorption coefficient of
          the hair material,
        \item $F(\eta', \eta'', \gamma_i)$ is the Fresnel 
          reflectance:
          \begin{align*}
            F(\eta', \eta'', \gamma_i) = 
            \frac{1}{2} \bigg( \frac{\eta'' \cos \gamma_i - \cos \gamma_t}{\eta'' \cos \gamma_i + \cos \gamma_t} \bigg)^2
            + \frac{1}{2} \bigg( \frac{\cos \gamma_i - \eta' \cos \gamma_t}{\cos \gamma_i + \eta' \cos \gamma_t}  \bigg)^2.
          \end{align*}          
        \item $F(1/\eta', 1/\eta'', \gamma_t)$ is the ``anti''-Fresnel reflectance:
          \begin{align*}
            F\bigg( \frac{1}{\eta'}, \frac{1}{\eta''}, 
            \gamma_t \bigg) 
            &= 
            \frac{1}{2} \bigg( \frac{ \frac{1}{\eta''} \cos \gamma_t - \cos \gamma_i}{\frac{1}{\eta''} \cos \gamma_t + \cos \gamma_i} \bigg)^2
            + \frac{1}{2} \bigg( \frac{\cos \gamma_t - \frac{1}{\eta'} \cos \gamma_i}{\cos \gamma_t + \frac{1}{\eta'} \cos \gamma_i}  \bigg)^2\\
            &= 
            \frac{1}{2} \bigg( \frac{ \cos \gamma_t - \eta'' \cos \gamma_i}{ \cos \gamma_t + \eta'' \cos \gamma_i} \bigg)^2
            + \frac{1}{2} \bigg( \frac{\eta' \cos \gamma_t - \cos \gamma_i}{\eta' \cos \gamma_t + \cos \gamma_i}  \bigg)^2\\
            &= F(\eta', \eta'', \gamma_i).
          \end{align*}
      \end{itemize}
            
    \item Note that, although we don't use this formula for $p = 2$,
      it is a component of the TRT approximation which is described
      in the next section.
      
  \end{itemize}
  
  \subsection{Approximation for TRT Mode}
  
  \begin{itemize}
    \item The TRT generates two caustics at 
      \begin{align*}
        h_c = \sqrt{ \frac{4 - \eta'^2}{3} }
      \end{align*}
      and at $-h_c$. 
      Therefore, if $\eta' \geq 2$, there will be no caustics.
      
    \item The Marschner model differentiates between the following
      two cases:
      \begin{itemize}
        \item If $\eta' < 2$, it tries to generate blurred 
          caustics at $\phi_c$ and $-\phi_c$
          
        \item If $\eta' \geq 2$, it tries to generate a caustic
          at $\phi_c = 0$ and fade it out smoothly.
      \end{itemize}
  \end{itemize}
  
  \subsubsection{Approximation for $\eta' < 2$ Case}
  
  \begin{itemize}
    \item First of all, we cannot evaluate $N_2(\theta_d, \phi_d)$ directly. Because, if $\phi_d = \pm \phi_c$, then $N_2(\theta_d, \phi_d)$ is undefined.
    
    \item Instead, in case $\phi_d = \pm \phi_c$, the Marschner
      model replaces $N_p(\theta_d, \phi_d)$ by the sum of
      two Gaussians:
      \begin{align}
        k_G A(\theta_d, 2, h_c) \Delta h (g(\phi_d; \phi_c, w_c) +
        g(\phi_d; -\phi_c, w_c)).
        \label{caustic-value}
      \end{align}
      The variables in the above equations are:
      \begin{itemize}
        \item $k_G$ is the caustic lobes scaling factor. Its value
          has range $0.5$ to $5$.
        \item $w_c$ is the width of the caustic lobe. It has
          value typically from $10^\circ$ to $25^\circ$.
        \item $g(x;\mu,\sigma)$ is the Gaussian distribution:
          \begin{align*}
            g(x;\mu,\sigma) = \frac{1}{\sigma \sqrt{2\pi}}
            \exp\bigg(-\frac{1}{2}\bigg( \frac{x-\mu}{\sigma} \bigg)^2 \bigg).
          \end{align*}
        \item $h_c$ is the $h$ value that corresponds to the
          position of the caustics: $h_c = h_\solve(\theta_d, 2, \phi_c).$
        \item $\Delta h$ is the estimate the width 
          of the interval of $h$ that maps 
          within $w_c$ of $\phi_c$:
      \begin{align}
        \Delta h = \min \bigg\{ \Delta h_M, 
        2 \sqrt{2w_c \Big| \frac{\dee^2 }{\dee h^2} \phi_\exit(\theta_d, 2, h_c) \Big|^{-1}  } \bigg\}
        \label{delta-h}
      \end{align}
      where $\Delta h_M$ is the maximum width of the $h$-interval,
      which is a parameter that the user has to set manually.
      \end{itemize}      
    \item To evaluate the above interval width, we need to evaluate
      the second derivative of the $\phi_\exit$ function,
      which is given below:
      \begin{align*}
        \frac{\dee \phi_\exit(\theta_d, 2, h) }{\dee h^2}
        &= \frac{\dee}{\dee h} \bigg\{ \frac{2}{\sqrt{1-h^2}} - \frac{4}{\eta' \sqrt{1-(h/\eta')^2}} \bigg\}\\
        &= 2 \frac{\dee}{\dee h}\bigg\{ \frac{1}{\sqrt{1 - h^2}} \bigg\}
        - \frac{4}{\eta'} \frac{\dee}{\dee h}\bigg\{ 
          \frac{1}{\sqrt{1 - (h/\eta')^2}}\bigg\}\\
        &= 2 \bigg( -\frac{1}{2} (1 - h^2)^{-3/2} (-2h) \bigg)
          - \frac{4}{\eta'} \bigg( -\frac{1}{2} (1 - (h/\eta')^2)^{-3/2} \bigg( - \frac{2h}{\eta'^2} \bigg) \bigg)\\
        &= \frac{2h}{(1-h^2)^{3/2}} - \frac{4h}{\eta'^3 (1 - (h/\eta')^2)^{3/2}}.
      \end{align*}
      
    \item In case, $\phi_d \neq \pm \phi_c$, the Marschner model
      attempts to interpolate smoothly between the value
      in \eqref{caustic-value} and $N_2(\theta_d, \phi_d)$.
      This is done by setting the azimuthal scattering function to:
      \begin{align*}
        N_2(\theta_d, \phi_d) \bigg(1 - \frac{g(\phi_d;\phi_c, w_c)}{g(\phi_c; \phi_c, w_c)}\bigg) \bigg( 1 - \frac{g(\phi_d;-\phi_c,w_c)}{g(-\phi_c;\phi_c,w_c)} \bigg) + 
        k_G A(\theta_d, 2, h_c) \Delta h (g(\phi_d; \phi_c, w_c) +
        g(\phi_d; -\phi_c, w_c)).
      \end{align*}
      The intuition is that, when $\phi_d$ is near $\pm \phi_c$,
      the contribution of \eqref{caustic-value} should be
      very strong. Otherwise, the constribution of 
      $N_2(\theta_d, \phi_d)$ should be very strong.
  \end{itemize}
  
  \subsubsection{Approximation for $\eta' \geq 2$ Case}
  
  \begin{itemize}
    \item The value should be $N_2(\theta_d, \phi_d)$.
      
    \item However, the caustic lobes given in \eqref{caustic-value}
    are present and fully used in the values $\eta' = 2-\varepsilon$.
    So, we should not use $N_2(\theta_d, \phi_d)$, but we should 
    face out the caustics as $\eta'$ increases.
    
    \item The Marschner model introduces an interpolation parameter
      $t$, which is $1$ if $\eta' < 2$, and $0$ if $\eta' > 2 + \Delta \eta'$:
      \begin{align*}
        t = \mathrm{smoothstep}(2, 2+\Delta \eta', \eta').
      \end{align*}
      Here, $\Delta \eta'$ is a parameter that the user sets.
      Its value is from $0.2$ to $0.4$. The $\mathrm{smoothstep}(a,b,x)$ is $1$ if $x < a$, and $0$ if $x > b$,
      and smooth between.
      
    \item The value of the azimuthal scattering function is given by:
    \begin{align*}
        N_2(\theta_d, \phi_d) \bigg(1 - t\frac{g(\phi_d;\phi_c, w_c)}{g(\phi_c; \phi_c, w_c)}\bigg) \bigg( 1 - t\frac{g(\phi_d;-\phi_c,w_c)}{g(-\phi_c;\phi_c,w_c)} \bigg) + 
        t k_G A(\theta_d, 2, h_c) \Delta h (g(\phi_d; \phi_c, w_c) +
        g(\phi_d; -\phi_c, w_c)).
      \end{align*}
    where $\phi_c$ should be set to 0 (because the caustic has
      merged), and $\Delta h$ should always be set to $\Delta h_M$.
  \end{itemize}
  
  \subsubsection{Pseudocode for the TRT Mode}
  
  We reproduce the pseudocode for the TRT mode here for completeness.
  
  \begin{verse}
    $\mathbf{function}\ N_{TRT}(\eta, \theta_d, \phi_d; w_c, k_G, \Delta \eta', \Delta h_M)$\\
    \quad $\eta' \gets \sqrt{\eta - \sin^2 \theta_d}/\cos \theta_d $\\
    \quad $\IF\ \eta' < 2\ \THEN$ \\
    \quad \quad Compute $h_c$, $\phi_c$ using $\eta'$.\\
    \quad \quad Compute $\Delta h$ using \eqref{delta-h}.\\
    \quad \quad $t \gets 1$\\
    \quad $\ELSE$\\
    \quad \quad $\phi_c \gets 0$\\
    \quad \quad $\Delta h \gets \Delta h_M$\\
    \quad \quad $t \gets \mathrm{smoothstep}(2, 2+\Delta\eta', \eta')$\\
    \quad $\FI$\\
    \quad $L \gets N_2(\theta_d, \phi_d)$\\
    \quad $L \gets L \cdot (1 - t g(\phi_d; \phi_c, w_c) / g(\phi_c;\phi_c, w_c)) (1 - t g(\phi_d; -\phi_c, w_c) g(-\phi_c; -\phi_c, w_c))$\\
    \quad $L \gets L + t k_G A(\theta_d, 2, h_c) \Delta h (g(\phi_d; \phi_c, w_c) + g(\phi_d; -\phi_c, w_c))$\\
    \quad $\mathbf{return}\ L$\\
    $\mathbf{end}$\\
  \end{verse}

  \subsubsection{Approximation for Eccentricity}
  
  \begin{itemize}
    \item The paper claims that the TRT caustics is affected
      by the slightest deviation in the eccentricity. 
      
    \item The Marschner model simulates the effect of eccentricity
      to the TRT mode by changing the index of refraction.
      
    \item Let $a$ be the cross section's eccentricity.
      The model computes the index of refraction $\eta^*(\phi_h)$
      to be fed to the procedure $N_{TRT}$ as follows:
      \begin{align*}
        \eta^*(\phi_h) = \frac{1}{2}((\eta^*_1 + \eta^*_2) 
          + \cos(2\phi_h)(\eta^*_1 - \eta^*_2))
      \end{align*}
      where
      \begin{itemize}
        \item $\phi_h = (\phi_i + \phi_r)/2$ is the \emph{azimuthal half angle}.
        \item $\eta^*_1$ and $\eta^*_2$ are given by:
        \begin{align*}
          \eta^*_1 &= 2(\eta-1)a^2 - \eta +2,\mbox{ and}\\
          \eta^*_2 &= 2(\eta-1)a^{-2} - \eta + 2.
        \end{align*}
      \end{itemize}
  \end{itemize}
  
  \section{Longitudinal Scattering Function}
  
  \begin{itemize}
    \item The longitudinal scattering functions of the
      Marschner models are very simple: they are Gaussians
      of the half azimuthal angle 
      $\theta_h = (\theta_i + \theta_r)/2$:
      \begin{align*}
        M_R(\theta_h) &= g(\theta_h; -\alpha_R, \beta_R)\\
        M_{TT}(\theta_h) &= g(\theta_h; -\alpha_{TT}, \beta_{TT})\\
        M_{TRT}(\theta_h) &= g(\theta_h; -\alpha_{TRT}, \beta_{TRT})
      \end{align*}
      where
      \begin{itemize}
        \item $\alpha_R$ is the shift of the R mode's 
          reflection peak,
        \item $\beta_R$ is the width of 
          the R mode's scattering lobe,
        \item $\alpha_{TT} = \alpha_R / 2$,
        \item $\beta_{TT} = \beta_R/2$,
        \item $\alpha_{TRT} = 3\alpha_R/2$, and
        \item $\beta_{TRT} = 2\beta_R$.
      \end{itemize}
  \end{itemize}
  
  \section{The Whole Scattering Model}
  
  The whole scattering model is given by the following equation:
  \begin{align*}
    S(\theta_i, \phi_i, \theta_r, \phi_r)
    &= M_R(\theta_h) N_0(\theta_d, \phi_d) / \cos^2 \theta_d\\
    &\phantom{{}={}}+ M_{TT}(\theta_h) N_1(\theta_d, \phi_d) / \cos^2 \theta_d\\
    &\phantom{{}={}} 
    + M_{TRT}(\theta_h) N_{TRT}(\eta^*(\phi_i), \theta_d, \phi_d) / \cos^2\theta_d  
  \end{align*}
  
  \section{Solving Cubic Equation}
  
  \begin{itemize}
    \item Computing $h_\solve(\theta_d, p, \phi_d)$ requires us to
      solve a cubic equation of the form $x^3 + Ax + B = 0$. 
      We reproduce the method to solve it here for completeness.
      
    \item We would like to find two numbers $s$ and $t$ such that
      \begin{align*}
        3uv &= A,\mbox{ and}\\
        u^3 - v^3 &= -B.
      \end{align*}
      If we find such a number, we have that
      $x = u - v$ is a solution to the equation. This is because
      \begin{align*}
        (u-v)^3 + 3uv(u-v) + v^3 - u^3 = 0.
      \end{align*}
      
    \item To find $u$ and $v$, we note that $u = A/3v$, and we have
      that
      \begin{align*}
        \bigg( \frac{A}{3v} \bigg)^3 - v^3 &= -B\\
        \frac{A^3}{27v^3} - v^3 &= -B\\
        A^3 - 27v^6 &= -27Bv^3\\
        -27 v^6 + 27 B v^3 + A^3 &= 0
      \end{align*}
      which is a quadratic equation of $v^3$. So, we solve for $v^3$,
      and then $u$, and then x.
      
  \end{itemize}
  
\bibliographystyle{plain}
\bibliography{marschner-model}	

\end{document}