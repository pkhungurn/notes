\documentclass[10pt]{article}
\usepackage{fullpage}
\usepackage{amsmath}
\usepackage[amsthm, thmmarks]{ntheorem}
\usepackage{amssymb}
\usepackage{graphicx}
\usepackage{epstopdf}
\usepackage{enumerate}
\usepackage{verse}
\usepackage{url}
\usepackage{bm}

\newtheorem{lemma}{Lemma}[section]
\newtheorem{theorem}[lemma]{Theorem}
\newtheorem{definition}[lemma]{Definition}
\newtheorem{proposition}[lemma]{Proposition}
\newtheorem{corollary}[lemma]{Corollary}
\newtheorem{claim}[lemma]{Claim}

\newcommand{\dee}{\mathrm{d}}
\newcommand{\In}{\mathrm{in}}
\newcommand{\Out}{\mathrm{out}}
\newcommand{\pdf}{\mathrm{pdf}}

\newcommand{\ve}[1]{\bm{#1}}
\newcommand{\mrm}[1]{\mathrm{#1}}
\newcommand{\etal}{{et~al.}}
\newcommand{\sphere}{\mathbb{S}^2}
\newcommand{\modeint}{\mathcal{M}}
\newcommand{\azimint}{\mathcal{N}}
\newcommand{\ra}{\rightarrow}
\newcommand{\mcal}[1]{\mathcal{#1}}
\newcommand{\likelihood}{\mathcal{L}}
\newcommand{\X}{\mathcal{X}}
\newcommand{\Y}{\mathcal{Y}}
\newcommand{\Z}{\mathcal{Z}}
\newcommand{\x}{\mathbf{x}}
\newcommand{\y}{\mathbf{y}}
\newcommand{\z}{\mathbf{z}}
\newcommand{\tr}{\mathrm{tr}}
\newcommand{\sgn}{\mathrm{sgn}}
\newcommand{\diag}{\mathrm{diag}}
\newcommand{\new}{\mathrm{new}}
\newcommand{\erf}{\mathrm{erf}}

\title{Ray Tracing Generalized Cylinders}
\author{Pramook Khungurn}

\begin{document}
	\maketitle
	
	This document is written as a companion in reading the paper ``Visible Light and X-Ray Ray Tracing of Generalized Cylinders'' by Hsu and Chelberg \cite{hsu:1994}.
	
\section{Generalized Cylinders}\label{sec:generalized_cylinders} % (fold)

\begin{itemize}
  \item A \emph{generalized cylinder} is a volumetric solid generated by sweeping an arbitrarily shaped closed 2D curve along an arbitrary three-dimensional space curve. The space curve is called the \emph{axis}.
  
  \item We denote the axis with the function $\ve{a}(u) = (a_x(u), a_y(u), a_z(u))$ where $u_i \leq u \leq u_f$.
  
  \item We assume that $\ve{a}'(u) \neq 0$ for all $u \in [u_i, u_f]$.
  
  \item The \emph{Frenet frame} gives an orientation of the axis curve at each point along it. The frame depends only on the local shape of the curve, not on the parameterization and the coordinate system in which the curve is defined. The frame consists of three orthogonal unit vector:
  \begin{align*}
    \hat{\ve{t}}(u) &= \frac{\ve{a}'(u)}{\| \ve{a}'(u) \|} & & \mbox{(the tangent)}\\
    \hat{\ve{b}}(u) &= \frac{\hat{\ve{t}}(u) \times \ve{a}''(u)}{\| \hat{\ve{t}}(u) \times \ve{a}''(u) \|} & & \mbox{(the binormal vector)}\\
    \hat{\ve{n}}(u) &= \hat{\ve{b}}(u) \times \hat{\ve{t}}(u) & & \mbox{(the normal vector)}
  \end{align*}
  
  \item The plane spanned by the normal vector and the binormal vector is called the \emph{normal plane}. It is in this normal plane that the cross section is defined.
  
  We will use $\hat{\ve{n}}(u)$ and $\hat{\ve{b}}(u)$ as the first and the second coordinate axis of the normal plane.
  
  \item The cross section is defined as a closed curve $\ve{cs}(v) = (cs_n(v), cs_b(v))$ where $v_i \leq v \leq v_f$.
  
  \item For each $u$, the \emph{contour} of the generalized cylinder is given by the curve $$\ve{c}(u,v) = (c_n(u,v), c_b(u,v)) = \ve{s}(\ve{cs}(v), u)$$ where $\ve{s}$ is the sweeping rule that specifies how the cross section changes along the axis of the genalized cylinder.
  
  \item The generalized cylinder is given by:
  \begin{align*}
    \ve{\Gamma}(u,v) = \ve{a}(u) + c_n(u,v)\hat{\ve{n}}(u) + c_b(u,v)\hat{\ve{b}}(u).
  \end{align*}
  
  \item One problem that might come up is that the Frenet frame is not defined when $\hat{\ve{t}}(u) \times \ve{a}''(u) = \ve{0}.$ This happens only when $\ve{a}(u)$ is locally a straight line at $u$. That is, either $\ve{a}(u)$ is a straight line or $u$ is an inflection point of the axis curve. The paper describes an implementation issue arising from this, and we shall cover it.
  
\end{itemize}

% section section_name (end)

\section{Ray Tracing Generalized Cylinders}\label{sec:ray_tracing_generalized_cylinders} % (fold)

\begin{itemize}
  \item The first paper that deals with ray tracing of generalized cylinders is by Bronsvoort and Klok \cite{Bronsvoort:1985}. This work is later extended to allowing the cross section to be scaled along two orthogonal directions \cite{Bronsvoort:1989}.
  
  \item Hsu and Chelberg \cite{hsu:1994} describes a ray tracing algorithm for generalized cylinders whose sweeping rule is invertible. This means that it can handle scaling and rotation, which is something that we want to use.
  
  \item The ray is given by a parameter curve $\ve{r}(w) = \ve{o} + w\ve{d}.$
  
  \item The ray tracing problem is find $(u,v,w)$ where $\ve{\Gamma}(u,v) = \ve{r}(w).$
  
  \item The approach is to first consider the intersection of the ray with each normal plane. These intersection points give rise to a 2D curve $\ve{\rho}(u)$.
  
  Now, if the cross section is constant, we can find the intersection point by finding the intersetion between $\ve{\rho}(u)$ and $\ve{cs}(v)$. This is exactly what is described in \cite{Bronsvoort:1985}.
  
  However, in our case, the cross section varies along with $u$, so we need a more sophisticated technique.
  
  \item Hsu and Chelberg proposes applying the inverse of the sweeping rule to $\ve{\rho}(u)$ to get another curve $\ve{\phi}(u)$. Then, we can intersect $\ve{\phi}(u)$ with $\ve{cs}(v)$.
  
\end{itemize}

\subsection{Computing $\ve{\rho}(u)$}

\begin{itemize}
  \item For each $u$, we project $\ve{v} = \ve{o} + w\ve{d} - \ve{a}(u)$ to the coordinate system defined by the Frenet frame at $u$. This is done by multiplying $\ve{v}(w,u)$ with the matrix $\ve{M}$ whose rows are the vectors $\hat{\ve{t}}(u)$, $\hat{\ve{n}}(u),$ and $\hat{\ve{b}}(u)$:
  \begin{align*}
    \ve{r}_l(w,u) &= \ve{M}\ve{v}
  \end{align*}
  where
  \begin{align*}
    \ve{M} = \begin{bmatrix}
      \hat{\ve{t}}(u)^T\\
      \hat{\ve{n}}(u)^T\\
      \hat{\ve{b}}(u)^T
    \end{bmatrix}.
  \end{align*}

  \item For each $u$, the intersection of the ray $\ve{r}(w)$ and the normal plane can be found using the equation:
  \begin{align*}
    \hat{\ve{t}}(u) \cdot (\ve{o} + w \ve{d} - \ve{a}(u)) = 0.
  \end{align*}
  Substituting $\hat{\ve{t}}(u) = \ve{a}'(u)/\| \ve{a}'(u) \|$, we have
  \begin{align*}
    \ve{a}'(u) \cdot (\ve{o} + w \ve{d} - \ve{a}(u)) = 0.
  \end{align*}
  So,
  \begin{align*}
    w = - \frac{\ve{a}'(u) \cdot (\ve{o} - \ve{a}(u))}{\ve{a}'(u) \cdot \ve{d}}
  \end{align*}
  provided that $\ve{a}'(u) \cdot \ve{d} \neq 0$. If this is not true, there are two cases:
  \begin{itemize}
    \item If $\ve{a}'(u) \cdot (\ve{o} - \ve{a}(u)) = 0$ as well, then the ray lies in the normal plane. The problem then becomes finding the intersection of the ray with the contour directly.
    
    \item If $\ve{a}'(u) \cdot (\ve{o} - \ve{a}(u)) \neq 0$, then there can be no intersection. We thus exclude this $u$ from the description of $\ve{\rho}(u).$
  \end{itemize}
  
  
  \item We can find the coordinate of the intersection point in the normal plane by substituting the above $w$ and $u$ to $\ve{r}_l(w, u)$. This gives:
  \begin{align*}
    \ve{\rho}(u) = \begin{bmatrix}
      \hat{\ve{n}}(u) \cdot \big( \ve{o} - \frac{\ve{a}'(u) \cdot(\ve{o} - \ve{a}(u))}{\ve{a}' \cdot \ve{d}} \ve{d} - \ve{a}(u) \big)\\
      \hat{\ve{t}}(u) \cdot \big( \ve{o} - \frac{\ve{a}'(u) \cdot(\ve{o} - \ve{a}(u))}{\ve{a}' \cdot \ve{d}} \ve{d} - \ve{a}(u) \big).
    \end{bmatrix}
  \end{align*}  
\end{itemize}

\subsection{Computing $\ve{\phi}(u)$}

\begin{itemize}
  \item Because we want to find $(u,v)$ such that $\ve{\rho}(u) = \ve{c}(u,v) = \ve{s}(\ve{cs}(v), u).$ Given that $\ve{s}(\cdot, u)$ is invertible for all $u$, the equation is equivalent to
  \begin{align*}
    \ve{s}^{-1}(\ve{\rho}(u), u) = \ve{cs}(v).
  \end{align*}
  The curve $\ve{\phi}(u)$ is thus defined as $\ve{s}^{-1}(\ve{\rho}(u), u)$.  
\end{itemize}

\subsection{Finding Intersection Points}

\begin{itemize}
  \item The curves $\ve{\phi}(u)$ and $\ve{cs}(v)$ are split at inflection points and points with horizontal or vertical tangents. As a result, the segments obtained are bounded within the rectangle defined by their end points.
  
  \item If each of the segments are subdivided, each subdivided segment is also bounded by the box defined by their end points.
  
  \item For each pair of $\ve{\phi}(u)$ and $\ve{cs}(v)$ segments, the following intersection routine is called:
  
  \begin{verse}
    {\bf procedure} {\sc Intersect}$(curve_1, curve_2)$\\
    \quad {\bf if not} {\sc Overlap}$(curve_1, curve_2)$\\
    \quad \quad {\bf return}\\
    \quad {\bf if} {\sc Is-Linear}$(curve_1)$ {\bf and} {\sc Is-Linear}$(curve_2)$\\
    \quad \quad {\sc Line-Intersect}$(curve_1, curve_2)$\\
    \quad {\bf else if} {\sc Is-Linear}$(curve_1)$ {\bf and not} {\sc Is-Linear}$(curve_2)$\\
    \quad \quad $curve_{21}, curve_{22} = $\ {\sc Divide-Curve}$(curve_2)$\\
    \quad \quad {\sc Intersect}$(curve_1, curve_{21})$\\
    \quad \quad {\sc Intersect}$(curve_1, curve_{22})$\\
    \quad {\bf else if} {\sc Is-Linear}$(curve_2)$ {\bf and not} {\sc Is-Linear}$(curve_1)$\\
    \quad \quad $curve_{11}, curve_{12} = $\ {\sc Divide-Curve}$(curve_2)$\\
    \quad \quad {\sc Intersect}$(curve_{11}, curve_2)$\\
    \quad \quad {\sc Intersect}$(curve_{12}, curve_2)$\\
    \quad {\bf else}\\
    \quad \quad $curve_{11}, curve_{12} = $\ {\sc Divide-Curve}$(curve_2)$\\
    \quad \quad $curve_{21}, curve_{22} = $\ {\sc Divide-Curve}$(curve_2)$\\
    \quad \quad {\sc Intersect}$(curve_{11}, curve_{21})$\\
    \quad \quad {\sc Intersect}$(curve_{12}, curve_{21})$\\
    \quad \quad {\sc Intersect}$(curve_{11}, curve_{22}))$\\
    \quad \quad {\sc Intersect}$(curve_{12}, curve_{22})$\\
  \end{verse}
  
  Here,
  \begin{itemize}
    \item {\sc Overlap}$(curve_1, curve_2)$ checks whether the bounding boxes of $curve_1$ and $curve_2$ overlaps.
    
    \item {\sc Is-Linear}$(curve)$ tests whether the curve segment is linear enough. We will cover how to do this later.
    
    \item {\sc Divide-Curve}$(curve, curve_1, curve_2)$ subdivides the curve at the midpoint of the range of parameter values.    
  \end{itemize}
  
  \item The aforementioned linearity test calculate the distance between:
  \begin{itemize}
    \item the intersection between the tangent lines at the end-points of the segment, and
    \item the line joining the end points.
  \end{itemize}
  If the distance is less than $\varepsilon$, the segment is considered linear.
  
  \item To compute the distance, we may consider the $x$ and the $y$ axes separately. Let the two end-points be denoted by $A$ and $B$. Let the $u$ parameters at $A$ and $B$ be $u_0$ and $u_1$, respectively. Also, let us refer to the x-coordinates of the two points as $x_0$ and $x_1$, and let the derivative of $x$ with respective to $u$ at $A$ and $B$ be $x'_0$ and $x'_1$, respectively.
  
  The $x$-coordinate of the tangent line at $A$ is given by:
  $$x = x_0 + (u - u_0) x'_0.$$
  Similarly, the $x$-coordinate of the tangent line at $B$ is given by:
  $$x = x_1 + (u-u_1)x'_1.$$
  Solving, we get
  \begin{align*}
    u^* &= \frac{x_0 - x_1 + u_1 x'_1 - u_0 x'_0}{x'_1 - x'_0}, \mbox{ and}\\
    x^* &= x_0 + x'_0(u^* - u_0).
  \end{align*}
  The equation of the chord is:
  \begin{align*}
    x(u_1 - u_0) - (x_1 - x_0)u + [(x_1 - x_0)u_0 - x_0(u_1-u_0)] = 0.
  \end{align*}
  The perpendicular distance of $(u^*, x^*)$ from the chord is given by:
  \begin{align*}
    d_x = \bigg| \frac{(u_1 - u_0)x^* + (x_0 - x_1)u^* + [(x_1 - x_0)u_0 - x_0(u_1 -u_0)]}{\sqrt{(u_1 - u_0)^2 - (x_0 - x_1)^2}} \bigg|.
  \end{align*}
  
  We can compute $d_y$ in a similar manner, and the total distance is given by $d = \sqrt{d_x^2 + d_y^2}$.
  
\end{itemize}

\subsection{Normal Calculation}

\begin{itemize}
  \item The surface normal at $(u,v)$ is given by
  \begin{align*}
    \hat{\ve{N}}(u,v) = \frac{\partial \ve{\Gamma}(u,v)}{\partial v} \times \frac{\partial \ve{\Gamma}(u,v)}{\partial u}
  \end{align*}
  where
  \begin{align*}
  \frac{\partial \ve{\Gamma}(u,v)}{\partial u}
  &= \ve{a}'(u) + \hat{\ve{n}}(u) \frac{\partial c_n(u,v)}{\partial u}
    + c_n(u,v) \hat{\ve{n}}'(u)
    + \hat{\ve{b}}(u) \frac{\partial c_b(u,v)}{\partial u}
    + c_b(u,v) \hat{\ve{b}}'(u)\\
  \frac{\partial \ve{\Gamma}(u,v)}{\partial v}
  &= \frac{\partial c_n(u,v)}{\partial v} \hat{\ve{n}}(u) + \frac{\partial c_b(u,v)}{\partial v} \hat{\ve{b}}(u).
  \end{align*}
  
  \item The derivatives $\hat{\ve{n}}'(u)$ and $\hat{\ve{b}}'(u)$ can be computed by the Frenet-Serret formualas:
  \begin{align*}    
    \hat{\ve{n}}'(u) &= -\kappa(u)\hat{\ve{t}}(u) + \tau(u) \hat{\ve{b}}(u)\\
    \hat{\ve{b}}'(u) &= - \tau(u) \hat{\ve{n}}(u)
  \end{align*}
  where $\kappa(u)$ is the \emph{curvature} and $\tau(u)$ is the \emph{torsion}. They are defined as:
  \begin{align*}
    \kappa(u) &= \frac{\| \ve{a}'(u) \times \ve{a}''(u) \|}{\| \ve{a}'(u) \|^3}, \mbox{ and}\\
    \tau(u) &= \frac{(\ve{a}' \times \ve{a}''(u)) \cdot \ve{a}'''(u) }{\| \ve{a}'(u) \times \ve{a}''(u) \|^2}.
  \end{align*}    
\end{itemize}

% section section_name (end)

\section{Hair Geometry}\label{sec:hair_geometry} % (fold)

\begin{itemize}
  \item The cross section is a circle:
    \begin{align*}
      \ve{cs}(v) = \begin{bmatrix}
        \cos v\\
        \sin v
      \end{bmatrix}
    \end{align*}
    where $v \in [0, 2\pi]$.
    
  \item We shall represent the axis as a Bezier curve $\ve{a}(u) = \ve{a}^{(0)} + \ve{a}^{(1)}u + \ve{a}^{(2)} u^2 + \ve{a}^{(3)} u^3$ where $u \in [0,1]$.
    
  \item We would like to allow the user to specify the following parameters:
    \begin{itemize}
      \item The major radius  $a(u) = a_0 + (a_1 - a_0)u$.
      \item The minor radius $b(u) = b_0 + (b_1 - b_0)u$.
      \item The in-plane rotation angle $\theta(u) = \theta_0 + (\theta_1 - \theta_0)u$.
    \end{itemize}
    The sweeping rule is given by:
    \begin{align*}
      \ve{s}(\ve{cs}(v), u) = S(u)\ve{cs}(v) =
      \begin{bmatrix}
        \cos(\theta(u)) & -\sin(\theta(u))\\
        \sin(\theta(u)) &  \cos(\theta(u))\\
      \end{bmatrix}
      \begin{bmatrix}
        a(u) & 0\\
        0 & b(u)
      \end{bmatrix}
      \ve{cs}(v).
    \end{align*}
    As such, the inverse of the sweeping rule is given by:
    \begin{align*}
      S^{-1}(u) &=
      \begin{bmatrix}
        1/a(u) & 0\\
        0 & 1/b(u)
      \end{bmatrix}
      \begin{bmatrix}
         \cos(\theta(u)) &   \sin(\theta(u))\\
        -\sin(\theta(u)) &  \cos(\theta(u))\\
      \end{bmatrix}.
    \end{align*}
    
  \item Now, if we want to intersect the ray $\ve{r}(w) = \ve{o} + w \ve{d}$, we first have to compute the points where $\ve{a}'(u) \cdot \ve{d} = 0.$ We have that
  \begin{align*}
    \ve{a}'(u) = \ve{a}^{(1)} + 2 \ve{a}^{(2)} u + 3 \ve{a}^{(3)} u^2
  \end{align*}
  So,
  \begin{align*}
    \ve{a}'(u) \cdot \ve{d} 
    &= (\ve{a}^{(1)} \cdot \ve{d}) + 2 (\ve{a}^{(2)} \cdot \ve{d}) u + 3 (\ve{a}^{(3)} \cdot \ve{d}) u^2.
  \end{align*}
  Therefore, $\ve{a}'(u) \cdot \ve{d} = 0$ is a quadratic equation.  
  
\end{itemize}
% section section_name (end)
		
\bibliographystyle{plain}
\bibliography{generalized-cylinder-rt}	

\end{document}
  