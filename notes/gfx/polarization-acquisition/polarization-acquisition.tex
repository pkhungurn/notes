\documentclass[10pt]{article}
\usepackage{fullpage}
\usepackage{amsmath}
\usepackage[amsthm, thmmarks]{ntheorem}
\usepackage{amssymb}
\usepackage{graphicx}
\usepackage{enumerate}
\usepackage{verse}
\usepackage{tikz}
\usepackage{verbatim}
\usepackage{hyperref}

\newtheorem{lemma}{Lemma}
\newtheorem{theorem}[lemma]{Theorem}
\newtheorem{definition}[lemma]{Definition}
\newtheorem{proposition}[lemma]{Proposition}
\newtheorem{corollary}[lemma]{Corollary}
\newtheorem{claim}[lemma]{Claim}
\newtheorem{example}[lemma]{Example}

\newcommand{\dee}{\mathrm{d}}
\newcommand{\Dee}{\mathrm{D}}
\newcommand{\In}{\mathrm{in}}
\newcommand{\Out}{\mathrm{out}}
\newcommand{\pdf}{\mathrm{pdf}}
\newcommand{\Cov}{\mathrm{Cov}}
\newcommand{\Var}{\mathrm{Var}}

\newcommand{\ve}[1]{\mathbf{#1}}
\newcommand{\mrm}[1]{\mathrm{#1}}
\newcommand{\etal}{{et~al.}}
\newcommand{\sphere}{\mathbb{S}^2}
\newcommand{\modeint}{\mathcal{M}}
\newcommand{\azimint}{\mathcal{N}}
\newcommand{\ra}{\rightarrow}
\newcommand{\mcal}[1]{\mathcal{#1}}
\newcommand{\X}{\mathcal{X}}
\newcommand{\Y}{\mathcal{Y}}
\newcommand{\Z}{\mathcal{Z}}
\newcommand{\x}{\mathbf{x}}
\newcommand{\y}{\mathbf{y}}
\newcommand{\z}{\mathbf{z}}
\newcommand{\tr}{\mathrm{tr}}
\newcommand{\sgn}{\mathrm{sgn}}
\newcommand{\diag}{\mathrm{diag}}
\newcommand{\Real}{\mathbb{R}}
\newcommand{\sseq}{\subseteq}
\newcommand{\ov}[1]{\overline{#1}}

\title{Appearance Acquisition using Polarization}
\author{Pramook Khungurn}

\begin{document}
	\maketitle

	This document is written as I read the paper ``Practical Modeling and Acquisition of Layered Facial Reflection'' \cite{Ghosh:2008} and ``Circularly Polarized Spherical Illumination Reflectometry'' \cite{Ghosh:2010}.

	\section{\cite{Ghosh:2008}}

	\subsection{Introduction}

	\begin{itemize}
		\item This paper acquire parameters of a light scattering model of a human face from $20$ photographs captured from a single viewpoint.  The photographs are taken under environment and projected lighting conditions.

		\item The acquisition time is several seconds, which, they claim, is good for a human subject.

		\item Their model decomposes scattered light into 4 components:
		\begin{enumerate}
			\item specular reflectance,
			\item single scattering,
			\item shallow multiple scattering, and
			\item deep multiple scattering. 
		\end{enumerate}

		\item Some parameters are estimated per pixel, some per region, and some for the entire face.
	\end{itemize}

	\subsection{Acquisition}

	\begin{itemize}
		\item Setup
		\begin{itemize}
			\item LED sphere with approximately 150 lights.  Each light is covered with a linear polarizer in the pattern of \cite{Ma:2007}.  (I need to read this later.)

			\item A vertically polarized LCD projector is aimed towards the center of the sphere.

			\item A stereo pair of radiometrically calibrated 10-Megapixel Canon 1D Mark III digial SLR.  The camera are placed on the sides of the projector.
			\begin{itemize}
				\item The right camera is used only for geometry measurement.  It is horizontally polarized.
				\item The left camera can switches between horizontal and vertical polarization. 
			\end{itemize}
		\end{itemize}

		\item The lights in the sphere needs to be tuned so that specular reflection is only visible through only horizontally polarized cameras.  This is done as follows:
		\begin{itemize}
			\item Put a plastic ball at the middle of the light sphere.
			\item Rotate the polarizer at each light until no highlights can be observed. 
		\end{itemize}

		\item There are two illuminators: the light sphere and the projector.  These can have different power outputs, which need to be calibrated with respect to each other.  Calibration is done as follows:
		\begin{itemize}
			\item Measure the responses of (1) the Macbeth chart and (2) 10 skin patches on different subjects.
			\item Compute a $3 \times 3$ color matrix that transforms the photographs into a common color space.
			\item They include the skin samples because using only the Macbeth chart leads to skin colors not matching well. 
		\end{itemize}

		\item The paper projects a black pattern and takes a photograph.  This photograph is subtracted from every photographs taken when the projector is used to light the subject.  This is to take into account the black level of the projector.

		\item They acquire the geometry and (specular) normals using the technique of \cite{Ma:2007}.  This requires:
		\begin{itemize}
			\item 4 photographs taken with 4 project color fringe patterns.  These are used for geometry reconstruction.
			\item 8 photographs under 4 gradient illumination conditions and 2 polarization directions. 
		\end{itemize}

		\item 8 more photographs are taken to infer scattering model parameters.  These are:
		\begin{itemize}
			\item 1 photograph for the black level reference for the video projector.
			\item 1 photograph with a cross-polarized grid of black dots projected from the front to measure the subsurface scattering parameters.
			\item 2 photographs of cross- and parallel-polarized front-lit images to model the specular and diffuse reflectance.
			\item 4 photographs of phase-shifted stripe patterns to separate shallow and deep multiple scattering.
		\end{itemize}

		\item All photographs are taken in 5 seconds.
  	\end{itemize}

  	\subsection{Skin Appearance Model}

  	\begin{itemize}
  		\item 4 phenomena:
  		\begin{itemize}
  			\item specular reflection,
  			\item single scattering,
  			\item shallow multiple scattering, and
  			\item deep multiple scattering.
  		\end{itemize}
  		\item Parameters are estimated at different granularities:
  		\begin{itemize}
  			\item Per pixel: albedos and surface normals.
  			\item Per region: specular roughness and scattering parameters.
  			\item Per face: the angular dependence of the scattering components. 
  		\end{itemize}
  		\item The regions are:
  		\begin{itemize}
  			\item forehead,
  			\item eyelids,
  			\item nose,
  			\item cheekbone,
  			\item lips, and
  			\item lower cheek regions.
  		\end{itemize}
  	\end{itemize}

  	\subsubsection{Specular Reflection and Single Scattering}

  	\begin{itemize}
  		\item Specular reflection and single scattering generally maintains the polarization of light.  On the other hand, multiple scattering depolarizes light.

  		\item The paper says they take photos under
  		\begin{itemize}
  			\item polarized spherical illumination, and
  			\item polarized front illumination.
  		\end{itemize}
  		For each illumination, they record cross- and parallel-polarized images.

  		However, does this mean they take 4 photographs instead of 2, as was stated previously?

  		\item Cross-polarized images should only included depolarized reflected light. Parallel-polarized images should include both polarized reflected light and depolarized light.

  		\item $(\mbox{paralell} - \mbox{cross})$ should yield an image of only polarized reflected light, which is specularly reflected light and some non--specularly reflected light which maintains polarization.
  		 The latter component should be dominated by single scattering because the probabiliy of maintaining polarization decreases exponentially with the number of bounces.
  
		\item The paper uses the distribution-based BRDF model \cite{Ashikhmin:2006}:
		\begin{align*}
			\rho_{\mathrm{sr}}(\ve{i},\ve{o}) = c\, \frac{p(\ve{h}) F(r_0, \ve{o}\cdot \ve{n})}{(\ve{i}\cdot\ve{n}) + (\ve{o}\cdot\ve{n}) - (\ve{i}\cdot\ve{n})(\ve{o}\cdot\ve{n})}
		\end{align*}
		where
		\begin{itemize}
			\item $\ve{i}$ is the incoming light direction,
			\item $\ve{o}$ is the outgoing light direction,
			\item $c$ is a scaling factor,
			\item $\ve{h}$ is the half vector,
			\item $p$ is the distribution of the microfacet normal,
			\item $F$ is the Fresnel term, and
			\item $r_0$ is most likely the amount of light reflected at incident angle $0$.
		\end{itemize}

		 \item The paper assumes the index of refraction to be $1.38$, so $r_0$ is a constant.\\
		 The parameters to be found are then the distribution $p$ and the normalization constant $c$.\\
		 The distribution $p$ is per region, but the constant $c$ is determined per pixel.

		 \item While $c$ is a simple constant, $p$ is a tabulated distribution acquired from measured data.  From my speculation, it is radially symmetric around $\ve{n}$, so it is a function of $\theta$, the azimuthal angle.

		\item The paper alternately esimates $p$ and $c$.  Starting with an initial constant $c$ for all pixels, it estimates $p$ using the difference image under front-lit illumination.  It then estimates $c$ using the difference image under spherical illumination.  Then, $p$ can be estimated again, then $c$, and so on.  However, the paper says one iteration is enough.

  		\item Let us talk about how to estimate $p$ given $c$.  In the difference image under front-lit configuration,  we have that $\ve{i} = \ve{o}$.  Assuming that the Fresnel term plays minimal role in back-scattered light, the model simplifies to:
  		\begin{align*}
  			\rho_{\mathrm{sr}}(\ve{o},\ve{o}) = c \, r_0 \, \frac{p(\ve{o})}{2(\ve{o} \cdot \ve{n}) - (\ve{o} \cdot \ve{n})^2}
  		\end{align*}

  		\item With the above formula, the value of $p(\ve{o})$ can be directly computed.  However, the paper only uses pixels where
  		\begin{itemize}
  			\item the brightness is above a certain threshold, and
  			\item the surface normals lie within a cone of $45^\circ$ from the viewing direction.
  		\end{itemize}
  		These pixels, according to the paper, are those where specular reflection dominates single scattering.  The reasons for $45^\circ$ are:
  		\begin{itemize}
  			\item specular lobes on faces are much sharper than $45^\circ$, and
  			\item single scattering is predominantly directed forward into the skin.
  		\end{itemize} 
  		This angular and intensity separation technique is also used in tissue optics literature.  

  		\item Given the distribution $p$, we now turn to the estimation of the scaling factor $c$.  The paper uses the difference image under constant spherical illumination.  It says that, at all pixels, specular reflectance dominates, and it says that the intensity at each pixel is given by:
  		\begin{align*}
  			c' 
  			&= \int_{H^2} \rho_{\mathrm{sr}}(\ve{i},\ve{o}) \max(\ve{i} \cdot \ve{n},0)\, \dee\ve{i} \\
  			&= \int_{H^2} c\, \frac{p(\ve{h}) F(r_0, \ve{o}\cdot \ve{n})}{(\ve{i}\cdot\ve{n}) + (\ve{o}\cdot\ve{n}) - (\ve{i}\cdot\ve{n})(\ve{o}\cdot\ve{n})} \max(\ve{i} \cdot \ve{n},0)\, \dee\ve{i} \\
  			&= c \int_{H^2} \frac{p(\ve{h}) F(r_0, \ve{o}\cdot \ve{n})}{(\ve{i}\cdot\ve{n}) + (\ve{o}\cdot\ve{n}) - (\ve{i}\cdot\ve{n})(\ve{o}\cdot\ve{n})} \max(\ve{i} \cdot \ve{n},0)\, \dee\ve{i}
  		\end{align*}
  		So, the constant $c$ is determined by:
  		\begin{align*}
  		c = c' \bigg( \int_{H^2} \frac{p(\ve{h}) F(r_0, \ve{o}\cdot \ve{n})}{(\ve{i}\cdot\ve{n}) + (\ve{o}\cdot\ve{n}) - (\ve{i}\cdot\ve{n})(\ve{o}\cdot\ve{n})} \max(\ve{i} \cdot \ve{n},0)\, \dee\ve{i} \bigg)^{-1}.
  		\end{align*}

  		\item The single scattering component is modeled with the first order single scattering BRDF \cite{Hanrahan:1993}.
  		\begin{align*}
  			\rho_{\mathrm{ss}}(\ve{i},\ve{o}) = \alpha \, T_{\mathrm{dt}} \, p(\cos\theta)\, \frac{1}{\ve{n} \cdot \ve{i} + \ve{n} \cdot \ve{o}}
  		\end{align*}
  		where
  		\begin{itemize}
  			\item $\alpha$ is the scattering albedo,
  			\item $T_{\mathrm{dt}}$ is the transmittance term,
  			\item $p$ is the Henyey--Greenstein phase function:
  			\begin{align*}
  				p(\cos \theta) = \frac{1-g^2}{4\pi(1-g+2g\cos\theta)^{3/2}},
  			\end{align*}
  			\item $\theta$ is the angle between $\ve{i}$ and $\ve{o}$, and
  			\item $g$ is the mean cosine of the phase function.
  		\end{itemize}
  		The paper sets $T_{\mathrm{dt}}$ as $\rho_{\mathrm{dt}}(\ve{i}) \rho_{\mathrm{dt}}(\ve{o})$ where:
  		\begin{align*}
  			\rho_{\mathrm{dt}}(\omega_o) = 1.0 - \int_{H^2} \rho_{\mathrm{sr}}(\omega_i, \omega_o) \max(\ve{n} \cdot \omega_i,0)\ \dee\omega_i
  		\end{align*}

  		\item The parameters to be found is $\alpha$ and $g$.  While $\alpha$ determined per pixel, a single value of $g$ is used for all pixels on the face.

  		\item The value of $\alpha$ is given by:
  		\begin{align*}
  			\alpha = (\mbox{pixel intensity of the difference image under spherical illumination}) - c.
  		\end{align*}

  		\item After determining $\alpha$, the model for specular reflectance and single scattering is almost complete because only $g$ is left.  As a result, $g$ can be determined by minimizing the RMS error between the difference image under spherical illumination and the image generated by the model.
  	\end{itemize}

  	\subsubsection{Multiple Scattering}

  	\begin{itemize}
  		\item If skin is modeled as a single-layered material, it looks too soft or waxy. It is better modeled as a two-layered subsurface scattering material:
  		\begin{itemize}
  			\item The top layer corresponds to the epidermal layer whose thickness is about $0.5$ mm.  The color is determined by the melanin content.
  			\item The bottom layer corresponds to the dermis, which is thicker layer and is red in color due to blood.
  		\end{itemize}
  		The top layer is designated as \emph{shallow}, and the bottom layer is designated as \emph{deep}.

  		\item \cite{Nayar:2006}
  		\begin{itemize}
  			\item This is a method to separate direct and indirect components of reflected light using high frequency illumination.

  			\item The frequency of the illumination patterns determines which part of the scattered light is direct and which part is indirect. 
  		\end{itemize}

  		\item In our case, the direct component corresponds to the shallow scattering, and the indirect component corresponds to the deep scattering.

  		\item If we make the frequency of the patterns to be on the order of the thickness of the epidermis, we can obtain an image containing only deep scattering and another image containing only shallow scattering.

  		\item The paper uses $4$ phase-shifted high-frequency patterns of $1.2$mm-wide stripes.  The images are taken with cross-polarization configuration to eliminate specular reflection and single scattering.

  		\item For each pixel, the paper computes $max$ and $min$ of the pixel intensity of the 4 images.  Then, we have that:
  		\begin{itemize}
  			\item $(max-min)$ is the direct/shallow scattering image.
  			\item $(2 \times min)$ is the indirect/deep scattering image.
  		\end{itemize}

  		\item Outgoing light due to multiple scattering is modeled as:
  		\begin{align*}
  			L_{\mathrm{ms}}(x_o, \omega_o) = \int_{A} \int_{H^2} T_{\mathrm{dt}} R_d(\| x_o - x_i \|) \cos \theta_i\ d\omega_i dA(x_i)
  		\end{align*}
  		where
  		\begin{align*}
  			R_d(\| x_o - x_i \|) = R_{\mathrm{deep}}(\| x_o - x_i \|) + R_{\mathrm{shallow}}(\| x_o - x_i \|).
  		\end{align*}

  		\item The paper models $R_{\mathrm{deep}}(\| x_o - x_i \|)$ with the dipole diffusion model, assuming an infinitely deep dermis.  On the other hand, $R_{\mathrm{shallow}}(\| x_o - x_i \|)$ is modeled with the multipole model.

  		\item The dipole model for the deep scattering component is given by:
  		\begin{align*}
  			R_{\mathrm{deep}}(\| x_o - x_i \|)
  			&= \frac{\alpha'}{4\pi} \bigg( z_r \Big( \sigma_{tr} + \frac{1}{d_r} \Big) \frac{e^{-\sigma_{tr} d_r}}{d_r^2} + z_v\Big( \sigma_{tr} + \frac{1}{d_v} \Big) \frac{e^{-\sigma_{tr}d_v}}{d_v^2} \bigg)
  		\end{align*}
      where
      \begin{itemize}
        \item $z_r$ is the distance of the real source to the surface,
        \item $d_r$ is the distance of the real source to $x_0$,
        \item $z_v$ is the distance of the virtual source to the surface, and
        \item $d_v$ is the distance of the virtual source to $x_0$.
      \end{itemize}
      The model thus requires us to estimate two parameters:
      \begin{itemize}
        \item the reduced albedo $\alpha'$, and
        \item the mean free path $l_d = 1/\sigma_{tr}$.
      \end{itemize}

      \item For optically dense materials, the following relation holds for $\alpha'$:
      \begin{align*}
        \mbox{diffuse albedo} = \frac{\alpha'}{2}( 1 + e^{-\frac{4}{3}A \sqrt{3(1-\alpha')}} ) e^{-\sqrt{3(1-\alpha')}}.
      \end{align*}
      where
      \begin{align*}
        A = \frac{1 + \rho_d}{ 1 - \rho_d}
      \end{align*}
      with $\rho_d$ being the reflectance of a rough specular surface due to hemispherical illumination.  (What the fuck is $\rho_d$ actually equal to?  What exactly is ``a rough specular surface?''  What the fuck with this crappy, unclear writing!  Can a SIGGRAPH publication be this low quality?)

      \item The diffuse albedo is computed per pixel, and this is basically the deep scattering image after factoring in the cosine falloff.

      \item The mean free path $l_d$ is computed per region.  The paper projects a (polarized) solid white pattern with black dots on the face.  The dots are $6$mm in diameter with $1$cm spacing between them, and the paper claims this distance exceeds the typical scattering distance of light through skin.  The paper uses the technique of \cite{Tariq:2006} to estimate $l_d$.  This involves creating a scattering profile, look at it, and find the parameters that matches it the best.

      \item When performing the above computation, the paper only uses pixels that are in the inner two-thirds of the projected black dots to discount the effects of shallow scattering.  Dot boundaries are determined by subtracting the photo under dot patterns from the photo under full-on illumination.  The pixels where the difference image change sign become the dot boundaries.

      \item After estimating the parameters for the deep scattering model, the paper goes on to the shallow one.  The paper removes effects of deep scattering from the scattering profile.  Then, it fits the following multipole model to the remaining profile:
      \begin{align*}
        R_{\mathrm{shallow}}(\| x_o - x_i \|) = \frac{\alpha'}{4} \sum_{i=-n}^n \bigg( \frac{z_{r,i}(1 + \sigma_{tr} d_{r,i})e^{-\sigma_{tr} d_{r,i}}}{d_{r,i}^3} - \frac{z_{v,i}(1 + \sigma_{tr} d_{v,i})e^{-\sigma_{tr} d_{v,i}}}{d_{v,i}^3} \bigg)
      \end{align*}
      (So, do we fit $\alpha'$ and $\sigma_{tr}$ again?)  The paper uses five dipoles.
  	\end{itemize}

    \section{\cite{Ghosh:2010}}

    \begin{itemize}
    \item With a fixed lighting setup of a circularly polarized spherical illumination, the paper takes 4 photographs: 3 with differently oriented linear polarizer in front of the camera, and one with a circular polarizer in front of the camera.  Then, it estimates per pixel the (1) diffuse albedo, (2) specular albedo, (3) index of refraction, and (4) specular roughness of isotropic BRDF.  The paper, however, assumes that the surface orientation is known at each pixel.

    \item If surface orientation is not known, additional 12 photos are needed to estimate it.

    \item There's also a previous work by Ghosh \etal\ \cite{Ghosh:2009}, which also uses polarization.  The work can determine per pixel scattering parameters.  However, it is limited to a single viewpoint because the polarizers has to be aligned.  The 2010 paper, however, uses circular polarized light, which does not depend on viewpoint.

    \item Polarization has been used in computer graphics for the following tasks:
    \begin{itemize}
        \item \textbf{Reflectance component separation}.  This is basically the separation of specular reflection from other components of scattered light.  This is based on the fact that specular reflection preserves polarization.  Methods that used polarization can be divided into to groups:
        \begin{itemize}
            \item \textit{Passive methods.}  These use unpolarized illumination.  They tend to have low SNR.
            \item \textit{Active methods.}  These use polarized illumination.  Previous works use linear polarizers.  However, such methods are either viewpoint dependent or require a large number of measurements.  The current paper uses circular polarization, and it hopes to improve the situation.
        \end{itemize} 
        
        \item \textbf{Material classification.}  Polarization can be used to separate dielectric materials from metalic materials.  Methods that do so use properties such as the Fresnel ratio, the polarization phase, and the polarization distributions around highlight.
        
        \item \textbf{Normal estimation.} Most previous work use unpolarized illumination.  They take a series of photographs while rotating the linear polarizer in front of the camera.  The angle of polarization determines the direction perpendicular to the plane of incidence.  For the direction in the plane of incidence, previous works use multiple cameras or degree of polarization.

        The current work, however, does not use polarization to determine surface normals.  It uses the standard photometric cues to do so.
        
        \item \textbf{Index of reflection estimation.} Most previous assumes known index of refraction, with the exception of a few.  The works by Dana \etal\ determines spatially varying index of refraction using bidirectional imaging and detection of Brewster's angle.  The current work uses a different approach.
    \end{itemize}
    \end{itemize}

    \subsection{Overview}

    \begin{itemize}
        \item The four photographs taken are observation of four Stokes parameters of an object under uniform spherical illumination (whatever that means).  The paper determines the shading parameters from the Stokes parameters, assuming surface orientation is known at each pixel.

        \item The inference process has three steps:
        \begin{itemize}
            \item Specular and diffuse albedo are separated by computing the degree of polarization from the Stokes parameters.
            \item The index of refraction is estimated from the observed ratio of circularly polarized specular reflections.
            \item Given the index of refraction, an inverse rendering approach is taken to estimate specular roughness.
        \end{itemize}         
    \end{itemize}

    \subsection{Polarization and Mueller Calculus}

    \begin{itemize}
        \item During light transport, the Stokes vector $\ve{s}$ that describes the state of a partially polarized ray is transformed into a new Stokes vector $\ve{s}'$ by the following linear operator:
        \begin{align*}
          \ve{s}' = C(\phi) D(\delta;\ve{n}) R(\theta;\ve{n}) C(-\phi) \ve{s}.
        \end{align*}
        where
        \begin{align*}
          C = \begin{bmatrix}
            1 & 0 & 0 & 0 \\
            0 & \cos 2\phi & -\sin 2\phi & 0 \\
            0 & \sin 2\phi & \cos 2\phi & 0 \\
            0 & 0 & 0 & 1
          \end{bmatrix}          
        \end{align*}
        is the Mueller rotation matrix,
        \begin{align*}
          R(\theta;\ve{n}) = \frac{1}{2} \begin{bmatrix}
            R_{\parallel} + R_{\perp} & R_{\parallel} - R_{\perp} & 0 & 0 \\
            R_{\parallel} - R_{\perp} & R_{\parallel} + R_{\perp} & 0 & 0 \\
            0 & 0 & 2\sqrt{R_{\parallel} R_{\perp}} & 0 \\
            0 & 0 & 0 & 2\sqrt{R_{\parallel} R_{\perp}} \\
          \end{bmatrix}
        \end{align*}
        is the Mueller reflection matrix with $R_{\parallel}$ and $R_{\perp}$ being the Fresnel coefficients for reflection, and
        \begin{align*}
          D = \begin{bmatrix}
            1 & 0 & 0 & 0 \\
            0 & 1 & 0 & 0 \\
            0 & 0 & \cos \delta & \sin\delta \\
            0 & 0 & -\sin\delta & \cos\delta
          \end{bmatrix}          
        \end{align*}
        is the Mueller matrix for a wave plate that retards the phase by $\delta$.

        \item The phase shift in the wave plate matrix differs depending on whether the material is dielectric or metalic.
        \begin{itemize}
          \item For dielectric material:
          \begin{itemize}
            \item $\delta = 180^\circ$ if $\theta < \theta_B$, where $\theta_B$ is the Brewster's angle.
            \item $\delta = 0^\circ$ if $\theta \geq \theta_B$.            
          \end{itemize}
          \item For metalic material, the phase shift varies continuously with $\theta$.
          \begin{itemize}
            \item It is $180^\circ$ at $\theta = 0^\circ$.
            \item It is $0^\circ$ at $\theta = 90^\circ$.
          \end{itemize}        
        \end{itemize}
        The shape of the falloff depends on the index of refraction $\eta$ and the extinction coefficient $\kappa$.  (I have no idea what $\kappa$ is.)

        \item The linear transformation above is only valid for mirror reflection.  Rough specular surface can be handled by a micro-facet BRDF model.  The output Stokes vector can then be compute with the standard BRDF integral:
        \begin{align*}
          \ve{s'}(\omega_o) = \int_{H^2} f(\ve{n}; \omega_i, \omega_o) \ve{s}(\omega_i)\ \dee\omega_i.
        \end{align*}
    \end{itemize}

    \subsection{Reflectance from Stokes Observations}

    \subsubsection{Diffuse/Specular Albedo}
    
    \begin{itemize}
      \item Polarized radiance can only be the result of specular reflection under polarized incident illumination.

      \item Diffuse reflection, on the other hand, is the aggregate result of multiple scattering events and hence depolarize the reflected incident radiance.

      \item It is useful to consider the degree of polarization:
      \begin{align*}
        \mathrm{DOP} = \frac{\sqrt{s_1^2 + s_2^2 + s_3^2}}{s_0}.
      \end{align*}
      It can be interpreted as the ratio of specular intensity to the observed total intensity under polarized illumination.

      As a result, the intensity of the observed s pecular reflectance is given by $\mathrm{DOP} \times s_0$, and the intensity of diffuse reflectance is given by $(1 - \mathrm{DOP}) \times s_0$.

      \item The paper uses uniform spherical (polarized) illumination.  As a result, $(1 - \mathrm{DOP}) \times s_0$ at each pixel is this equal to the diffuse albedo $\rho_d$, and $\mathrm{DOP} \times s_0$ is the specular albedo $\rho_s$.      
    \end{itemize}

    \subsection{Index of Refraction for Mirror-Like Specular Dielectric Material}

    \begin{itemize}
      \item In this case, tthe exact amount of total reflected radiance is governed by the Fresnel equations.  The reflection coefficients are given by:
      \begin{align*}
        R_{\perp} &= \left( \frac{\cos\theta - \eta \sqrt{1 - \sin^2\theta^2/\eta^2 }}{\cos\theta + \eta \sqrt{1 - \sin^2 \theta/\eta^2 }} \right) ^2 \\
        R_{\parallel} &= \left( \frac{\sqrt{1 - \sin^2 \theta / \eta^2} - \eta \cos\theta }{ \sqrt{1 - \sin^2 \theta / \eta^2} + \eta \cos\theta } \right)^2.
      \end{align*}
      As a result, we can right $\eta$ as a function of the reflection coefficients:
      \begin{align*}
        \eta^2 = \frac{(1 + \sqrt{R_{\parallel}})(1 + \sqrt{R_{\perp}})}{(1 - \sqrt{R_{\parallel}})(1 - \sqrt{R_{\perp}})}.        
      \end{align*}
      So, we may compute $\eta$ from $R_{\parallel}$ and $R_\perp$.

      \item For a mirror-like material, only a single incident direction contributes to the observed specular radiance for a given point.  The observed Stokes vector is that case is:
      \begin{align*}
        \ve{s}' = C(\phi) D(\delta; \ve{n}) R(\theta; \ve{n}) C(-\phi) \ve{s}
      \end{align*}
      where $\ve{s} = (1,0,0,1)^T$ for uniform spherical RCP illumination.

      The first and last component of the resulting vector, $s'_0$ and $s'_3$, are not affected by $C(\phi)$ and $C(-\phi)$.  So,
      \begin{align*}
        s_0 &= \frac{1}{2} (R_\parallel + R_\perp) + \rho_d \\
        s_3 &= \pm \sqrt{R_\parallel R_\perp}
      \end{align*}
      where the sign flip is due to the retardation matrix:
      \begin{itemize}
        \item $\theta \leq \theta_B$ yields a negative sign, and
        \item $\theta > \theta_B$ yields a positive sign.
      \end{itemize}
      Substituting for $R_\perp$ yields the following equation:
      \begin{align*}
        s_3 = \pm \sqrt{2 (s_0 - \rho_d) R_\parallel - R_\parallel^2}.
      \end{align*}
      Squaring both sides, you get a quadratic equation, which you can solve for two solutions.  One of them corresponds to $R_\parallel$ and the other to $R_\perp$.  We can distinguish which is which by using the fact that $R_\perp \geq R_\parallel$.  Once $R_\parallel$ and $R_\perp$ are known, we can compute $\eta$.
    \end{itemize}

    \subsection{Index of Refraction for Mirror-Like Specular Metallic Materails}

    \begin{itemize}
      \item The index of refraction for metallic materias is a complex number, where the real part $\eta$ is the refractive index, and the imaginary part $\kappa$ is the extinction coefficient.

      \item Estimating the above parameters from sphercial illumination is difficult.  Using point source does not solve the problem because (1) the measurements become view-independent, and (2) only a small portion of the surface will specularly reflect the incident light toward the camera, making it difficult to estimate the index of refraction for all surface points.

      \item The paper takes the approach of approximating the complex index of refraction by a single real number (only $\eta$) yields good approximation.  So, they just used the same procedure as the last section to acquire $\eta$
    \end{itemize}

    \subsection{Index of Refraction for Rough Specular Materials}

    \begin{itemize}
      \item The apply the same procedure as the above section again.

      \item This is an approximation, so it's better to understand its limitation.
      \begin{itemize}
        \item We generally get good approximating for $\eta$ when the view angle is less than the Brewster's angle.
        \item We also get good approximating when roughness is low.  (Of course!)  Things break down at high roughness.
      \end{itemize}
    \end{itemize}

    \subsection{Specular Roughness}

    \begin{itemize}
      \item Given a priori reflectance model $f$, estimated index of refration $\eta$, surface normal $\ve{n}$, and a postulated specular roughness value $m$, we can compute the magnitude of circular polarization under uniform spherical illumination using Monte Carlo integration:
      \begin{align*}
        \tilde{s}_3^m = \int_{H^2} f(\ve{n}; m, \theta) \cos\theta \cos\delta \sqrt{R_\parallel(\eta;\theta) R_\perp(\eta;\theta)}\ \dee\omega.
      \end{align*}
      The value $\tilde{s}_3^m$ varies smoothly with $\eta$ and $m$.  Moreover, it is monotonic in both parameters.  

      \item As such, given $\eta$, we can easily binary search for $m$ so that $\tilde{s}_3^m$ matches the observation.  Moreoever, if $\eta$ is inaccurate, we know that the error would not be dramatic.

      \item The paper approximates $m$ by:
      \begin{enumerate}
        \item Compute $\tilde{s}_3 = s_3 / \sqrt{s_1^2 + s_2^2 + s_3^3}$.
        \item Find the optimal roughness $m = \arg\min_m \| \tilde{s}_3^m - \tilde{s}_m \|_2$.
      \end{enumerate}
    \end{itemize}

    \subsection{Measurements}

    \begin{itemize}
      \item For each lighting configuration, the paper takes 4 photographs with the following polarizer in front of the camera:
      \begin{itemize}
        \item $P_H$: a linear polarizer rotated $0^\circ$.
        \item $P_{45}$: a linear polarizer rotated $45^\circ$.
        \item $P_V$: a linear polarizer rotated $90^\circ$.
        \item $P_\circ$: a circular polarizer.
      \end{itemize}

      \item The Stokes parameters are computed as:
      \begin{align*}
        s_0 &= P_H + P_V \\
        s_1 &= P_H - P_V \\
        s_2 &= 2P_{45} - s_0 \\
        s_3 &= s_0 - 2P_\circ.
      \end{align*}

      \item The paper acquires surface normals with a technique used in \cite{Ma:2007}.

      Let
      \begin{itemize}
        \item $F_s$ be the specular image taken under full-on spherical illumination.
        \item $X_s$ be the specular image taken under $x$-gradient illumination.
        \item $Y_s$ be the specular image taken under $y$-gradient illumination.
        \item $Z_s$ be the specular image taken under $z$-gradient illumination.
      \end{itemize}
      Then, let $\overline{\ve{n}} = 2(X_s, Y_s, Z_s) - F_s$.  The normal is just $\overline{\ve{n}} / \| \overline{\ve{n}} \|$.

      \item We can compute the specular image as $\mathrm{DOP} \times s_0$.  So, for each lighting configuration, we can take the above 4 photos to get all the information.  This is 12 extra photos from those already used to estimate all the shading parameters.

      \item The paper discusses 3 lighting setups that can generate the desired lighting conditions:
      \begin{itemize}
        \item LED sphere,
        \item reflective dome, and
        \item polarized CRT.
      \end{itemize}
      Since I'm lazy, I'm not going to talk about them.
    \end{itemize}

	\bibliographystyle{apalike}
	\bibliography{polarization-acquisition}  
\end{document} 