\documentclass[10pt]{article}
\usepackage{fullpage}
\usepackage{amsmath}
\usepackage[amsthm, thmmarks]{ntheorem}
\usepackage{amssymb}
\usepackage{graphicx}
\usepackage{enumerate}
\usepackage{verse}
\usepackage{tikz}
\usepackage{verbatim}
\usepackage{hyperref}

\newtheorem{lemma}{Lemma}
\newtheorem{theorem}[lemma]{Theorem}
\newtheorem{definition}[lemma]{Definition}
\newtheorem{proposition}[lemma]{Proposition}
\newtheorem{corollary}[lemma]{Corollary}
\newtheorem{claim}[lemma]{Claim}
\newtheorem{example}[lemma]{Example}

\newcommand{\dee}{\mathrm{d}}
\newcommand{\Dee}{\mathrm{D}}
\newcommand{\In}{\mathrm{in}}
\newcommand{\Out}{\mathrm{out}}
\newcommand{\pdf}{\mathrm{pdf}}
\newcommand{\Cov}{\mathrm{Cov}}
\newcommand{\Var}{\mathrm{Var}}

\newcommand{\ve}[1]{\mathbf{#1}}
\newcommand{\mrm}[1]{\mathrm{#1}}
\newcommand{\etal}{{et~al.}}
\newcommand{\sphere}{\mathbb{S}^2}
\newcommand{\modeint}{\mathcal{M}}
\newcommand{\azimint}{\mathcal{N}}
\newcommand{\ra}{\rightarrow}
\newcommand{\mcal}[1]{\mathcal{#1}}
\newcommand{\X}{\mathcal{X}}
\newcommand{\Y}{\mathcal{Y}}
\newcommand{\Z}{\mathcal{Z}}
\newcommand{\x}{\mathbf{x}}
\newcommand{\y}{\mathbf{y}}
\newcommand{\z}{\mathbf{z}}
\newcommand{\tr}{\mathrm{tr}}
\newcommand{\sgn}{\mathrm{sgn}}
\newcommand{\diag}{\mathrm{diag}}
\newcommand{\Real}{\mathbb{R}}
\newcommand{\sseq}{\subseteq}
\newcommand{\ov}[1]{\overline{#1}}

\title{Multi-Scale Modeling and Rendering of Granular Materials}
\author{Pramook Khungurn}

\begin{document}
  \maketitle

  This document is created as I read the paper ``Multi-Scale Modeling and Rendering of Granular Materials'' \cite{Meng:2015}.

  \section{Overview}
  \begin{itemize}
  	\item The parameter presents a stochastic model for grains.
  	
  	\item It renders grains with three separate algorithms:
  	\begin{itemize}
  		\item Explicit path tracing (EPT) for the finest scale.
  		\item Volume path tracing a homogeneous medium (VPT).
  		\item Diffusion approximnation (DA).  		
  	\end{itemize}  	
  	The paper describes a criteria for switching between them.

  	\item VPT and DA requires suitable parameters.  The paper describes an algorithm for deriving these parameters without performing a per-scene precomputation.  	  	
  \end{itemize}

  \subsection{Stochastic Granular Model}
  \begin{itemize}
  	\item The input for the model are:
  	\begin{itemize}
  		\item A surface of the boundary of the aggregate object (i.e., the macrogeometry).
  		\item A model for the geometry and material properties of individual grains.
  		\item The size and packing rate of the grains.
  		\item Mixing weights if there are multiple types of grains.
  	\end{itemize}
  	
  	\item Assumptions: 
  	\begin{itemize}
  	  	\item Grains are randomly positioned and oriented.
  	  	\item Grains are non-overlapping.
  	\end{itemize}
  	
  	\item Grain arrangement via tiled sphere packing.
  	\begin{itemize}
  		\item They say each grain is contained in a non-overlapping sphere.

  		\item To get the actual arrangement of grains, they procedurally pack the spheres instead of the grains.  
  		\begin{itemize}
  			\item This limits the packing rate for non-spherical grains, but they say this simplifies things.
  		\end{itemize}

  		\item The building block of sphere packing is a \emph{tiled}, which is a cuboid of non-overlapping spheres.

  		\item The tiles are generated by the algorithm of Skoge \etal\ \cite{Skoge:2006} and are repeated throughout the volume.  	
  	\end{itemize}

  	\item For each sphere,
  	\begin{itemize}
  		\item a grain type is chosen randomly according to the mixing weights specified by the user, and
  		\item the grain's orientation is randomized.
  	\end{itemize}
  	To ensure deterministic appearance, the paper sets the random seed for each grain to be based on the tile ID and the sphere ID.  	
  \end{itemize}

  \subsection{Explicit Path Tracing}
  \begin{itemize}
  	\item At the finest level of detail, the paper traces paths through the geometry directly.

  	\item The boundary mesh is voxelized.
  	\begin{itemize}
  		\item Each voxel corresponds to a tile described in the last section.
  		\item Each voxel is either ``fully outside'', ``partially inside'', or ``fully inside.''
  	\end{itemize}
  	 
  	\item When a ray hits a voxel that is at least partially inside, the path tracer intersects the ray with the bounding spheres inside the voxel.

  	\item When intersecting a bounding sphere in a voxel that is partially inside, the path tracer needs to determine whether the sphere is inside or outside the shape.  It does so by tracing the ray from the center of the sphere and see if the ray intersects the bounding surface on the inside or not.  If so, it is in; otherwise, it is not.

  	\item After the hitting bounding sphere is identified, the grain inside is instantialized.  The ray is then transformed to the grain's local coordinate system, and then intersect with the ray.  	
  \end{itemize}

  \subsection{Volumetric Path Tracing}
  \begin{itemize}
  	\item This is basically the standard volume path tracing in a homogeneous medium.

  	\item The parameters needed for this are:
  	\begin{itemize}
  		\item the extinction coefficient $\sigma_t$,
  		\item the scattering albedo $\alpha_s$, and
  		\item the phase function $\Phi$ (isotropic).
  	\end{itemize}

  	\item Unlike DPT, VPT can do emitter sampling, which significantly reduce variance.
  \end{itemize}

  \subsection{Diffusion Approximation}
  \begin{itemize}
  	\item VPT can be costly for long paths.

  	\item To make the computation even cheaper, they transition to a diffusion approximation.

  	\item The paper switches to DA by sampling a location on the boundary mesh and estimating the diffusion transport.

  	\item Components:
  	\begin{itemize}
  		\item A technique based on \cite{Li:2005}.
  		\item d'Eon \etal's improved diffusion model \cite{d'Eon:2011}.
  		\item A Monte Carlo integration scheme \cite{Habel:2013}.
  		\item Multipole expansion to account for finite thickness \cite{Donner:2005}.
  		\item An original virtual source placement procedure.
  	\end{itemize}

  	\item The parameters for this step is the reduced medium parameter ($\sigma_t'$ and $\alpha_s'$), which are obtained from the parameters of VPT using Jensen's \etal's method in the dipole paper \cite{Jensen:2001}. 
  \end{itemize}

  \section{RTE Parameters}

  We need to derive RTE parameters that are consistent with the behavior of EPT.  This is non-trivial because the scattering elements are not points.  The paper introduces a light transport model called the \emph{teleportation transport} (TT), in order to account for non-point scatterings.  Then, it uses the model to derive RTE parameters.

  \subsection{Teleportation Transport Model}

  \begin{itemize}
  	\item The TT model is created to characterize paths generated by EPT in granular material.

  	\item The TT model consists of two steps occur in alternation.
  	\begin{enumerate}
  		\item {\bf Inter-grain transportation} decides how far along a ray to move before the next interaction with a grain bounding sphere.
  		\item {\bf Intra-grain transportation} deals with interaction between the ray and the grain inside and ``transporting'' the ray to the point where it exits the bounding sphere.
  	\end{enumerate}  	
  \end{itemize}

  \subsubsection{Inter-Grain Transport}
  \begin{itemize}
  	\item We have to determine the free-flight distances from one sphere to the next.

  	\item An idea that we might use: just tabulate this distribution like Moon \etal\ did \cite{Moon:2007}. 

  	\item To avoid precomputation, they use a model by Dixmier \cite{Dixmier:1978}:
  	\begin{align*}
  		p_b(z) = \sigma_b e^{-\sigma_b z}\mbox{, with } \sigma_b = \frac{3}{4R}\frac{f}{1-f}.
  	\end{align*}
  	Here, $z$ is the free-flight distance, $R$ is the radius of the sphere, and $f$ is the packing rate of the sphere.  	
  \end{itemize}

  \subsubsection{Intra-Grain Transport}
  \begin{itemize}
  	\item The paper defines the \emph{teleportation scattering distribution function} (TSDF) $S(\ve{x}_i, \omega_i \ra \ve{x}_o, \omega_o)$ where $\ve{x}_i$ and $\ve{x}_o$ are points on the sphere and $\omega_i$ and $\omega_i$ are directions.

  	\item This function can be computed by tracing rays through the sphere and having the ray interact with the grain.

  	\item The paper avoids using this function and tries to derive the RTE parameters from it instead.
  \end{itemize}

  \subsection{RTE \& Diffusion Parameters}

  \begin{itemize}
  	\item {\bf Phase function and albedo}
  	\begin{itemize}
  		\item Due to symmetry of the random grain rotations, the phase function will depend solely on the cosine of the deflecting angle $\cos \theta = \omega_i \cdot \omega_o$.

  		\item The paper tabulate this 1D function. 

  		\item The integral of the 1D distribution over the outgoing directions of the unit hemisphere is the effective albedo $\sigma_s$.
  	\end{itemize}
  	

  	\item {\bf Combined free flight distribution}
  	\begin{itemize}
  		\item We need to extend the spherical free-flight distance $\lambda_b = 1 / \sigma_f$ with what can happen inside the sphere.

  		\item There are two cases:
  		\begin{enumerate}
  			\item The ray that goes into the sphere might not hit the grain inside and just straight through the sphere.
  			\item The ray interacts hits the grain and leaves the sphere.  			
  		\end{enumerate}

  		\item Let $\beta$ be the probability that the ray that goes into the sphere does not hit anything.  This can be estimated by just tracing rays.  This is called the \emph{hit probability}.

  		\item Let the $\lambda_\delta$ be the mean length of the portion of unscattered rays inside the sphere.  This can be computed by simulation as well.

  		\item Here's what can happens after a ray leaves a sphere before it hits something inside the sphere.
  		\begin{itemize}
  			\item It travels a distance of $\lambda_b$ and then hits something inside.\\
  			The contribution of this case is $\beta \lambda_b$.

  			\item It travels a distance of $\lambda_b$, hits nothing inside so travels distance $\lambda_\delta$, and then travels a distance of $\lambda_b$ again, before hitting something inside.\\
  			The contribution of this case is $\beta (1-\beta) (2\lambda_b + \lambda_\delta)$.

  			\item It travels, not hit, travels, not hit, travels, and then hits.\\
  			The contribution of this case is $\beta (1-\beta)^2 (3\lambda_b + 2\lambda_\delta)$.
  		\end{itemize}
  		
  		\item So, the distance before the ray hits something is:
  		\begin{align*}
  			\lambda_\beta 
  			= \beta \sum_{i=1}^n (1 - \beta)^i [i(\lambda_b + \lambda_\delta) + \lambda_b]
  			= (\lambda_b + \lambda_\delta) \frac{1 - \beta}{\beta} + \lambda_b.
  		\end{align*}

  		\item Once the ray hits something inside, it has probability $\alpha_s$ of coming out.  With simulation, we can estimate the mean teleport vector $\ve{x}_o - \ve{x}_i$.  The length of which we denote $\lambda_s$. 

  		\item The mean free-flight length is then given by:
  		\begin{align*}
  			\lambda_t = \lambda_\beta + \alpha_s \lambda_s.
  		\end{align*}

  		\item The probability distribution of the free-flight distance is then given by:
  		\begin{align*}
  			p_t(z) = \sigma_t e^{-\sigma_t z}
  		\end{align*}
  		where $\sigma_t = 1 / \lambda_t$.  		
  	\end{itemize}

  	\item The diffusion prameters are given by:
  	\begin{align*}
  		\sigma'_s &= (1 - g)\alpha_s \sigma_t \\
  		\sigma'_t &= \sigma'_s + (1 - \alpha_s)\sigma_t \\
  		\alpha'_s &= \sigma'_s / \sigma'_t,
  	\end{align*}
  	where $g$ is the mean scattering cosine computed from the phase function.
  \end{itemize}

  \section{Switching between Rendering Techniques}

  \begin{itemize}
    \item Rays are assume to start outside the granular material volume.

    \item At first, renderings are performed with EPT.

    \item {\bf EPT $\ra$ VPT}
    \begin{itemize}
      \item Spawn a bundle of $N = 16$ rays per pixel.  These rays are traced in lock-step.

      \item At each bounce $k$, suppose there are $N_k$ rays left.  (Ray can be terminated by Russian roulette and other reasons.)  Compute the standard deviation $\sigma_k$ of $N_k$ of the vertex positions of the $N_k$ rays.

      \item The algorithm switches to VPT when:
      \begin{align*}
        \sigma_k > \tau \frac{N_k}{N}
      \end{align*}
      where $\tau$ is a user-specified multiple of the maximum grain radius $r$.

      \item The paper uses $\tau = 4$.
    \end{itemize}

    \item {\bf VPT $\ra$ DA}
    \begin{itemize}
      \item Measure the minimum distance between the vertex position $\ve{x}_k$ at the $k$th bounce to the boundary surface.

      \item The paper switches to DA once the distance above is greater than $\min(1/\sigma'_t, 0.5/\sigma_{tr} )$.  (See the definition of this in the Stam paper \cite{Stam:1995}.)

      \item The criteria above allows the algorithm to switch to DA more quickly in material with lower albedo.      
    \end{itemize}

    \item {\bf Russian roulette}
    \begin{itemize}
      \item EPT converges much more slower than VPT and DA.  The latter two algorithms converges very fast, and not so many samples are needed.

      \item So, we can cut time spending on VPT and DA by introducing Russian roulette when we switch.

      \item The paper introduces the probably of continuing with VPT and DA, which they call $P_a$.

      \item The optimal $P_a$ is scene dependent, but the paper introduces a way to compute it with some precomputation.

      \item The paper renders an image $1\%$ of the original size with the same number of pixels as the original image.

      \item In this rendering, for each pixel $(x,y)$ in the small image:
      \begin{itemize}
        \item the sample variance of the low order (EPT) contributions $V_L(x,y)$,
        \item the sample variance of the high order (VPT and DA) contributions $V_H(x,y)$,
        \item the CPU time for computing the low order contributions $t_L(x,y)$, and
        \item the CPU time for computing the high order contributions $t_H(x,y)$.
      \end{itemize}

      \item The variance of the combined image as a function of the number of samples $n$ and the acceptance rate $P_a$ is:
      \begin{align*}
        V \approx \frac{1}{n} \bigg( V_L + \frac{V_H}{P_a} \bigg)
      \end{align*}
      where $V_H$ and $V_L$ are averages of $V_H(x,y)$ and $V_L(x,y)$ across the image.

      \item The total time needed to render is:
      \begin{align*}
        t = n (t_L + P_a t_H)
      \end{align*}
      where $t_L$ and $t_H$ are averages across the image as well.

      \item Solving for $P_a$ that miniminzes $t$, we have :
      \begin{align*}
        P_a = \sqrt{\frac{V_H t_L}{V_L t_H}}.
      \end{align*}
      (It is unclear what constraint they optimize against.  Do they set $V$ to be a specific value and use the constraint $V \leq (V_L + V_H/P_a)/n$?)
    \end{itemize}
  \end{itemize}

  \bibliographystyle{apalike}
  \bibliography{granular-rendering}  
\end{document}